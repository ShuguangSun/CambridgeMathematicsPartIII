\chapter{Stochastic Processes}

\section{Stochasticity}
\begin{defn}
  There is a one-to-one correspondence between the class $Q$ of all
  probability measures equivalent to $P$ and the class $\Lambda$ of
  all $F$-adapted (or $F$ predicatable) processes $\lambda_{t}$
  satisfying
  \begin{equation}
    P \left(\int_{0}^{T^{*}} |\lambda_{u}|^{2} \left(\int_{0}^{\cdot}
        \lambda_{u} dW_{u} \right) \right) = 1
  \end{equation}
  and the set of all probability measures on $\mathbb{R}^{n}$.
\end{defn}

\begin{thm}
  There is a one-to-one correspondence between $Q$ and $P$. 
\end{thm}

Then \emph{apples} and \textbf{bananas}

\begin{proof}
  This follows as a corollary to 14.2.  In general, this is a simple
  result that follows naturally from the text.
\end{proof}

\section{Processes}
\label{sec:processes}

Consider the variables $X, Y$ and ignore the $\delta$ inherent in
their makeup. Then, the set of variables $\alpha_{i}, i = 1, 2, ...$
form a commutative semigroup with identity $\alpha_{0}$ -
associativity, distribution, etc fall out neatly from the group
axioms.

\begin{defn}
  Very impressive stuff
\end{defn}

\begin{proof}
  Follows naturally from the theorem
\end{proof}

%%% Local Variables:
%%% TeX-master: "master"
%%% End: