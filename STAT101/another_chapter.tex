\chapter{Another Chapter}
\label{cha:another-chapter}

\section{Group Theory}
\label{sec:group-theory}

\begin{defn}
  A monoid is a set $S$ equipped with a single operation $\cdot$
  obeying the following axioms:
  \begin{enumerate}
  \item \textbf{Closure} for all $a, b \in S$, $a \cdot b \in S$.
  \item \textbf{Associativity} For all $a,b,c \in S$, $(a \cdot b)
    \cdot c = a \cdot (b \cdot c)$.
  \item \textbf{Identity} There exists an element $e$ in $S$ such that
    $e \cdot a = a = a \cdot e$ for all $a \in S$.
  \end{enumerate}
\end{defn}

\begin{defn}
  A group is a set equipped with an operation $\cdot$ obeying the
  axioms of associateivity, existence of inversese, and existence of
  an identity.
\end{defn}

\begin{defn}
  An abelian group is a group where the operation $\cdot$ is
  commutative.
\end{defn}

\begin{defn}
  A cyclic group is a group that can be generated by a signel element
  $\langle x \rangle = \{ x^{n} | x \in \mathcal{Z} \}$.
\end{defn}

\begin{defn}
  A subset $H$ of a group $G$ is a subgroup of $G$ if and only if $H$
  is non-empty and for all $x, y \in G$,
  \begin{enumerate}
  \item If $x, y \in H$, then $x \cdot y \in H$, and
  \item If $x \in H$, then $x^{-1} \in H$.
  \end{enumerate}
\end{defn}

\begin{align*}
  a & = b \\
    & = c
\end{align*}

\begin{align*}
  a & = b \\
    & = c
\end{align*}

\section{Alpha}
\label{sec:alpha}

\citet{asdf} suggest an alternative interpretation.

%%% Local Variables: 
%%% mode: latex
%%% TeX-master: "master"
%%% End: 
