
\chapter{Conditional Expectation}
\label{cha:cond-expect}

Let $(\Sigma, \mathcal{F}, \mathbb{P})$ be a probability space.
$\Sigma$ is a set, $\mathcal{F}$ is a $\sigma$-algebra on $\Omega$,
and $\mathbb{P}$ is a probability measure on ($\Omega, \mathcal{F}$).

\begin{defn}
  \label{defn:1}
  $\mathcal{F}$ is a $\sigma$-algebra on $\Sigma$ if it satisfies
  \begin{itemize}
  \item $\emptyset, \Sigma \in \mathcal{F}$
  \item $A \in \mathcal{F} ==> A^{c} \in \mathcal{F}$
  \item $(A_n)_{n \geq 0}$ is a collection of sets in $\mathcal{F}$
    then $\cup_{n} A_{n} \in \mathcal{F}$.
  \end{itemize}
\end{defn}

\begin{defn}
  \label{defn:2}
  $\mathbb{P}$ is a probability measure on $(\Omega, \mathcal{F})$ if
  \begin{itemize}
  \item $\mathbb{P}: \mathcal{F} \rightarrow [0, 1]$ is a set function.
  \item $\mathbb{P}(\emptyset) = 0$, $\mathbb{P}(\Omega) = 1$,
  \item $(A_{n})_{n \geq 0}$ is a collection of pairwise disjoint
      sets in $\mathcal{F}$, then $\mathbb{P}(\cup_{n} A_{n}) =
      \sum_{n} \mathbb{P}(A_{n})$.
  \end{itemize}
\end{defn}

\begin{defn}
  \label{defn:3}
  The Borel $\sigma$-algebra $\mathbb{B}(\mathbb{R})$ is the
  $\sigma$-algebra generated by the open sets of $\mathbb{R}$.  Call
  $\mathcal{O}$ the collection of open subsets of $\mathbb{R}$, then
  \begin{equation}
    \label{eq:1}
    \mathbb{B}(\mathbb{R}) = \cap \{ \xi: \xi \text{ is a sigma
      algebra containing $\mathcal{O}$} \}
  \end{equation}
\end{defn}

\begin{defn}
  \label{defn:4}
  $\mathcal{A}$ a collection of subsets of $\Omega$, then we write
  $\sigma(\mathcal{A}) = \cap \{ \xi: \xi \text{ a sigma algebra
    containing $\mathcal{A}$} \}$
\end{defn}

\begin{defn}
  \label{defn:5}
  $X$ is a random variable on $(\Omega, \mathcal{F})$  if $X: \Omega
  -> \mathbb{R}$ is a function with the proerty that $X^{-1}(V) \in
  \mathcal{F}$ for all $V$ open sets in $\mathbb{R}$.
\end{defn}

Exercise: If $X$ is a random varaible then $\{ B \subseteq \mathbb{R}, X^{-1}(B)
\in \mathcal{F} \}$ is a $\sigma$-algebra and contains
$\mathbb{B}(\mathbb{R})$.

If $(X_{i}, i \in I)$ is a collection of random varaibles, then we
write $\sigma(X_{i}, i \in I) = \sigma(\{ \omega \in \Omega:
X_{i(\omega) \in \mathcal{B} \}, i \in I, B \in
  \mathbb{B}(\mathbb{R}))) }$
and it is the smallest $\sigma$ algebra that makes all the $X_{i}$'s
measurable.

\begin{defn}
  \label{defn:6}
  First we define it for the positive simple random variables.

  \begin{equation}
    \label{eq:2}
    \mathbb{E}[\sum_{i=1}^{n} c_{i} \mathbf{1}(A_{i})] =
    \sum_{i=1}^{n} \mathbb{P}(A_{i}).
  \end{equation}
  with $c_{i}$ positive constants, $(A_{i}) \in \mathcal{F}$.

  We can extend this to any positive random variable $X \geq 0$ by
  approximation $X$ as the limit of piecewise constant functions.

  For a general $X$, we write $X = X^{+} - X^{-}$ with $X^{+}= \max(X,
  0), X^{-} = \max(-X, 0)$.
\end{defn}

If at least one of $\mathbb{E}[X^{+}]$ or $\mathbb{E}[X^{-}]$ is
finite, then we define $\mathbb{E}[X] = \mathbb{E}[X^{+}] +
\mathbb{E}[X^{-}]$.

We call $X$ integrable if $\mathbb{E}[|X|] < \infty$.

\begin{defn}
  \label{defn:7}
  Let $A, B \in \mathcal{F}, \mathbb{P}(B) > 0$. Then
  \begin{align*}
    \mathbb{P}(A | B) = \frac{\mathbb{P}(A \cap B)}{\mathbb{P}(B)} \\
    \mathbb{E}[X | B] = \frac{\mathbb{E}[X \mathbb{1}(B)]}{\mathbb{P}(B)}
  \end{align*}
\end{defn}

Goal - we want to define $\mathbb{E}[X | \mathcal{G}]$ that is a
random variable measurable with respect to the $\sigma$-algebra
$\mathcal{G}$.

\subsection{Discrete Case}
\label{sec:discrete-case}

Suppose $\mathcal{G}$ is a $\sigma$-algebra countably generated
$(B_{i)_{i \in \mathbb{N}}}$ is a collection of pairwise disjoint sets
in $\mathcal{F}$ with $\cup B_{i} = \Omega$. Let $\mathcal{G} =
\sigma(B_{i}, i \in \mathbb{N})$.

It is easy to check that $\mathcal{G} = \{ \cup_{j \in J} B_{j}, J
\subseteq {N} \}$.

Let $X$ be an integrable random variable.  Then
\begin{align*}
  X' = \mathbb{E}[X | \mathcal{G}] = \sum_{i \in \mathbb{N}}
  \mathbb{E}[X | \mathcal{B_{i}}] \mathbb{1}(B_{i})
\end{align*}

\begin{enumerate}
\item $X'$ is $\mathcal{G}$-measurable (check).
\item
  \begin{equation}
    \label{eq:3}
    \mathbb{E}[|X'|] \leq \mathbb{E}[|X|]
  \end{equation} and so $X'$ is integrable.
\item $\forall G \in \mathcal{G}$, then
  \begin{equation}
    \label{eq:4}
    \mathbb{E}[X 1(G)] = E[X' 1(G)]
  \end{equation} (check).
\end{enumerate}

\section{Existence and Uniqueness}
\label{sec:existence-uniqueness}

\begin{defn}
  \label{defn:8}
  $A \in \mathcal{F}$, $A$ happens almost surely (a.s.) if
  $\mathbb{P}(A) = 1$.
\end{defn}

\begin{thm}[Monotone Convergence Theorem]
  If $X_{n} \geq 0$ is a sequence of random variables and $X_{n}
  \uparrow X$ as $n \rightarrow \infty$ a.s, then
  \begin{equation}
    \label{eq:5}
    E[X_{n}] \uparrow E[X]
  \end{equation} almost surely as $n \rightarrow \infty$.
\end{thm}

\begin{thm}[Dominated Convergence Theorem]
  If $(X_{n})$ is a sequence of random variables such that $|X_{n}|
  \leq Y$ for $Y$ an integrable random variable, then if $X_{n} \cas
  X$ then $E[X_{n}] \cas E[X]$. 
\end{thm}

\begin{defn}
  \label{defn:9}
  For $p \in [1, \infty)$, $f$ measruable functions, then
  \begin{align}
    \| f \|_{p} = E[|f|^{p}]^{\frac{1}{p}} \\
    \| f \|_{\infty} = \inf \{ \lambda : |f| \leq \lambda a.e. \}
  \end{align}
\end{defn}

\begin{defn}
  \label{defn:10}
  \begin{align*}
    L^{p} = L^{p}(\Omega, \mathcal{F}, \mathbb{P}) = \{ f : \| f
    \|_{p} < \infty \} \\
  \end{align*}
  Formally, $L^{p}$ is the collection of equivalence classes where two
  functions are equivalent if they are equal a.e.  We will just
  represent an element of $L^{p}$ by a function, but remember that
  equality is a.e.
\end{defn}

\begin{thm}
  The space $(L^{2}, \| \cdot \|_{2})$ is a Hilbert space with
  $\langle U, V \rangle> = E[UV]$.

  Suppose $\mathcal{H}$ is a closed subspace, then $\forall f \in
  L^{2}$ there exists a unique $g \in \mathcal{H}$ such that $\| f - g
  \|_{2} = \inf \{ \| f- h \|_{2}, h \in \mathcal{H}$ and $\langle f
  -g, h \range = 0$ for all $h \in \mathcal{H}$.

  We call $g$ the orthogonal projection of $f$ onto $\mathcal{H}$.
\end{thm}

%%% Local Variables: 
%%% mode: latex
%%% TeX-master: "master"
%%% End: 
