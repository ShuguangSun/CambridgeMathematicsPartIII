
\chapter{Discrete Time Martingales}
\label{cha:discr-time-mart}

Let $(\Omega, \mathcal{F}, \Prob)$ be a probabilty space and $(E,
\xi)$ a measurable space.  Usually $E = \R, \R^{d}, \mathbb{C}$.  For
us, $E = \R$.  A sequence $X = (X_{n})_{n \geq 0}$ of random variables
taking values in $E$ is called a \textbf{stochastic process}.

A \textbf{filtration} is an increasing family $(\mathcal{F}_{n})_{n
  \geq 0}$ of sub-$\sigma$-algebras of $\mathcal{F}_{n}$, so
$\mathcal{F}_{n} \subseteq \mathcal{F}_{n+1}$.

Intuitively, $\mathcal{F}_{n}$ is the information available to us at
time $n$.  To every stochsatic process $X$ we assoicate a filtration
called the natural filtration
\begin{equation}
  \label{eq:24}
  (\mathcal{F}^{X}_{n})_{n \geq 0}, \mathcal{F}_{n}^{X} =
  \sigma(X_{k}, k \leq n)
\end{equation}

A stochastic process $X$ is called adapted to $(\mathcal{F}_{n})_{n
  \geq 0}$ if $X_{n}$ is $\mathcal{F}_{n}$-measurable for all $n$.

A stochastic process $X$ is called integrable if $X_{n}$ is integrable
for all $n$.

\begin{defn}
  \label{defn:discrete_time_martingales:1}
  An adapted integrable process $(X_{n})_{n \geq 0}$ taking values in
  $\R$ is called a
  \begin{enumerate}
  \item \textbf{martingale} if $\E{X_{n} | \mathcal{F}_{m}} = X_{m}$
    for all $n \geq m$.
  \item \textbf{super-martingale} if $\E{X_{n} | \mathcal{F}_{m}} \leq
    X_{m}$ for all $n \geq m$.
  \item \textbf{sub-martingale} if $\E{X_{n} | \mathcal{F}_{m}} \geq
    X_{m}$ for all $n \geq m$.
  \end{enumerate}
\end{defn}

\begin{remark}
  A (sub,super)-martingale with respect to a filtration
  $\mathcal{F}_{n}$ is also a (sub, super)-martingale with respect to
  the natural filtration of $X_{n}$ (by the tower property)
\end{remark}


\begin{exmp}
  \label{defn:discrete_time_martingales:2}
  Suppose $(\xi_{i})$ are IID random variables with $\E{\xi_{i}} = 0$.
  Set $X_{n} = \sum_{i=1}^{n} \xi_{i}$.  Then $(X_{n})$ is a martingale.
\end{exmp}

\begin{exmp}
  \label{defn:discrete_time_martingales:3}
  As above, but let $(\xi_{i})$ be IID with $\E{\xi_{i}} = 1$. Then
  $X_{n} = \Pi_{i=1}^{n} \xi_{i}$ is a martingale.
\end{exmp}

\begin{defn}
  \label{defn:discrete_time_martingales:4}
  A random variables $T: \Omega \rightarrow \mathbb{Z}_{+} \cup \{ \infty \}$
  is called a stopping time if $\{ T \leq n \} \in \mathcal{F}_{n}$
  for all $n$.  Equivalently, $\{ T = n \} \in \mathcal{F}_{n}$ for
  all $n$.
\end{defn}

\begin{exmp}
  \label{defn:discrete_time_martingales:5}
  \begin{enumerate}
  \item Constant times are trivial stopping times.
  \item $A \in \mathcal{B}(\R)$.  Define $T_{A} = \inf \{ n \geq 0 |
    X_{n} \in A \}$, with $\inf \emptyset = \infty$.  Then $T_{A}$ is a
    stopping time.
  \end{enumerate}
\end{exmp}

\begin{proposition}
  Let $S, T, (T_{n})$ be stopping times on the filtered probability
  space $(\Omega, \mathcal{F}, (\mathcal{F}_{n}), \Prob)$.  Then $S \wedge
  T, S \vee T, \inf_{n} T_{n}, \liminf_{n} T_{n}, \limsup_{n} T_{n}$
  are stopping times. 
\end{proposition}

\begin{notation}
  $T$ stopping time, then $X_{T}(\omega) = X_{T(\omega)}(\omega)$. The
  stopped process $X^{T}$ is defined by $X^{T}_{t} = X_{T \wedge t}$.

  $\mathcal{F}_{T} = \{ A \in \mathcal{F} | A \cap {T \leq T} \in
  \mathcal{F}_{t}, \forall t \}$.
\end{notation}

\begin{proposition}
  $(\Omega, \mathcal{F}, (\mathcal{F}_{n}), \Prob)$, $X = (X_{n})_{n
    \geq 0}$ is adapted.
  \begin{enumerate}
  \item $S \leq T$, stopping times, then $\mathcal{F}_{S} \subseteq \mathcal{F}_{T}$
  \item $X_{T} \I{T < \infty}$ is $\mathcal{F}_{T}$-measurable.
  \item $T$ a stopping time, then $X^{T}$ is adapted
  \item If $X$ is integrable, then $X^{T}$ is integrable.
  \end{enumerate}
\end{proposition}

\begin{proof}
  Let $A \in \xi$.  Need to show that $\{ X_{T} \I{T < \infty} \in A
  \} \in \mathcal{F}_{T}$.
  \begin{align}
    \label{eq:25}
    \{ X_{T} \I{T < \infty} \} \cap \{ T \leq t \} = \cup_{s \leq t}
    \left( \underbrace{\{ T = s \}}_{\mathcal{F}_{s} \subseteq
        \mathcal{F}_{t}} \cap \underbrace{\{ X_{s} \in A \}}_{\in
        \mathcal{F}_{s} \subseteq \mathcal{F}_{t}} \right) \in \mathcal{F}_{t}
  \end{align}
\end{proof}


\section{Optional Stopping}
\label{sec:optional-stopping}

\begin{thm}
  \label{defn:discrete_time_martingales:6}
  Let $X$ be a martingale.
  \begin{enumerate}
  \item If $T$ is a stopping time, then $X^{T}$ is also a martingale.
    In particular, $\E{X_{T \wedge t}} = \E{X_{0}}$ for all $t$.
  \end{enumerate}
\end{thm}

\begin{proof}
  By the tower property, it is sufficient to check
  \begin{align*}
    \label{eq:26}
    \E{X_{T \wedge t} | \mathcal{F}_{t - 1}} &= 
    \E{\sum_{i=1}^{t-1} X_{s} \underbrace{\I{T=s}}_{\in
        \mathcal{F}_{s} \subseteq \mathcal{F}_{t-1}}  |
      \mathcal{F}_{t-1}} + \E{X_{t} \I{T > t -1} | \mathcal{F}_{t-1}}
    \\
    &= \sum_{s=0}^{t-1} \I{T=s} X_{s} + \I{t>t-1}X_{t-1} = X_{T \wedge (t-1)}
  \end{align*}  Since it is a martingale, $\E{X_{T \wedge t}} = \E{X_{0}}$.
\end{proof}


%%% Local Variables: 
%%% mode: latex
%%% TeX-master: "master"
%%% End: 
