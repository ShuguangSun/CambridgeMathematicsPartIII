\chapter{Introduction}
\label{cha:introduction}

\section{Basic Concepts}
\label{sec:basic-concepts}

\begin{thm}[The Delta Method]
  Let $Y_{n}$ be a sequence of random vectors in $\R^{d}$ such that
  for some $\mu \in \R^{d}$ and a random vector $Z$, we have
  $n^{\frac{1}{2}}(Y_{n} - \mu) \cd Z$.  If $g: \R^{d} \rightarrow \R$
  is differentiable at $\mu$, then $n^{\frac{1}{2}}(g(Y_{n}) - g(\mu))
  \cd \grad g(\mu)^{T} Z$.
\end{thm}
\begin{proof}
  For $d = 1$.  Let $g'(\mu) = \grad g(\mu)$, and let $h: \R
  \rightarrow \R$, by
  \begin{equation}
    \label{eq:134}
    h(y) =
    \begin{cases}
      \frac{g(y) - g(\mu)}{y - \mu}  & y \neq \mu \\
      g'(\mu) & y = \mu
    \end{cases}
  \end{equation}  Then by the continuous mapping theorem and Sltusky's
  theorem, $n^{\frac{1}{2}}(g(Y_{n}) - g(\mu)) = h(Y_{n})
  n^{\frac{1}{2}}(Y_{n} - \mu) \cd g'(\mu) Z$.
\end{proof}

\subsection{Parametric vs Nonparametric models}
\label{sec:param-vs-nonp}

A statistical model postulates a family of possible data generating
mechanisms.  Examples include:
\begin{enumerate}
\item Let $X_{1}, \dots, X_{n} \sim T(m, \theta)$ \iid, with $m$ known
  and $\theta \in (0, \infty) = \Theta$ an unknown parameter.
\item Let $Y_{i} = \alpha + \beta x_{i}+ \epsilon_{i}, i = 1, \dots,
  n$ where $x_{i}$ are known and $\epsilon_{i}$ are \iid with
  $\E{e_{i}} = 0, \Var{\epsilon_{i}} = \sigma^{2}$.  Here, the unknown
  parameter is $\theta =
  \begin{pmatrix}
    \alpha \\
    \beta \\
    \sigma^{2}
  \end{pmatrix} \in \R \times \R \times (0, \infty) = \Theta$.
\end{enumerate}

If the parameter space $\Theta$ is finite dimensional, we speak of a
\textbf{parametric model}.  In such situations, typically we can
estimate $\theta$ using the MLE $\hat \theta_{n}$, and have $\hat
\theta_{n} - \theta = O_{p}(n^{-\frac{1}{2}})$.\footnote{Definition of
  $O_{p}$ - TODO}

This assumes the model contains the true data generating process, if
not, inference can be misleading.

Examples of nonparametric models include:
\begin{enumerate}
\item \label{item:1} Let $X_{1}, \dots, X_{n}, i = 1, \dots, n$ be
  \iid with arbitrary distribution function $F$.
\item \label{item:2} Let $X_{1}, \dots, X_{n}, i = 1, \dots, n$ be
  \iid with twice continuously differentiable density $f$.
\item \label{item:3} Let $Y_{i} = m(x_{i}) + v(x_{i})^{\frac{1}{2}}, i
  = 1, \dots, n$ where $m$ is twice continuously differentiable and
s  $\epsilon_{1}, \dots, \epsilon_{n}$ are \iid with $\E{\epsilon_{i}}
  = 0, \Var{\epsilon_{i}} = 1$.
\end{enumerate}

Such infinite-dimensional models are much less vulnerable to model
misspecification, typically, however we pay a price for our generality
in terms of a slower convergence rate - e.g. $O_{p}(n^{-\frac{2}{3}})$
in problems \ref{item:2} and \ref{item:3} above.

\subsection{Estimating an arbitrary distribution function}
\label{sec:estim-an-arbitr}

Let $X_{1}, \dots, X_{n}$ be \iid on a probability space $(\Omega,
\mathcal{F}, \Prob)$ with distribution function $F$.  The
\textbf{empirical distribution function} $\hat F_{n}$ is defined by
\begin{equation}
  \label{eq:1}
  \hat F_{n}(x) = \frac{1}{n} \sum_{i=1}^{n} \I{X_{i} \leq x}.
\end{equation}


\begin{thm}[Glivenko-Cantelli (1933) - The Fundamental Theorem of Statistics]
  \label{defn:Introduction:2}
  \begin{equation}
    \label{eq:2}
    \sup_{x \in \R} \left| \hat F_{n}(x) - F(x) \right| \cas 0.
  \end{equation}
\end{thm}

\begin{proof}
  Given $\epsilon > 0$, choose a partition $-\infty = x_{0} < x_{1} <
  \dots < s_{k} = \infty$ such that, for each $i = 1, \dots, k$, we
  have $F(x_{i}-) - F(x_{i-1}) \leq \epsilon$, where $F(x-) = \lim_{y
    \uparrow x} F(y)$.

  Note that any point at which $F$ jumps by more than $\epsilon$ must
  be in the partition.  By the strong law of large numbers, there
  exists an event $\Omega_{\epsilon}$ with $\Prob{\Omega_{\epsilon}} =
  1$ such that for all $\omega \in \Omega_{\epsilon}$, there exists
  $n_{0} = n_{0}(\omega, \epsilon)$ with
  \begin{align}
    \label{eq:3}
    \left| \hat F_{n}(x_{i}) - F(x_{i}) \right| \leq \epsilon, i = 1,
    \dots, k - 1, n \geq n_{0}, \\
    \left| \hat F_{n}(x_{i}-) - F(x_{i}-) \right| \leq \epsilon, 1 =
    i, \dots, k-1, n \geq n_{0}.
  \end{align}

  Now, fix $x \in \R$, and find $i \in \{ 1, \dots, k \}$ with $x \in
  [x_{i-1}, \dots, x_{i})$.  Then for $\omega \in \Omega_{\epsilon}$
  and $n \geq n_{0}$,
  \begin{align}
    \label{eq:4}
    \hat F_{n}(x) - F(x) \leq \hat F_{n}(x_{i}-) - F(x_{i-1}) = \hat
    F_{n}(x_{i}-) - F(x_{i}-) + F(x_{i}-) - F(x_{i-1}) \leq \epsilon +
    \epsilon = 2\epsilon
  \end{align}

  Similarly, we have
  \begin{align}
    \label{eq:5}
    F(x) - \hat F_{n}(x) \leq F(x_{i}-) - \hat F_{n}(x_{i-1}) =
    F(x_{i}-) - F(x_{i-1}) + F(x_{i-1}) - \hat F_{n}(x_{i-1}) \leq
    \epsilon + \epsilon = 2 \epsilon
  \end{align}

  We deduce that
  \begin{align}
    \label{eq:6}
    \Prob{\sup_{x \in \R} \left| \hat F_{n}(x) - F(x) \right|
    \rightarrow 0} &= \Prob{\cap_{m=1}^{\infty} \cup_{n_{0} =
      1}^{\infty} \cap_{n=n_{0}}^{\infty} \{ \sup_{x \in \R} \left|
      \hat F_{n}(x) - F(x) \right| \leq \frac{1}{m} \}} \\
  &= \lim_{m \rightarrow \infty} \Prob{\Omega_{\frac{1}{2m}}} = 1
\end{align}
\end{proof}

\begin{thm}[Dvortesky-Kiefer-Wolfowizt]
  \label{defn:Introduction:1}
  Let $X_{1}, \dots, X_{n} \sim F$ \iid.  Then for every $\epsilon >
  0$,
  \begin{align}
    \label{eq:7}
    \Prob{\sup_{x \in \R}|\hat F_{n}(x) - F(x)| \geq \epsilon} \leq 2 e^{-2n\epsilon^{2}}.
  \end{align}
\end{thm}

An application is to consider the problem of finding a confidence band
for $F$ at $1-\alpha$. Given $\alpha \in (0, 1)$, set $\epsilon_{n}
=(-\frac{1}{2n} \log \frac{\alpha}{2})^{\frac{1}{2}}$.  Then by
\ref{defn:Introduction:1},
\begin{align}
  \label{eq:8}
  (\max(\hat F_{n}(x) - \epsilon_{n}, 0), \min(\hat F_{n}(x), 1))
\end{align} is a $1-\alpha$ confidence interval for $F$.

In fact, let $U_{1}, \dots, U_{n} \sim U(0, 1)$ \iid, and let $\hat
G_{n}$ denote their empirical distribution function.  Then
\begin{align}
  \label{eq:9}
  \hat G_{n}(F(x)) = \frac{1}{n} \sum_{i=1}^{n} \I{U_{i} \leq F(x)} =
  \frac{1}{n} \sum_{i=1}^{n} \I{F^{-1}(u_{i}) \leq x} = \frac{1}{n}
  \sum_{i=1}^{n} \I{X_{i} \leq x} = \hat F_{n}(x)
\end{align}

It follows that
\begin{align}
  \label{eq:10}
  \sup_{x \in \R} |\hat F_{n}(x)- F(x)| = \sup_{x \in R} |\hat
  G_{n}(F(x)) - F(x)| \leq \sup_{t \in (0, 1)} |\hat G_{n}(t) - t |
\end{align} with equality if $F$ is continuous.  We deduce that, if
$F$ is continuous, the distribution of $\sup_{x \in \R} | \hat
F_{n}(x) - F(x)|$ does not depend on $F$.

Other examples include Uniform Laws of Large Numbers (ULLN).  Let
$X, X_{1}, X_{2}, \dots$ be \iid taking values in a measurable space
$(\mathcal{X}, \mathcal{A})$, and let $\mathcal{G}$ denote a class of
measurable functions on $\mathcal{X}$.  We say that $\mathcal{G}$
satisfies a ULLN if
\begin{equation}
  \label{eq:11}
  \sup_{g \in \mathcal{G}} | \frac{1}{n} \sum_{i=1}^{n} g(X_{i}) -
  \E{g(X)}| \cas 0.
\end{equation}
Thus Theorem 1 shows that the class $\mathcal{G} = \{ \I{\cdot \leq
  x}: x \in \R \}$ satisfies a ULLN.  In general, proving a ULLN
amounts to controlling the \textbf{size} of $\mathcal{G}$, which can
be done by using the idea of entropy (c.f. Statistical Theory).

Further results start with the observation that
\begin{align}
  \label{eq:12}
  n^{\frac{1}{2}} (\hat F_{n} - F(x)) \cd N(0, F(x)(1-F(x)))
\end{align} by the central limit theory.  This result can be
strengthened by studying $\{ n^{\frac{1}{2}}(\hat F_{n}(x) - F(x)), x
\in \R \}$ as a stochastic process.

\begin{proposition}
  Let $U_{1}, \dots, U_{n} \sim U(0, 1)$ \iid.  Let $Y_{1}, \dots,
  Y_{n+1} \sim \textsc{Exp}(1)$ \iid and let $S_{j} = \sum_{i=1}^{j}
  Y_{i}$ for $j = 1, \dots, n+1$.  Then
  \begin{align}
    \label{eq:13}
    U_{j} =^{d} \frac{S_{j}}{S_{n+1}} \sim \textsc{Beta}(j, n-j+1).
  \end{align}
\end{proposition}

\begin{defn}
  \label{sec:let-x_1-dots}
  For $p \in (0, 1]$, the quartile function is defined by $F^{-1}(p) =
  \inf \{x \in \R: F(x) \geq p \}$ and is left-continuous.

  The sample quartile function is $\hat F_{n}^{-1}(p) = \inf \{ x \in
  \R: \hat F_{n}(x) \geq p \} $. 
\end{defn}

\begin{thm}
  Let $U_{1}, U_{2}, \dots, U_{n} \sim U(0, 1)$ \iid and $p \in (0,
  1)$.  Then
  \begin{equation}
    \label{eq:135}
    \sqrt{n}(U_{\lceil np \rceil} - p) \cd N(0, p(1-p)).
  \end{equation}
\end{thm}

\begin{proof}
  Let $Y_{1}, \dots, Y_{n}$ \iid \textsc{Exp}(1), let $V_{n} = Y_{1} + \dots +
  Y_{\lceil np \rceil}$ and $W_{n} = Y_{\lceil np \rceil + 1}, \dots,
  Y_{n+1}$.  Note that $V_{n}, W_{n}$ are independent and
  \begin{equation}
    \label{eq:136}
    \frac{V_{n}}{V_{n} + W_{n}} =^{d} U_{\upperbound{np}}
  \end{equation} by previous proposition.
  Then
  \begin{align}
    \label{eq:137}
    \sqrt{n}(\frac{V_{n}}{n} - p) =
    \frac{\sqrt{\upperbound{np}}}{\sqrt{n}} (\frac{V_{n} -
      \upperbound{np}}{\sqrt{\upperbound{np}}} ) +
    \frac{\upperbound{np} - np}{\sqrt{n}} \cd N(0, p)
  \end{align} by the CLT and Slutsky's theorem.

  Similarly, $\sqrt{n}(\frac{W_{n}}{n} - q) \cd N(0, q)$, where $q = 1
  - p$, then by the delta method, with $g(x, y) = \frac{x}{x+y}$,
  \begin{align}
    \label{eq:138}
    \sqrt{n}(U_{\upperbound{np}} - p) &=^{d}
    \sqrt{n}(g(\frac{V_{n}}{n}, \frac{W_{n}}{n}) - g(p, q)) \\
    &\cd N(0, \grad g(p, q)^{T}
    \begin{pmatrix}
      p & 0 \\
      0 & q
    \end{pmatrix} \grad g(p, q)) \\
    \label{eq:140}
    &=^{d} n(0, p(1-p))
  \end{align}
\end{proof}

\begin{thm}
  Let $p \in (0, 1)$ and let $X_{1}, \dots, X_{n} \iid F$ where $F$ is
  differentiable at $F^{-1}(p)$ with positive derivative
  $f(F^{-1}(p))$.  Then
  \begin{align}
    \label{eq:141}
    \sqrt{n}(X_{\upperbound{np}} - F^{-1}(p)) \cd N(0, \frac{p(1-p)}{f(F^{-1}(p))^{2}} )
  \end{align}
\end{thm}

\begin{proof}
  Let $U_{1}, \dots, U_{n} \iid U(0, 1)$ so that
  $F^{-1}(U_{\upperbound{np}}) =^{d} X_{\upperbound{np}}$.  Then by
  the previous theorem and the delta method with $g = F^{-1}$,
  \begin{align}
    \label{eq:142}
    \sqrt{n} (X_{\upperbound{np}} - F^{-1}(p)) &=^{d}
    \sqrt{n}(g(U_{\upperbound{np}}) - g(p)) \\
    \label{eq:143}
    \cd N(0, \frac{p(1-p)}{f(F^{-1}(p))^{2}} )
  \end{align}
\end{proof}

\section{Density Estimators}
\label{sec:density-estimators}

\begin{defn}[Histogram Estimator]
  \label{sec:density-estimators-1}
  \begin{align}
    \label{eq:144}
    \tilde f_{b}(x) = \frac{1}{nb} \sum_{i=1}^{n} \I{X_{i} \in [x_{k},
      x_{k+1})}
  \end{align}
\end{defn}

\begin{defn}[Kernel Density Estimator]
  \label{sec:density-estimators-2}
  \begin{align}
    \label{eq:145}
    \hat f_{h}(x) = \frac{1}{nh} \sum_{i=1}^{n} K(\frac{x - X_{i}}{h}).
  \end{align} where $K: \R \rightarrow \R$ satisfies $\int_{\R} K(x)
  dx = 1$ is called the kernel, $h > 0$ is the bandwidth.

  Write $K_{h}(x) = \frac{1}{h} K(\frac{x}{h})$ so that
  \begin{align}
    \label{eq:146}
    \hat f_{h}(x) = \frac{1}{n} \sum_{i=1}^{n} K_{h}(x - X_{i}).
  \end{align}
\end{defn}

\begin{defn}[MSE]
  \label{sec:density-estimators-3}
  \begin{align}
    \label{eq:147}
    MSE(\hat f_{h}(x)) &= \E{(\hat f_{h}(x) - f(x))^{2}} \\
    \label{eq:148}
    &= \E{(\hat f_{h}(x) - \E{\hat f_{h}(x)}^{2})^{2}} + (\E{\hat
      f_{h}(x)} - f(x))^{2}.
  \end{align}

  Write $(f \star g)(x) = \int_{\R} f(x - y) g(y) dy$
\end{defn}

\begin{thm}
  For the KDE, we can write
  \begin{align}
    \label{eq:149}
    Bias(\hat f_{h}(x)) &= \E{K_{h}(x - X_{1})} - f(x) \\
    &= \int_{\R} K_{h}(x - y) f(y) dy - f(x) \\
    \label{eq:150}
    &= (K_{h} \star f)(x) - f(x)
  \end{align}

  Similarly,
  \begin{align}
    \label{eq:152}
    \Var(\hat f_{h}(x)) = \frac{1}{n} ((K_{h}^{2} \star f)(x) - (K_{h}
    \star f)(x)^{2})
  \end{align}
\end{thm}



Usually, we prefer to choose $h$ to minimize some expression measuring
how well $\hat f_{h}$ estimates $f$ as a function.  We therefore
define the Mean Integrated Squared Error ($MSIE$) as
\begin{align}
  \label{eq:14}
  MSIE(\hat f_{h}) &= \E{\int_{-\infty}^{\infty} \{ \hat f_{h}(x) -
    f(x) \}^{2} dx} \\
  &= \int_{-\infty}^{\infty} MSE(\hat f_{h}(x)) dx \\
  &= \int_{\infty}^{\infty} ((K_{h} \star f)(x) - f(x))^{2} +
  \frac{1}{h} ((K^{2}_{n} \star f)(x) - (K_{h} \star f)^{2}(x)) dx
\end{align} which is justified by Fubini's theorem as the integrand
is non-negative.
Although exact, this expression depends on $h$ in a complicated way.
We therefore seek asymptotic approximation to clarify this dependence
and facilitate an asymptotically optimal choice of $h$.

\section{Asymptotic MSE and MSIE approximation}
\label{sec:asynptotic-mse-msie}

We need the following conditions:
\begin{enumerate}
\item \label{item:4} $f$ is twice differentiable, $f'$is bounded, and $R(f) =
  \int_{-\infty}^{\infty} f''(x)^{2} dx < \infty$.
\item \label{item:5} $h = h_{n}$ is a non-random sequence with $h \rightarrow 0$ and
  $nh \rightarrow \infty$ as $n \rightarrow \infty$.
\item \label{item:6} $K$ is non-negative, $\int_{-\infty}^{\infty} K(x) dx = 1$,
  $\int_{-\infty}^{\infty} x K(x) dx = 0$, $\mu_{2}(K) =
  \int_{-\infty}^{\infty} x^{2} K(x) dx < \infty$, and $R(x) < \infty$.
\end{enumerate}

\begin{thm}
  \label{defn:Introduction:3}
  Assume that the previous conditions hold. Then, for all $x \in \R$,
  \begin{align}
    \label{eq:15}
    MSE(\hat f_{n}(x)) = \frac{R(K) f(x)}{nh} + \frac{1}{4} h^{4}
    \mu_{2}^{2}(K) f''(x)^{2} + o(\frac{1}{nh} + h^{4})
  \end{align} as $n \rightarrow \infty$.
\end{thm}

\begin{proof}
  We first claim that $f$ is bounded. Otherwise, there would exists
  $(x_{n})$ such that $f(x_{n}) \geq n$. Since $f$ is a density, the
  exists $x_{n, l} \in [x_{n} - \frac{2}{n}, x_{n}]$ such that
  $f(x_{n, l}) \leq \frac{n}{2}$ and there exists $x_{n, m} \in
  [x_{n}, x_{n} + \frac{2}{n}]$ such that $f(x_{n, m}) \leq
  \frac{n}{2}$. y the mean value theorem, there exists $x^{\star}_{n,
    l} \in [x_{n, l}, x_{n}]$ such that $f'(x^{\star}_{n, l}) \geq
  \frac{n^{2}}{4}$ and there exists $x^{\star}_{n, m} \in [x_{n},
  x_{n, m}]$ such that $f'(x^{\star}_{n, m}) \leq -\frac{n^{2}}{4}$.
  By the mean value theorem again, we have that there exists
  $x_{n}^{\star \star} \in [x_{n, l}^{\star}, x_{n, m}^{\star}]$ such
  that $f''(x_{n}^{\star \star}) \leq -\frac{n^{3}}{8}$, contradicting
  boundedness of $f''$.

  We can therefore define $C_{0} = \sup_{x \in \R} f(x)$ and $C_{2} =
  \sup_{x \in \R} |f''(x)|$.

  Now,
  \begin{align}
    \label{eq:16}
    \E{\hat f_{h}(x)} &= \int_{-\infty}^{\infty} \frac{1}{h}
    K(\frac{x-y}{h}) f(y) dy \\
    &= \int_{-\infty}^{\infty}  K(z) f(x - hz) dz \\
    &= \int_{-\infty}^{\infty} K(z)(f(x) - hz f'(x) + \frac{1}{2} h^{2}
    z^{2} f''(x)) dz + REM_{1} \\
    &= f(x) + \frac{1}{2} h^{2} \mu_{2}(K) f''(x) + REM_{1}.
  \end{align}

  To control the remainder, given $\epsilon > 0$, choose $\delta > 0$
  such that
  \begin{align}
    \label{eq:17}
    |f(x - hz) - (f(x) - hz f'(x) + \frac{1}{2} h^{2} z^{2} f''(x))|
    \leq \epsilon h^{2} z^{2}
  \end{align} for all $|hz| \leq \delta$.

  Then
  \begin{align}
    \label{eq:18}
    |REM_{1}| &= | \int_{-\infty}^{\infty} K(z) f(x - hz) dz -
    \int_{-\infty}^{\infty} K(x)(f(x) + \frac{1}{2} h^{2} z^{2} f''(x))
    dz | \\
    &\leq | \int_{|z| > \frac{\delta}{h}} K(z) f(x - hz) dz| +
    \int_{|z| \leq \frac{\delta}{h}} K(z)|f(x-hz) - (f(z) +
    \frac{1}{2}h^{2} z^{2} f''x)| dz \\
    &\qquad+ | \int_{|z| > \frac{\delta}{h}}
    K(z) (f(x) + \frac{1}{2} h^{2} z^{2} f''(x)) dz| \\
    &\leq C_{0} \frac{h^{2}}{\delta^{2}} \int_{|z| > \frac{\delta}{h}}
    z^{2} K(x) dz \\
    &\qquad+ \epsilon h^{2} \int_{|z| \leq \frac{\delta}{h}}
    z^{2} K(z) dz + C_{0} \frac{h^{2}}{\delta^{2}} \int_{|z| >
      \frac{\delta}{h}} z^{2} K(z) dz + \frac{1}{2} h^{2} C_{2}
    \int_{|z| > \frac{\delta}{h}} z^{2} K(z) dz \\
    &\leq \epsilon h^{2}{1 + \mu_{2}(K)}
  \end{align}
  since $\int_{-\infty}^{\infty} z K(z) dx = 0$, Markov's inequality, etc.
  Thus,
  \begin{equation}
    \label{eq:19}
    BIAS(\hat f_{h}(x)) = \frac{1}{2} h^{2} \mu_{2}(K) f''(x) + o(h^{4}).
  \end{equation}

  For the variance,
  \begin{align}
    \label{eq:20}
    \Var{\hat f_{h}(x)} &= \frac{1}{nh^{2}} \int_{-\infty}^{\infty}
    K^{2}(\frac{x-y}{h}) f(y) dy - \frac{1}{n} \{\E{\hat f_{h}(x)}
    \}^{2} \\
    &= \frac{1}{nh} \int_{-\infty}^{\infty} K^{2}(z) f(x - hz) dz -
    \frac{1}{n} (f(x) + o(1))^{2} \\
    &= \frac{1}{nh} \int_{-\infty}^{\infty} K^{2}(z) f(x) dz + REM_{2} +
    O(\frac{1}{n}) \\
    &= \frac{R(K) f(x)}{nh} + REM_{2} + O(\frac{1}{n})
  \end{align}

  To control $REM_{2}$, given $\epsilon > 0$, choose $y > 0$ such that
  $|f(x - hz) - f(x)| \leq \epsilon$ for $|hz| \leq y$.  Then
  \begin{align}
    \label{eq:21}
    nh | REM_{2} | &= | \int_{-\infty}^{\infty} K^{2}(z) (f(x - hz) -
    f(x)) dz | \\
    &\leq \epsilon \int_{|z| \leq \frac{y}{h}} K^{2}(z) + 2 C_{0}
    \int_{|z| > \frac{y}{h}} K^{2}(z) dz \\
    &\leq \epsilon(R(K) + 1)
  \end{align}
  for large $n$.

  We deduce that $\Var{\hat f_{h}(x)} = \frac{R(K) f(x)}{nh} +
  o(\frac{1}{nh})$ and 
  \begin{align}
    \label{eq:22}
    MSE( \hat f_{h}(x)) = \frac{R(K) f(x)}{nh} + \frac{1}{4} h^{4}
    \mu_{2}^{2}(K) f''(x)^{2} + o(\frac{1}{nh} + h^{2})
  \end{align}

  The hope is that to compute the MSIE, we can just integrate the MSE
  over range of the RV.  We need to be careful - in general we cannot
  integrate asymptotic pointwise estimates - need to understand
  dependency on $x$.

  With mild additional conditions and further work (see the example
  sheet), it can be shown that
  \begin{align}
    \label{eq:23}
    MISE(\hat f_{h})= \frac{R(K)}{nh} + \frac{1}{4} h^{4}
    \mu_{2}^{2}(K)R(f'') + o(\frac{1}{nh} + h^{4})
  \end{align}

  We see that asymptotically the integrated variance term decreases with
  $h$ while the integrated squared bias term increases with $h$.  This
  is the \textbf{bias-variance tradeoff}.

  This \textbf{bias-variance tradeoff} summarizes the critical role of
  the bandwidth.
\end{proof}

Consider now minimizing the asymptotic MISE (AMISE) $\frac{R(K)}{nh} +
\frac{1}{4} h^{4} \mu_{2}^{2}(K) R(f'')$ with respect to $h$, yielding
the asymptotically optimal bandwidth
\begin{align}
  \label{eq:24}
  h_{AMISE} = (\frac{R(K)}{\mu_{2}^{2}(K) R(f'') n})^{\frac{1}{5}}
\end{align}

Substituting back, we obtain
\begin{align}
  \label{eq:25}
  AMISE(\hat f_{AMISE}) = \frac{5}{4} R(K)^{\frac{4}{5}}
  \mu_{2}(K)^{\frac{2}{5}} R(f'')^{\frac{1}{5}} n^{-\frac{4}{5}}.
\end{align}

Notice the slower rate than the typical $O(n^{-1})$
parametric rate.  Notice that for the ``rough'' densities, with larger
$R(f'')$, we should use a smaller bandwidth, and these densities are
harder to estimate.


\section{Pointwise asymptotic distribution}
\label{sec:pointw-asympt-distr}

\begin{thm}
  \label{defn:Introduction:4}
  Assume the previous assumptions \ref{item:4}, \ref{item:5},
  \ref{item:6} and that $K$ is bounded.  Then, for all $x \in \R$,
  \begin{equation}
    \label{eq:26}
    n^{\frac{2}{5}}(\hat f_{h_{AMISE}}(x) - f(x)) \cd
    N(\frac{1}{2}\mu_{2}(K) f''(x), R(K) f(x))
  \end{equation}
\end{thm}

\begin{proof}
  First, observe that from the proof of the previous theorem,
  \begin{align}
    \label{eq:27}
    n^{\frac{2}{5}} (\E{\hat f_{h_{AMISE}}(x) - f(x)}) \rightarrow
    \frac{1}{2} \mu_{2} K f''(x)
  \end{align}

  For the stochastic term, let $Y_{ni} = \frac{1}{h^{\frac{1}{2}}}
  K(\frac{x - X_{i}}{h})$.  We have
  \begin{align}
    \label{eq:28}
    \Var{Y_{ni}} &= \frac{1}{h} \int_{-\infty}^{\infty}
    K^{2}(\frac{x-y}{n}) f(y) - h(\E{\hat f_{h}(x)})^{2} \\
    &= \int_{-\infty}^{\infty} K^{2}(z) f(x - hz) dz - h(f(x) +
    o(1))^{2} \\
    &\rightarrow R(K) f(x)
  \end{align} as $n \rightarrow \infty$.

  Moreover,
  \begin{align}
    \label{eq:29}
    \E{Y^{2}_{ni} \I{|Y_{ni} \geq \epsilon n^{\frac{1}{2}}}} &=
    \int_{-\infty}^{\infty} \frac{1}{n} K^{2}(\frac{x - y}{h}) f(y)
    \I{K(\frac{x - y}{h}) \geq \epsilon (nh)^{\frac{1}{2}}} dy \\
    &= 0
  \end{align} for $n$ large enough such that $\sup_{z \in R} K(z) < e
  (nh)^\frac{1}{2}$.

  Thus by the Linderberg-Feller central limit theorem, we have our
  required result.
\end{proof}

\section{Bandwidth Selection}
\label{sec:bandwidth-selection}

Since $h_{AMISE}$ depends on $f$ through $R(f'')$, we still require
practical bandwidth selection algorithms.

\subsection{Normal Scale rules}
\label{sec:normal-scale-rules-1}

If $f$ is the $N(0, \sigma^{2})$ density, then $R(f'') = \frac{3}{8
  \sqrt{\pi}} \sigma^{-5}$.    The normal scale rate $\hat h_{NS}$
consists of replacing $R(f'')$ in $h_{AMISE}$ with $\frac{3}{8
  \sqrt{\pi}} \hat \sigma^{-5}$, where $\hat \sigma$ is an estimate of
$\sigma$.  This tends to over-smooth.

\subsection{Least-squares Cross-Validation}
\label{sec:least-squares-cross}

Recall that
\begin{equation}
  \label{eq:30}
  MISE(\hat f_{h}) = \E{\int_{-\infty}^{\infty} \hat f(x)^{2} dx} - 2
  \E{\int_{-\infty}^{\infty} \hat f_{h}(x) f(x)} +
  \int_{-\infty}^{\infty} f(x)^{2} dx.
\end{equation}

Observe that it suffices to minimize the sum of the first two terms.
This depend on the unknown $f$, but an unbiased estimate is given by
$LSCV(h)$, with
\begin{align}
  \label{eq:31}
  LSCV(h) = \int_{-\infty}^{\infty} \hat f_{h}(x)^{2} dx - \frac{2}{n}
  \sum_{i=1}^{n} f_{-i, h}(x_{i})
\end{align} with
\begin{align}
  \label{eq:32}
  \hat f_{-i, h}(x) = \frac{1}{(n-1)h} \sum_{j \neq i} K(\frac{x - x_{j}}{h})
\end{align}

Minimization of $LSCV(h)$ yields $\hat h_{LSCV}$.

\subsection{Biased Cross-Validation}
\label{sec:bias-cross-valid}

Under regularity conditions,
\begin{align}
  \label{eq:34}
  \E{R(\hat f_{h})} = R(f'') + \frac{R(K'')}{n h^{5}} + O(h^{2}).
\end{align}

We can therefore define
\begin{align}
  \label{eq:36}
  BCV(h) = \frac{R(K)}{nh} + \frac{1}{4} \mu_{2}^{2}(K) \widetilde{R(f'')}
\end{align}

where
\begin{align}
  \label{eq:35}
  \widetilde{R(f'')} = R(\hat f_{h_{1}}) - \frac{R(K'')}{nh_{1}^{5}}
\end{align} with $h_{1}$ a ``pilot'' bandwidth (c.f Ward and Jones,
1995).  Minimization of $BCV(h)$ yields $\hat h_{BCV}$.

\subsection{Solve-the-equation Rules}
\label{sec:solve-equation-rules}

Under smoothness assumptions, we can integrate by parts to obtain
\begin{align}
  \label{eq:37}
  R(f'') = \int_{-\infty}^{\infty} f''''(x) f(x) dx = \E{f''''(X)}
\end{align}

We can therefore estimate $R(f'')$ by using
\begin{align}
  \label{eq:38}
  \hat R_{h_{2}} = \frac{1}{n} \sum_{i=1}^{n} \hat
  f_{h_{2}}''''(x_{i})
\end{align} where again $h_{2}$ is a pilot bandwidth.  By exploiting
the relationship between $h_{AMISE}$ and the $AMISE$-optimal bandwidth
for estimating $R(f'')$ in this way, we obtain an equation which can
be solved numerically to yield $\hat h_{SJE}$.


\section{Other Topics}
\label{sec:other-topics}

\subsection{Choice of Kernel}
\label{sec:choice-kernel}

The choice of kernel is coupled with the choice of bandwidth, because
if we replace $K(x)$ by $\frac{1}{2} K(\frac{1}{2})$ and we halve the
bandwidth, the estimate is unchanged.  We therefore fix the scale by
setting $\mu_{2}(K) = 1$.  Minimizing $AMISE(\hat f_{h})$ over $K$ the
amounts to minimizing $R(K)$ subject to
\begin{align}
  \label{eq:40}
  \int_{-\infty}^{\infty} K(x) dx = 1 \\
  \int_{-\infty}^{\infty} x K(x) dx = 0 \\
  \mu_{2}(K) = 1 \\
  K(x) \geq 0
\end{align}

The solution is given by the Epanechnikov kernel (1969).
\begin{align}
  \label{eq:41}
  K_{E}(x) = \frac{3}{4 \sqrt{5}}(1 - \frac{x^{2}}{5}) \I{|x| \leq \sqrt{5}}
\end{align}

The ratio $\frac{R(K_{E})}{R(K)}$ is called the \textbf{efficiency} of
a kernel $K$, because it represents the ratio of the sample sizes
needed to obtain the same $AMISE$ when using $K_{E}$ compared with $K$.


\begin{center}
% BEGIN RECEIVE ORGTBL testtbl
\begin{tabular}{lr}
Kernel & Efficiency \\
\hline
Epachnikov & 1.0 \\
Normal & 0.951 \\
Triangular & 0.986 \\
Uniform & 0.930 \\
\end{tabular}
% END RECEIVE ORGTBL testtbl
\end{center}

\subsection{Derivative Estimation}
\label{sec:deriv-estim-1}

A natural estimator of the $r$-th derivative $f^{(r)}$ of $f$ is given
by
\begin{align}
  \label{eq:33}
  \hat f^{(r)}_{h}(x) = \frac{1}{n h^{r+1}} \sum_{i=1}^{n} K(\frac{x -
  x_{i}}{h})
\end{align}
obtained from differentiating the standard KDE for $\hat f$.

Under regularity conditions,
\begin{align}
  \label{eq:39}
  MSE(\hat f^{(r)}_{h}(x)) = \frac{R(K^{(r)})}{nh^{2r+1}}f(x) +
  \frac{1}{4} h^{4} \mu_{2}^{2} f^{(r-2)}(x)^{2} + o(\frac{1}{nh} + h^{4}).
\end{align}

This leads to an optimal bandwidth of order $n^{-\frac{1}{2r + 5}}$
and a rate of converge of $n^{-\frac{4}{2r+5}}$.

The intuition is that estimating derivatives of densities is harder
than estimating densities themselves.

\subsection{Higher Order Kernels}
\label{sec:higher-order-kernels}

It is possible to make the dominant integrated squared bias term of
$MISE(\hat f_{h})$ vanish by choosing $\mu_{2}(K) = 0$.  This means
we have to allow the Kernel to take negative values, so the resulting
estimate need not be a density.

We can set $\hat f_{h}(x) = \max (\hat f_{h}(x), 0)$ and then
renormalize, but then we lose smoothness.  Nevertheless, we define $K$
to be a $k$-th order kernel if writing $\mu_{j}(K) =
\int_{-\infty}^{\infty} x^{j} K(x) dx$, we have
\begin{align}
  \label{eq:42}
  \mu_{0}(K) = 1
\end{align} and $\mu_{j}(K) = 0$ for $j = 1, \dots, k - 1$,
$\mu_{k}(K) \neq 0$, and
\begin{align}
  \label{eq:43}
  \int_{-\infty}^{\infty} |x|^{k} |K(x)| dx < \infty
\end{align} If $f$ has $k$ continuous bounded derivatives with
$R(f^{(k)}) < \infty$, then it is shown (example sheet) that
$h_{AMISE} = c n^{-\frac{1}{2k+1}}$ and
\begin{align}
  \label{eq:44}
  AMISE(\hat f_{h_{AMISE}}) = O(n^{-\frac{2k}{2k+1}})
\end{align}

Thus, under increasingly strong smoothness assumptions, convergence
rates arbitrarily close to the parametric rate of $O(n^{-1})$ can be
obtained.

The practical benefit of higher order kernels is not always apparent,
and the negativity/smoothness/bandwidth selection problems mean that
they are rarely used in practice.

\subsection{Local Bandwidths}
\label{sec:local-bandwidths}

Choosing $h = h(x)$  is problematic, because the resulting estimate
need not be a density.  However, we can define
\begin{align}
  \label{eq:45}
  \hat f(x) = \frac{1}{n} \sum_{i=1}^{n} K_{h(x)}(x - X_{i})
\end{align}

Theory suggests that we should choose $h(X_{i}) = h_{0}
f^{-\frac{1}{2}}(X_{i})$, and, with four derivatives and a second
order kernel, one can attain a ``fourth-order kernel'' rate of
$O(n^{-\frac{8}{9}})$.  There is no negativity problem, but we do
require pilot bandwidth selection.  Difficult to tune well and rarely
used in practice.

\subsection{Transformation Methods}
\label{sec:transf-meth}

It may be that $f$ is difficult to estimate, but it may be that we
can construct a strictly increasing, continuously differentiable
function $t$ on the support of $f$, such that, setting $Y_{i} =
t(X_{i})$, the density of $Y_{1}, \dots, Y_{n}$ is easier to estimate.
We ``back transform'' the estimate to obtain
\begin{align}
  \label{eq:46}
  \bar f_{h}(x) = \frac{1}{n} \sum_{i=1}^{n} K_{h}(t(x) - t(X_{i})) t'(x)
\end{align}

\subsection{Multi-Dimensional Density Estimation}
\label{sec:multi-dimens-dens}

The general $d$-dimensional kernel estimator is of the form
\begin{align}
  \label{eq:47}
  \hat f_{H}(x) = \frac{1}{n} (\det H)^{\frac{1}{2}} \sum_{i=1}^{n}
  K(H^{-\frac{1}{2}}(x - X_{i}))
\end{align} where $H$ is positive definite symmetric bandwidth matrix.
The difficulties of choosing the $\frac{1}{2} d(d+1)$ independent
entries mean that we often restrict attention to the diagonal $H$, or
even $H = h^{2} I$.  In this latter case,
\begin{align}
  \label{eq:48}
  AMISE(\hat f_{h^{2}I}) = \frac{R(K)}{n h^{d}} + \frac{1}{4} h^{4}
  \mu_{2}^{2}(K) \int_{\R^{d}} \{ \Delta_{f}(x) \}^{2} dx
\end{align} where $\Delta_{f}(x) = \sum_{j=1}^{d} \frac{\partial^{2}
  f}{\partial x_{j}^{2}}(x)$ is the Laplacian of $f$ at $x$.  This leads
to an
\begin{align}
  \label{eq:49}
  AMISE(\hat f_{h^{2}_{AMISE}I}) = O(n^{-\frac{4}{d + 4}})
\end{align}

Thus the ``curse of dimensionality'', together with bandwidth
selection problems, means that this is only really feasible for $d
\leq 4$.

%%% Local Variables:
%%% mode: latex
%%% TeX-master: "master"
%%% End:

\iffalse
#+ORGTBL: SEND testtbl orgtbl-to-latex :splice nil :skip 0
| Kernel     | Efficiency |
|------------+------------|
| Epachnikov |        1.0 |
| Normal     |      0.951 |
| Triangular |      0.986 |
| Uniform    |      0.930 |
\fi