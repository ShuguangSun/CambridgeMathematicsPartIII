\documentclass[8pt, reqno, oneside, twocolumn]{amsart} 

% Utility packages
\usepackage{etoolbox}
\newtoggle{tufte}
\togglefalse{tufte}
\newtoggle{summary}
\togglefalse{summary}
\usepackage{amsthm}
\usepackage{amsmath}
\usepackage{amssymb}
% \usepackage[landscape]{geometry}
\usepackage[top=1cm, bottom=1cm, left=1cm, right=1cm]{geometry}
% \usepackage[inner=1cm, outer=1cm]{geometry}
\usepackage{setspace}
\usepackage{graphicx}
\usepackage{enumerate}
\usepackage{listings}
\usepackage{booktabs}
\usepackage{hyperref}
\usepackage{thmtools}
\usepackage{xparse}
\usepackage{xspace}
\usepackage{enumitem}
\usepackage{todonotes}
\usepackage{dcolumn}
\usepackage{array}
\usepackage{booktabs}
\usepackage{colortbl, xcolor}
\usepackage{longtable}
\usepackage{natbib}
\usepackage{python}
% \usepackage{a4wide}

\hypersetup{colorlinks=true}
\numberwithin{equation}{section}

\declaretheorem[numbered=no,name=Theorem,style=definition]{thm}
\declaretheorem[numbered=no,name=Lemma]{lem}
\declaretheorem[numbered=no,name=Corollary]{corollary}
\declaretheorem[numbered=no,name=Definition]{defn}
\declaretheorem[numbered=no,name=Proposition]{proposition}
\declaretheorem[numbered=no,name=Example]{exmp}
\declaretheorem[numbered=no,name=Exercise]{exer}
\declaretheorem[numbered=no,name=History]{history}
\declaretheorem[numbered=no,name=Question]{question}
\declaretheorem[numbered=no,name=Remark]{remark}
\declaretheorem[numbered=no,name=Notation]{notation}
\declaretheorem[numbered=no,name=Theorem]{boxthm}

\let\proof\relax
\declaretheorem[style=remark,numbered=no,qed=\qedsymbol]{proof}

\DeclareMathOperator*{\argmax}{arg\,max}
\DeclareMathOperator*{\dist}{dist}
\DeclareMathOperator*{\V}{V}
\DeclareMathOperator*{\saddlepoint}{sp}
\DeclareMathOperator*{\divergence}{div}
\DeclareMathOperator*{\sym}{Sym}
\DeclareMathOperator*{\vol}{Vol}
\DeclareMathOperator*{\argmin}{arg\,min}
\DeclareMathOperator*{\proj}{\Pi}
\DeclareMathOperator*{\bnd}{bnd}
\DeclareMathOperator*{\supp}{supp}
\DeclareMathOperator*{\aff}{aff}
\DeclareMathOperator*{\rint}{rint}
\DeclareMathOperator*{\lev}{lev}
\DeclareMathOperator*{\tr}{tr}
\DeclareMathOperator*{\interior}{int}
\DeclareMathOperator*{\epi}{epi}
\DeclareMathOperator*{\dom}{dom}
\DeclareMathOperator*{\vecspan}{span}
\DeclareMathOperator*{\con}{con}
\DeclareMathOperator*{\cl}{cl}
\DeclareMathOperator*{\grad}{\nabla}
\DeclareMathOperator*{\crit}{crit}
\DeclareMathOperator*{\diag}{diag}
\DeclareMathOperator*{\res}{Res}
\DeclareMathOperator*{\boundary}{boundary}
\DeclareMathOperator*{\range}{range}
\DeclareMathOperator*{\Tr}{Tr}
\DeclareMathOperator*{\sign}{sign}
\setlist[enumerate]{label=(\roman*)}

\newlist{exercises}{enumerate}{1}
\setlist[exercises]{label=Ex. \arabic*}


\setcounter{secnumdepth}{2}
\setcounter{tocdepth}{2}
% \onehalfspacing

\author{Andrew Tulloch}


% Shortcuts
% convergence in probability
\newcommand{\cd}{\overset{d}{\rightarrow}}    
% convergence in distribution
\newcommand{\cp}{\overset{p}{\rightarrow}}
% convergence almost surely
\newcommand{\cas}{\overset{as}{\rightarrow}}
\newcommand{\R}{\mathbb{R}}
\newcommand{\N}{\mathbb{N}}
\newcommand{\Z}{\mathbb{Z}}
\newcommand{\np}{\textsc{np}\xspace}
\newcommand{\iid}{\textsc{iid}\xspace}
\newcommand{\knn}{$k$-\textsc{nn}\xspace}

% Expectation
\NewDocumentCommand{\E}{gg}{%
  \IfNoValueTF{#1}
    {%
      \mathbb{E}
    }    
    {%
      \IfNoValueTF{#2}
      {%
        \mathbb{E}\!\left(#1\right)
      }
      {%
        \mathbb{E}_{#2}\!\left(#1\right)
      }
    }%
}

% Probability with measure P
\NewDocumentCommand{\Prob}{gg}{%
  \IfNoValueTF{#1}
    {%
      \mathbb{P}
    }    
    {%
      \IfNoValueTF{#2}
      {%
        \mathbb{P}\!\left(#1\right)
      }
      {%
        \mathbb{P}_{#2}\!\left(#1\right)
      }
    }%
}

% Probability with measure Q
\NewDocumentCommand{\Q}{g}{%
  \IfNoValueTF{#1}
    {%
      \mathbb{Q}
    }    
    {%
      \mathbb{Q}\!\left(#1\right)
    }%
}


% Indicator
\NewDocumentCommand{\I}{g}{%
  \IfNoValueTF{#1}
    {%
      \mathbb{I}
    }    
    {%
      \mathbb{I}\!\left(#1\right)
    }%
}


\NewDocumentCommand{\Var}{g}{%
  \IfNoValueTF{#1}
    {%
      \mathbb{V}
    }    
    {%
      \mathbb{V}\!\left(#1\right)
    }%
}


\NewDocumentCommand{\Cov}{gg}{%
  \IfNoValueTF{#1}
    {%
      \text{Cov}
    }   
    {%
      \text{Cov}\!\left(#1, #2\right)
    }%
}

% Inner product
\NewDocumentCommand{\IP}{g}{%
    {%
      \left \langle #1 \right \rangle
    }%
}
%%% Local Variables: 
%%% mode: latex
%%% End: 


\title{Nonparametric Statistics Summary}

\begin{document}

\maketitle

\section{Basics}
\begin{thm}[The Delta Method]
  Let $Y_{n}$ be a sequence of random vectors in $\R^{d}$ such that
  for some $\mu \in \R^{d}$ and a random vector $Z$, we have
  $n^{\frac{1}{2}}(Y_{n} - \mu) \cd Z$.  If $g: \R^{d} \rightarrow \R$
  is differentiable at $\mu$, then $n^{\frac{1}{2}}(g(Y_{n}) - g(\mu))
  \cd \grad g(\mu)^{T} Z$.
\end{thm}
\begin{proof}
  For $d = 1$.  Let $g'(\mu) = \grad g(\mu)$, and let $h: \R
  \rightarrow \R$, by
  \begin{equation}
    \label{eq:134}
    h(y) =
    \begin{cases}
      \frac{g(y) - g(\mu)}{y - \mu}  & y \neq \mu \\
      g'(\mu) & y = \mu
    \end{cases}
  \end{equation}  Then by the continuous mapping theorem and Sltusky's
  theorem, $n^{\frac{1}{2}}(g(Y_{n}) - g(\mu)) = h(Y_{n})
  n^{\frac{1}{2}}(Y_{n} - \mu) \cd g'(\mu) Z$.
\end{proof}

Let $X_{1}, \dots, X_{n}$ be \iid on a probability space $(\Omega,
\mathcal{F}, \Prob)$ with distribution function $F$.  The
\textbf{empirical distribution function} $\hat F_{n}$ is defined by
\begin{equation}
  \label{eq:1}
  \hat F_{n}(x) = \frac{1}{n} \sum_{i=1}^{n} \I{X_{i} \leq x}.
\end{equation}

\begin{thm}[Glivenko-Cantelli (1933) - The Fundamental Theorem of Statistics]
  \label{defn:Introduction:2}
  \begin{equation}
    \label{eq:2}
    \sup_{x \in \R} \left| \hat F_{n}(x) - F(x) \right| \cas 0.
  \end{equation}
\end{thm}

\begin{thm}
  \label{defn:Introduction:1}
  Let $X_{1}, \dots, X_{n} \sim F$ \iid.  Then for every $\epsilon >
  0$,
  \begin{align}
    \label{eq:7}
    \Prob{\sup_{x \in \R}|\hat F_{n}(x) - F(x)| \geq \epsilon} \leq 2 e^{-2n\epsilon^{2}}.
  \end{align}
\end{thm}

\begin{defn}
  \label{sec:let-x_1-dots}
  For $p \in (0, 1]$, the quartile function is defined by $F^{-1}(p) =
  \inf \{x \in \R: F(x) \geq p \}$ and is left-continuous.

  The sample quartile function is $\hat F_{n}^{-1}(p) = \inf \{ x \in
  \R: \hat F_{n}(x) \geq p \} $. 
\end{defn}

\begin{thm}
  Let $U_{1}, U_{2}, \dots, U_{n} \sim U(0, 1)$ \iid and $p \in (0,
  1)$.  Then
  \begin{equation}
    \label{eq:135}
    \sqrt{n}(U_{\lceil np \rceil} - p) \cd N(0, p(1-p)).
  \end{equation}
\end{thm}

\begin{proof}
  Let $Y_{1}, \dots, Y_{n}$ \iid \textsc{Exp}(1), let $V_{n} = Y_{1} + \dots +
  Y_{\lceil np \rceil}$ and $W_{n} = Y_{\lceil np \rceil + 1}, \dots,
  Y_{n+1}$.  Note that $V_{n}, W_{n}$ are independent and
  \begin{equation}
    \label{eq:136}
    \frac{V_{n}}{V_{n} + W_{n}} =^{d} U_{\upperbound{np}}
  \end{equation} by previous proposition.
  Then
  \begin{align}
    \label{eq:137}
    \sqrt{n}(\frac{V_{n}}{n} - p) =
    \frac{\sqrt{\upperbound{np}}}{\sqrt{n}} (\frac{V_{n} -
      \upperbound{np}}{\sqrt{\upperbound{np}}} ) +
    \frac{\upperbound{np} - np}{\sqrt{n}} \cd N(0, p)
  \end{align} by the CLT and Slutsky's theorem.

  Similarly, $\sqrt{n}(\frac{W_{n}}{n} - q) \cd N(0, q)$, where $q = 1
  - p$, then by the delta method, with $g(x, y) = \frac{x}{x+y}$,
  \begin{align}
    \label{eq:138}
    \sqrt{n}(U_{\upperbound{np}} - p) &=^{d}
    \sqrt{n}(g(\frac{V_{n}}{n}, \frac{W_{n}}{n}) - g(p, q)) \\
    &\cd N(0, \grad g(p, q)^{T}
    \begin{pmatrix}
      p & 0 \\
      0 & q
    \end{pmatrix} \grad g(p, q)) \\
    \label{eq:140}
    &=^{d} n(0, p(1-p))
  \end{align}
\end{proof}

\begin{thm}
  Let $p \in (0, 1)$ and let $X_{1}, \dots, X_{n} \iid F$ where $F$ is
  differentiable at $F^{-1}(p)$ with positive derivative
  $f(F^{-1}(p))$.  Then
  \begin{align}
    \label{eq:141}
    \sqrt{n}(X_{\upperbound{np}} - F^{-1}(p)) \cd N(0, \frac{p(1-p)}{f(F^{-1}(p))^{2}} )
  \end{align}
\end{thm}

\begin{proof}
  Let $U_{1}, \dots, U_{n} \iid U(0, 1)$ so that
  $F^{-1}(U_{\upperbound{np}}) =^{d} X_{\upperbound{np}}$.  Then by
  the previous theorem and the delta method with $g = F^{-1}$,
  \begin{align}
    \label{eq:142}
    \sqrt{n} (X_{\upperbound{np}} - F^{-1}(p)) &=^{d}
    \sqrt{n}(g(U_{\upperbound{np}}) - g(p)) \\
    \label{eq:143}
    \cd N(0, \frac{p(1-p)}{f(F^{-1}(p))^{2}} )
  \end{align}
\end{proof}

\begin{defn}
  Usually, we prefer to choose $h$ to minimize some expression measuring
how well $\hat f_{h}$ estimates $f$ as a function.  We therefore
define the Mean Integrated Squared Error ($MSIE$) as
\begin{align}
  \label{eq:14}
  MSIE(\hat f_{h}) &= \E{\int_{-\infty}^{\infty} \{ \hat f_{h}(x) -
    f(x) \}^{2} dx} \\
  &= \int_{-\infty}^{\infty} MSE(\hat f_{h}(x)) dx \\
  &= \int_{\infty}^{\infty} ((K_{h} \star f)(x) - f(x))^{2} +
  \frac[1]{h} ((K^{2}_{n} \star f)(x) - (K_{h} \star f)^{2}(x)) dx
\end{align} which is justified by Fubini's theorem as the integrand
is non-negative.
JAlthough exact, this expression depends on $h$ in a complicated way.
We therefore seek asymptotic approximation to calify this dependnence
and faciliate an asymptotically optimal choice of $h$.
\end{defn}

We need the following conditions:
\begin{enumerate}
\item \label{item:4} $f$ is twice differentiable, $f'$is bounded, and $R(f) =
  \int_{-\infty}^{\infty} f''(x)^{2} dx < \infty$.
\item \label{item:5} $h = h_{n}$ is a non-random sequecne with $h \rightarrow 0$ and
  $nh \rightarrow \infty$ as $n \rightarrow \infty$.
\item \label{item:6} $K$ is non-negative, $\int_{-\infty}^{\infty} K(x) dx = 1$,
  $\int_{-\infty}^{\infty} x K(x) dx = 0$, $\mu_{2}(K) =
  \int_{-\infty}^{\infty} x^{2} K(x) dx < \infty$, and $R(x) < \infty$.
\end{enumerate}

\begin{thm}
  \label{defn:Introduction:3}
  Assume that the previous conditions hod. Then, for all $x \in \R$,
  \begin{align}
    \label{eq:15}
    MSE(\hat f_{n}(x)) = \frac{R(K) f(x)}{nh} + \frac{1}{4} h^{4}
    \mu_{2}^{2}(K) f''(x)^{2} + o(\frac{1}{nh} + h^{4})
  \end{align} as $n \rightarrow \infty$.
\end{thm}


Consider now minimizing the asymptotic MISE (AMISE) $\frac{R(K)}{nh} +
\frac{1}{4} h^{4} \mu_{2}^{2}(K) R(f'')$ with respect to $h$, yielding
the asymptotically optimal bandwidth
\begin{align}
  \label{eq:24}
  h_{AMISE} = (\frac{R(K)}{\mu_{2}^{2}(K) R(f'') n})^{\frac{1}{5}}
\end{align}

Substituting back, we obtain
\begin{align}
  \label{eq:25}
  AMISE(\hat f_{AMISE}) = \frac{5}{4} R(K)^{\frac{4}{5}}
  \mu_{2}(K)^{\frac{2}{5}} R(f'')^{\frac{1}{5}} n^{-\frac{4}{5}}.
\end{align}

Notice the slower rate than the typical $O(n^{-1})$
parametric rate.  Notice that for the ``rough'' densities, with larger
$R(f'')$, we should use a smaller bandwidth, and these densities are
harder to estimate.

\begin{thm}
  \label{defn:Introduction:4}
  Assume the previous assumptions \ref{item:4}, \ref{item:5},
  \ref{item:6} and that $K$ is bounded.  Then, for all $x \in \R$,
  \begin{equation}
    \label{eq:26}
    n^{\frac{2}{5}}(\hat f_{h_{AMISE}}(x) - f(x)) \cd
    N(\frac{1}{2}\mu_{2}(K) f''(x), R(K) f(x))
  \end{equation}
\end{thm}

\begin{thm}
  If $f$ is the $N(0, \sigma^{2})$ density, then $R(f'') = \frac{3}{8
    \sqrt{\pi}} \sigma^{-5}$. The normal scale rate $\hat h_{NS}$
  consists of replacing $R(f'')$ in $h_{AMISE}$ with $\frac{3}{8
    \sqrt{\pi}} \hat \sigma^{-5}$, where $\hat \sigma$ is an estimate
  of $\sigma$. This tends to oversmooth.
\end{thm}

The choice of kernel is coupled with the choice of bandwidth, because
if we replace $K(x)$ by $\frac{1}{2} K(\frac{1}{2})$ and we halve the
bandwidth, the estimate is unchanged.  We therefore fix the scale by
setting $\mu_{2}(K) = 1$.  Minimizing $AMISE(\hat f_{h})$ over $K$ the
amounts to minimizing $R(K)$ subject to
\begin{align}
  \label{eq:40}
  \int_{-\infty}^{\infty} K(x) dx = 1 \\
  \int_{-\infty}^{\infty} x K(x) dx = 0 \\
  \mu_{2}(K) = 1 \\
  K(x) \geq 0
\end{align}

The solution is given by the Epanechnikov kernel (1969).
\begin{align}
  \label{eq:41}
  K_{E}(x) = \frac{3}{4 \sqrt{5}}(1 - \frac{x^{2}}{5}) \I{|x| \leq \sqrt{5}}
\end{align}

The ratio $\frac{R(K_{E})}{R(K)}$ is called the \textbf{efficiency} of
a kernel $K$, because it represents the ratio of the sample sizes
needed to obtain the same $AMISE$ when using $K_{E}$ compared with $K$.


\begin{center}
% BEGIN RECEIVE ORGTBL testtbl
\begin{tabular}{lr}
Kernel & Efficiency \\
\hline
Epachnikov & 1.0 \\
Normal & 0.951 \\
Triangular & 0.986 \\
Uniform & 0.930 \\
\end{tabular}
% END RECEIVE ORGTBL testtbl
\end{center}

\begin{thm}
  A natural estimator of the $r$-th derivative $f^{(r)}$ of $f$ is given
by
\begin{align}
  \label{eq:33}
  \hat f^{(r)}_{h}(x) = \frac{1}{n h^{r+1}} \sum_{i=1}^{n} K(\frac{x -
  x_{i}}{h})
\end{align}
obtained from differentiating the standard KDE for $\hat f$.

Under regularity conditions,
\begin{align}
  \label{eq:39}
  MSE(\hat f^{(r)}_{h}(x)) = \frac{R(K^{(r)})}{nh^{2r+1}}f(x) +
  \frac{1}{4} h^{4} \mu_{2}^{2} f^{(r-2)}(x)^{2} + o(\frac{1}{nh} + h^{4}).
\end{align}

This leads to an optimal bandwidth of order $n^{-\frac{1}{2r + 5}}$
and a rate of converge of $n^{-\frac{4}{2r+5}}$.

\end{thm}

\begin{thm}
  It is possible to make the dominant integrated squared bias term of
  $MISE(\hat f_{h})$ vanish by choosing $\mu_{2}(K) = 0$. This means
  we have to allow the Kernel to take negative values, so the
  resulting estimate need not be a density.

  We can set $\hat f_{h}(x) = \max (\hat f_{h}(x), 0)$ and then
  renormalize, but then we lose smoothness. Nevertheless, we define
  $K$ to be a $k$-th order kernel if writing $\mu_{j}(K) =
  \int_{-\infty}^{\infty} x^{j} K(x) dx$, we have
  \begin{align}
    \label{eq:42}
    \mu_{0}(K) = 1
  \end{align} and $\mu_{j}(K) = 0$ for $j = 1, \dots, k - 1$,
  $\mu_{k}(K) \neq 0$, and
  \begin{align}
    \label{eq:43}
    \int_{-\infty}^{\infty} |x|^{k} |K(x)| dx < \infty
  \end{align} If $f$ has $k$ continuous bounded derivatives with
  $R(f^{(k)}) < \infty$, then it is shown (example sheet) that
  $h_{AMISE} = c n^{-\frac{1}{2k+1}}$ and
  \begin{align}
    \label{eq:44}
    AMISE(\hat f_{h_{AMISE}}) = O(n^{-\frac{2k}{2k+1}})
  \end{align}

  Thus, under increasingly strong smoothness assumptions, convergence
  rates arbitrarily close to the parametric rate of $O(n^{-1})$ can be
  obtained.

  The practical benefit of higher order kernels is not always
  apparent, and the negativity/smoothness/bandwidth selection problems
  mean that they are rarely used in practice.
\end{thm}

\section{Nonparameteric Regression}
\label{sec:nonp-regr}

Assume a fixed design.  The local polynomial estimator $\hat m_{h}(x;
p)$ of degree $p$ with kernel $K$ with a bandwidth $h$ is constructed
by fitting a polynomial of degree $p$ using weighted least squares.
The weight $K_{h}(x_{i} - x)$ is associated with the weight $(x_{i},
Y_{i})$.

More precisely, $\hat m_{h}(x; p) = \hat \beta_{0}$ where $\hat \beta =
(\hat \beta_{0}, \hat \beta_{1}, \dots, \hat \beta_{p})$ whicch is
minimizing
\begin{align}
  \label{eq:52}
  \sum_{i=1}^{n} (Y_{i} - \beta_{0} - \beta_{1}(x_{i} - x) + \dots +
  \beta_{p}(x_{i} - x)^{p})^{2} K_{h}(x_{i} - x)
\end{align} where $\beta \in \R^{p+1}$

The theory of weighted least squares gives
\begin{align}
  \label{eq:53}
  (X^{T} K X) \hat \beta = X^{T} K y
\end{align}

For $p = 0$, then a simple expression (Nodorya-Watson, local constant)
exists:
\begin{align}
  \label{eq:54}
  \hat m_{h}(x; 0) = \frac{\sum_{i=1}^{n} K_{h}(x_{i} -
    x)Y_{i}}{\sum_{i=1}^{n} k_{h}(x_{i} - x)}
\end{align}

For $p= 1$, we call this a local linear estimator, and we have the
explicit result
\begin{align}
  \label{eq:55}
  \hat m_{h}(x; 1) = \frac{1}{n} \sum_{i=1}^{n} \frac{S_{2, h}(x) -
    S_{1, h}(x)(x_{i} - x)}{S_{2, h}(x) S_{0, h}(x) - S_{1, h}(x)^{2}}
  K_{h}(x_{i} - x) Y_{i}
\end{align}
with
\begin{align}
  \label{eq:56}
  S_{r, h}(x) = \frac{1}{n} \sum_{i=1}^{n} (x_{i} - x)^{r} K_{h}(x_{i}
  - x)
\end{align}

All local polynomial estimators of the form
\begin{align}
  \label{eq:57}
  \sum_{i=1}^{n} W(x_{i}, x) Y_{i}
\end{align}
This type of estimator is called a linear estimator.   This set of
weights $\{ W(x_{i}, x) \}$ is called the \textbf{effective kernel}.

\subsection{Mean Squared Error Approximations}

For convenience, let $x_{i} = \frac{i}{n}$.  We consider the following
conditions:

\begin{enumerate}
\item $m$ is twice continuosuly differentiable on $[0, 1]$ and is bounded, $v$ is continuous.
\item $h = h_{n}, h_{n} \rightarrow 0$, $nh \rightarrow \infty$.
\item $K$ is a nonnegative probability density, symmetric, has zeros
  outside of $[-1, 1]$. $R(K) = \int K^{2}(x) dx < \infty$, and
  $\mu_{2}(K) = \int x K^{2}(x) < \infty$.
\end{enumerate}

\begin{thm}
  \label{defn:nonparametric_regression:1}
  Under the conditions previously, for $x \in (0, 1)$, we have
  \begin{align}
    \label{eq:58}
    MSE(\hat m_{h}(x; 1)) = \frac{1}{nh} R(K) v(x) + \frac{1}{4} h^{4}
    (m''(x))^{2} \mu_{2}(K) + o(\frac{1}{nh} + h^{4})
  \end{align}
\end{thm}

\subsection{Splines}
\label{sec:splines}


Let $n \geq 3$, and consider for a fixed homoscedastic design
\begin{align}
  \label{eq:64}
  Y_{i} = m(x_{i}) + \sigma \epsilon_{i}
\end{align} where $\epsilon_{i}$ are \iid with $\E{\epsilon_{i}} = 0$,
$\Var{\epsilon_{i}} = 1$.

Another natural idea to estimate the regression curve $m$ is to
balance the fidelity of the fit to the data and the roughness of the
resulting curve.  This can be done by minimizing
\begin{align}
  \label{eq:65}
  \sum_{i=1}^{n} (Y_{i} - \tilde g(x_{i}))^{2} + \lambda \int \tilde
  g''(x)^{2} dx
\end{align} over $\tilde g \in S_{2}[a, b]$, the set of twice
continuously differentiable functions on $[a, b]$. $\lambda$ is a
regularization parameter. As $\lambda \rightarrow \infty$, the curve
is very close to the linear regression line. As $\lambda \rightarrow
0$, the resulting curve closely fits the observations.

\begin{defn}
  \label{defn:nonparametric_regression:2}
  A cubic spline is a function $g: [a, b] \rightarrow \R$ satisfies
  \begin{enumerate}
  \item $g$ is a cubic polynomial on $[(a, x_{1}), (x_{1}, x_{2}),
    \dots, (x_{n}, b)]$.
  \item $g$ is twice continuously differentiable on $[a, b]$.
  \end{enumerate}
\end{defn}

\begin{proposition}
  For a given $\mathbf{g} = (g_{1}, \dots, g_{n}^{T})$, there exists a unique
  natural cubic spline $g$ with knots $x_{1}, \dots, x_{n}$ - so
  $g(x_{i}) = g_{i}$ for $i = 1, \dots , n$.  Moreover, there exists a
  nonnegative definite matrix $K$ such that
  \begin{equation}
    \label{eq:59}
    \int_{a}^{b} g''(x)^{2}dx = g^{T}K g
  \end{equation}
  We call $g$ the \textbf{natural cubic spline} interploant to $g$ at
  $x_{1}, \dots, x_{n}$.
\end{proposition}

\begin{thm}
  \label{defn:nonparametric_regression:3}
  For any $\tilde g \in S_{2}[a, b]$ satisfying $\tilde g(x_{i})=
  g_{i}, i=1, \dots, n$, the cubic spline interpolant to $g$ at
  $\mathbf{g} = g_{1}, \dots, g_{n}$ uniquely minimizes
  \begin{align}
    \label{eq:66}
    \int_{a}^{b} \tilde g''(x)^{2} dx
  \end{align} over $\tilde g \in S_{2}[a, b]$.
\end{thm}

Recall that $Y_{i} = m(x_{i}) + \sigma \epsilon_{i}$, $m \in
S_{2}[a,b]$, $0 < x_{1} < \dots < x_{n} < b$.  We seek to minimize
\begin{align}
  \label{eq:70}
  \mathcal{G}_{\lambda}(\tilde g) = \sum_{i=1}^{n} (Y_{i} - \tilde
  g(x_{i}))^{2} + \lambda \int_{a}^{b}  \tilde g''(x)^{2} dx 
\end{align} over $\tilde g \in S_{2}[a, b]$.

\begin{thm}
  \label{defn:nonparametric_regression:4}
  For each $\lambda > 0$, there is a unique solution $\hat g$ minimizing
  $\mathcal{G}(\tilde g)$over $\tilde g \in S_{2}[a, b]$.  This is the
  natural cubic spline
  \begin{equation}
    \label{eq:71}
    \hat g = (I + \lambda K)^{-1} Y
  \end{equation}  
\end{thm}

Cross validation method validates the estimated curve without the
$i$-th observation by comparing the $i$-th value
\begin{align}
  \label{eq:74}
  CV(\lambda) = \frac{1}{n} \sum_{i=1}^{n} (Y_{i} - \hat g_{-i, \lambda}(x_{i}))^{2}
\end{align}

where $\hat g_{-i, \lambda}$ is chosen by minimizing
$\mathcal{G}_{\lambda}$ over all data points except the $i$-th,

\begin{align}
  \label{eq:76}
  \sum_{j \neq i}^{n} (Y_{j} - \tilde g(x_{j}))^{2} + \lambda
  \int_{a}^{b} \tilde g''(x)^{2} dx
\end{align} (*)

\begin{thm}
  \label{defn:nonparametric_regression:5}
  \begin{align}
    \label{eq:75}
    CV(\lambda) = \frac{1}{n} \sum_{i=1}^{n} (\frac{Y_{i} - \hat
      g_{\lambda}(x_{i})}{1 - A_{ii}})^{2}
  \end{align}
  where $A = (I + \lambda K)^{-1}$ and
  \begin{equation}
    \label{eq:80}
    \int_{-\infty}^{\infty}  \hat g_{\lambda}''(x)^{2} dx = \hat
    g_{\lambda}(\mathbf{x})^{T} K \hat g_{\lambda}(\mathbf{x})
  \end{equation}
\end{thm}

\section{Nearest Neighbour Classification}
\label{sec:near-neighb-class}

\begin{defn}
  \label{sec:near-neighb-class-1}
  A function $g: \R^{d} \rightarrow \{ 0, 1 \}$ is called a
  classifier. If the distribution of $(X, Y)$ are known, we can
  minimize the risk $\Prob{g(X) \neq Y} = L(g)$ over $g: \R^{d}
  \rightarrow \{ 0, 1 \}$. The minimizer $g^{\star}$ is called a Bayes
  classifier, and $L(g^{d})$ is called the Bayes risk.
\end{defn}

\begin{lem}
  For a classifier $\tilde g$ whcih has the form
  \begin{equation}
    \label{eq:90}
    \tilde g(x) =
    \begin{cases}
      1 & \hat \nu(x) > \frac{1}{2} \\
      0 & \text{otherwise}
    \end{cases}
  \end{equation} we have
  \begin{equation}
    \label{eq:93}
    \Prob{\tilde g(X) \neq Y} - L^{\star} \leq 2 \E{\| \hat \nu(X) - \nu(X)}
  \end{equation}
\end{lem}

\begin{defn}[$k$-nearest neighbour classification]
  \label{defn:nearest_neighbour_classification:1}
  A \knn classifier $g_{n}$ is defined by
  \begin{align}
    \label{eq:94}
    g_{n}(x) =
    \begin{cases}
      1 & \sum_{i=1}^{n} W_{ni}(X) \I{Y_{i} = 1} > \sum_{i=1}^{n}
      W_{ni}(X) \I{Y_{i} = 0} \\
      0 & \text{otherwise}
    \end{cases}
  \end{align}
  which is equivalent to
  \begin{align}
    \label{eq:96}
    \sum_{i=1}^{n} W_{ni}(X) \I{Y_{i} = 1} > \frac{1}{2} \iff
    \sum_{i=1}^{n} W_{ni}(X) Y_{i} > \frac{1}{2}
  \end{align}
  
  where
  \begin{align}
    \label{eq:95}
    W_{ni}(X) = \frac{1}{k} 
  \end{align} if $X_{i}$ is a $k$-nearest neighbour of $X$, and zero otherwise.
\end{defn}

\begin{defn}
  \label{defn:nearest_neighbour_classification:2}
  For a certain distirbution of $(X, Y)$, we say $g_{n}$ is consistent
  if $\Prob{g_{n}(X) \neq Y} - L^{\star} \rightarrow 0$ .

  We say $g_{n}$ is strongly consistent if
  \begin{equation}
    \label{eq:97}
    \Prob{\lim_{n \rightarrow \infty} L(g_{n}) = L(g^{\star})} = 1
  \end{equation}
\end{defn}

\begin{thm}
  \label{defn:nearest_neighbour_classification:3}
  If $k \rightarrow \infty$, $\frac{k}{n} \rightarrow 0$, then for all
  distributions of $(X, Y)$, the \knn estimates $g_{n}$ are consistent.
\end{thm}

\section{Minimax Lower Bounds}
\label{sec:minimax-lower-bounds}


\begin{defn}
  \label{sec:minimax-lower-bounds-1}
  As a first attempt to understand a nonparametric estimation problem,
  we consider a minimax risk,
  \begin{equation}
    \label{eq:105}
    R(\Theta) = \inf_{\tilde \theta} \sup_{\theta \in \Theta}
    \E_{\theta} L(\tilde j, \theta).
  \end{equation}
\end{defn}

\begin{defn}
  \label{sec:minimax-lower-bounds-2}
  If we can find our $\hat \theta^{\star}$, which minimizes
  $\sup_{\theta \in \Theta} \E_{\theta} L(\tilde \theta, \theta)$ we
  call $\hat \theta^{\star}$ our minimax estimator. However, it is
  very difficult to find $\hat \theta^{\star}$. Let $c \gamma_{n} \leq
  R(\Theta) \leq C \gamma_{n}$, we call $\gamma_{n}$ is minimax rate
  of convergence.
\end{defn}

\begin{lem}[Le Cam's two points lemma]
  Let $\mathcal{P}$ be probability measures on $(\mathcal{X}, \mathcal{A})$,
  and let $(\Theta, d)$ be the pseudo-metric space, with
  \begin{equation}
    \label{eq:108}
    d: \Theta \times \Theta \rightarrow [0, \infty)
  \end{equation} given by
  \begin{equation}
    \label{eq:109}
    d(\theta_{1}, \theta_{2}) = d(\theta_{2}, \theta_{1}),
    d(\theta_{1}, \theta_{2}) + d(\theta_{2}, \theta_{3}) \geq A
    d(\theta_{1}, \theta_{3})
  \end{equation}

  Let $\theta: \mathcal{P} \rightarrow \Theta$, $\theta(P)$ is the
  parameter of interest ($P \in \mathcal{P}$).  With $\theta_{0} =
  \theta(P_{0})$ , $\theta_{1} = \theta(P_{1})$, under two conditions,
  \begin{enumerate}
  \item $d(\theta_{0}, \theta_{1} ) \geq \delta > 0$,
  \item $h^{2}(P_{0}, P_{1}) \leq C < 1$
  \end{enumerate}
  where $h^{2}(P_{0}, P1)$ is the Hellinger distance $\int (\sqrt
  {dP_{0}} - \sqrt{dP_{1}})^{2}$,
  the we have for all estimators $\tilde \theta$,
  \begin{equation}
    \label{eq:110}
    \sup_{P \in \mathcal{P}} \E_{P} d(\tilde \theta, \theta(P)) \geq
    \frac{A \delta}{2}(1 - \sqrt{C})
  \end{equation}
\end{lem}

\begin{thm}[Nonparametric regression]
  \label{defn:minimax_lower_bounds:1}
  Let $Y_{i} = m(x_{i}) + \epsilon_{i}$, $\epsilon_{i} \sim N(0, 1)$,
  $x_{i} = \frac{i}{n}$, $m \in \Theta$ with $\Theta$ the set of all
  twice continuously differentiable functions on $[0, 1]$, $m''(x) <
  \infty$.
  Then for any estimator $\tilde m$ and any $x_{0} \in [0, 1]$,
  \begin{equation}
    \label{eq:115}
    \sup_{m \in \Theta} \E{(\tilde m(x) - m(x_{0}))^{2}} \geq C n^{-\frac{4}{5}}
  \end{equation}
\end{thm}

\section{Extreme Value Theory}
\label{sec:extreme-value-theory}

Let $X_{n}$ be an IID sample from a distribution function $F$, and
denote $X_{(n)} = \max \{ X_{1}, \dots, X_{n} \}$ as the maximum order
statistic.

Without any normalization, $X_{(n)} \rightarrow x_{\star} = \inf \{ x:
F(x) = 1 \}$.

This is not overly interesting, since the limit distribution is
degenerate (we call $F$ non-degenerate if there does not exists $a \in
\R$ such that $F(x) = \I{x \geq a}$)

We may ask if there exists $\{ a_{n} \} > 0$, $\{ b_{n} \} > 0$, and a
nondegenerate $G$ such that
\begin{equation}
  \label{eq:118}
  \Prob{\frac{X_{(n)} - b_{n}}{a_{n}} \leq x} \rightarrow G(x)  
\end{equation} for all continuity points $x$ of $G$

Classical extreme value theory starts by asking:
\begin{enumerate}
\item What kind of $G$ appears in the limit of \eqref{eq:118}?
\item Cna we characterize $F$ such that \eqref{eq:118} holds for a
  specific limit distribution $G$?
\end{enumerate}

For the first question, we have the Extremal Types theorem. For the
second question, we have the ``domain of attraction'' problem.

Recall that $\Prob{X_{(n)} \leq x} = F(x)^{n}$.  We say that $F$ is in
the domain of attraction of $G$ ($F \in D(G)$) if there exists $\{ a_{n} \} > 0$, $\{
b_{n} \} $ and a non-degenerate $G$ such that
\begin{equation}
  \label{eq:119}
  \Prob{\frac{X_{(n)} - b_{n}}{a_{n}} \leq x} = [\text{$F(a_{n} x +
    b_{n})^{n} \rightarrow G(x)$ for all continuity points $x$ of $G$}].
\end{equation} and write $F(a_{n} x + b_{n})^{n} \hookrightarrow G(x)$.

We say that $G_{1}$ and $G_{2}$ are of same type if $G_{1}(ax + b) =
G_{2}(x)$ for some $a > 0, b$.

The next lemma shows that if $F \in D(G_{1})$ and $F \in D(G_{2})$,
then $G_{1}$ and $G_{2}$ are of the same type.

\begin{lem}
  Suppose $X_{n}$ is an IID sample from $F$ and there exists $\{ a_{n}
  \} > 0, \{ b_{n} \}$ and non-degenerate $G$ such that $F(a_{n} x
  +b_{n})^{n} \hookrightarrow G(x)$ .  Then there exists $\{ \alpha_{n} \} > 0,
, \{ \beta_{n} \}$ and non-degenerate $G_{\star}$ such that
$F(\alpha_{n} x _{ \beta_{n}})^{n} \hookrightarrow G_{\star}(x)$. if and
only if $\frac{\alpha_{n}}{a_{n}} \rightarrow a$ for some $a > 0$, and
$\frac{\beta_{n} - \beta}{a_{n}} \rightarrow b$ for some $b$.

Then we can let $G_{\star}(x) = G(ax + b)$.
\end{lem}


\begin{defn}
  \label{defn:extreme_value_theory:1}
  $G$ is \textbf{max-stable} if for every $n \in \N$, there exists
  $\{ a_{n} \} > 0, \{ b_{n} \} $ such that $G^{n}(a_{n}x + b_{n}) = G(x)$
\end{defn}

\begin{thm}
  \label{defn:extreme_value_theory:2}
  $D(g)$ is non-empty if and only if $G$ is max-stable.
\end{thm}


\begin{thm}
  \label{defn:extreme_value_theory:3}
  If $F \in D(G)$, then $G$ must belong to the following distributions
  (within type):

  \begin{enumerate}
  \item Frechet - $G_{1, \alpha}(x) = \exp(-x^{-\alpha})$, $x > 0$,
    $\alpha > 0$
  \item Negative Weibull - $G_{2, \alpha} = \exp(- (-x)^{\alpha})$, $x
    < 0$, $\alpha > 0$
  \item Gumbel - $G_{3}(x) = \exp(-\exp(-x))$, $x \in \R$.
  \end{enumerate}

  Conversely, these distributions can appear as such limits in \eqref{eq:118}.
\end{thm}

\begin{remark}
  \begin{enumerate}
  \item Using $X_{(1)} = -\max \{ -X_{1}, \dots, -x_{n} \} $, we have
    equivalent theorems in terms of normalized minima.
  \item Sometimes, we cannot have nondegenreate $G$ of normalized
    maxima  - for example $X_{1}, \dots, X_{n} \sim
    Bern(\frac{1}{2})$, $X_{(n)}$.
  \item We can combine these three types into Generalized Extreme
    Value Distribution (GEV) -
    \begin{equation}
      \label{eq:122}
      G(x; \mu, \sigma, \gamma) = \exp(- (1 + \gamma(\frac{x-\mu}{\sigma}))^{-\frac{1}{\gamma}})
    \end{equation} with $1 + \gamma(\frac{x - \mu}{\sigma}) > 0$, $\mu
    \in \R, \gamma \in \R, \sigma > 0$.
  \end{enumerate}

  We have Frechet corresponds to $\gamma > 0$, $\alpha =
  \frac{1}{\gamma} $, NW is $\gamma < 0$, $\alpha =
  -\frac{1}{\gamma}$, and Gumbel corresponds to the case where $\gamma
  \rightarrow 0$.
\end{remark}

\subsection{Necessary and Sufficient Conditions for Convergence}
\label{sec:necess-suff-cond}

\begin{defn}
  \label{sec:necess-suff-cond-1}
  We say a function $l: [C, \infty] \rightarrow (0, \infty)$ is ``slowly
varying'' if $\lim_{x \rightarrow \infty} \frac{l(tx)}{l(x)} = 1$ for
all $t > 0$.  For example, $l (x) = \log x, \log \log x, (\log
x)^{\alpha}$.  

We say a function $r_{\alpha}: [C, \infty) \rightarrow (0, \infty)$ is
``regularly varying'' with an index $a \in \R$ if $r_{\alpha}(x) =
x^{-\alpha}l(x)$ where $l$ is slowly varying - so $r_{2}(x) = x^{-2}
\log x$.

We define an \textbf{expected residual lifetime} as
\begin{align}
  \label{eq:120}
  R(x) = \E{X - x | X > x} = \frac{1}{1 - F(x)} \int_{x}^{x_{\star}}
  (1 - F(y)) dy
\end{align} where $x_{\star} = \inf \{ x : F(x) = 1 \}$, and
$\overline F(x) = 1 - F(x)$
\end{defn}

\begin{thm}
  \label{defn:extreme_value_theory:4}
  $F \in D(G_{1, \alpha})$ if and only if $x_{\star} = \infty$,
  $\overline F(x) = x^{-\alpha} l(x)$ where $l$ is slowly varying.  We
  can choose $b_{n} = 0$, $a_{n} = F^{-1}(1 - \frac{1}{n})$ for which
  $F^{n}(a_{n} x + b_{n}) \hookrightarrow G_{1, \alpha}(x)$ is
  satisfied.

  $F \in D(G_{2, \alpha})$ if and only if $x_{\star} < \infty$,
  $\overline F(x_{\star} - \frac{1}{x} ) = x^{-\alpha} l(x)$, with $l$
  slowly varying for $x > 0$.  We can choose $b_{n} = x_{\star}$,
  $a_{n} = x_{\star} - F^{-1}(1 - \frac{1}{n})$ for convergence.

  $F \in D(G_{3})$ if and only if
  \begin{align}
    \label{eq:124}
    \frac{\overline F(x + t R(x))}{\overline F(x)}  \rightarrow e^{-t}
  \end{align} We can choose $b_{n} = F^{-1}(1 - \frac{1}{n})$, $a_{n}
  = R(b_{n})$.
\end{thm}

\begin{lem}[$\star$]
  Suppose there exists $a_{n} > 0$, $b_{n}$ such that $n(1 - F(a_{n}
  x + b_{n})) \rightarrow{ u(x)}$.  Then
  \begin{align}
    \label{eq:125}
    F^{n}(a_{n} x + b_{n}) \hookrightarrow \exp(-u(x))
  \end{align}
\end{lem}


\bibliographystyle{plainnat}
\bibliography{../../common/bibliography}
\end{document}
