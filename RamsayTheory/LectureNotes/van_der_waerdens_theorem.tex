
\chapter{Van Der Waerden's Theorem}
\label{cha:van-der-waerdens}

\begin{thm}
  \label{defn:van_der_waerdens_theorem:1}
  Whenever $\N$ is two-coloured, there exists a monochromatic
  arithmetic progression of length $m$, for any $m \in \N$.
\end{thm}

\begin{thm}
  \label{defn:van_der_waerdens_theorem:2}
  Let $k \in \N$. Then there exists $n \in \N$ such that whenever
  $[n]$ is $2$-coloured, there exists a monochromatic arithmetic
  progression of length $m$.
\end{thm}

One idea in the proof is - we show that $\forall m, k \in \N$,
whenever $[n]$ is $k$-coloured, there exists a monochromatic
arithmetic progression of length $m$.\sidenote{Harder results
  \textbf{could} be easier to prove - if the proof is by induction!}

Write $W(m, k)$ for the least such $n$ (if it exists) - a ``Wan Der
Waerden's number''.  Let $A_{1}, \dots, A_{r}$ be arithmetic
progressions of length $m-1$, say
\begin{equation}
  \label{eq:1}
  A_{i}= \{ a_{i}, a_{i} + d_{i}, \dots, a_{i}+ (m-2) * d_{i} \}
\end{equation}

We say $A_{1}, \dots, A_{r}$ are \textbf{focused} at $f$ if $a_{i} +
(m-1) d_{i} = f$ for all $i$ - for example, $\{ 1, 4\}$ and $\{ 5, 6
\}$ are focused at 7.  If each $A_{i}$ are monochromatic (for a given
colouring), with no two $A_{i}$ the same colour, say that $A_{i},
\dots, A_{r}$ are colour-focused at $f$.\sidenote{So if we have a
  $r$-colouring, and $A_{1}, \dots, A_{r}$ are colour-focused.  Then,
  we get a monochromatic arithmetic progression of length $m$ - by
  asking, what colour is the focus $f$.}

\begin{proposition}
  Let $k \in \N$.  Then there exists $n \in \N$ such that whenever
  $[n]$ is $k$-coloured, there exists a monochromatic arithmetic
  progression of length 3.\sidenote{This will be subsumed by Van Der
    Waerden's theorem}
\end{proposition}

\begin{lem}
  We claim the following result - for all $r \leq k$, there exists $n$
  such that whenever $[n]$ is $k$-coloured, there exists a
  monochromatic arithmetic progression of length 3 or there exist $r$
  colour-focused arithmetic progressions of length 2.
\end{lem}

\begin{proof}
  Proceed by induction on $r$.  This is true for $r = 1$ (setting $n =
  k + 1$.)  We'll show that if $n$ is suitable for $r - 1$ then
  \begin{equation}
    \label{eq:2}
    (k^{2n} + 1) \cdot 2n
  \end{equation} is suitable for $r$.  Indeed, given a $k$-colouring
  of $[(k^{2n} + 1) 2n]$ with no monochromatic arithmetic progression
  of length 3.

  Break up $[(k^{2n} + 1)2n]$ into intervals $B_{1}, \dots, B^{k^{2n}
    + 1}$ of length $2m$ - so $B_{i} = [2m(i-1)  +1, 2ni]$ for $i = 1,
  2, \dots, k^{2n} +1$.Now, there are $k^{2m}$ ways to $k$-colour a
  block. Thus, there exist two blocks coloured identically - say
  $B_{s}$ and $B_{s+t}$.  

  ...
  \todo{Complete this proof}
\end{proof}

\todo{Missed lecture...}

\begin{thm}[Strengthened Van Der Warden]
  \label{defn:van_der_waerdens_theorem:3}
  Let $m \in \N$.  Then whenever $\N$ is finitely coloured there
  exists an arithmetic progression that (together with it's common
  difference) is monochromatic.
\end{thm}

\begin{proof}
  Induction on $k$, the number of colours.  Given $n$ suitable for $k
  - 1$ (whenever $[n]$ is $k-1$ coloured there exists a monochromatic
  arithmetic progression with common difference of length $n$), then
  $W(n(m-1) + 1, k)$ is suitable for $k$.

  Given $k$-colouring of $[W(n(m-1) + 1, k)]$, there exists a
  monochromatic arithmetic progression of length $n(m-1) + 1$ - say
  $a, a + d, a + 2d, \dots, a + n(m-1)d$. If $d$ or $2d$ or \dots{} is
  the same color as the arithmetic progression, we are done.
  Otherwise, $\{ d, 2d, \dots, nd \}$ is $k-1$coloured, so we are done
  by induction.
\end{proof}

\begin{remark}
  \begin{enumerate}
  \item Henceforth, we do not care about bounds.
  \item The case $k = 2$ is Shur's theorem - whenever $\N$ is finitely
    coloured, there exist $x, y, z$ monochromatic with $x + y = z$.
  \end{enumerate}
\end{remark}

%%% Local Variables: 
%%% mode: latex
%%% TeX-master: "master"
%%% End: 
