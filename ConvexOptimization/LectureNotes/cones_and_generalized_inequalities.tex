\chapter{Cones and Generalized Inequalities}
\label{cha:cones-gener-ineq}

\begin{defn}
  \label{defn:cones_and_generalized_inequalities:1}
  $K \subseteq \R^{n}$ is a cone if and only if $0 \in K$ and $\lambda x
  \in K$ for all $x \in K, \lambda \geq 0$.
\end{defn}

\begin{defn}
  \label{defn:cones_and_generalized_inequalities:1}
  A cone in \textbf{pointed} if and only if
  \begin{equation}
    \label{eq:27}
     x_{1} + \dots + x_{n} = 0 \iff x_{1} = \dots = x_{n} = 0
  \end{equation}
\end{defn}

\begin{exmp}
  \label{defn:cones_and_generalized_inequalities:1}
  \begin{enumerate}
  \item $R^{n}$ is a pointed cone.
  \item $\{ (x_{1}, x_{2}) \in \R^{2} | x_{2} > 0 \}$ is not a cone.
  \item $\{ (x_{1}, x_{2}) \in \R^{2} | x_{2} \geq 0 \}$ is a cone,
    but is not pointed.
  \item $\{ x \in \R^{n} | x_{i} \geq 0, i \in 1, \dots, n \}$ is a
    pointed cone.
  \end{enumerate}
\end{exmp}

\begin{proposition}
  $K \subseteq \R^{n}$ be any set.  Then the following are equivalent:
  \begin{enumerate}
  \item $K$ is a convex cone.
  \item $K$ is a cone and $K + K \subseteq K$.
  \item $K \neq \emptyset$ and $\sum_{i=0}^{n} \alpha_{i} x_{i} \in K$
    for all $x_{i} \in K$, $\alpha_{i} \geq 0$.
  \end{enumerate}
\end{proposition}

\begin{proof}
  Exercise.
\end{proof}

\begin{proposition}
  If $K$ is a convex cone, then $K$ is pointed if and only if $K \cap
  (-K) = \{ 0 \}$.
\end{proposition}

\begin{proof}
  Exercise.
\end{proof}

\begin{defn}
  \begin{equation}
    \partial f(x) = \{ v \in \R^{n} | f(x) + \langle v, y - x \rangle
    \leq f(y) \forall y \}
  \end{equation}
  
  \begin{equation}
    f'(x) = \lim_{t \rightarrow 0} \frac{f(x + t) - f(x)}{t}
  \end{equation}
\end{defn}

\begin{proposition}
  Let $f, g: \R^{n} \rightarrow \bar \R$ be convex.  Then
  \begin{enumerate}
  \item \label{item:5} $f$ differentiable at $x$ implies $\partial f(x) = \{ \nabla
    f(x) \}$.
  \item \label{item:6} $f$ differentiable at $x$, $g(x) \in \R$ implies
    \begin{equation}
      \partial(f + g)(x) = \nabla f(x) + \partial g(x)
    \end{equation}
  \end{enumerate}
\end{proposition}

\begin{proof}
  We have
  \begin{enumerate}
  \item Equivalent to \ref{item:6} with $g = 0$.
  \item To show $LHS \supseteq RHS$, if $v \in \partial g(x)$, then
    $\langle v, y - x \rangle \leq g(y)$, which by Theorem 3.12 in
    notes gives
    \begin{align}
      \label{eq:28}
      \langle \nabla f(x), y - x \rangle + f(x) \leq f(y) \\
      \langle v + \nabla f(x), y - x \rangle + (f + g)(x) \leq (f +
      g)(y) \\
      \iff v + \nabla f(x) \in \partial (f + g)
    \end{align}

    The other direction is given as
    \begin{align}
      \label{eq:29}
      \liminf{z \rightarrow x} \frac{f(z) + g(z) - f(x) - g(x) -
        \langle v, z - x \rangle}{\| z - x \|} &\geq 0 \\
      \Rightarrow \liminf_{z \rightarrow x} \frac{g(z) - g(x) - \langle v -
        \nabla f(x), z - x \rangle}{\| z - x \|} &\geq 0 \\
      \Rightarrow \liminf_{t \downarrow 0} \frac{g(z(t)) - g(x) -
        \langle v - \nabla f(x), (1 - t) x + ty - y \rangle}{t \| y -
        x \|} &\geq 0 \\
      \Rightarrow \liminf_{t \downarrow 0} \frac{t(g(y) - g(x)) - t
        \langle v - \nabla f(x), y - x \rangle}{t \| y - x \|} &\geq 0
    \\
    \Rightarrow g(y) - g(x) - \langle v - \nabla f(x), y - x \rangle
    &\geq 0
    \Rightarrow v - \nabla f(x) &\in \partial g(x) \\
    v &\in \nabla f(x) - \partial g(x)
    \end{align}
  \end{enumerate}
\end{proof}

\begin{thm}
  \label{defn:cones_and_generalized_inequalities:2}
  Let $f: \R^{n} \rightarrow \bar R$ be proper.  Then
  \begin{equation}
    \label{eq:30}
    x \in \argmin f \iff 0 \in \partial f(x)
  \end{equation}
\end{thm}

\begin{proof}
  \begin{equation}
    \label{eq:31}
    0 \in \partial f(x) \iff 
    \underbrace{\langle 0, y -x \rangle}_{0} + f(x) \leq f(y)
  \end{equation}
\end{proof}

\begin{defn}
  \label{defn:cones_and_generalized_inequalities:3}
  For a convex set $C \subseteq \R^{n}$ and point $x \in C$,
  \begin{equation}
    \label{eq:33}
    N_{C}(x) = \{ v \in \R^{n} | \langle v, y -x \rangle \leq 0
    \forall y \in G \}
  \end{equation} is the ``normal cone'' of $x$.  By convention,
  $N_{C}(x) = \emptyset$ for all $x \notin C$.
\end{defn}

\begin{proposition}
  Let $C$ be convex and $C \neq \emptyset$.  Then
  \begin{equation}
    \label{eq:34}
    \partial \I{C}(x) = N_{C}(x)
  \end{equation}
\end{proposition}

\begin{proof}
  For $x \in C$, we have
  \begin{equation}
    \label{eq:35}
    \partial \I{C} = \{ v | \I{C}(x) + \langle v, y - x \rangle \leq
    \I{C}(y) \forall y \in C \} = N_{C}(x)
  \end{equation}

  For $x \notin C$, \todo{Fill in proof here}.
\end{proof}


\begin{proposition}
  $C$, closed, $C \neq \emptyset$ and convex, $x \in \R^{n}$.  Then
  \begin{equation}
    \label{eq:36}
    y = \Pi_{C}(x) \iff x - y \in N_{C}(y)
  \end{equation}
\end{proposition}

\begin{proof}
  \begin{equation}
    \label{eq:37}
    y \in \Pi_{C}(x) \iff y \text{ minimizes } \underbrace{\frac{1}{2} \| y - x \|^{2} + \I{C}(y')}_{g(y)}.
  \end{equation}

  If and only if $0 \in \partial g(y)$ if and only if $0 \in y - x
  + \partial \I{C}(y)$
\end{proof}

\begin{proposition}
  Let $f: \R^{n} \rightarrow \bar \R$ be convex proper.  Then
  \begin{equation}
    \label{eq:38}
    \partial f(x) =
    \begin{cases}
      \emptyset & x \notin \text{dom} f \\
      \{ v \in \R^{n} | (u, -1) \in N_{\epi f} (x, f(x))\} & x
      \in \text{dom} f
    \end{cases} 
  \end{equation}

  If $x \in \text{dom} f$,
  \begin{equation}
    \label{eq:39}
    N_{\text{dom} f} = \{ v \in \R^{n} | (x, 0) \in N_{\epi f}
    (x, f(x)) \}
  \end{equation}
\end{proposition}

\begin{exmp}
  \label{defn:cones_and_generalized_inequalities:4}
  Let subdifferential of $f: \R \rightarrow \R, f(x) = |x|$ is
  \begin{equation}
    \label{eq:40}
    \partial f(x) = \begin{cases}
      \sign x & x \neq 0 \\
      [-1, 1] & x = 0
    \end{cases}
  \end{equation}
\end{exmp}


\begin{defn}
  \label{defn:cones_and_generalized_inequalities:5}
  For $C \subseteq \R^{n}$, define the \textbf{affine hull} as
  \begin{align}
    \aff(C) &= \cap_{\text{$A$ affine, $C \subseteq A$}} A \\
    \rint(C) &= \{ x \in \R^{n} | \text{there exists an open
      neigbourhood $N$ of $x$ such that $N \cap \text{aff}(C)
      \subseteq C$} \}
  \end{align}
\end{defn}

\begin{exmp}
  \label{defn:cones_and_generalized_inequalities:6}
  \begin{align}
    \label{eq:42}
    \aff(\R^{n}) &= \R^{n} \\
    \aff(\R^{n}) &= \R^{n} \\
    \rint([0, 1]^{2}) &= (0, 1)^{2} \\
    \rint([0, 1] \times \{ 0 \}) &= (0, 1) \times \{ 0 \}
  \end{align}
\end{exmp}

\begin{exer}
  We know \begin{align}
    \label{eq:32}
    \text{int} A \cap B \subseteq \int A \cap \int B
  \end{align}

  Does
  \begin{equation}
    \label{eq:41}
    \rint(A \cap B) \subseteq \rint A \cap \rint B
  \end{equation}

\end{exer}

\begin{proposition}
  Let $f: \R^{n} \rightarrow \bar \R$ be convex.  Then
  \begin{enumerate}
  \item
    \begin{enumerate}
    \item $g(x) = f(x + y) \Rightarrow \partial g(x) = \partial
      f(x+y)$
    \item $g(x) = f(\lambda x) \Rightarrow \partial g(x) =
      \lambda \partial f(x)$
    \item $g(x) = \lambda f(x) \Rightarrow \partial g(x) =
      \lambda \partial f(x)$
    \end{enumerate}
  \item $f: \R^{n} \rightarrow \bar \R$ proper, convex, $A \in \R^m
    \times n$ such that
    \begin{equation}
      \label{eq:43}
      \{  Ay | y \in \R^{m} \} \cap \rint \dom f \neq \emptyset
    \end{equation}

    Then for $x \in \dom (f \circ A)$ we have
    \begin{equation}
      \label{eq:44}
      \partial (f \circ A)(x) = A^{T} \partial f(Ax)
    \end{equation}
  \item Let $f_{1}, \dots f_{m}: \R^{n} \rightarrow \bar \R$ be
    proper, convex, and $\rint \dom f_{1} \cap \dots \cap \rint \dom
    f_{m} \neq \emptyset$.  Then
    \begin{equation}
      \label{eq:45}
      \partial f(x) = \partial f_{1}(x) + \dots + \partial f_{m}(x)
    \end{equation}
  \end{enumerate}
\end{proposition}

%%% Local Variables: 
%%% mode: latex
%%% TeX-master: "master"
%%% End: 
