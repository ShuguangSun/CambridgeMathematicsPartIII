b\chapter{First-order Methods}
\label{cha:first-order-methods}

Idea is that we do gradient descent on our objective function $f$,
with step size $\tau_{k}$.

\begin{defn}
  \label{defn:first_order_methods:1}
  Let $f: \R^{n} \rightarrow \bar \R$, then
  \begin{enumerate}
  \item
    \begin{equation}
      \label{eq:1}
      F_{\tau_{k}f}(x^{k}) = x^{k} - \tau_{k} \partial f(x^{k}) = (I -
      \tau_{k} \partial f)(x^{k})
    \end{equation}
  \item
    \begin{equation}
      \label{eq:2}
      B_{\tau_{k} f}(x^{k}) = (I + \tau_{k} \partial f)^{-1}(x^{k})
    \end{equation} so
    \begin{align}
      \label{eq:3}
    \end{align} \todo{Fill in}
  \end{enumerate}
\end{defn}

\begin{proposition}
  If $f: \R^{n} \rightarrow \bar \R$ is proper, lowersemicontinuous,
  and convex, for $\tau > 0$, 

  \begin{equation}
    \label{eq:62}
    B_{\tau_{k} f}(x) = \argmin_{y} \{
    \frac{1}{2\tau_{k}} \| y - x \|_{2}^{2} + f(y) \}
  \end{equation}
\end{proposition}

\begin{proof}
  \begin{align}
    \label{eq:72}
    y \in \argmin \{ ... \} &\iff 0 \in \frac{1}{\tau_{k}}(y -x
    ) \partial f(y) \\
    &\
  \end{align}
\end{proof}

%%% Local Variables: 
%%% mode: latex
%%% TeX-master: "master"
%%% End: 
