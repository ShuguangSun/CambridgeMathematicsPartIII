
\chapter{Example Sheet 3}
\label{cha:example-sheet-3}

\begin{exercises}
\item
\item
  We first show $\con \{ \grad f_{i} | i \in I(x) \}
  \subseteq \partial f(x)$.

  First, note that we have for all $x, z$ and $k \in I(x)$,
  \begin{align}
    \label{eq:41}
    f(z) \geq f_{k}(z) \geq f_{k}(x) + \IP{\grad f_{k}(x), z -x} =
    f(x) + \IP{\grad f_{k}(x), z-x}
  \end{align} 
  and so $\grad f_{k}(x) \in \partial f(x)$. 

  Now, let $g$ be a convex combination of $\grad f_{k}(x), k \in
  I(x)$.  Then we have
  \begin{align}
    \label{eq:42}
    f(x) + \IP{g, z-x} &= f(x) + \IP{\sum_{k} \lambda_{k} \grad f_{k},
      z-x} \\
    &= f(x) + \sum_{k} \IP{\lambda_{k} \grad f_{k}, z - x} \\
    &\leq f(x) + \sum_{k} \lambda_{k} (f(z) - f(x)) \\
    &= f(x) = f(z) - f(x) \\
    &= f(z)
  \end{align} as required.

  We must now show $\partial f(x) \subseteq \con \{ \grad f_{i} | i
  \in I(x) \}$.

  Recall that $\partial f(x) = \{ v | (v, -1) \in N_{\epi f}(x, f(x))
  \}$.

  Then we claim
  \begin{align}
    \label{eq:51}
    N_{\epi \{ \max_{i} f_{i}  \}} (x, f(x)) = \sum_{i=1}^{n} N_{\epi
      f_{i}} (x, f_{i}(x)).
  \end{align} We show

  \todo{Fill in}

\item
  We have
  \begin{align}
    \label{eq:43}
    y \in B_{\tau f^{\star}}(x^{\star}) \\
    \iff y \in (I + \tau \partial f^{\star})^{-1}(x^{\star}) \\
    \iff y \in (I + \tau (\partial f^{-1})^{-1})^{-1}(x^{\star}) \\
    \iff x^{\star} \in (I + \tau (\partial f)^{-1})(y) \\
    \iff 0 \in y - x^{\star} + \tau (\partial f)^{-1}(y) \\
    \iff \frac{x^{\star} - y}{\tau} \in (\partial f)^{-1}(y) \\
    \iff y \in \partial f (\frac{x^{\star} -y}{\tau}) \\
    \iff 0 in y - \partial f(\frac{x^{\star}-y}{\tau}) \\
    \iff 0 \in y + \frac{1}{\tau} \partial f(y - x^{star}) \\
    \iff 0 \in (I + \frac{1}{\tau} \partial f(\cdot - x^{\star}))(y) \\
    \iff y \in (I + \frac{1}{\tau} \partial f(\cdot - x^{\star}))^{-1}(0) \\
    \iff y \in (I + \frac{1}{\tau} \partial f)^{-1}(-x^{\star})
  \end{align}
\item
  Consider $f_{z}(x, u) = k(x) + h(z + u -x)$.  Then
  \begin{align}
    \label{eq:44}
    p(u) = \inf_{x} f_{z}(x, u) \\
    &= \inf_{y} k(y) + h(z + u - y) \\
    &= F(z + u)
  \end{align} as required.

  Thus $p(0) = F(z)$. By properness of $h, z$, $F(z) = p(0) \in \R$,
  and by lsc of $h, z$, $F(z) = p(0)$ is lsc.  Thus strong duality
  holds.

  Consider the dual objective.  First, we compute $f^{\star}(v, y)$.
  We have
  \begin{align}
    \label{eq:45}
    f^{\star}(v, y) = \IP{-z, y} + k^{\star}(y + v) + h^{\star}(y).  
  \end{align}

  Then $\psi(y) = - f^{\star}(0, y) = \IP{z, y} - k^{\star}(y) -
  h^{\star}(y)$.

  Thus, we have $\sup_{y} \psi(y) = \sup_{y} \IP{z, y} - h^{\star}(y)
  - k^{\star}(y) = (h^{\star} + k^{\star})^{\star}(z)$.
\item
  Given an $LP$ of the form $\max \IP{c, x}$ s.t $Ax \leq b$, an
  $SOCP$ of the form $\max {c, x}$ s.t $\| A_{i} x + b_{i}\|_{2} \leq
  \IP{c_{i}, x} + d_{i}$, $Fx = g$, and an $SDP$ of the form $\inf
  \IP{c, x}$ s.t. $Ax - b$ is positive semidefinite.

  Note that by setting $A_{i}, b_{i} = 0$, we obtain that $LP \subseteq
  SOCP$.  Now, note by setting 
\item
\item
\item
\item
\item
  Let our pre-Hilbert space $\mathcal{G}$ be given as the span of $\kappa_{x}$,
  and let $f, g \in G$.  Thus $f = \sum_{i=1}^{n} a_{i} \kappa_{x_{i}}, g
  = \sum_{j=1}^{m} b_{j} \kappa_{x'_{j}}$.  Then let our inner product
  on $G$ be given as
  \begin{align}
    \label{eq:46}
    \IP{f, g}_{\mathcal{G}} = \sum_{i=1}^{n} \sum_{j=1}^{m} a_{i}
    \overline{b_{j}} \kappa(x_{i}, x'_{j})
  \end{align}  This trivially satisfies the properties of the norm -
  linearity, conjugate symmetric, and positive definite.

  Now, let $\mathcal{H}$ be the metric space completion of
  $\mathcal{G}$.  By Hilbert space theory, $\mathcal{G}$ is dense in
  $\mathcal{H}$, and we can write every element of $\mathcal{H}$ in
  the form
  \begin{align}
    \label{eq:47}
    \sum_{i=1}^{\infty} a_{i} \kappa_{x_{i}}.
  \end{align} with appropriate $L^{2}$ condition on $a_{i}$.

  Let $f = \sum_{i=1}^{\infty} a_{i} \kappa_{x_{i}}$.  Then
  \begin{align}
    \label{eq:48}
    \IP{k_{x}, f} = \sum_{i=1}^{\infty} a_{i} \kappa(x_{i}, x) = f(x).
  \end{align} as required.

  Let $\kappa$ be a Mercel kernel, and let $\mathcal{H}$ be the
  Hilbert space constructed before.  Then
  \begin{align}
    \label{eq:49}
    \nu : \mathcal{F} \rightarrow \mathcal{H} \\
    \nu(x) \mapsto \kappa_{x}
  \end{align} satisfies this requirement, with
  \begin{align}
    \label{eq:50}
    \IP{\nu(x), \nu(x')} = \IP{\kappa_{x}, \kappa_{x'}} = \kappa(x, x')
  \end{align}
\end{exercises}


%%% Local Variables: 
%%% mode: latex
%%% TeX-master: "master"
%%% End: 
