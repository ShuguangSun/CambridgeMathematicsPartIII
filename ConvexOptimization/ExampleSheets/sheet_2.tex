\chapter{Example Sheet 2}
\label{cha:example-sheet-2}

\begin{exercises}
\item The first direction is trivial. Assume $0 \in \int (C - D)$ and
  a separating hyperplane $(b, \beta)$ exists. Then there exists $\epsilon
  > 0$ such that $B_{\epsilon}(0) \subseteq \int (C - D)$. Let $b_{i}$
  be some non-zero element of $b$. Thus, there exists $(x^{1}, y^{1}),
  (x^{2}, y^{2}), (x^{3}, y^{3}) \in C \times D$ such that $\IP{b,
    x^{1} - y^{1}} = 0$, $\IP{b, x^{2} - y^{2}} = \epsilon b_{i}$ and
  $\IP{b, x^{3} - y^{3}} = -\epsilon b_{i}$.

  Note however, that the condition of the separating hyperplane is
  such that $\IP{b, x - y} \leq 0$ for all $(x, y) \in C \times D$.
  By contradiction, we have that no such hyperplane exists.

  \todo{Opposite direction}
\item
  At points of continuity of $f$, the subgradient is simply the singleton set
  $\{ \grad f \}$.  Thus, for $x \neq 0$, $\partial f(x) = \{
  \frac{x}{\| x \|}$.  At $x = 0$, we seek the set of $v \in \R^{n}$
  such that
  \begin{equation}
    \label{eq:31}
    V = \| x \| \geq 0 + \IP{v, x} = \IP{v, x}
  \end{equation} for all $x \in \R^{n}$.

  We claim that $V = B_{1}$.  First, let $v \in B_{1}$. Then by
  definition of the norm as
  \begin{equation}
    \label{eq:32}
    \| x \| = \sup_{v \leq 1} \IP{v, x},
  \end{equation} we have $v \in V$.

  Now, let $v \in V$.  Then taking $x = v$ in \eqref{eq:32}, we have
  $\| v \| \geq \| v \|^{2}$, and so $\| v \| \leq 1$.  Thus $v \in
  B_{1}$.

  Hence, $V = B_{1}$.
\item The problem is convex (sum of composition of convex function $f:
  x \mapsto x^{2}$ with affine transform $g: x \mapsto x - a^{i}$).
  The function is convex, continuous, level bounded, and proper.
  Thus, by Theorem 2.14, $\inf f$ is nonempty.

  Optimality conditions at $x \in \R^{n}$ are equivalent to requiring
  that $0 \in \partial f(x)$.  Taking derivatives, this is equivalent
  to
  \begin{align}
    \label{eq:33}
    0 =
    \begin{cases}
      \sum_{i=1}^{m} \frac{w_{i}(x - a_{i})}{\|x - a_{i} \|} & x \neq
      a_{i} \forall i \\
      \sum_{i=1}^{m} v_{i} & \| v_{i} \| \leq 1, a_{i} = x
    \end{cases}
  \end{align}
  with obvious interpolation between the two solutions.

  In the case where $n = 1$, then the $L^{1}$ and $L^{2}$ norms are
  equal, and this is just computing the weighted medians of the
  $a_{i}$.  Can just compute

  \todo{What are the applications for this technique in image
    processing}
\item Note that $g(x)$ is affine sum of convex functions, and so is
  convex. Let $x$ minimize $g$.  Then $0 \in \partial g(x)$, and we
  have
  \begin{equation}
    \label{eq:34}
    \partial_{i} g(x) =
    \begin{cases}
      1 + \mu \grad_{i} f(x) & x_{i} > 0 \\
      [-1 + \mu \grad_{i} f(x), 1 + \mu \grad_{i} f(x)] & x_{i} = 0 \\
      -1 + \mu \grad_{i} f(x) & x_{i} < 0
    \end{cases}
  \end{equation} and thus if $0 \in \partial g(x)$, we must have
  \begin{align}
    \label{eq:35}
    \begin{cases}
      x_{i} = x_{i} - \grad_{i} f(x) - \frac{1}{\mu} & x_{i} > 0 \\
      | \grad_{i} f(x) | \leq \frac{1}{\mu} & x_{i} = 0 \\
      x_{i} = x_{i} - \grad_{i} f(x) + \frac{1}{\mu} & x_{i} > 0
    \end{cases} 
  \end{align} which is equivalent to the shrinkage operation.
\item Note that $K$ is a closed convex cone.  As such, we have that
  $K^{\star \star} = \cl K = K$. We have
  \begin{align}
    \label{eq:37}
    K^{\star \star} &= \{ w \in \R^{n} | \IP{w, x} \leq 0 \forall x \in
    K^{\star} \} \\
    &= \{ w \in \R^{n} | \IP{w, Ax} \leq 0 \forall x \geq 0 \} \\
    &= \{ w \in \R^{n} | \IP{A^{T}w, x} \leq 0 \forall x \geq 0 \} \\
    &= \{ w \in \R^{n} | A^{T} w \leq 0 \}
  \end{align}

  By uniqueness, we have our result.

  Now, consider Farkas's lemma. Consider the cone $K$ as above, and
  consider the two cases, $b \in K^{\star}$ and $b \notin K^{\star}$.
  In the first case, we have that there exists $x \geq 0$ such that
  $Ax = b$. In the second case, we have $b \notin K^{\star \star
    \star} = K^{\star} = \{ w | \IP{w, x} \leq 0 \forall x \in K \}$,
  and so there must exist $x \in K$ such that $\IP{b, x} > 0$, which
  is equivalent to requiring that $A^{T} x \leq 0$ and $\IP{b, x} >
  0$, and so letting $y = -x$, we have our alternative.

  \todo{Is this correct? It seems like we must be taking a shortcut
    since the separating hyperplane theorem is so deep and this seems
    to require only elementary manipulation.}
\item
  \begin{enumerate}
  \item $\aff C$ is a closed set, as  
  \item $\cl C$ is the smallest closed set containing $C$.  Then $\cl
    D$ is a closed set containing $D$.  Thus, $\cl D$ contains $C$,
    and so $\cl C \subseteq \cl D$.
  \item $\int D$ is the largest open set contained in $D$. Then $\int
    C$ is an open set contained in $D$.  Thus, $\int C$ 

  \item \todo{proof}
  \end{enumerate}
\item
  \begin{enumerate}
  \item Recall that the affine
  \end{enumerate}
\item 
\item
\item
\item
\item
\item
\item
\item
\item We can show that $f(u)$ is convex, lower semicontinuous, and
  proper. Since $f(u) \geq \frac{1}{2} \| u - g \|_{2}^{2}$, we have
  level boundedness.  Thus we are guaranteed the existence of a
  solution.

  Since $\| v \| = \sup_{x \neq 0} \frac{\IP{x, v}}{\| x \|}_{2}$.

  We can then find a product of scaled unit balls $D$ such that $f(u) = \| u
  - g \|_{2}^{2} + (\delta_{D})^{\star}(Lu)$.

  If we form the perturbed function $f'$, we have
  \begin{align}
    \label{eq:36}
    f'(u, w) = k(u) + h(Lu + w) \\
    (f')^{\star}(v, y) = k^{\star}(-L^{T}y + v) +
    \underbrace{h^{\star}}_{\delta_{D}(y)}(y) 
  \end{align} with
  \begin{align}
    \label{eq:38}
    k^{\star}(z) = \sup_{u} \IP{z, u} - \frac{1}{2}\| u - g \|_{2}^{2}
    \\
    &= \frac{1}{2} \| z - g \|_{2}^{2} - \frac{1}{2} \| g \|_{2}^{2}
  \end{align} which is a special case of the dual of $\frac{1}{2} \| x
  \|^{2}$ is $\frac{1}{2} \| u \|^{2}$.

  Then
  \begin{align}
    \label{eq:39}
    f^{\star}(v, y) = \| -L^{T} y + v - g \|_{2}^{2} + \frac{1}{2} \|
    g \|_{2}^{2} + \delta_{D}(y) \\
    \psi(y) = -f^{\star}(0, y) = - \frac{1}{2} \| - L^{T} y - g
    \|_{2}^{2} + \frac{1}{2} \| g \|_{2}^{2} - \delta_{D}(y)
  \end{align} and so we have transformed our problem into a quadratic.

  \begin{align}
    \label{eq:40}
    p(w) = \inf_{u} f'(u, w) \\
    q(v) = \inf_{y} f'^{\star}(v, y)
  \end{align}
  Then $(u, y)$ is a primal-dual solution if and only if $(0, y)
  \in \partial f^{-1}(u, 0) \iff (0, y) \in (u - g + L^{T} \partial
  (\delta_{D})^{\star}(Lu), \partial (\delta_{D})^{\star}())$
\end{exercises}

%%% Local Variables: 
%%% mode: latex
%%% TeX-master: "master"
%%% End: 
