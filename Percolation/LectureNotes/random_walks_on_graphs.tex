
\chapter{Random Walks on Graphs}
\label{cha:random-walks-graphs}

Our basic setting is the (hyper-)cubic lattice on $\R^{d}, d \geq 1$.
This is the graph with vertex set $\Z^{d}$, edges $\IP{x, y} \iff \| x
- y \|_{1} = 1$, and edge set denoted $E^{d}$.  A lattice is $L^{d} =
(\Z^{d}, E^{d})$.

\section{Percolation}
\label{sec:percolation}

Let $0 < p < 1$. Let $e \in E^{d}$, and with probability $p$
independently for each edge, declare $e$ to be \textbf{open} else
\textbf{closed}. Consider $x \leftrightarrow y$ if there exists an
open path from $x$ to $y$. The \textbf{open cluster} at $x$ is $C_{x}
= \{ y: x \leftrightarrow y \}$.

\begin{boxthm}
  For a given $p$, what can be said about the $C_{x}$?
\end{boxthm}

For $p = 1$, $C_{x} = Z_{d}$.  For $p = 0$, $C_{x} = \{ x \}$.

\begin{defn}[Percolation probability]
  \label{defn:random_walks_on_graphs:1}
  Let $\theta(p) = \Prob{|C_{\theta}|} = \Prob_{p}$. Note that
  $\theta$ is non-decreasing.

  Let $p_{c} = \sup \{ p : \theta(p) = 0 \}$.

  It is known that $\theta$ is $C^{\infty}$ on $(p_{c}, 1]$, and that
  $\theta$ is right-continuous on $[0, 1]$.

  It is believed that $\theta$ is concave on $(p_{c}, 1]$, and that
  $\theta$ is real-analytic on $(p_{c}, 1]$, and that $\theta(p_{c}) =
  0$ (known for $d = 2$, and $d \geq 16$).
\end{defn}

\begin{defn}
  \label{defn:random_walks_on_graphs:2}
  Probability theorey.  Let $\Omega = \{ 0, 1\}^{E^{d}}$,
  $\mathcal{F}$ be the $\sigma$-filed generated by the
  finite-dimensional cylinder ... of form .
  $\{ \omega \in \Omega: \omega = \text{$\xi$ on $\mathcal{F}$}\} =
  E_{F}(\xi)$.
  ...
  \todo{Fill in from lecture notes}
\end{defn}

\begin{thm}
  \label{defn:random_walks_on_graphs:3}
  For $d \geq 2$, $0 < p_{c} < 1$.
\end{thm}

\todo{Fill in lecture notes from Chapter 3 of Probability on Graphs}

Consider $\Z^{d}$, with $\kappa_{n} = \mu^{n(1 + o(1))}$ as $n
\rightarrow \infty$, $\mu = \mu(\Z^{d})$

We have $\kappa_{n} \sim A n^{c} \mu^{n}$ for some $A = A(d), c =
c(d)$ where $a_{n} \sim b_{n}$ means $\frac{a_{n}}{b_{n}} \rightarrow
1$.

$c_{n}$ is called the \textbf{critical exponent}.  People are hoping
to show that for $d=2$, $c=\frac{11}{32}$.  $c$ is expected to be
\textbf{universal} in that it depends on $d$ but not each
$d$-dimensional graph.


\section{Coupling}
\label{sec:coupling}

Let $L^{d} = (\Z^{d}, E^{d})$ consider $P_{p}$ on $\Omega = \{ 0,
1\}^{E^d}$.

Let $(U_{e}, e \in E)$ be independent uniform random variables $U(0,
1)$.

Let $p \in (0, 1)$.  Then
\begin{equation}
  \label{eq:1}
  \mu_{p}(e) =
  \begin{cases}
    0 & U_{e} \geq p \\
    1 & U_{e} < p
  \end{cases}  
\end{equation}
if $p_{1} \leq p_{2}$ then $\mu_{p_{1}}(e) \leq \mu_{p_{2}}(e)$.

$\mu_{p}: 0 < p < 1$ is a coupling of percolations, containing all
interesting, ``universal'' in $p$.

\begin{thm}
  \label{defn:random_walks_on_graphs:4}
  For any increasing function $f: \Omega \rightarrow \R$,
  \begin{equation}
    \label{eq:2}
    \E{f}{p_{1}} \leq \E{f}{p_{2}}
  \end{equation} for $p_{1} \leq p_{2}$.
\end{thm}

\begin{exmp}
  \label{defn:random_walks_on_graphs:5}
  For example, $u, v \in \Z^{d}$, $f(u) = \I{u \leftrightarrow v}$.
  Then $\Prob{u \leftrightarrow v}{p_{1}} \leq \Prob{u \leftrightarrow
  v}{p_{2}}$.
\end{exmp}

\section{Oriented/Directed Percolations}
\label{sec:orient-perc}

Consider th standard percolation, and define
$\overrightarrow{\theta}(p) = \Prob(\text{there exists an infinite
  directed path through the origin}){p}$ . Then
$\overrightarrow{p_{c}} = \sup \{ p: \overrightarrow{\theta}(p) = 0
\}$.  As $\overrightarrow{\theta}(p) \leq \theta(p)$, we have
$\overrightarrow{p_{c}} \geq p_{c}$.


\section{Correlation Inequalities}
\label{sec:corr-ineq}

Consider a set $E$ be nonempty and finite, and $\Omega = \{ 0, 1
\}^{E}$.  The sample space $\Omega$ is partially ordered by
$\omega_{1} \leq \omega_{2}$ if $\omega_{1}(e) \leq \omega_{2}(e)$ for
all $e \in E$.

Event $A \subseteq \Omega$ is called \textbf{increasing} if $w \in A$,
$w \leq w' \Rightarrow w' \in A$ and decreasing if $\overline A =
\Omega \backslash A$ is increasing.

\begin{defn}
  \label{defn:random_walks_on_graphs:6}
  With two probability measures $\mu_{1}, \mu_{2}$, we write $\mu_{1}
  \leq_{st} \mu_{2}$ if $\mu_{1}(A) \leq \mu_{2}(A)$ for all
  increasing events $A$.

  Equivalently, $\mu_{1} \leq_{st} \mu_{2}$ if and only if $\mu_{1}(f)
  = \sum_{\Omega} f(\omega) \mu_{1}(\omega) \leq \mu_{2}(f)$ for all
  increasing functions $f: \Omega \rightarrow \R$.
\end{defn}

%%% Local Variables: 
%%% mode: latex
%%% TeX-master: "master"
%%% End: 
