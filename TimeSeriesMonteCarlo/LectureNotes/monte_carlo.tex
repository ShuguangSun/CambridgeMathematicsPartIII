
\chapter{Monte Carlo Inference}
\label{cha:monte-carlo-infer}

\begin{exer}
  What are the expected number of samples obtained from running the
rejection sampling algorithm $N$ times?
\end{exer}


\todo{Bunch of content, straight from the lecture notes}

\section{Quantile Approximation by Monte Carlo Methods}
\label{sec:quant-appr-monte}


\begin{thm}
  \label{defn:monte_carlo:1}
  If $F$ is continuous and strictly increasing, then $\hat c_{\alpha}
  \cas c_{\alpha}$
\end{thm}

\begin{proof}
  Let $\epsilon > 0$.

  Then $\hat F_{B}(c_{\alpha - \epsilon}) = \frac{1}{B}
  \sum_{i=1}^{B} \I{X_{i} \leq c_{\alpha - \epsilon}}$.

  Thus, by the strong law of lare numbers, $\hat F_{B}(c_{\alpha -
    \epsilon}) \cas \alpha - \epsilon$.

  In the same way, $\hat F_{B}(c_{\alpha + \epsilon}) \cas \alpha +
  \epsilon$.

  Thus, for all $\delta > 0$, there exists $\Omega_{1}$ with $P(\Omega)
  \geq 1 - \epsilon$, such that there exists $N_{1}$ with $\forall B > N_{1}$,
  on $\Omega_{1}$,
  \begin{equation}
    |\hat F_{B}(c_{\alpha - \epsilon}) - (\alpha - \epsilon)| \leq \frac{\epsilon}{4} 
  \end{equation}
  and similarly on $\Omega_{2}$,
  \begin{equation}
    |\hat F_{B}(c_{\alpha + \epsilon}) - (\alpha + \epsilon)| \leq \frac{\epsilon}{4} 
  \end{equation}

  On $\Omega_{1} \cap \Omega_{2}$, and for $B > \max \{N_{1}, N_{2}
  \}$,
  \begin{align}
    \label{eq:71}
    \hat F_{B}(c_{\alpha - \epsilon}) \leq \alpha -
    \frac{3\epsilon}{4}
    \hat F_{B}(c_{\alpha + \epsilon}) \geq \alpha + \frac{3\epsilon}{4}
  \end{align}

  We now that by definition, $\hat F_{B}(\hat c_{\alpha}) = \alpha$.

  On $\Omega_{1} \cap \Omega_{2}$,
  \begin{equation}
    \label{eq:71}
    \hat F_{B}(c_{\alpha - \epsilon}) \leq \hat F_{B}(\hat c_{\alpha})
  \end{equation}

  Since $\hat F$ is increasing, $c_{\alpha - \epsilon} \leq \hat
  c_{\alpha} \leq c_{\alpha + \epsilon}$.  Since $F$ is continuous,
  for $\epsilon$ small enough, $|c_{\alpha - \epsilon} - c_{\alpha +
    \epsilon}|$ can be made arbitrarily small.

  Also, $P(\Omega_{1} \cap \Omega_{2}) \geq 1 - 2 \delta$.  So we
  proved that for any $\epsilon_{1} > 0$, there exists $\Omega_{3} =
  \Omega_{1} \cap \Omega_{2}$ such that $P(\Omega_{3}) \geq 1 - 2
  \delta$, and there exists $N_{3} = \max \{ \Omega_{1}, \Omega_{2}
  \}$ such that for all $B \geq n_{3}$,
  \begin{equation}
    \label{eq:71}
    |\hat c_{\alpha} - c_{\alpha}| \leq \epsilon_{1}.
  \end{equation}
  Thus $\hat c_{\alpha} \cas c_{\alpha}$.
\end{proof}



%%% Local Variables: 
%%% mode: latex
%%% TeX-master: "master"
%%% End: 
