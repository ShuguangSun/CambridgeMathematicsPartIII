\section{Spectral Analysis}
\label{sec:spectral-analysis}

Let $X_{t}$ be a zero-mean stationary time series with autocovariance
function $\gamma(\cdot)$ satisfying $\sum_{h=-\infty}^{\infty}
|\gamma(h)| < \infty$.

\begin{defn}
  \label{defn:spectral_analysis:1}
  The spectral density of $X_{t}$ is the function $f(\cdot)$ defined
  by
  \begin{align}
    \label{eq:47}
    f(\lambda) = \frac{1}{2 \pi} \sum_{h=-\infty}^{\infty}
    e^{-ih\lambda} y(h)
  \end{align}  The summability implies that the series converges
  absolutely.
\end{defn}

\begin{thm}
  \label{defn:spectral_analysis:2}
  \begin{enumerate}
  \item $f$ is even
  \item $f(\lambda) \geq 0$ for all $\lambda \in (-\pi, \pi]$.
  \item $\gamma(k) = \int_{-\pi}^{\pi} e^{-k\lambda} f(\lambda) d
    \lambda = \int_{-\pi}^{\pi} \cos (k \lambda) f(\lambda) d\lambda$.
  \end{enumerate}
\end{thm}

\begin{defn}
  \label{defn:spectral_analysis:3}
  A function $f$ is the \textbf{spectral density} of a stationary time
  series $X_{t}$ with autocovariance function $\gamma(\cdot)$ if
  \begin{enumerate}
  \item $f(\lambda) \geq 0$ for all $\lambda \in (0, \pi]$,
  \item $\gamma(h) = \int_{-\pi}^{\pi} e^{ih\lambda} f(\lambda)
    d\lambda$ for all integers $h$.
  \end{enumerate}
\end{defn}

\begin{lem}
  If $f$ and $g$ are two spectral density corresponding to the
  autocovariance function $\gamma$, then $f$ and $g$ have the same
  Fourier coefficients and hence are equal.
\end{lem}

\begin{thm}
  \label{defn:spectral_analysis:4}
  A real-valued function $f$ defined on $(-\pi, \pi]$ is the spectral
  density of a stationary process if and only if
  \begin{enumerate}
  \item $f(\lambda) = f(-\lambda)$,
  \item $f(\lambda) \geq 0$
  \item $\int_{-\pi}^{\pi} f(\lambda) d\lambda < \infty$.
  \end{enumerate}
\end{thm}

\begin{thm}
  \label{defn:spectral_analysis:5}
  An absolutely summable function $\gamma(\cdot)$ is the
  autocovariance function of a stationary time series if and only if
  it is even and
  \begin{equation}
    \label{eq:70}
    f(\lambda) = \frac{1}{2 \pi} \sum_{h=-\infty}^{\infty}
    e^{-ih\lambda} \gamma(h) \geq 0
  \end{equation} for all $\lambda \in (-\pi, \pi]$, in which case
  $f(\cdot)$ is the spectral density of $\gamma(\cdot)$.
\end{thm}

\begin{thm}[Spectral Representation of the ACVF]
  \label{defn:spectral_analysis:6}
  A function $\gamma(\cdot)$ defined on the integers is the ACVF of a
  stationary time series if and only if there exists a
  right-continuous, nondecreasing, bounded function $F$ on $[-\pi,
  \pi]$ with $F(-\pi) = 0$ such that
  \begin{equation}
    \label{eq:72}
    \gamma(h) = \int_{-\pi}^{\pi} e^{ih\lambda} dF(\lambda)
  \end{equation} for all integers $h$.
\end{thm}

\begin{remark}
  The function $F$ is a \textbf{generalized distribution function} on
  $[-\pi, \pi]$ in the sense that $G(\lambda) =
  \frac{F(\lambda)}{F(\pi)}$ is a probability distribution function on
  $[-\pi, \pi]$. Note that since $F(\pi) = \gamma(0) = \Var{X_{1}}$,
  the ACF of $X_{t}$ has the spectral representation function
  \begin{equation}
    \label{eq:73}
    \rho(h) = \int_{-\pi}^{\pi} e^{ih \lambda} dG(\lambda)
  \end{equation}

  The function $F$ is called the spectral distribution function of
  $\gamma(\cdot)$. If $F(\lambda)$ can be expressed as $F(\lambda) =
  \int_{-\pi}^{\lambda} f(y) dy$ for all $\lambda \in [-\pi, \pi]$,
  then $f$ is the spectral density function and the time series is
  said to have a continuous spectrum. If $F$ is a discrete function,
  then the time series is said to have a discrete spectrum.
\end{remark}

\begin{thm}
  \label{sec:spectral-analysis-1}
  A complex valued function $\gamma(\cdot)$ is the autocovariance
  function of a stationary process $X_{t}$ if and only if either
  \begin{enumerate}
  \item $\gamma(h) = \int_{-\pi}^{\pi} e^{-ihv} dF(v) $ for all $h =
    0, \pm 1, \dots$ where $F$ is a right-continuous, non-decreasing,
    bounded function on $[-\pi, \pi]$ with $F(-\pi) = 0$, or
  \item $\sum_{i, j =1}^{n} a_{i} \gamma(i - j) \overline a_{j} \geq
    0$ for all positive integers $n$ and all $a = (a_{1}, \dots, a_{n}
    \in \mathbb{C}^{n})$.
  \end{enumerate}
\end{thm}

\subsection{The Spectral Density of an ARMA Process}
\label{sec:spectral-density-an}

\begin{thm}
  \label{sec:spectral-density-an-1}
  If $Y_{t}$ is any zero-mean, possibly complex-valued stationary
  process with spectral distribution function $F_{Y}(\cdot)$ and
  $X_{t}$ is the process $X_{t} = \sum_{j = -\infty}^{\infty} \psi_{j}
  Y_{t-j}$ where $\sum_{j=-\infty}^{\infty} |\psi_{j}| < \infty$, then
  $X_{t}$ is stationary with spectral distribution function
  $F_{X}(\lambda) = \int_{-\pi, \lambda} | \sum_{j=-\infty}^{\infty}
  \psi_{j} e^{-ijv}|^{2} dF_{Y}(v)$ for $-\pi \leq \lambda \leq \pi$.

  If $Y_{t}$ has a spectral density $f_{Y}(\cdot)$, then $X_{t}$ has a
  spectral density $f_{X}(\cdot)$ given by $f_{X}(\lambda) =
  |\Psi(e^{-i\lambda})|^{2} f_{Y}(\lambda)$ where $\Psi(e^{-i\lambda})
  = \sum_{j=-\infty}^{\infty} \psi_{j} e^{-ij\lambda}$.
\end{thm}

\begin{thm}
  \label{sec:spectral-density-an-3}
  Let $X_{t}$ be an ARMA($p, q$) process, not necessarily causal or
  invertible satisfying $\phi(B)X_{t} = \theta(B) Z_{t}$, $Z_{t} \sim
  WN(0, \sigma^{2})$ where $\phi(z) = 1 - \phi_{1} z - \dots -
  \phi_{p} z^{p}$ and $\theta(z) = 1 + \theta_{1} z + \dots +
  \theta_{q} z^{q}$ have no common zeroes and $\phi(z)$ has no zeroes
  on the unit circle.  Then $X_{t}$ has spectral density
  \begin{equation}
    \label{eq:71}
    f_{X}(\lambda) = \frac{\sigma^{2}}{2 \pi} \frac{|\theta(e^{-i\lambda})|^{2}}{||\phi(e^{-i\lambda})|^{2}}
  \end{equation} for $-\pi \leq \lambda \leq \pi$.
\end{thm}

\begin{thm}
  \label{sec:spectral-density-an-4}
  The spectral density of the white noise process is constant,
  $f(\lambda) = \frac{\sigma^{2}}{2 \pi}$.
\end{thm}


\subsection{The Periodogram}
\label{sec:periodogram}

\begin{defn}
  \label{sec:periodogram-1}
  The periodogram of $(x_{1}, \dots, x_{n})$ is the function
  \begin{equation}
    \label{eq:74}
    I_{n}(\lambda) = \frac{1}{n} | \sum_{t=1}^{n} x_{t} e^{-it\lambda}|^{2}
  \end{equation}
\end{defn}

\begin{thm}
  \label{sec:periodogram-2}
  If $x_{1}, \dots, x_{n}$ are any real numbers and $\omega_{k}$ is
  any of the nonzero Fourier Frequencies $\frac{2 \pi k}{n}$ in
  $(-\pi, \pi]$, then $I_{n}(\omega_{k}) = \sum_{|h| < n}^{} \hat
  \gamma(h) e^{-ih \omega_{k}}$ where $\hat \gamma(h)$ is the sample
  ACVF of $x_{1}, \dots, x_{n}$.
\end{thm}

\begin{thm}
  \label{sec:periodogram-4}
  Let $X_{t}$ be the linear process $X_{t} = \sum_{j=-\infty}^{\infty}
  \psi_{j} Z_{t-j}$, $Z_{t} \sim IID(0, \sigma^{2})$, with
  $\sum_{j=-\infty}^{\infty} |\psi_{j}| < \infty$. Let
  $I_{n}(\lambda)$ be the periodogram of $X_{1}, \dots, X_{n}$, and
  let $f(\lambda)$ be the spectral density of $X_{t}$.
  \begin{enumerate}
  \item If $f(\lambda) > 0$for all $\lambda \in [-\pi, \pi]$ and if $0
    < \lambda_{1} < \dots < \lambda_{m} < \pi$, then the random vector
    $(I_{n}(\lambda_{1}), \dots, I_{n}(\lambda_{m}))$ converges in
    distribution to a vector of independent and exponentially
    distributed random variables, the $i$-th component which has mean
    $2\pi f(\lambda_{i})$, $i = 1 \dots, m $.
  \item If $\sum_{j=-\infty}^{\infty} |\psi_{j}| |j|^{\frac{1}{2}} <
    \infty$, $\E{Z^{4}_{1}} = \nu \sigma^{4} < \infty$, $\omega_{j} =
    \frac{2 \pi j}{n} \geq 0$, and $\omega_{k} = \frac{2 \pi k}{n}
    \geq 0$, then
    \begin{align}
      \label{eq:75}
      \Cov{I_{n}(\omega_{j}), I_{n}(\omega_{k})} =
      \begin{cases}
        2(2 \pi)^{2} f^{2}(\omega_{j}) + O(n^{-\frac{1}{2} }) &
        \omega_{j} = \omega_{k} = \{ 0, \pi \} \\
        (2 \pi)^{2} f^{2}(\omega_{j})+ O(n^{-\frac{1}{2} }) & 0 <
        \omega_{j} = \omega_{k} < \pi \\
        O(n^{-1}) & \omega_{j} \neq \omega_{k}
      \end{cases}
    \end{align}
  \end{enumerate}

\end{thm}

\begin{defn}
  \label{sec:periodogram-3}
  The estimator $\hat f(\omega) = \hat f(g(n, \omega))$ with $\hat
  f(\omega_{j})$ defined by $\frac{1}{2 \pi} \sum_{|k| \leq m_{n}}^{}
  W_{n}(k) I_{n}(w_{j+k})$ with $m \rightarrow \infty$, $\frac{m}{n}
  \rightarrow 0$, $W_{n}(k) = W_{n}(-k)$, $W_{n}(k) \geq 0$ for all
  $k$, and $\sum_{|k| \leq m}^{} W_{n}(k) = 1$, and $\sum_{|k|}^{}
  W_{n}^{2}(k) \rightarrow 0$ as $n \rightarrow \infty$ is called a
  \textbf{discrete spectral average estimator} of $f(w)$.
\end{defn}

\begin{thm}
  \label{sec:periodogram-5}
  Let $X_{t}$ be the linear process $X_{t} = \sum_{j=-\infty}^{\infty}
  \psi_{j} Z_{t-j}$, $Z_{t} \sim IID(0, \sigma^{2})$, with
  $\sum_{j=-\infty}^{\infty} |\psi_{j}| |j|^{\frac{1}{2}} < \infty$
  and $\E{Z_{1}^{4}} < \infty$.  If $\hat f$is a discrete spectral
  average estimator of the spectral density $f$, then for $\lambda,
  \omega \in [0, \pi]$,
  \begin{enumerate}
  \item $\lim_{n \rightarrow \infty} \E{\hat f(\omega)} = f(\omega)$
  \item
    \begin{align}
      \label{eq:76}
      \lim_{n \rightarrow \infty} \frac{1}{\sum_{|j| \leq m}^{}
        W_{n}^{2}(j)} \Cov{\hat f(\omega), \hat f(\lambda)} =
      \begin{cases}
        2 f^{2}(\omega) & w = \lambda = \{ 0, \pi \} \\
        f^{2}(\omega) & 0 < \omega = \lambda < \pi \\ \\
        0 & \omega \neq \lambda.
      \end{cases}
    \end{align}
  \end{enumerate}
\end{thm}

%%% Local Variables: 
%%% mode: latex
%%% TeX-master: "master"
%%% End: 
