\documentclass[8pt, reqno, oneside, twocolumn]{amsart} 

% Utility packages
\usepackage{etoolbox}
\newtoggle{tufte}
\togglefalse{tufte}
\newtoggle{summary}
\togglefalse{summary}
\usepackage{amsthm}
\usepackage{amsmath}
\usepackage{amssymb}
% \usepackage[landscape]{geometry}
\usepackage[top=1cm, bottom=1cm, left=1cm, right=1cm]{geometry}
% \usepackage[inner=1cm, outer=1cm]{geometry}
\usepackage{setspace}
\usepackage{graphicx}
\usepackage{enumerate}
\usepackage{listings}
\usepackage{booktabs}
\usepackage{hyperref}
\usepackage{thmtools}
\usepackage{xparse}
\usepackage{xspace}
\usepackage{enumitem}
\usepackage{todonotes}
\usepackage{dcolumn}
\usepackage{array}
\usepackage{booktabs}
\usepackage{colortbl, xcolor}
\usepackage{longtable}
\usepackage{natbib}
\usepackage{python}
% \usepackage{a4wide}

\hypersetup{colorlinks=true}
\numberwithin{equation}{section}

\declaretheorem[numbered=no,name=Theorem,style=definition]{thm}
\declaretheorem[numbered=no,name=Lemma]{lem}
\declaretheorem[numbered=no,name=Corollary]{corollary}
\declaretheorem[numbered=no,name=Definition]{defn}
\declaretheorem[numbered=no,name=Proposition]{proposition}
\declaretheorem[numbered=no,name=Example]{exmp}
\declaretheorem[numbered=no,name=Exercise]{exer}
\declaretheorem[numbered=no,name=History]{history}
\declaretheorem[numbered=no,name=Question]{question}
\declaretheorem[numbered=no,name=Remark]{remark}
\declaretheorem[numbered=no,name=Notation]{notation}
\declaretheorem[numbered=no,name=Theorem]{boxthm}

\let\proof\relax
\declaretheorem[style=remark,numbered=no,qed=\qedsymbol]{proof}

\DeclareMathOperator*{\argmax}{arg\,max}
\DeclareMathOperator*{\dist}{dist}
\DeclareMathOperator*{\V}{V}
\DeclareMathOperator*{\saddlepoint}{sp}
\DeclareMathOperator*{\divergence}{div}
\DeclareMathOperator*{\sym}{Sym}
\DeclareMathOperator*{\vol}{Vol}
\DeclareMathOperator*{\argmin}{arg\,min}
\DeclareMathOperator*{\proj}{\Pi}
\DeclareMathOperator*{\bnd}{bnd}
\DeclareMathOperator*{\supp}{supp}
\DeclareMathOperator*{\aff}{aff}
\DeclareMathOperator*{\rint}{rint}
\DeclareMathOperator*{\lev}{lev}
\DeclareMathOperator*{\tr}{tr}
\DeclareMathOperator*{\interior}{int}
\DeclareMathOperator*{\epi}{epi}
\DeclareMathOperator*{\dom}{dom}
\DeclareMathOperator*{\vecspan}{span}
\DeclareMathOperator*{\con}{con}
\DeclareMathOperator*{\cl}{cl}
\DeclareMathOperator*{\grad}{\nabla}
\DeclareMathOperator*{\crit}{crit}
\DeclareMathOperator*{\diag}{diag}
\DeclareMathOperator*{\res}{Res}
\DeclareMathOperator*{\boundary}{boundary}
\DeclareMathOperator*{\range}{range}
\DeclareMathOperator*{\Tr}{Tr}
\DeclareMathOperator*{\sign}{sign}
\setlist[enumerate]{label=(\roman*)}

\newlist{exercises}{enumerate}{1}
\setlist[exercises]{label=Ex. \arabic*}


\setcounter{secnumdepth}{2}
\setcounter{tocdepth}{2}
% \onehalfspacing

\author{Andrew Tulloch}


% Shortcuts
% convergence in probability
\newcommand{\cd}{\overset{d}{\rightarrow}}    
% convergence in distribution
\newcommand{\cp}{\overset{p}{\rightarrow}}
% convergence almost surely
\newcommand{\cas}{\overset{as}{\rightarrow}}
\newcommand{\R}{\mathbb{R}}
\newcommand{\N}{\mathbb{N}}
\newcommand{\Z}{\mathbb{Z}}
\newcommand{\np}{\textsc{np}\xspace}
\newcommand{\iid}{\textsc{iid}\xspace}
\newcommand{\knn}{$k$-\textsc{nn}\xspace}

% Expectation
\NewDocumentCommand{\E}{gg}{%
  \IfNoValueTF{#1}
    {%
      \mathbb{E}
    }    
    {%
      \IfNoValueTF{#2}
      {%
        \mathbb{E}\!\left(#1\right)
      }
      {%
        \mathbb{E}_{#2}\!\left(#1\right)
      }
    }%
}

% Probability with measure P
\NewDocumentCommand{\Prob}{gg}{%
  \IfNoValueTF{#1}
    {%
      \mathbb{P}
    }    
    {%
      \IfNoValueTF{#2}
      {%
        \mathbb{P}\!\left(#1\right)
      }
      {%
        \mathbb{P}_{#2}\!\left(#1\right)
      }
    }%
}

% Probability with measure Q
\NewDocumentCommand{\Q}{g}{%
  \IfNoValueTF{#1}
    {%
      \mathbb{Q}
    }    
    {%
      \mathbb{Q}\!\left(#1\right)
    }%
}


% Indicator
\NewDocumentCommand{\I}{g}{%
  \IfNoValueTF{#1}
    {%
      \mathbb{I}
    }    
    {%
      \mathbb{I}\!\left(#1\right)
    }%
}


\NewDocumentCommand{\Var}{g}{%
  \IfNoValueTF{#1}
    {%
      \mathbb{V}
    }    
    {%
      \mathbb{V}\!\left(#1\right)
    }%
}


\NewDocumentCommand{\Cov}{gg}{%
  \IfNoValueTF{#1}
    {%
      \text{Cov}
    }   
    {%
      \text{Cov}\!\left(#1, #2\right)
    }%
}

% Inner product
\NewDocumentCommand{\IP}{g}{%
    {%
      \left \langle #1 \right \rangle
    }%
}
%%% Local Variables: 
%%% mode: latex
%%% End: 


\title{Time Series and Monte-Carlo Summary}

\begin{document}

\maketitle

\section{Monte-Carlo}
\label{sec:monte-carlo}

\begin{thm}
  \label{sec:monte-carlo-2}
  Consider a discrete random variable $X$ with $M$ mass points $\{
  m_{1}, \dots, m_{M} \}$ with probability $p_{1}, \dots, p_{M}$, such
  that $p_{} \geq 0$, $\sum p_{i} = 1$.  Then the CDF of such a random
  variable is $F(x) = \Prob{X \leq x} = \sum_{i=1}^{\lfloor x \rfloor
    x} p_{i}$.  Write $F(m_{i}) = F_{i}$, and note that $p_{i} = F_{i}
  - F_{i-1}$.  Then we have that to draw from $F$, we simulate $U \sim
  \mathcal{U}_{[0, 1]}$, and set $X = m_{i}$ if $F_{i-1} \leq U \leq F_{i}$.
\end{thm}

\begin{thm}
  \label{sec:monte-carlo-3}
  Let $X \sim F$ be a scalar random variable on $X \subset \R$.
  Assume $F$ is strictly increasing and continuous.  Then $U = F(X)
  \sim \mathcal{U}_{[0, 1]}$.
\end{thm}

\begin{proof}
  $\Prob{U \leq u} = \Prob{F(x) \leq u} = \Prob{X \leq F^{-1}(u)} =
  F(F^{-1}(u)) = u$.

  Thus $X = F^{-1}(U) \sim F$, as $\Prob{X \leq x} = \Prob{F^{-1}(U)
    \leq x} = \Prob{U \leq F(x)} = F(x)$.
\end{proof}

\begin{defn}
  \label{sec:monte-carlo-4}
  Assume that we can generate a uniform random variable $U$ on some
  domain $\mathcal{U}$.  Then let $\mu$ the density of this uniform
  random variable.  Assume we now wish to simulate a random variable
  $X$ uniformly on some domain $\mathcal{X} \subset \mathcal{U}$ ($\mu
  (\cdot | \mathcal{X})$).  Then do to so, simply simulate $U$
  uniformly on $\mathcal{U}$, and if $U \in \mathcal{X}$, set $X = U$,
  otherwise reject $U$ and repeat.  Then if $\mu(X) = p > 0$, $X$ is
  distributed uniformly on $\mathcal{X}$.
\end{defn}

\begin{thm}
  \label{sec:monte-carlo-5}
  Suppose we have two density's $f, g$ and $g$ defined on $X \subset
  \R$. Suppose that we can samply according to $g$ and there exists $M
  \in [1, \infty)$ such that $f(x) \leq M g(x)$ for all $x \in
  \mathcal{X}$. Then $g$ is an envelope density that dominates $f$. If
  we then simulate $Y \sim g$ , and some $U \sim \mathcal{U}_{[0,
    1]}$, and if $U \leq \frac{f(Y)}{Mg(Y)}$, set $X = Y$, otherwise
  re-sample.

  Then the output $X$ has density $f$.
\end{thm}

\begin{thm}
  \label{sec:monte-carlo-6}
  To generate a Gaussian random variable, we can use the
  \textbf{Box-Muller} method.  This proceeds by generating $U \sim
  \mathcal{U}_{[0, 1]}$, and setting $\theta = 2 \pi U$, and then
  simulating $V \sim \mathcal{U}_{[0, 1]}$, and setting $R^{2} - 2
  \log V$.  We can then set $X_{1}, X_{2} = R \sin \theta, R \cos
  \theta$.

  Alternatively, we can generate $U, V \in \mathcal{U}_{[0, 1]}$
  uniformly on the unit disk, and set $W = U^{2} + V^{2}$ (by
  rejecting sampling), and set $X_{1} = U \sqrt{-2 \frac{\log W}{W}
  }$, $X_{2} = V \sqrt{-2 \frac{\log W}{W}}$.
\end{thm}

\section{Monte-Carlo Method and Non-Parametric Inference}
\label{sec:monte-carlo-method}

\begin{defn}
  \label{sec:monte-carlo-method-1}
  The plug-in estimator is defined as
  \begin{equation}
    \hat F_{n}(x) = \frac{1}{n} \sum_{i=1}^{n} \I{X_{i} \leq x}
  \end{equation}
\end{defn}

\begin{thm}
  \label{sec:monte-carlo-method-2}
  Consider estimating $\theta(F) = \E_{F} \phi(X) =
  \int_{\mathcal{X}}^{} \phi(x) f(x) dx$, with plug-in estimator
  $\theta(\hat F_{n}) = \E_{\hat F_{n}} \phi(X) = \frac{1}{n}
  \sum_{i=1}^{n} f(X_{i})$.

  Assume that $\int_{\mathcal{X}} \phi(x)^{2} f(x) dx < \infty$.  Then
  $\theta(\hat F_{n})$ is such that
  \begin{align}
    \label{eq:1}
    \E_{F^{n}}(\theta(\hat F_{n})) = \theta(F), \\
    \label{eq:2}
    \V_{F^{n}}(\theta(\hat F_{n})) = \frac{1}{n} (\int_{\mathcal{X}}
    (\phi(x) - \theta(F))^{2} f(x) dx)
  \end{align}
\end{thm}

\begin{thm}
  \label{sec:monte-carlo-method-3}
  Assume that $\int_{\mathcal{X}} \phi(x)^{2} f(x) dx < \infty$.  Then
  the following statements hold:
  \begin{enumerate}
  \item $\theta(\hat F_{n})$ converges almost surely to $\theta(F)$
  \item $\sqrt{n} (\theta(\hat F_{n}) - \theta(F))$ converges in
    distribution to $Normal(0, \int_{\mathcal{X}} (\phi(x) -
    \theta(F))^{2} f(x) dx)$
  \end{enumerate}
\end{thm}

\begin{thm}
  \label{sec:monte-carlo-method-4}
  Let $g$ be a density that is strictly positive whenever $f \phi$ is
  non-zero.  Let $X_{1}, \dots, X_{n}$ be IID samples from $g$, and
  define the following estimate of  $\theta(F)$ as $\hat \theta_{g} =
  \frac{1}{n} \sum_{i=1}^{n} w_{i} \phi(X_{i})$ where $w_{i} =
  \frac{f(X_{i})}{g(X_{i})}$.

  Then $\hat \theta_{g}$ is such that
  \begin{align}
    \label{eq:3}
    \E_{G^{n}}(\hat \theta_{g}) = \theta(F), \\
    \label{eq:4}
    \V_{G^{n}}(\hat \theta_{g}) = \frac{1}{n} (\int_{\mathcal{X}}
    \frac{\phi(x)^{2} f(x)^{2}}{g(x)} dx - \theta(F)^{2})
  \end{align}
\end{thm}

\begin{thm}
  \label{sec:monte-carlo-method-5}
  Note that the variance of this estimate depends crucially on the
  function $\psi(x) = \frac{\phi(x)^{2} f(x)^{2}}{g(x)}$, and the
  density that minimizes the variance of $\hat \theta_{g}$ is
  $g^{\star} = \frac{f |\phi|}{\int_{\mathcal{X}} f |\phi|}$.
\end{thm}

\begin{defn}
  \label{sec:monte-carlo-method-6}
  If we sample randomly on our space $\mathcal{X}$ according to $f$,
  we create a source of randomness and thus of error.  By using
  \textbf{stratified sampling} we reduce the error by reducing the
  amount of randomness in the picking of the points.

  \begin{enumerate}
  \item Divide the domain into $K$ strata $\Omega_{i}$ that are
    measurable according-to $f$ and form a partition of the domain, and
    we know exactly $w_{i} = \Prob_{f}(\Omega_{i})$.
  \item Sample exactly $T_{i}$ in each stratum $\Omega_{i}$ according
    to $f | \Omega_{i}$.  The numbers $T_{i}$ are deterministic with
    $\sum_{}^{} T_{i}$.  Writing the conditional empirical mean in
    stratum $\Omega_{i}$ as $\mu_{i} = \frac{1}{T_{i}} \sum_{j=1}^{n}
    X_{j} \I{X_{j} \in \Omega_{i}}$, and return the weighted estimate
    of the integral $\hat \mu = \sum_{i=1}^{K} w_{i} \hat \mu_{i}$.
  \end{enumerate}
\end{defn}

\begin{thm}
  \label{sec:monte-carlo-method-7}
  $\E{\hat \mu} = \mu$, and $\V \hat \mu = \sum_{i=1}^{K}
  \frac{w_{i}^{2} \sigma^{2}_{i}}{T_{i}} $ where $\mu_{i} =
  \frac{1}{w_{i}} \int_{\Omega_{i}} \phi(x) f(x) dx$ is the
  conditional mean, and $\sigma^{2}_{i} = \frac{1}{w_{i}}
  \int_{\Omega_{i}}^{} (\phi(x) - \mu_{i})^{2} f(x) dx $ is the
  conditional variance in stratum $\Omega_{i}$.
\end{thm}

\begin{defn}
  \label{sec:monte-carlo-method-8}
  Uniform stratified sampling takes $T_{i}^{u} = w_{i} n$.  This
  \textbf{consolidates} the random sampling while preserving the shape
  of the density $f$.  With this choice of $T_{i}$, we have $\E{\hat
    \mu_{u}} = \mu$, and $\V \hat \mu_{u} = \sum_{i=1}^{K} \frac{w_{i}
  \sigma_{i}^{2}}{n}$.
\end{defn}

\begin{thm}
  \label{sec:monte-carlo-method-9}
  If we solve the minimization problem
  \begin{align}
    \label{eq:5}
    \min_{(T_{i})_{i}} \V \hat \mu = \sum_{i=1}^{K} \frac{w_{i}^{2}
      \sigma_{i}^{2}}{T_{i}}
  \end{align} such that $T_{i} \geq 0$, $\sum_{i}^{} T_{i} = n$
  we obtain the unique solution
  \begin{align}
    \label{eq:6}
    T_{i}^{\star} = \frac{w_{i} \sigma_{i}}{\sum_{j}^{} w_{j}
      \sigma_{j}} n, 
  \end{align} known as \textbf{oracle stratified sampling}.  In this
  case, $\E{\hat \mu^{\star}} = \mu$, and $\V \hat \mu^{\star}_{n} =
  \frac{1}{n} (\sum_{i=1}^{K} w_{i} \sigma_{i})^{2}$.
\end{thm}

\begin{thm}
  \label{sec:monte-carlo-method-10}
  If we have Lipschitz function $\phi$ that we are integrating uniformly
  over $[0, 1]$, we have there exists $C > 0$ such that for any $u,
  v$, $|\phi(u) - \phi(v)| \leq C |u - v$.  If we uniformly divide
  $[0, 1]$ into $K$ sections, we have $\Omega_{i} = [\frac{i}{K},
  \frac{i+1}{K}]$, with $w_{i} = \frac{1}{K}$.  Then we have $\V \hat
  \mu_{u} \leq \sum_{i=1}^{n} \frac{1}{n} \int_{\Omega_{i}}^{}
  \frac{C^{2}}{n^{2}} dx = \frac{C^{2}}{n^{3}}$, which is a significant
  gain over the Monte-Carlo estimate of $\frac{1}{n}$.
\end{thm}


\begin{defn}
  \label{sec:monte-carlo-method-11}
  To estimate the $c_{\alpha}$ quantile with tail probability $\alpha$
  of a distribution $F$, we can do this with Monte-Carlo by
  \begin{enumerate}
  \item Choose $B \in \N$.
  \item Restrict $\alpha \in k = \{ 1, \dots, B \} $, with $\alpha =
    \frac{k}{B+1} $.
  \item Simulate $T_{1}, \dots, T_{B}$ according to $F$.
  \item Let $\hat c_{\alpha} = T_{(k)}$ where $T_{(k)}$ is the $k$-th
    order statistic of $T_{1}, \dots, T_{B}$. 
  \end{enumerate}
\end{defn}

\begin{thm}
  \label{sec:monte-carlo-method-12}
  If $F$ corresponds to a density $f$, then $\E{F(\hat c_{\alpha})} = \alpha$.
\end{thm}

\begin{defn}
  \label{sec:monte-carlo-method-13}
  Suppose we are interested in the distribution $K_{n}(F)$ of a root
  or pivot $R_{n}(X, F)$ where $X  = (X_{1}, \dots, X_{n})$ (e.g. the
  distribution of the statistic $T(X_{1}, \dots, X_{n})$ in hypothesis
  tests).  The boostrapboostrap estimator of $K_{n}(F)$is $K_{n}(\hat F)$.
\end{defn}

\begin{defn}
  \label{sec:monte-carlo-method-14}
  To approximate the Boostrap estimator by Monte-Carlo, we compute the
  following:
  \begin{enumerate}
  \item Draw $B$ independent boostrap samples $X^{\star}_{b} = (X_{b,
      1}^{\star}, \dots, X_{b, n}^{\star})$ from $\hat F^{n}$.
  \item Approximate $K_{n}(\hat F)$ by the empirical distribution
    function of $(R_{n}(X^{\star}_{b}, \hat F))_{b}$.
  \end{enumerate}
\end{defn}

\section{Bayesian Inference and Associated Methods}
\label{sec:bayes-infer-assoc}

\begin{defn}
  \label{sec:bayes-infer-assoc-1}
  Our objective is to generate samples according to the posterior
  $\pi(\theta | \overline X) = L(X, \theta) p(\theta)$.

  The solution is the generate a Markov chain $(\theta^{(1)},
  \theta^{(2)}, \dots, \theta^{(t)}, \dots)$ such that $\pi$ is the
  stationary distribution.
\end{defn}

\begin{defn}
  \label{sec:bayes-infer-assoc-2}
  The Gibbs sampler proceeds as follows. Let $\overline \theta =
  (\theta_{1}, \dots, \theta_{p})$ be the parameter of interest and
  $\pi(\theta_{i} | \overline \theta_{(-i)}) = \pi_{i}(\overline
  \theta_{(-i)})$ be the conditional posterior distributions. The
  Gibbs sampler works as follows:
  \begin{enumerate}
  \item Set the initial vector $\theta^{(0)}$,
  \item At time $t+1$:
    \begin{enumerate}
    \item Set $\theta_{1}^{(t+1)} \sim \pi_{1}(\theta_{2}^{(t)},
      \theta_{3}^{(t)}, \dots) = \pi_{1}(\overline \theta^{(t)}_{(-1)})$
    \item Set $\theta_{2}(t+1) = \pi_{2}(\theta_{1}^{t+1},
      \theta_{3}^{(t)}, \dots, \theta_{p}^{(t)}) =  \pi_{2}(\overline
      \theta^{(t)}_{(-2)})$
    \item \dots
    \item Set $\theta_{p}^{(t+1)} \sim \pi_{p}(\theta^{(t+1)}_{(-p)})$
    \end{enumerate}
  \item Collect $T$ samples using this iteration.
  \item Throw away the first $b$ samples and consider only the last samples.
  \end{enumerate}

  This method generates a Markov chain whose stationary distribution
  is the posterior under not-too-strong assumptions.

  Note that the samples $\overline \theta^{(t)}$ are \textbf{not independent}.
\end{defn}


\begin{defn}
  \label{sec:bayes-infer-assoc-3}
  The Metropolis Hastings algorithm is a sequential form of rejection
  sampling.  It is an Markov Chain Monte-Carlo method to construct a
  Markov chain whose stationary distribution is $\pi$.

  Consider a distribution $\pi$ defined on a domain $\mathcal{X}$.
  Assume that $\pi$ is such that for any atom $x \in \mathcal{X}$, $\{
  x \} $ is measurable according to $\pi$.  For any $x \in
  \mathcal{X}$, define a transition measure $\mu(\cdot | x)$ on
  $\mathcal{X}$.  
  The method proceeds as follows:
  \begin{enumerate}
  \item Set the initial vector $\theta^{(0)}$.
  \item At time $t+1$,
    \begin{enumerate}
    \item Simulate $X \sim \mu(\cdot | \theta^{(t)})$ and $U \sim
      \mathcal{U}_{[0, 1]}$
    \item If $U \leq \frac{\pi(X) \mu(\theta^{(t)} |
        X)}{\pi(\theta^{(t)}) \mu(X | \theta^{(t)})}$, then
      $\theta^{(t+1)} = X$, otherwise $\theta^{(t+1)} = \theta^{(t)}$.
    \end{enumerate}
  \item Collect $T$ samples like that.
  \item Throw away the first $b$ samples, consider the last samples
    (and possibly sub-sample to reduce correlations.)
  \end{enumerate}

  Note that the initial state is important (particular for unbounded
  distributions), the transition probability is important (for fast
  convergence), and the number of samples discarded is
  problem-dependent.
\end{defn}

\begin{defn}
  \label{sec:bayes-infer-assoc-4}
  Given MCMC methods produce a correlated chain of samples - so $t$
  samples do not provide the sample information as $t$ IID samples.  A
  common notion for measuring this is the \textbf{effective sample
    size}.  For any $l \geq 0$, define $\gamma(l)$ as the correlation
  between two samples of lag $l$.  Defining $\rho(l) =
  \frac{\gamma(l)}{\gamma(1)}$, the effective sample size $\tilde T$of
  a chain of length $t$ is
  \begin{equation}
    \label{eq:8}
    \tilde T = \frac{t}{1 + 2 \sum_{l=1}^{T-1} \rho(l)}.
  \end{equation}
\end{defn}

\bibliographystyle{plainnat}
\bibliography{../../common/bibliography}
\end{document}
