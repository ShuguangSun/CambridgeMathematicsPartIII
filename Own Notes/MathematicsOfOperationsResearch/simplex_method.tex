\chapter{Simplex Method}
\label{cha:simplex-method}

\section{Basic Solutions}
\label{sec:basic-solutions}

Maximize $c^{T}x$ subject to $Ax = b$, $x \geq 0$, $A \in \R^{m \times
n}$.

Call a solution $x \in R^{n}$ of $Ax = b$ \textbf{basic} if it has at
most $m$ non-zero entries, that is, there exists $B \subseteq \{1, \dots, n
\}$ with $|B| = m$ and $x_{i} = 0$ if $i \notin B$.

\newcommand{\bfs}{\textsc{bfs}{}}
A basic solution $x$ with $x \geq 0$ is called a \textbf{basic
  feasible solution} (\bfs).

\section{Extreme Points and Optimal Solutions}
\label{sec:extr-points-optim}

We make the following assumptions:

\begin{enumerate}
\item The rows of $A$ are linearly independent
\item Every set of $m$ columns of $A$ are linearly independent.
\item Every basic solution is non-degenerate - that is, it has exactly
  $m$ non-zero entries.
\end{enumerate}

\begin{thm}
  \label{defn:simplex_method:1}
  $x$ is a \bfs of $Ax = b$ if and only if it is an extreme point of
  the set  $X(b) = \{ x: Ax = b, x \geq 0 \}$.
\end{thm}

\begin{thm}
  \label{defn:simplex_method:2}
  If the problem has a finite optimum (feasible and bounded), then it has an optimal solution
  that is a \bfs.
\end{thm}


\section{The Simplex Tableau}
\label{sec:simplex-tableau}

Let $A \in \R^{m \times n}, b \in \R^{m}$.  Let $B$ be a basis (in the
\bfs sense), and $x \in R^{n}$, such that $Ax = b$.  Then
\begin{equation}
  \label{eq:20}
  A_{B} x_{B} + A_{N} x_{N} = b
\end{equation} where $A_{B} \in R^{m \times m}$ and $A_{N} \in R^{m
  \times ( n - m)}$ respectively consist of the columns of $A$ indexed
by $B$ and those not indexed by $B$. Moreover, if $x$ is a basic
solution, then there is a basic $B$ such that $x_{N} = 0$ and $A_{B}
x_{B} = b$, and if $x$ is a basic feasible solution, there is a basis
$B$ such that $x_{n} = 0$, $A_{B} x_{B} = b$, and $x_{B} \geq 0$.

For every $x$ with $Ax = b$ and every basis $B$, we have
\begin{equation}
  \label{eq:21}
  x_{B} = A_{B}^{-1} (b - A_{N} x_{N})
\end{equation} as we assume that $A_{B}$ has full rank.  Thus,

\begin{align}
  \label{eq:23}
  f(x) & = c^{T} x = c_{B}^{T} x_{B} + c^{T}_{N} x_{N}             \\
       & = c_{B}^{T} A_{B}^{-1}(b - A_{N} X_{N}) + c^{T}_{N} x_{N} \\
       & = C_{B}^{T} A_{B}^{-1} b + (c_{N}^{T} - c_{B}^{T} A_{B}^{-1} A_{N}) x_{N}
\end{align}

Assume we can guarantee that $A_{B}^{-1} b = 0$.  Then $x^{\star}$ with
$x_{B}^{\star} = A_{B}^{-1} b$ and $x_{N}^{\star} = 0$ is a \bfs with
\begin{equation}
  \label{eq:24}
  f(x^{\star}) = C_{B}^{T} A_{B}^{-1} b
\end{equation}

Assume that we are maximizing $c^{T} x$.  There are two different
cases:
\begin{enumerate}
\item If $C_{N}^{T} - C_{B}^{T} A_{B}^{-1} A_{N} \leq 0$, then $f(x)
  \leq c_{B}^{T} A_{B}^{-1} b$ for every feasible $x$, so $x^{\star}$
  is optimal.
\item If $(c_{N}^{T} - c_{B}^{T} A_{B}^{-1} A_{N})_{i} > 0$, then we
  can increase the objective value by increasing the corresponding row
  of $(x_{N})_{i}$.
\end{enumerate}

%%% Local Variables: 
%%% mode: latex
%%% TeX-master: "master"
%%% End: 
