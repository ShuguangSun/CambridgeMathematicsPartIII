\chapter{Discrete Time Models}
\label{cha:discrete-time-models}

\section{Standing Assumptions}
\label{sec:standing-assumptions}

\begin{enumerate}
\item Zero dividends
\item Zero tick size
\item Zero transaction costs
\item Infinitely divisible transactions
\item No short-selling constraints
\item No bid-ask spread
\item No market impact (infinitely deep market)
\end{enumerate}

\section{Setup}
\label{sec:setup}

Consider a probability space $(\Omega, \mathcal{F}, \mathbb{P})$.

\begin{defn}
  \label{defn:1}
  A random variable is a measurable map $X: \Omega \rightarrow \mathbb{R}$
\end{defn}

\begin{defn}
  \label{defn:2}
  A stochastic process $Y = (Y_{t})_{t \in I}$ is a collection of
  random variables.  For us, $I = \{ 0, 1, \dots \}$ or $[0, \infty)$. 
\end{defn}


\begin{defn}
  \label{defn:3}
  A filtration $\mathbb{F} = (\mathcal{F})_{t \geq 0}$ is a collection
  of sub-$\sigma$-algebras on $\mathcal{F}$ such that $\mathcal{F}_{s}
  \subseteq \mathcal{F}_{t}$ for all $0 \leq s \leq t$ (discrete and
  continuous time).
\end{defn}

\begin{exmp}
  Tossing coins.
  \begin{enumerate}
  \item $\Omega = \{ HH, HT, TH, TT \}$
  \item $\mathcal{F}$ is all 16 subsets of $\Omega$
  \item $\mathbb{P}(A) = \frac{|A|}{4}$
  \end{enumerate}

  Possible filtration
  \begin{enumerate}
  \item $\mathcal{F}_{0} = \{ \emptyset, \Omega \}$
  \item $\mathcal{F}_{1} = \{ \emptyset, \Omega, \{ HH, HT \}, \{ TH,
    TT \} \}$
  \item $\mathcal{F}_{2} = \mathcal{F}$
  \end{enumerate}
\end{exmp}

\begin{defn}
  \label{defn:4}
  A process $Y$ is adapted if and only if $Y_{t}$ is $\mathcal{F}_{t}$-measurable.
\end{defn}

Throughout the course, $\mathcal{F}_{0}$ is assumed trivial.

\begin{defn}
  \label{defn:discrete_time_models:1}
  Given a filtration $\mathbb{F} = (\mathcal{F}_{t})_{t \geq 0}$ in
  discrete time, a process $X = (X_{t})_{t \geq 1}$ is predictable if
  and only if $X_{t}$ is $\mathcal{F}_{t-1}$-measurable.

  Sometimes we need $X_{0}$ to be defined, so we just ask that $X_{0}$
  is $\mathcal{F}_{0}$-measurable.
\end{defn}

\begin{defn}
  \label{defn:discrete_time_models:2}
  Given $P = (P_{t})_{t \geq 0}$ prices process in discrete time.  An
  investment/consumption strategy is a predictable process $(H, c)$
  where $H_{t}$ takes values in $R^{n}$ and $c_{t} \geq 0$ and
  satisfies the \textbf{self-financing condition}
  \begin{equation}
    \label{eq:1}
    H_{t - 1} - P_{t - 1} = H_{t} \cdot P_{t} + c_{t}
  \end{equation} for all $t \geq 1$.
\end{defn}

$H_{t}$ models the portfolio during $(t - 1, t]$, and $c_{t}$ models
the consumption during $(t-1, t]$.

\begin{notation}
  $X_{t}(H) = H_{t} \cdot P_{t}$ is the wealth at time $t$.
  Note that given $H$, we can find $C$ by solving the self-financing
  condition.
\end{notation}

If $c_{t} = 0$ a.s. for all $t$ then $H$ is a pure investment
strategy. 
\begin{exmp}
  \label{defn:discrete_time_models:3}
  Given an initial wealth $x > 0$, find $(H, c)$ to maximize
  \begin{equation}
    \label{eq:2}
    \sum_{i=1}^{T} \E{U(c_{t}))}
  \end{equation} subject to $X_{T}(H) = 0$ where $T > 0$ is not
  random.

  Assume that $U$ is strictly increasing, strongly concave, and
  bounded from above.
\end{exmp}

\section{A Detour into Martingales}
\label{sec:deto-into-mart}


\begin{proposition}
  Let $X$ be integrable and $\mathcal{G} \subseteq \mathcal{F}$. Then
  there exists an integrable, $\mathcal{G}$-measurable random variable
  $\bar{X}$ such that
  \begin{equation}
    \label{eq:3}
    \E{X \I{G}} = \E{\bar{X} \I{G})}
  \end{equation} for all $G \in \mathcal{G}$.  Moreover, it is unique
  in the sense that if $\bar{\bar{X}}$ has the same property, then $\bar
  X = \bar{\bar{X}}$.
\end{proposition}

\begin{defn}
  \label{defn:discrete_time_models:4}
  Such $\bar X$ is written $\E{X | \mathcal{G}}$, the conditional
  expectation of $X$ given $\mathcal{G}$.
\end{defn}

Useful properties of conditional expectation:
\begin{enumerate}
\item If $X$ is $\mathcal{G}$-measurable, then $\E{X | \mathcal{G}} = X$.
\item If $X$ is independent of $\mathcal{G}$ (that is, $X$ and $\I{G}$
  are independent for all $G \in \mathcal{G}$), then $\E{X |
    \mathcal{G}} = \E{X}$.
\item (Tower property) If $\mathcal{H} \subseteq \mathcal{G}
  \subseteq \mathcal{F}$, then
  \begin{equation}
    \label{eq:4}
    \E{\E{X | \mathcal{G}} | \mathcal{H}} = \E{ \E{X | \mathcal{H}} |
      \mathcal{G}} = \E{X | \mathcal{H}}
  \end{equation}
\item (Slot property) If $Y$ is $\mathcal{G}$-measurable and $XY$ is
  integrable, then
  \begin{equation}
    \label{eq:5}
    \E{XY | \mathcal{G}} = Y \E{X | \mathcal{G}}
  \end{equation}
\end{enumerate}

\begin{defn}
  \label{defn:discrete_time_models:5}
  A martingale $(X_{t})_{t \geq 0}$ with respect to a filtration
  $\mathbb{F}$ has the properties
  \begin{itemize}
  \item $\E{|X_{t}|} < \infty$ for all $t$,
  \item $\E{X_{t} | \mathcal{F}_{s}} = X_{s}$ for all $0 \leq s \leq t$.
  \end{itemize}

  Note that $X$ is automatically adapted.
\end{defn}

\begin{exer}
  Suppose $X$ is an integrable discrete-time process such that
  $\E{X_{t}| \mathcal{F}_{t-1}} = X_{t-1}$ for all $t \geq 1$. Show
  that $X$ is a martingale.
\end{exer}

\begin{exmp}
  \label{defn:discrete_time_models:6}
  Let $\xi_{i}$, $i = 1, 2, ...$ be independent, integrable random
  variables with $\E{\xi_{i}} = 0$.  Let $\mathcal{F}_{t} =
  \sigma(\xi_{1}, \dots, \xi_{t}), X_{t} = \xi_{1} + \xi_{2} + \dots +
    \xi_{t}$.

    Then $X$ is a martingale.
\end{exmp}

\begin{exmp}
  \label{defn:discrete_time_models:7}
  Let $\xi$ be integrable and let $\mathbb{F}$ be a filtration, and
  $X_{t} = \E{\xi | \mathcal{F}_{t}}$
\end{exmp}

\begin{proof}
  Integrability comes from integrability of conditional expectations.

  \begin{align*}
    \E{X_{t} | \mathcal{F}_{s}} & = \E{\E{\xi | \mathcal{F}_{t}} |
      \mathcal{F}_{s}}                                        \\
                                & = \E{\xi | \mathcal{F}_{s}} \\ 
                                & = X_{s}
  \end{align*}
\end{proof}


\begin{exmp}
  \label{defn:discrete_time_models:8}
  Suppose $X$ is a discrete-time martingale and $Y$ is predictable and
  bounded.  Let $Z_{t} = \sum_{s=1}^{t} Y_{s} (X_{s} - X_{s-1})$. Then
  $Z$ is a martingale.
\end{exmp}

\begin{proof}
  Integrability checked by integrability of $X$ and boundedness of
  $Y$.

  $Z_{t-1}$ is $\mathcal{F}_{t-1}$ measurable since measurability
  respects algebraic operations.
  \begin{align*}
    \E{Z_{t} | \mathcal{F}_{t-1}} & = \E{Z_{t-1} + Y_{t}(X_{t} -
      X_{t-1}) | \mathcal{F}_{t-1}} \\
                                  & = Z_{t-1} + \underbrace{Y_{t}}_{\text{slot property}} \E{\underbrace{X_{t} - X_{t-1}}_{=0} | \mathcal{F}_{t-1}}
  \end{align*}
\end{proof}

\begin{thm}
  \label{defn:discrete_time_models:9}
  Suppose $u: [0, \infty) \rightarrow \R$ is strictly increasing,
  strictly concave, differentiable, bounded from above.  Suppose there
  exists investment strategy $H^{\star}$ and consumption
  $c^{\star}_{t}= (H^{\star}_{t-1} - H^{\star}_{t}) \cdot P_{t-1}$,
  and a state price density $Y^{\star}$ such that $u'(c^{\star}_{t}) =
  Y^{\star}_{t-1}$.  Then $(H^{\star}, c^{\star})$ is optimal for the
  problem $\max \sum_{t=1}^{T} \E{u(c_{t})}$, subject to $X_{0}(H) =
  x, X_{T}(H) = 0$.
\end{thm}

\begin{proof}
  We consider the case where $\Omega$ is finite.

  Let $L(H, c, Y) = \E{\sum_{t=1}^{T} \left( u(c_{t}) +
      Y_{t+1}(H_{t+1} P(t+1)
      - c_{t} - H_{t} \cdot P_{t-1}) \right)}$
  Note that $L(H, c, Y)$ is the objective when $(H, c)$ is feasible.
  Then
  \begin{align*}
    L(H, c, Y) &= \E{\sum_{t=1}^{T} \left(u(c_{t}) - c_{t}
        Y_{t-1}\right)}
    +  Y_{0} X - Y_{t-1} H_{t} P_{t-1} + \sum_{t=1}^{T-1} H_{t}(Y_{t}
    P_{t} - Y_{t-1} P_{t-1})
  \end{align*}

  First note that $u(c^{\star}_{t}) - Y^{\star}_{t-1} c^{\star}_{t}
  \geq u(c_{t}) - Y^{\star}_{t-1} c_{t}$ since $u'(c^{\star}_{t}) =
  Y^{\star}_{t-1}$ (first order condition for the maximum of the
  concave function $c \mapsto u(c) - yc$).

  Second, by definition, $YP$ is a martingale, and by finiteness of
  $\Omega$, the predictable process $H$ is bounded. Therefore, $M_{t}
  = \sum_{s=1}^{t} H_{s}(Y_{s} P_{s} - Y_{s-1} P_{s-1})$ is a
  martingale and $E(M_{t}) = M_{s} = 0$.

  Putting this together, $L(H, c, Y^{\star}) \leq L(H^{\star},
  c^{\star}, Y^{\star})$.
\end{proof}

\begin{thm}
  \label{defn:discrete_time_models:10}
  An absolute arbitrage is an investment/consumption strategy $(H, c)$
  such that $X_{0}(H) = 0, X_T(H) = 0$, at some non-random time horizon
  $T > 0$, and $\Prob{\sum_{t=1}^T c_t > 0} > 0$.
\end{thm}

\begin{defn}
  \label{defn:discrete_time_models:11}
  A numeraire asset is one whose price is strictly positive almost surely.
\end{defn}

\begin{exmp}
  \label{defn:discrete_time_models:12}
  Here is a market without a numeraire. $P_{0} = 1, P_{0} = -1, P_{2}
  = 1$.

  Arbitrage:
  \begin{align*}
    H_{1} = -1, c_{1} = 1
    X_{1}= 1, c_{2} = 1, H_{2} = 0
    X_{2}= 0
  \end{align*}
\end{exmp}

\begin{exer}
  Suppose $H_{1}$ is an arbitrage and the market has a numeraire.
  Then there exists a pure investment strategy $H'$ and a time horizon $T'$ such that
  $X_{0}(H') = 0, X_{T'}(H') \geq 0$ a.s., and $\Prob{X_{T'}(H') > 0}
    > 0$.
\end{exer}

\begin{thm}
  \label{defn:discrete_time_models:13}
  A market model has no arbitrage if and only if there exists a state
  price density.
\end{thm}

\begin{proof}
  $T = 1$ case.  Suppose there exists a state price density
  $(Y_{t})_{t = 0, 1}$ without loss $Y_{0} = 1$.  Let $Y = Y_{1}$ for
  clarity, $Y > 0$ a.s.

  Suppose $(H_{t})_{t = 1} = H_{1} = H$ (non-random vector) is a
  candidate arbitrage, so $H \cdot P_{0} \leq 0$ and $H \cdot P_{1}
  \geq 0$ a.s.  We must show $H \cdot P_{0} = 0 = H \cdot P_{1}$ a.s.

  Since $Y > 0$, $H \cdot P_{1} \geq 0 \Rightarrow \E{Y H P_{1}} \geq
    0$, but $H \underbrace{\E{Y P_{1}}}_{\text{state price density}} =
    H P_{0} \leq 0$.

  By the pigeonhole principle, if $Z \geq 0$ a.s and $E(Z) = 0$, then
  $Z = 0$ a.s.

  Thus, $Y H \cdot P_{1} = 0$ a.s., and since $Y > 0$ a.s., $H_{0}
  P_{1} = 0 = H P_{0} = 0$ a.s.

  Now consider the other direction.  Let $\mathcal{Y} = \{ Y > 0 a.s,
  \E{Y \| P_{1} \|} < a \}$.  $\mathcal{Y}$ is non-empty since $Y_{0}
  = e^{-\| P_{1} \|} \in \mathcal{Y}$ and $\mathcal{Y}$ is convex.
  Let $\mathcal{C} = \{ \E{YP_{1}}, y \in \mathcal{Y} \}$.  Suppose
  $P_{0} \notin \mathcal{C}$.

  By the separating hyperplane theorem, there exists $H \in \R^{n}$ such
  that
  \begin{enumerate}
  \item For all $c \in \mathcal{C}$, $H(c - P_{0}) \geq 0$.
  \item There exists $c^{\star} \in \mathcal{C}$, $H(c^{\star} - P_{0}) > 0$.
  \end{enumerate}

  This implies
  \begin{enumerate}
  \item For all $Y \in \mathcal{Y}$, $\E{YH \cdot P_{1}} \geq H \cdot
    P_{0}$
  \item There exists $Y^{\star} \in \mathcal{Y}$, $\E{Y H \cdot P_{1}} > H
    \cdot P_{0}$.
  \end{enumerate}
\end{proof}

%%% Local Variables: 
%%% mode: latex
%%% TeX-master: "master"
%%% End: 