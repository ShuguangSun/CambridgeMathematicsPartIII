\chapter{Conditional Expectation}
\label{cha:cond-expect}

Let $(\Omega, \mathcal{F}, \Prob)$ be a probability space.
$\Omega$ is a set, $\mathcal{F}$ is a $\sigma$-algebra on $\Omega$,
and $\mathbb{P}$ is a probability measure on ($\Omega, \mathcal{F}$).

\begin{defn}
  \label{defn:1}
  $\mathcal{F}$ is a $\sigma$-algebra on $\Omega$ if it satisfies
  \begin{itemize}
  \item $\emptyset, \Omega \in \mathcal{F}$
  \item $A \in \mathcal{F} ==> A^{c} \in \mathcal{F}$
  \item $(A_n)_{n \geq 0}$ is a collection of sets in $\mathcal{F}$
    then $\cup_{n} A_{n} \in \mathcal{F}$.
  \end{itemize}
\end{defn}

\begin{defn}
  \label{defn:2}
  $\Prob$ is a probability measure on $(\Omega, \mathcal{F})$ if
  \begin{itemize}
  \item $\Prob: \mathcal{F} \rightarrow [0, 1]$ is a set function.
  \item $\Prob{\emptyset} = 0$, $\Prob{\Omega} = 1$,
  \item $(A_{n})_{n \geq 0}$ is a collection of pairwise disjoint
      sets in $\mathcal{F}$, then $\Prob{\cup_{n} A_{n}} =
      \sum_{n} \Prob{A_{n}}$.
  \end{itemize}
\end{defn}

\begin{defn}
  \label{defn:3}
  The Borel $\sigma$-algebra $\mathbb{B}(\mathbb{R})$ is the
  $\sigma$-algebra generated by the open sets of $\mathbb{R}$.  Call
  $\mathcal{O}$ the collection of open subsets of $\mathbb{R}$, then
  \begin{equation}
    \label{eq:1}
    \mathbb{B}(\mathbb{R}) = \cap \{ \xi: \xi \text{ is a sigma
      algebra containing $\mathcal{O}$} \}
  \end{equation}
\end{defn}

\begin{defn}
  \label{defn:4}
  $\mathcal{A}$ a collection of subsets of $\Omega$, then we write
  $\sigma(\mathcal{A}) = \cap \{ \xi: \xi \text{ a sigma algebra
    containing $\mathcal{A}$} \}$
\end{defn}

\begin{defn}
  \label{defn:5}
  $X$ is a random variable on $(\Omega, \mathcal{F})$  if $X: \Omega
  -> \mathbb{R}$ is a function with the proerty that $X^{-1}(V) \in
  \mathcal{F}$ for all $V$ open sets in $\mathbb{R}$.
\end{defn}

\begin{exer}
  If $X$ is a random varaible then $\{ B \subseteq \mathbb{R}, X^{-1}(B)
  \in \mathcal{F} \}$ is a $\sigma$-algebra and contains
  $\mathbb{B}(\mathbb{R})$.
\end{exer}

If $(X_{i}, i \in I)$ is a collection of random variables, then we
write $\sigma(X_{i}, i \in I) = \sigma(\{ \omega \in \Omega:
X_{i(\omega) \in \mathcal{B} \}, i \in I, B \in
  \mathbb{B}(\mathbb{R}))) }$
and it is the smallest $\sigma$-algebra that makes all the $X_{i}$'s
measurable.

\begin{defn}
  \label{defn:6}
  First we define it for the positive simple random variables.

  \begin{equation}
    \label{eq:2}
    \E{\sum_{i=1}^{n} c_{i} \mathbf{1}(A_{i})} =
    \sum_{i=1}^{n} \Prob{A_{i}}.
  \end{equation}
  with $c_{i}$ positive constants, $(A_{i}) \in \mathcal{F}$.

  We can extend this to any positive random variable $X \geq 0$ by
  approximation $X$ as the limit of piecewise constant functions.

  For a general $X$, we write $X = X^{+} - X^{-}$ with $X^{+}= \max(X,
  0), X^{-} = \max(-X, 0)$.
\end{defn}

If at least one of $\E{X^{+}}$ or $\E{X^{-}}$ is
finite, then we define $\E{X} = \E{X^{+}} +
\E{X^{-}}$.

We call $X$ integrable if $\E{|X|} < \infty$.

\begin{defn}
  \label{defn:7}
  Let $A, B \in \mathcal{F}, \Prob{B} > 0$. Then
  \begin{align*}
    \Prob{A | B} = \frac{\Prob{A \cap B}}{\Prob{B}} \\
    \mathbb{E}[X | B] = \frac{\E{X \mathbb{1}(B)}}{\Prob{B}}
  \end{align*}
\end{defn}

Goal - we want to define $\E{X | \mathcal{G}}$ that is a
random variable measurable with respect to the $\sigma$-algebra
$\mathcal{G}$.

\section{Discrete Case}
\label{sec:discrete-case}

Suppose $\mathcal{G}$ is a $\sigma$-algebra countably generated
$(B_{i)_{i \in \mathbb{N}}}$ is a collection of pairwise disjoint sets
in $\mathcal{F}$ with $\cup B_{i} = \Omega$. Let $\mathcal{G} =
\sigma(B_{i}, i \in \mathbb{N})$.

It is easy to check that $\mathcal{G} = \{ \cup_{j \in J} B_{j}, J
\subseteq {N} \}$.

Let $X$ be an integrable random variable.  Then
\begin{align*}
  X' = \E{X | \mathcal{G}} = \sum_{i \in \mathbb{N}}
  \E{X | \mathcal{B}_{i}} \I{B_{i}}
\end{align*}

\begin{enumerate}
\item $X'$ is $\mathcal{G}$-measurable (check).
\item
  \begin{equation}
    \label{eq:3}
    \E{|X'|} \leq \E{|X|}
  \end{equation} and so $X'$ is integrable.
\item $\forall G \in \mathcal{G}$, then
  \begin{equation}
    \label{eq:4}
    \E{X \I{G} = \E{X' \I{G}}}
  \end{equation} (check).
\end{enumerate}

\section{Existence and Uniqueness}
\label{sec:existence-uniqueness}

\begin{defn}
  \label{defn:8}
  $A \in \mathcal{F}$, $A$ happens almost surely (a.s.) if
  $\Prob{A} = 1$.
\end{defn}

\begin{thm}[Monotone Convergence Theorem]
  If $X_{n} \geq 0$ is a sequence of random variables and $X_{n}
  \uparrow X$ as $n \rightarrow \infty$ a.s, then
  \begin{equation}
    \label{eq:5}
    \E{X_{n}} \uparrow \E{X}
  \end{equation} almost surely as $n \rightarrow \infty$.
\end{thm}

\begin{thm}[Dominated Convergence Theorem]
  If $(X_{n})$ is a sequence of random variables such that $|X_{n}|
  \leq Y$ for $Y$ an integrable random variable, then if $X_{n} \cas
  X$ then $\E{X_{n}} \cas \E{X}$. 
\end{thm}

\begin{defn}
  \label{defn:9}
  For $p \in [1, \infty)$, $f$ measruable functions, then
  \begin{align}
    \| f \|_{p} = E[|f|^{p}]^{\frac{1}{p}} \\
    \| f \|_{\infty} = \inf \{ \lambda : |f| \leq \lambda a.e. \}
  \end{align}
\end{defn}

\begin{defn}
  \label{defn:10}
  \begin{align*}
    L^{p} = L^{p}(\Omega, \mathcal{F}, \mathbb{P}) = \{ f : \| f
    \|_{p} < \infty \} \\
  \end{align*}
  Formally, $L^{p}$ is the collection of equivalence classes where two
  functions are equivalent if they are equal a.e.  We will just
  represent an element of $L^{p}$ by a function, but remember that
  equality is a.e.
\end{defn}

\begin{thm}
  The space $(L^{2}, \| \cdot \|_{2})$ is a Hilbert space with
  $\langle U, V \rangle> = \E{UV}$.

  Suppose $\mathcal{H}$ is a closed subspace, then $\forall f \in
  L^{2}$ there exists a unique $g \in \mathcal{H}$ such that $\| f - g
  \|_{2} = \inf \{ \| f- h \|_{2}, h \in \mathcal{H}$ and $\langle f
  -g, h \rangle = 0$ for all $h \in \mathcal{H}$.

  We call $g$ the orthogonal projection of $f$ onto $\mathcal{H}$.
\end{thm}


\begin{thm}
  Let $(\Omega, \mathcal{F}, \P)$ be an underlying probability space,
  and let $X$ be an integrable random variable, and let $\mathcal{G}
  \subset \mathcal{F}$ sub $\sigma$-algebra.  Then there exists a
  random variable $Y$ such that
  \begin{enumerate}
  \item $Y$ is $\mathcal{G}$-measurable
  \item If $A \in \mathcal{G}$,
    \begin{equation}
      \label{exmp:general:1}
      \E{X \I{A} = \E{Y \I{A}}}
    \end{equation} and $Y$ is integrable.
  \end{enumerate}
  Moreover, if $Y'$ also satisfies the above properties, then $Y = Y'$ a.s.
\end{thm}
\begin{remark}
  $Y$ is called a version of the conditioanl expectiation of $X$ given
  $\mathcal{G}$ and we write $\mathcal{G} = \sigma(Z)$ as $Y = \E{X |
    \mathcal{G}}$.
\end{remark}


\begin{remark}
  (b) could be replaced by the following condition: for all $Z$
  $\mathcal{G}$-measurable, bounded random variables,
  \begin{equation}
    \label{exmp:general:2}
    \E{XZ} = \E{YZ}
  \end{equation}
\end{remark}

\begin{proof}

  \textbf{Uniqueness} - let $Y'$ satisfy (a) and (b).  If we consider
  $\{Y' - Y > 0 \} = A$, $A$ is $\mathcal{G}$ measurable.
  From (b),
  \begin{align*}
    \E{(Y' - Y)\I{A} = \E{X\I{A}}} - \E{X
      \I{A}} = 0
  \end{align*} and hence $\Prob{Y' - Y > 0)} = 0$ which implies that $Y'
    \leq Y$ a.s.  Similarly, $Y' \geq Y$ a.s.

  \textbf{Existence} - Complete the following three steps:
  \begin{enumerate}
  \item $X \in L^2(\Omega, \mathcal{F}, \P)$ is a Hilbert space with
    $\langle U, V \rangle = \E{UV}$.
    The space $L^{2}(\Omega, \mathcal{G}, \P)$ is a closed
    subspace.
    \begin{equation}
      \label{exmp:general:3}
      X_{n} \rightarrow X (L^{2}) => X_{n} \cp X => \exists subseq
      X_{n_{k}} \cas X => X' = \limsup X_{n_{k}}
    \end{equation}

  We can write
  \begin{align*}
    L^{2}(\Omega, \mathcal{F}, \P) = L^{2}(\Omega, \mathcal{G}, \P) +
    L^{2}(\Omega, \mathcal{G}, \P)^{\perp} \\
    X = Y + Z
  \end{align*}  Set $Y = \E{X | \mathcal{G}}$, $Y$ is
  $\mathcal{G}$-measurable, $A \in \mathcal{G}$.
  
  \begin{align*}
    \E{X \I{A}} = E{Y \I{A}} + \underbrace{E{Z \I{A}}}_{=0}
  \end{align*}
  
\item   If $X \geq 0$ then $Y \geq 0$ a.s. Consider $A = \{ Y < 0 \}$, then
  \begin{equation}
    \label{exmp:general:4}
    0 \leq \E{X \I{A}} = \E{Y \I{A}} \leq 0
  \end{equation}
  Thus $\Prob{A} = 0 \Rightarrow Y \geq 0$ a.s.

  Let $X \geq 0$, Set $0 \leq X_{n} = max(X, n) \leq n$, so $X_{n} \in
  L^{2}$ for all $n$. Write $Y_{n} = \E{X_{n} | \mathcal{G}}$, then
  $Y_{n} \geq 0$ a.s., $Y_{n}$ is increasing a.s..  Set $Y = \limsup
  Y_{n}$.  So $Y$ is $\mathcal{G}$-measurable.  We will show $Y = \E{X
    | \mathcal{G}}$ a.s.  For all $A \in \mathcal{G}$, we need to
  check $\E{X \I{A}} = \E{Y \I{A}}.$ We know that
  $\E{X_{n} \I{A}} = \E{Y_{n} \I{A}}$, and $Y_{n}
  \uparrow Y$ a.s.  Thus, by monotone convergence theorem, $\E{X
    \I{A}} = \E{Y\I{A}}$.
  
  If $X$ is integrable, setting $A = \Omega$, we have $Y$ is
  integrable.
\item $X$ is a general random variable, not necessarily in $L^{2}$ or
  $\geq 0$.  Then we have that $X = X^{+} + X^{-}$.  We define $\E{X |
    \mathcal{G}} = \E{X^{+} | \mathcal{G}} - \E{X^{-} | \mathcal{G}}$.
  This satisfies (a), (b).
\end{enumerate}
\end{proof}

\begin{remark}
  If $X \geq 0$, we can always define $Y = \E{X | \mathcal{G}}$ a.s.
  The integrability condition of $Y$ may not be satisfied.
\end{remark}

\begin{defn}
  \label{defn:general:1}
  Let $\mathcal{G}_{0}, \mathcal{G}_{1}, \dots$ be sub $\sigma$-algebas of
  $\mathcal{F}$.  Then they are called independent if for all $i, j
  \in \mathbb{N}$,
  \begin{equation}
    \label{exmp:general:5}
    \Prob{G_i \cap \dots \cap G_j} = \Pi_{i = 1}^{n} \Prob{G_{i}}
  \end{equation}
\end{defn}

\begin{thm}
  \begin{enumerate}
  \item If $X \geq 0$ then $\E{X | \mathcal{G}} \geq 0$
  \item $\E{\E{X | \mathcal{G}}} = \E{X}$ ($A = \Omega$)
  \item $X$ is $\mathcal{G}$-measurable implies $\E{X | \mathcal{G}} = X$ a.s.
  \item $X$ is independent of $\mathcal{G}$, then $\E{X|\mathcal{G}} = \E{X}$.
  \end{enumerate}
\end{thm}

\begin{thm}[Fatou's lemma] $X_{n} \geq 0$, then for all $n$,
  \begin{equation}
    \label{exmp:general:6}
    \E{\liminf X_{n}} \leq \liminf \E{X_{n}}
  \end{equation}
\end{thm}

\begin{thm}[Conditional Monotone Convergence]  Let $X_{n} \geq 0$,
  $X_{n} \uparrow X$ a.s.  Then
  \begin{equation}
    \label{exmp:general:7}
    \E{X_{n} | \mathcal{G}} \uparrow \E{X | \mathcal{G}} a.s.
  \end{equation} 
\end{thm}

\begin{proof}
  Set $Y_{n} = \E{X_{n} | \mathcal{G}}$.  Then $Y_{n} \geq 0$ and
    $Y_{n}$ is increasing.  Set $Y  = \limsup Y_{n}$.  Then $Y$ is
    $\mathcal{G}$-measurable.  
\end{proof}

\begin{thm}[Conditional Fatou's Lemma]  $X_{n} \geq 0$, then
  \begin{equation}
    \label{exmp:general:8}
    \E{\liminf X_{n} | \mathcal{G}} \leq \liminf \E{X_{n} |
      \mathcal{G}} a.s.
  \end{equation}
\end{thm}

\begin{proof}
  Let $X$ denote the limit inferior of the $X_{n}$. For every natural number
  $k$ define pointwise the random variable $Y_k= \inf_{n \geq k} X_n$.
  Then the sequence $Y_{1}, Y_{2}, \dots$ is increasing and converges pointwise
  to $X$. For $k \leq n$, we have $Y_{k} \leq X_{n}$, so that 
  \begin{equation}
    \label{eq:6}
    \E{Y_k | \mathcal{G}} \leq \E{X_n | \mathcal{G}} a.s
  \end{equation}

  by the monotonicity of conditional expectation, hence
  \begin{equation}
    \label{eq:7}
    \E{Y_k|\mathcal{G}} \leq \inf_{n \geq k} \E{X_n | \mathcal{G}} a.s.
  \end{equation}

  because the countable union of the exceptional sets of probability
  zero is again a null set. Using the definition of X, its
  representation as pointwise limit of the Yk, the monotone convergence
  theorem for conditional expectations, the last inequality, and the
  definition of the limit inferior, it follows that almost surely

  \begin{align}
  \E{\liminf_{n \rightarrow \infty} X_{n} | \mathcal{G}} 
 & = \E{X |
    \mathcal{G}}                                         \\
 & = \E{\lim_{k \rightarrow \infty} Y_{k} | \mathcal{G}} \\
 & = \lim_{k \rightarrow \infty} \E{Y_{k} | \mathcal{G}} \\
 & \leq \lim_{k \rightarrow \infty} \inf_{n \geq k} \E{X_{n} |
    \mathcal{G}}                                         \\
 & = \liminf_{n \rightarrow \infty} \E{X_{n} | \mathcal{G}}
 \end{align}
\end{proof}


\begin{thm}[Conditional dominated convergence] TODO
  
\end{thm}


%%% Local Variables: 
%%% mode: latex
%%% TeX-master: "master"
%%% End: 
