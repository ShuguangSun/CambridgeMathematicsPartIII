\documentclass[8pt, reqno, oneside, twocolumn]{amsart} 

% Utility packages
\usepackage{etoolbox}
\newtoggle{tufte}
\togglefalse{tufte}
\newtoggle{summary}
\togglefalse{summary}
\usepackage{amsthm}
\usepackage{amsmath}
\usepackage{amssymb}
% \usepackage[landscape]{geometry}
\usepackage[top=1cm, bottom=1cm, left=1cm, right=1cm]{geometry}
% \usepackage[inner=1cm, outer=1cm]{geometry}
\usepackage{setspace}
\usepackage{graphicx}
\usepackage{enumerate}
\usepackage{listings}
\usepackage{booktabs}
\usepackage{hyperref}
\usepackage{thmtools}
\usepackage{xparse}
\usepackage{xspace}
\usepackage{enumitem}
\usepackage{todonotes}
\usepackage{dcolumn}
\usepackage{array}
\usepackage{booktabs}
\usepackage{colortbl, xcolor}
\usepackage{longtable}
\usepackage{natbib}
\usepackage{python}
% \usepackage{a4wide}

\hypersetup{colorlinks=true}
\numberwithin{equation}{section}

\declaretheorem[numbered=no,name=Theorem,style=definition]{thm}
\declaretheorem[numbered=no,name=Lemma]{lem}
\declaretheorem[numbered=no,name=Corollary]{corollary}
\declaretheorem[numbered=no,name=Definition]{defn}
\declaretheorem[numbered=no,name=Proposition]{proposition}
\declaretheorem[numbered=no,name=Example]{exmp}
\declaretheorem[numbered=no,name=Exercise]{exer}
\declaretheorem[numbered=no,name=History]{history}
\declaretheorem[numbered=no,name=Question]{question}
\declaretheorem[numbered=no,name=Remark]{remark}
\declaretheorem[numbered=no,name=Notation]{notation}
\declaretheorem[numbered=no,name=Theorem]{boxthm}

\let\proof\relax
\declaretheorem[style=remark,numbered=no,qed=\qedsymbol]{proof}

\DeclareMathOperator*{\argmax}{arg\,max}
\DeclareMathOperator*{\dist}{dist}
\DeclareMathOperator*{\V}{V}
\DeclareMathOperator*{\saddlepoint}{sp}
\DeclareMathOperator*{\divergence}{div}
\DeclareMathOperator*{\sym}{Sym}
\DeclareMathOperator*{\vol}{Vol}
\DeclareMathOperator*{\argmin}{arg\,min}
\DeclareMathOperator*{\proj}{\Pi}
\DeclareMathOperator*{\bnd}{bnd}
\DeclareMathOperator*{\supp}{supp}
\DeclareMathOperator*{\aff}{aff}
\DeclareMathOperator*{\rint}{rint}
\DeclareMathOperator*{\lev}{lev}
\DeclareMathOperator*{\tr}{tr}
\DeclareMathOperator*{\interior}{int}
\DeclareMathOperator*{\epi}{epi}
\DeclareMathOperator*{\dom}{dom}
\DeclareMathOperator*{\vecspan}{span}
\DeclareMathOperator*{\con}{con}
\DeclareMathOperator*{\cl}{cl}
\DeclareMathOperator*{\grad}{\nabla}
\DeclareMathOperator*{\crit}{crit}
\DeclareMathOperator*{\diag}{diag}
\DeclareMathOperator*{\res}{Res}
\DeclareMathOperator*{\boundary}{boundary}
\DeclareMathOperator*{\range}{range}
\DeclareMathOperator*{\Tr}{Tr}
\DeclareMathOperator*{\sign}{sign}
\setlist[enumerate]{label=(\roman*)}

\newlist{exercises}{enumerate}{1}
\setlist[exercises]{label=Ex. \arabic*}


\setcounter{secnumdepth}{2}
\setcounter{tocdepth}{2}
% \onehalfspacing

\author{Andrew Tulloch}


% Shortcuts
% convergence in probability
\newcommand{\cd}{\overset{d}{\rightarrow}}    
% convergence in distribution
\newcommand{\cp}{\overset{p}{\rightarrow}}
% convergence almost surely
\newcommand{\cas}{\overset{as}{\rightarrow}}
\newcommand{\R}{\mathbb{R}}
\newcommand{\N}{\mathbb{N}}
\newcommand{\Z}{\mathbb{Z}}
\newcommand{\np}{\textsc{np}\xspace}
\newcommand{\iid}{\textsc{iid}\xspace}
\newcommand{\knn}{$k$-\textsc{nn}\xspace}

% Expectation
\NewDocumentCommand{\E}{gg}{%
  \IfNoValueTF{#1}
    {%
      \mathbb{E}
    }    
    {%
      \IfNoValueTF{#2}
      {%
        \mathbb{E}\!\left(#1\right)
      }
      {%
        \mathbb{E}_{#2}\!\left(#1\right)
      }
    }%
}

% Probability with measure P
\NewDocumentCommand{\Prob}{gg}{%
  \IfNoValueTF{#1}
    {%
      \mathbb{P}
    }    
    {%
      \IfNoValueTF{#2}
      {%
        \mathbb{P}\!\left(#1\right)
      }
      {%
        \mathbb{P}_{#2}\!\left(#1\right)
      }
    }%
}

% Probability with measure Q
\NewDocumentCommand{\Q}{g}{%
  \IfNoValueTF{#1}
    {%
      \mathbb{Q}
    }    
    {%
      \mathbb{Q}\!\left(#1\right)
    }%
}


% Indicator
\NewDocumentCommand{\I}{g}{%
  \IfNoValueTF{#1}
    {%
      \mathbb{I}
    }    
    {%
      \mathbb{I}\!\left(#1\right)
    }%
}


\NewDocumentCommand{\Var}{g}{%
  \IfNoValueTF{#1}
    {%
      \mathbb{V}
    }    
    {%
      \mathbb{V}\!\left(#1\right)
    }%
}


\NewDocumentCommand{\Cov}{gg}{%
  \IfNoValueTF{#1}
    {%
      \text{Cov}
    }   
    {%
      \text{Cov}\!\left(#1, #2\right)
    }%
}

% Inner product
\NewDocumentCommand{\IP}{g}{%
    {%
      \left \langle #1 \right \rangle
    }%
}
%%% Local Variables: 
%%% mode: latex
%%% End: 


\title{Mathematics of Operations Research Summary}

\begin{document}

\maketitle

\section{Optimization}
\label{sec:optimization}


\begin{defn}
  \label{sec:optimization-2}
  An optimization problem has the standard form, $\min f(x)$ s.t.
  $h(x) = b, x \in X$.  The set $X(b) = \{ x \in X : h(x) = b \} $ is
  called the feasible set, and a problem is feasible if $X(b)$ is
  non-empty and is bounded if $f(x)$ is bounded from below no $X(b)$.
  A vector $ x^{\star}$ is optimal if it is in the feasible set and
  minimizes $f$ among all vectors in the feasible set.
\end{defn}

\begin{defn}
  \label{sec:optimization-3}
  The \textbf{Lagrangian} associated with \ref{sec:optimization-2} is
  \begin{equation}
    \label{eq:1}
    L(x, \lambda) = f(x) - \lambda^{T} (h(x) - b)
  \end{equation}

\end{defn}

\begin{thm}
  \label{sec:optimization-1}
  Let $x \in X$ and $\lambda \in \R^{m}$ such that $L(x, \lambda) =
  \inf_{x' \in x} L(x', \lambda)$ and $h(x) = b$.  Then $x$ is an
  optimal solution 
\end{thm}

\begin{thm}
  \label{sec:optimization-4}
  To minimize $f(x)$ subject to $h(x) \leq b, x \in X$,
  \begin{enumerate}
  \item Introduce a vector $z$ of slack variables to obtain the
    equivalent problem $\min f(x)$ s.t. $h(x) + z = b, x \in X, z \geq
    0$.
  \item Compute $L(x, z, \lambda) = f(x) - \lambda^{T}(h(x) + z - b)$
  \item Define $Y = \{ \lambda \in \R^{m} | \inf_{x \in X, z \geq 0}
    L(x, z, \lambda) > -\infty \} $
  \item For each $\lambda \in Y$, minimize $L(x, z, \lambda)$ subject
    only to regional constraints - so finding $x^{\star}(\lambda),
    z^{\star}(\lambda)$ satisfying $L(x^{\star}(\lambda),
    z^{\star}(\lambda), \lambda) = \inf_{x \in X, z \geq 0} L(x, z,
    \lambda)$.
  \item Find $\lambda^{\star} \in Y$ such that $(x^{\star}(\lambda),
    z^{\star}(\lambda^{\star}))$ is feasible - so
    $x^{\star}(\lambda^{\star}) \in X$, $z^{\star}(\lambda^{\star})
    \geq 0$, and $h(x^{\star}(\lambda^{\star})) +
    z^{\star}(\lambda^{\star}) = b$.  By \ref{sec:optimization-3},
    $x^{\star}(\lambda^{\star})$ is optimal.
  \end{enumerate}
\end{thm}

\begin{defn}
  \label{sec:optimization-5}
  Denote $\phi(b) = \inf_{x \in X(b)}f(x)$ the solution of our
  optimization problem, and define the \textbf{Lagrange dual function}
  $G: \R^{m} \rightarrow \R$ as the minimum value of the Lagrangian
  over $X$, so $g(\lambda) = \inf_{x \in X} L(x, \lambda)$. For all
  $\lambda \in \R^{m}$,
  \begin{equation}
    \label{eq:2}
    \inf_{x \in X(b)} f(x) = \inf_{x \in X(b)} L(x, \lambda) \geq
    \inf_{x \in X} L(x, \lambda) = g(\lambda)
  \end{equation}
  so we have a lower bound on the optimal value of our problem.  Thus,
  the dual problem is to \textbf{maximize the lower bound}, thus $\max
  g(\lambda)$ s.t. $\lambda \in Y$, where $Y = \{ \lambda \in \R^{m} |
g(\lambda) > -\infty \} $.
\end{defn}

\begin{thm}
  \label{sec:optimization-6}
  We have the \textbf{weak duality theorem}, 
  $\inf_{x \in X(b)} \geq \max_{\lambda \in Y} g(\lambda)$.

  The primal problem satisfies \textbf{strong duality} if this holds
  with equality --- so there exists $\lambda$ such that $\phi(b) = g(\lambda)$.
\end{thm}

\begin{defn}
  \label{sec:optimization-8}
  Call a hyperplane $\alpha: \R^{m} \rightarrow \R$ a
  \textbf{supporting hyperplane} to $\phi$ at $b$ if $\alpha(c) =
  \phi(b) - \lambda^{T}(b-c)$ and $\phi(c) \geq \phi(b) -
  \lambda^{T}(b-c)$ for all $c \in \R^{m}$.
\end{defn}

\begin{thm}
  \label{sec:optimization-10}
  There exists a (no-vertical) supporting hyperplane to $\phi$ at $b$
  if and only if the problem satisfies strong duality.
\end{thm}

\section{Linear Programming}
\label{sec:linear-programming}

\begin{thm}
  \label{sec:linear-programming-1}
  Suppose that $\phi$ is convex and $b \in \R$ lies in the interior of
  the set of points where $\phi$ is finite.  Then there exists a
  (non-vertical) supporting hyperplane to $\phi$ at $b$.
\end{thm}

\begin{thm}
  \label{sec:linear-programming-2}
  Consider the optimization problem, $\min f(x)$ s.t. $h(x) \leq b, x
  \in X$, and let $\phi$ be given by $\phi(b) = \inf_{x \in X(b)}
  f(x)$.  Then $\phi$ is convex when $X, f$ and $h$ are convex.
\end{thm}

\begin{defn}
  \label{sec:linear-programming-3}
  A linear program is in \textbf{general form} when written as $\min
  \{ c^{T} x | Ax \geq x, x \geq 0 \}$.

  A linear program of the form $\min \{ c^{T}x | Ax = b, x \geq 0 \} $
  is said to be in \textbf{standard form}.
\end{defn}

\begin{thm}
  \label{sec:linear-programming-4}
  A linear program in general form can be written with slack variables
  as $\min \{ c^{T} x | Ax - z = b, x, z \geq 0 \} $.  Then $X = \{
(x, z) : x \geq 0, z \geq 0 \} $, and the Lagrangian is $L((x, z),
\lambda) = c^{T} x - \lambda^{T}(Ax - z - b) = (c^{T} - \lambda^{T}
A)x + \lambda^{T} z + \lambda^{T}b$ with finite minimum over $X$ if
and only if $\lambda \in Y = \{ \mu \in \R^{m} | c^{T}- \mu^{T} A \geq
0, \mu \geq 0 \} $.  Thus $g(\lambda) = \inf_{(x, z) \in X} L((x, z),
\lambda) = \lambda^{T} b$.  The dual problem is thus $\max \{ b^{T}
\lambda | A^{T} \lambda \leq c, \lambda \geq 0 \}$.  

Analogously, the dual of the \textbf{standard form} is $\max b^{T}
\lambda | A^{T} \lambda \leq c$.
\end{thm}

\begin{thm}
  \label{sec:linear-programming-5}
  Let $x, \lambda$ be feasible solutions for the primal in general
  form and dual of general form, respectively.  Then $x, \lambda$ are
  optimal if and only if they satisfy complementary slackness - so
  $(c^{T} - \lambda^{T} A) x = 0$ and $\lambda^{T}(Ax - b) = 0$.
\end{thm}

\begin{thm}
  \label{sec:linear-programming-6}
  Suppose $f, h$ are continuously differentiable on $\R^{n}$, and
  there exists unique function $x^{\star}: \R^{m} \rightarrow \R^{n}$
  and $\lambda^{\star}: \R^{m} \rightarrow \R^{m}$ such that for each
  $b \in \R^{m}$, $h(x^{\star}(b)) = b$, $\lambda^{\star}(b) \leq 0$
  and $f(x^{\star}(b)) = \phi(b) = \inf \{ f(x) -
  \lambda^{\star}(b)^{T}(h(x) - b) | x \in \R^{n}\} $.  If $x^{\star}$
  and $\lambda^{\star}$ are continuously differentiable, then
  \begin{equation}
    \label{eq:3}
    \frac{\partial \phi}{\partial b_{i}} (b) = \lambda^{\star}_{i}(b).
  \end{equation}
\end{thm}

\section{The Simplex Method}
\label{sec:simplex-method}

\begin{defn}
  \label{sec:simplex-method-2}
  Consider the problem $\max c^{T} x$ s.t. $Ax = b, x \geq 0$, where
  $A \in R^{m \times n}$ and $b \in R^{m}$.  Call a solution $x \in
  \R^{n}$ of the equation $Ax = b$ \textbf{basic} if at most $m$ of
  its entries are non-zero - so there exists a set $B \subseteq \{ 1,
  \dots, n \}$ with $|B| = m$ such that $x_{i} = 0$ if $i \notin B$.
  The set $B$ is called the \textbf{basis}, and $x_{i}$ is
  \textbf{basic} if $i \in B$ and \textbf{non-basic} if $i \notin B$.
  A basic solution $x$ that also satisfies $x \geq 0$ is called a
  \textbf{basic feasible solution}.
\end{defn}

\begin{thm}
  \label{sec:simplex-method-3}
  $x$ is a basic feasible solution of $Ax = b$ if and only if it is an
  extreme point of the set $X(b) = \{ x : Ax = b, x \geq 0 \} $.
\end{thm}

\begin{thm}
  \label{sec:simplex-method-4}
  If the linear program $\max c^{T} x$ s.t. $Ax = b, x \geq 0$ is
  feasible and bounded, then it has an optimal solution that is also a
  basic feasible solution.
\end{thm}



\begin{defn}
  \label{sec:simplex-method-1}
  The simplex method consists of the following steps.

  \begin{equation}
    \label{eq:4}
    \begin{Bmatrix}
      (a_{ij}) & a_{i0} \\
      a_{0j} & a_{00}
    \end{Bmatrix}
  \end{equation}

  \begin{enumerate}
  \item\label{item:2} Find an initial BFS with basis $B$.
  \item\label{item:3} Check whether $a_{0j} \leq 0$ for every $j$.  If yes, the
    current solution is optimal, so stop.
  \item\label{item:4} Choose $j$ such that $a_{0j} > 0$, and choose $i \in \{ i' |
    a_{i'j} > 0 \} $ to minimize $a_{i0} / a_{ij}$.  If $a_{ij} \leq
    0$ for all $i$, the problem is unbounded, so stop.  If multiple
    rows minimize $\frac{a_{i0}}{a_{ij}}$, the problem has a
    degenerate BFS.
  \item\label{item:5} Update the tableau by multiplying row $i$ by
    $\frac{1}{a_{ij}}$ and adding $-\frac{a_{ij}}{a_{ij}}$ multiples
    of row $i$ to each row $k \neq i$. Then return to step \ref{item:3}.
  \end{enumerate}
\end{defn}



\section{Advanced Simplex Procedures}
\label{sec:advanc-simpl-proc}

\begin{defn}
  \label{sec:advanc-simpl-proc-1}
  Two phase simplex method can be used where it is difficult to find
  an initial BFS.  We proceed as follows:

  \begin{enumerate}
  \item Bring the constraints into equality form.  For each constraint
    in which the slack variable and the right hand side have opposite
    signs, or in which there is no slack variable, add a new
    artificial variable that has the same sign as the right hand side.
  \item Minimize the sum of the artificial variables, starting from
    the BFS where the absolute value of the artificial variable for
    each constraint, or the slack variable in the case there is no
    artificial variable, is equal to that of the right hand side.
  \item If some artificial variable has a positive value in the
    optimal solution, the original problem is infeasible --- stop.
  \item Now solve the original problem, starting from the BFS in Phase I.
  \end{enumerate}
\end{defn}

\begin{defn}
  \label{sec:advanc-simpl-proc-2}
  We first find a dual-feasible solution, and proceed by selecting a
  row $i$ such that $a_{i0} < 0$ and a column $j \in \{ j' | a_{ij'} <
0 \} $ that minimized $-\frac{a_{0j}}{a_{ij}}$.  We can then pivot
just like in the primal algorithm.
\end{defn}

\todo{Cutting plane method?}

\section{Complexity of Problems and Algorithms}
\label{sec:compl-probl-algor}

\begin{defn}
  \label{sec:compl-probl-algor-1}
  A problem is in $P$ if there exists a Turing machine $M$ and $k \in
  \N$ with the following property - for every $x \in \{ 0, 1
  \}^{\star}$, if $M$ is started with input $x$, then after
  $\mathcal{O}(|x|^{k})$ steps it halts with output $f(x)$.

  $L \subseteq \{ 0, 1 \}^{\star} $ is in $NP$ if there exists a
  Turing machine $M$ and $k \in \N$ with the property that for every
  $x \in \{ 0, 1 \}^{\star} $, $x \in L$ if and only if there exists a
  certificate $y \in \{ 0, 1 \}^{\star} $ with $|y| =
  \mathcal{O}(|x|^{k})$ such that $M$ accepts $(x, y)$ after
  $\mathcal{O}(|x|^{k})$ steps.
\end{defn}

\section{The Complexity of Linear Programming}
\label{sec:compl-line-progr}

\begin{thm}
  \label{sec:compl-line-progr-1}
  Consider the linear problem of minimizing $-x_{n}$ subject to
  $\epsilon \leq x_{1} \leq 1$, $\epsilon x_{i-1} \leq x_{i} \leq 1 -
  \epsilon x_{i-1}$ for $i = 2, \dots, n$. Then there exists a
  pivoting rule and an initial BFS such the simplex methods requires
  $2^{n} - 1$ iterations before terminating.
\end{thm}

\begin{defn}
  \label{sec:compl-line-progr-2}
  Given a symmetric positive definite matrix $D \in R^{n \times n}$
  and $z \in \R^{n}$, the set of points
  \begin{align}
    \label{eq:5}
    E = E(z, D) = \{ x \in \R^{n} | (x - z)^{T} D^{-1}(x - z) \leq 1 \} 
  \end{align} is called an ellipsoid with center $z$.
\end{defn}

\begin{thm}
  \label{sec:compl-line-progr-3}
  Let $E = E(z, D)$ be an ellipsoid in $\R^{n}$ and $a \in \R^{n}$
  non-zero.  Consider the half-space $H = \{ x \in \R^{n} | a^{T} x
  \geq a^{T}z \} $, and let $z' = z + \frac{1}{n+1}
  \frac{Da}{sqrt{a^{T}Da}}$, $D' = \frac{n^{2}}{n^{2} - 1}(D -
  \frac{2}{n+1}\frac{Daa^{T}D}{a^{T}Da})$.  Then $D'$ is symmetric and
  positive definite, and so $E' = E(z', D')$ is an ellipsoid.
  Moreover, $E \cap H \subseteq E'$ and $\vol E' <
  \exp(-\frac{1}{2(n+1)}) \vol E$.
\end{thm}

\begin{thm}
  \label{sec:compl-line-progr-4}
  Let $S: \R^{n} \rightarrow \R^{n}$ be an affine transformation given
  by $S(x) = Dx + b$, and let $L \subseteq \R^{n}$.  Then $\vol S(L) =
  |\det D| \vol L$.
\end{thm}

\section{The Ellipsoid Method}
\label{sec:ellipsoid-method}

\begin{defn}
  \label{sec:ellipsoid-method-1}
  Consider a polytope $P = \{ x \in \R^{n} | Ax \geq b \} $, with $A \in
  Z^{m \times n}$ and $b \in Z^{m}$.  Assume that $P$ is bounded and
  either empty or full-dimensional ($\vol P > 0$).  The ellipsoid
  method proceeds as follows to decide whether $P$ is non-empty.

  
  \begin{enumerate}
  \item Let $U$ be the largest absolute value among the entries of $A$
    and $b$, and define $x_{0} = 0$, $D_{0} = n(nU)^{2n}I$, $E_{0} =
    E(x_{0}, D_{0})$, $V = (2n)^{n}(nU)^{n^{2}}$, $v =
    n^{-n}(nU)^{-n^{2}(n+1)}$, $t^{\star} = \lceil 2(n+1) \log
    \frac{V}{v} \rceil$.\todo{Notify lecturer about $x_{0}$ not
      specified in notes}
  \item For $t = 0, \dots, t^{\star}$, do
    \begin{enumerate}
    \item If $t = t^{\star}$, stop ($P$is empty)
    \item If $x_{t} \in P$, then stop ($P$ is non-empty)
    \item Find a violated constraint (a row $j$ such that $a_{j}^{T}
      x_{t} < b_{j}$).
    \item Let $E_{t+1} = E(x_{t+1}, D_{t+1})$, with $x_{t+1} = x_{t} =
      \frac{1}{n+1} \frac{D_{t} a_{j}}{\sqrt{a_{j}^{T} D_{t} a_{j}}}$,
      $D_{t+1} = \frac{n^{2}}{n^{2} - 1}(D_{t} - \frac{2}{n+1}
      \frac{D_{t} a_{j} a_{j}^{T} D_{t}}{a_{j}^{T} D_{t} a_{j}})$.
    \end{enumerate}
  \end{enumerate}
\end{defn}

\todo{Proof of correctness?}

\section{Graphs and Flows}
\label{sec:graphs-flows}

\begin{defn}
  \label{sec:graphs-flows-1}
  $x \in \R^{n \times n}$ is a \textbf{minimum cost flow} of $G$ if it
  is an optimal solution of the following optimization problem: $\min
  \sum_{(i, j) \in E}^{} c_{ij} x_{ij}$ s.t. $b_{i} + \sum_{j: (j, i)
    \in E} x_{ji} = \sum_{j: (i, j) \in E}^{} x_{ij}$ for all $i \in
  V$ (source/sink balance), and $\underline m_{ij} \leq x_{ij}
  \overline m_{ij}$ for all $(i, j) \in E$ (capacity constraint).

  Note that $\sum_{i \in V}^{} b_{i} = 0$ is required for any feasible
  flows to exist.

  The minimum cost flow is an LP, with constraints of the form $Ax =
  b$, where
  \begin{equation}
    \label{eq:7}
    a_{ik} =
    \begin{cases}
      1 & \text{$k$-th edge starts at vertex $i$} \\
      -1 & \text{$k$-th edge ends at vertex $i$} \\
      0 & \text{otherwise}
    \end{cases}
  \end{equation}
\end{defn}

\begin{defn}
  \label{sec:graphs-flows-3}
  As solution $x$ to the minimum cost flow problem for a connected
  network $G = (V, E)$ is a \textbf{spanning tree solution} if there
  exists a spanning tree $(V, T)$ of $G$ and two sets $L, U \subseteq
  E$ with $L \cap U = \emptyset$ and $L \cup U = E \backslash T$ such
  that $x_{ij} = \underline m_{ij}$ if $(i, j) \in L$, and $x_{ij} =
  \overline m_{ij}$ if $(i, j) \in U$. These constraints determine the
  values for $x_{ij}$ for $(i, j) \in T$.
\end{defn}

\begin{thm}
  \label{sec:graphs-flows-2}
  A flow vector is a basic solution of a minimum cost flow problem if
  and only if it is a spanning tree solution.
\end{thm}

\begin{defn}
  \label{sec:graphs-flows-4}
  The \textbf{Lagrangian} of the minimum cost flow problem is
  \begin{align}
    \label{eq:6}
    L(x, \lambda) &= \sum_{(i, j) \in E}^{} c_{ij} x_{ij} - \sum_{i \in
    V}^{} \lambda_{i} (\sum_{j: (i, j) \in E}^{} x_{ij} - \sum_{j: (j,
    i) \in E}^{} x_{ji} - b_{i}) \\
  &= \sum_{(i, j) \in E}^{} (c_{ij} - \lambda_{i} + \lambda_{j})
  x_{ij} + \sum_{i \in V}^{\lambda_{i}}  b_{i}.
\end{align}

Let $\overline c_{ij} = c_{ij} - \lambda_{i} + \lambda_{j}$ be the
\textbf{reduced cost} of edge $(i, j) \in E$. We have the
complementary slackness conditions $\overline c_{ij} > 0$ implies
$x_{ij} = \underline m_{ij}$, $\overline c_{ij} < 0$ implies $x_{ij} =
\overline m_{ij}$, and $\underline m_{ij} < x_{ij} < \overline m_{ij}$
implies $\overline c_{ij} = 0$.

Assume that $x$ is a BFS associated with sets $T$, $U$, and $L$.  Then
the system of equations $\lambda_{|V|} = 0$, $\lambda_{i} -
\lambda_{j} = c_{ij}$ for all $(i, j) \in T$ has a unique solution,
which allows us to compute $\overline c_{ij}$ for all edges $(i, j)
\in E$.  Note that by construction, $\overline c_{ij} = 0$ for all $(i,
j) \in T$.
\end{defn}

\begin{thm}
  \label{sec:graphs-flows-6}
  If $\overline c_{ij} \geq 0$  for all $(i, j) \in L$, and $\overline
  c_{ij} \leq 0$ for all $(i, j)\in U$, then $(x, \lambda)$ is
  dual-feasible and therefore optimal. Otherwise, find and edge $(i,
  j)$ that violates these conditions, and observe this edge with the
  edges in $T$ forms a unique cycle $C$.  Since $(i, j)$ is the only
  edge in $C$ with non-zero reduced cost, we can decrease the
  objective by pushing flow along $C$ to increase $x_{ij}$ if
  $\overline c_{ij}$ is negative and decrease $x_{ij}$ if $\overline
  c_{ij}$ is positive.  

Let $\underline B \subseteq C$ denote the set of edges whose flow is
to decrease, and $\overline B \subseteq C$ the set of edges whose flow
is to increase.  If the problem is uncapacitated and $\underline B =
\emptyset$ or $\overline B = \emptyset$, the problem is unbounded.
Otherwise, changing the flow by $\min \{ \min_{(k, l) \in \underline
  B} \{ x_{kl} - \underline m_{kl} \}, \min_{(k, l) \in \overline B}
\{ \overline m_{kl} - x_{kl} \}  \} $ decreases this object as much as
possible while maintaining prime feasibility.  After this change,
there will be an edge $(k, l) \in C$  whose flow is either $\underline
m_{k,l}$ or $\overline m_{k,l}$.  If $(k, l) \in T$, we obtain a new
BFS with spanning tree $(T \backslash \{ (k, l) \}) \cup \{ (i, j)
\}$.  If instead $(k, l) = (i, j)$, we obtain a new BFS where $(i, j)$
has moved from $U$ to $L$, or vice versa.

To find an initial BFS, set all $\underline m_{ij} = 0$ (by
introducing flows of forced capacity $\overline m'_{ij} = \underline
m'_{ij} = \underline m_{ij}$)  Introduce a dummy vertex $d \notin V$
and uncapacitated dummy edges $E' = \{ (i, d) | i \in V, b_{i} \geq 0
\} \cup \{ (d, i) | i \in V, b_{i} < 0\}  $ with cost equal to
$\sum_{(i, j) \in E}^{} c_{ij}$.  A dummy edge has positive flow in an
optimal solution of the new problem if and only if the original
problem is infeasible. A feasible spanning tree solution is obtained
by setting $T = E'$, $x_{id} = b_{i}$ for all $i \in V$ with $b_{i} >
0$, $d_{di} = -b_{i}$ for all $i \in V$ with $b_{i} < 0$, and $x_{ij}
= 0$ otherwise.
\end{thm}

\begin{thm}
  \label{sec:graphs-flows-5}
  Consider a minimum cost flow problem that is feasible and bounded.
  If $b_{i}$is integral for all $i \in V$ and $\underline m_{ij}$ and
  $\overline m_{ij}$are integral for all $(i, j) \in E$, then there
  exists an integral optimal solution.  If $c_{ij}$is integral for all
  $(i, j)\in E$, then there exists an integral optimal solution to the dual.
\end{thm}

\section{Transportation and Assignment Problems}
\label{sec:transp-assignm-probl}

\begin{defn}
  \label{sec:transp-assignm-probl-1}
  We are given a set of supplies producing $s_{i}$ units of a good and
  a set of consumers with demands $d_{j}$, with $\sum_{i=1}^{n} s_{i}
  = \sum_{j=1}^{m} d_{j}$.  The cost of transporting  from supplier
  $i$ to $j$is $c_{ij}$.  We formulate the problem of a minimum cost
  flow on the bipartite network as $\min \sum_{i=1}^{} \sum_{j=1}^{m}
  c_{ij} x_{ij}$ s.t. $\sum_{i=1}^{n} x_{ij} = d_{j}$, $\sum_{j=1}^{m}
  x_{ij} = s_{i}$, and $x_{ij} \geq 0$ for all $i, j$.
\end{defn}

\begin{thm}
  \label{sec:transp-assignm-probl-2}
  Every minimum cost flow problem with finite capacities or
  non-negative costs has an equivalent transportation problem.
\end{thm}

\todo{Fill in network simplex tableau description?}

\begin{thm}
  \label{sec:transp-assignm-probl-3}
  In the assignment problem, we have to assign exactly one agent to
  one job - so the problem is $\min \sum_{i=1}^{n} \sum_{j=1}^{n}
  c_{ij} x_{ij}$ subject to $x_{ij} \in \{ 0, 1 \}$, $\sum_{j=1}^{n}
  x_{ij} = 1$, $\sum_{i=1}^{n} x_{ij} = 1$ - since all solutions to
  the LP relation are spanning tree solutions (and thus integral), the
  network simplex method yields an optimal solution of the original
  problem when applied to the LP relaxation.
\end{thm}

\section{Maximum Flows and Perfect Matchings}
\label{sec:maxim-flows-perf}

\begin{defn}
  \label{sec:maxim-flows-perf-1}
  In a flow network $(V, E)$ with a single source $1$, a single sink
  $n$, and finite capacities $\underline m_{ij} = 0, \overline m_{ij} = C_{ij}$ for all $(i,
  j) \in E$.    The maximum flow problem seeks to maximum $\delta$
  such that
  \begin{align}
    \label{eq:8}
    \sum_{j: (i, j) \in E}^{} x_{ij} - \sum_{j: (j, i) \in E}^{}
    x_{ji} =
    \begin{cases}
      \delta & i = 1 \\
      -\delta & i = n \\
      0 & \text{otherwise}
    \end{cases}
  \end{align} and $0 \leq x_{ij} \leq C_{ij}$ for all $(i, j) \in E$.
\end{defn}

\begin{defn}
  The capacity of a cut $S \subseteq V$ is given by
  \begin{align}
    \label{eq:10}
    C(S) = \sum_{(i, j) \in E \cap (S \times (V \backslash S))}^{} C_{ij}.
  \end{align}
\end{defn}

\begin{thm}
  \label{sec:maxim-flows-perf-2}
  Let $\delta$ be the optimal solution of
  \ref{sec:maxim-flows-perf-1} for a network $(V, E)$ with capacities
  $C_{ij}$.  Then
  \begin{align}
    \label{eq:9}
    \delta = \min \{ C(S) | S \subseteq V, 1 \in S, n \in V \backslash
S \} 
  \end{align}
\end{thm}

\begin{thm}
  \label{sec:maxim-flows-perf-3}
  The \textbf{Ford-Fulkerson algorithm} proceeds to find a maximum
  flow by pushing along an augmenting path, until such a path cannot
  be found.

  \begin{enumerate}
  \item Start with a feasible flow $x$.
  \item\label{item:8} If there is no augmenting path from $1$ to $n$, stop.
  \item\label{item:6} Else, pick some augmenting path from $1$ to $n$, push a
    maximum amount of flow along this path without violating any
    constraints.  Then go to step \ref{item:8}.
  \end{enumerate}
\end{thm}

\begin{defn}
  \label{sec:maxim-flows-perf-5}
  Consider an alternative formulation of maximum flow as a minimum
  cost flow, $\min -x_{n1}$ subject to $\sum_{j:(i, j) \in E'}^{}
  x_{ij} - \sum_{j: (j, i) \in E'}^{} x_{ji} = 0$ for all $i \in V$,
  $0 \leq x_{ij} \leq C_{ij}$ for all $(i, j) \in E$, $x_{n1} \geq 0$,
  where $E' = E \cup \{ (n, 1) \}$.  The Lagrangian is
  \begin{equation}
    \label{eq:12}
    L(x, \lambda) = (-1 - \lambda_{n} + \lambda_{1}) x_{n1} -
    \sum_{(i, j) \in E}^{} (\lambda_{i} - \lambda_{j}) x_{ij}.
  \end{equation} with bounded minimum when $x_{n1} > 0$ only if
  $\lambda_{1} - \lambda_{n} = 1$.  

  \todo{Follow on with this explanation?}
\end{defn}

\begin{defn}
  \label{sec:maxim-flows-perf-6}
  A matching of a graph is a set of edges that do not share any
  vertices.  A matching is perfect if it covers every vertex - $|M| =
  \frac{|V|}{2}$.  A graph is $k$-regular if every vertex has degree $k$.
\end{defn}

\begin{thm}
  \label{sec:maxim-flows-perf-7}
  A bipartite graph $G = (L \biguplus R, E)$ with $|L| = |R|$ has a
  perfect matching if and only if $|N(X)| \geq |X|$ for every $X
  \subseteq L$, where $N(X) = \{ j \in R | i \in X, (i, j) \in E \}$.
\end{thm}

\section{Shortest Paths and Minimum Spanning Trees}
\label{sec:short-paths-minim}

\begin{thm}
  \label{sec:short-paths-minim-1}
  Let $\lambda_{i}(k)$ be the length of a shortest path from $i$ to
  $t$that uses at most $k$ edges.  Then $\lambda_{t}(k) = 0$ for all
  $k \geq 0$, and $\lambda_{i}(0) = \infty$, $\lambda_{i}(k) =
  \min_{j: (i, j) \in E}(c_{ij} + \lambda_{j}(k-1))$ for all $i \in V
  \backslash \{ t \}$ and $k \geq 1$.  This is the
  \textbf{Bellman-Ford} algorithm.
\end{thm}

\begin{thm}
  \label{sec:short-paths-minim-2}
  Consider a graph with vertices $V$ and edge lengths $c_{ij} \geq 0$
  for all $i, j \in V$ . Fix $t \in V$ and let $\lambda_{i}$ denote
  the length of a shortest path from $i \in V$ to $t$.  Let $j \in V
  \backslash \{ t \} $such that $c_{jt} = \min_{i \in V \backslash \{
    t \}} c_{it}$. Then $\lambda_{j} = c_{jt}$ and $\lambda_{j} = \min
  c_{jt} = \min_{i \in V \backslash \{ t \}} \lambda_{i}$.
\end{thm}

\begin{thm}
  \label{sec:short-paths-minim-3}
  Let $(V, E)$ be a graph with edge costs $c_{ij}$ for all $(i, j) \in
  E$. Let $U \subseteq V$ and $(u, v) \in U \times (V \backslash U)$
  such that $c_{uv} = \min_{(i, j) \in U \times (V \backslash U)}
  c_{ij}$. Then there exists a spanning tree of minimum cost that
  contains $(u, v)$.
\end{thm}

\section{Semidefinite Programming}
\label{sec:semid-progr}

\begin{defn}
  \label{sec:semid-progr-1}
  Let $\IP{C, X} = \tr CX = \sum_{i=1}^{n} \sum_{j=1}^{n} c_{ij}
  x_{ij}$ for some $C, X \in \S^{n}$, the set of symmetric matrices.
  A semidefinite program takes the form $\min \IP{C, X}$ subject to
  $\IP{A_{i}, X} = b_{i}$ for all $i \in \{ 1, \dots, m \} $, and $X
  \succeq 0$ (positive semidefinite) where $C, A_{1}, \dots, A_{m} \in
  \S^{n}$ and $b \in \R^{m}$.
  
Equivalently, $\min c^{T} x$ subject to $B_{0} + \sum_{i=1}^{k} x_{i}
B_{i} \succeq 0$, where $B_{i} \in S^{n}$ and $c \in \R^{k}$.
\end{defn}

\begin{thm}
  \label{sec:semid-progr-2}
  The Lagrangian of \ref{sec:semid-progr-1} can be written as
  \begin{equation}
    \label{eq:13}
    L(X, \lambda, Z) = \IP{C, X} - \sum_{i=1}^{m} \lambda_{i}
    (\IP{A_{i}, X} - b_{i}) - \IP{Z, X},
  \end{equation} where the last term takes into account the constraint
  $X \succeq 0$. Then $g(\lambda, Z) = \inf_{X \in S^{n}} L(X,
  \lambda, Z) = \lambda^{T} b$ if $C - \sum_{i=1}^{m} \lambda_{i}
  A_{i} - Z = 0$ , and $-\infty$ otherwise.  Eliminating $Z$, we
  obtain the dual of \ref{sec:semid-progr-1} (another semidefinite
  program), $\max \lambda^{T} b$ subject to $C - \sum_{i=1}^{m}
  \lambda_{i} A_{i} \succeq 0$.
\end{thm}

\begin{thm}
  \label{sec:semid-progr-3}
  Consider the primal/dual linear programs $\min \{ c^{T} x | Ax = b,
  x \geq 0 \} $ and $\max \{ b^{T} \lambda | A^{T} \lambda +z = c, z
  \geq 0 \} $  If we use \textbf{primal-dual interior point methods},
  we can augment the objective with a barrier function and solve with
  Newton's method.  We augment with $\min c^{T} x - \mu \sum_{i=1}^{n}
  \log x_{i} | Ax = b$ and $\max \{ b^{T} \lambda + \mu \sum_{j=1}^{m}
  \log z_{j} | A^{T} \lambda + z = c \} $ for a parameter $\mu > 0$.

  Then $(x, \lambda, z)$ is optimal for the modified primal/dual
  problems if $Ax = b, x \geq 0$, $A^{T} \lambda + z = c$, $z \geq 0$,
  $x_{i} z_{i} = \mu$, for all $i = 1, \dots, n$.
\end{thm}

\section{Branch and Bound}
\label{sec:branch-bound}

\begin{defn}
  \label{sec:branch-bound-1}
  Assume we wish to solve $\min f(x)$ s.t. $x \in X$ for some feasible
  region $X$., we use divide and conquer on $X_{i}$ with
  $\cup_{i=1}^{k} X_{i} = X$, and solve $\min_{x \in X} f(x) =
  \min_{i=1}^{k} \min_{x \in X_{i}} f(x)$.

  If we have a lower and upper bound on the optimal solution -
  functions $l$ and $u$ such that for all $X' \subseteq X$, $l(X')
  \leq \min_{x \in X'} f(x) \leq u(X')$ , then we can efficiently
  prune our search space.

  The algorithm proceeds as follows:
  \begin{enumerate}
  \item Set $U = \infty, L = \{ X \} $.
  \item\label{item:1} Pick $Y \in L$, remove from $L$, and split into $k \geq 2$
    sets $Y_{1}, \dots, Y_{k}$.
  \item For $i \in \{ 1, \dots, k \} $, compute $l(Y_{i})$.  If this
    yields $x \in X$ with $l(Y_{i}) = f(x) < U$ , set $U$ to $f(x)$.
    If $l(Y_{i}) < U$ but no $x \in X$ is found, add $y_{i}$ to $L$.
    If $L = \emptyset$, stop, the optimum value is $U$.  Otherwise, go
    back to \ref{item:!}.
  \end{enumerate}
\end{defn}

\begin{thm}
  \label{sec:branch-bound-2}
  Applied to an integer program, we can obtain a lower bound by
  solving the LP relaxation, and so eliminate large chunks of the
  search tree.
\end{thm}

\begin{defn}
  \label{sec:branch-bound-3}
  The traveling salesman problem is the problem of finding a cycle in
  $G$ that visits each vertex exactly once, of minimum overall cost.

  TO proceed, we can encode it as an integer program by introducing
  $x_{ij} \in \{ 0, 1 \} $, whether the tour traverses edge $i$, and
  variables $t_{i} \in \{ 0, \dots, n-1 \} $ indicating the position
  of the vertex $i$ in the tour. If $x_{ij} = 1$, then $t_{j} = t_{i}
  + 1$. If $x_{ij} = 0$, then $t_{j} \geq t_{i} - (n-1)$. We need to
  constrain that there is exactly one edge entering and one edge
  leaving - $\sum_{i=1}^{n} x_{ij} = 1$ for $j = 1, \dots, n$, and
  $\sum_{j=1}^{n} x_{ij} = 1$ for $i = 1, \dots, n$.
\end{defn}

\section{Heuristic Algorithms}
\label{sec:heuristic-algorithms}

\begin{defn}
  \label{sec:heuristic-algorithms-1}
  Assume we want solve the problem $\min c(x)$ s.t. $x \in X$, and
  that for any feasible solution $x \in X$, the cost $c(x)$ and a
  \textbf{neighborhood} $N(x) \subseteq X$ can be computed
  efficiently.  Then \textbf{local search} proceeds as follows:
  \begin{enumerate}
  \item Find an initial feasible solution $x \in X$.
  \item\label{item:7} Find a solution $y \in N(x)$ such that $c(y) < c(x)$.
  \item If there is no solution - stop and return $x$, otherwise set
    the current solution $x$ to $y$ and return to \ref{item:7}.
  \end{enumerate}
\end{defn}

\begin{defn}
  \label{sec:heuristic-algorithms-2}
  Simulated annealing considers a neighbor $y$ of the current
  solution $x$ and moves to the new solution with probability $p_{xy}
  = \min(1, \exp(-\frac{c(y) - c(x)}{T}))$ where $T \geq 0$ is a
  parameter (the \textbf{temperature}) that can vary over the time.

  It can be shown that with detailed balance, we have
  \begin{align}
    \label{eq:14}
    \pi_{x} = \frac{e^{-\frac{c(x)}{T}}}{\sum_{z \in X}^{} e^{-\frac{c(z)}{T}}} 
  \end{align} for every $x \in X$ is a distribution and satisfies
  detailed balance, and so must be the stationary distribution.
  We have $\frac{\pi_{Y}}{1 - \pi_{Y}} \rightarrow \infty$ as $T
  \rightarrow 0$, where $Y \subseteq X$ is the set of solutions with
  the minimum cost. We then decrease $T$ for the Markov chain to reach
  the stationary distribution -e.g. $T = \frac{c}{\log t}$.
\end{defn}

\section{Approximation Algorithms}
\label{sec:appr-algor}

\begin{defn}
  A function $g: \{ 0. 1 \}^{\star} \rightarrow \{ 0, 1 \}^{\star}$ is
  an $\alpha$ approximation for $o$ if for some $\alpha \geq 1$, if
  for all $x \in P$, $o(x, g(x)) \leq \alpha o(x, f(x))$. In what
  follows we will be interested in algorithms that compute the
  function $g$ in polynomial time, and refer to such an algorithm as a
  polynomial-time $\alpha$-approximation
  algorithm.\label{sec:appr-algor-1}

  The class $APX$ are the class of problems where the exists an
  $\alpha$ approximation for some $\alpha$.  The class $PTAS \subseteq
  APX$ are problems that possess an $(1 + \epsilon)$ approximation
  algorithm for any $\epsilon > 0$.
\end{defn}

\begin{thm}
  \label{sec:appr-algor-2}
  The max-cut problem asks for a cut on an undirected graph that
  maximizes the number of edges crossing from one side to the other.
  There exists a $\frac{1}{2}$ approximation with a simple greedy
  algorithm.

  Use a universal hash function to obtain $n^{2}$ pairwise-independent
  samples, we can apply this to randomly cut a variable with
  probability $\frac{1}{2}$, and so obtain a cut with cost $\E{Q} =
  \E{\sum_{(i, j) \in E}^{} \Prob{Q_{i} \neq Q_{j}}} = \sum_{(i, j)
    \in E}^{} \E{\Prob{Q_{i} \neq Q_{j}}} = \frac{|E|}{2}$.
\end{thm}

\section{Non-Cooperative Games}
\label{sec:non-coop-games}

\begin{defn}
  \label{sec:non-coop-games-1}
  A normal-form game is a tuple $\Gamma = (N, (A_{i})_{i \in N},
  (p_{i})_{i \in N})$, where $N$ is a finite set of players, and
  $A_{i}$ is a non-empty  and finite set of actions available to $i$,
  and $p_{i}: (\times_{i \in N} A_{i}) \rightarrow \R$ is a function
  mapping each combination of actions to a payoff for $i$.

  A two-player game can be represented by $P, Q \in \R^{m \times n}$,
  where $p_{ij}, q_{ij}$ are the payoffs of players $1, 2$ when player
  $1$ plays action $i$ and player $2$ plays action $j$.

  The set of possible \textbf{strategies} of the two players are given
  as $X = \{ x \in \R^{m}_{\geq 0} | \sum_{i=1}^{m} x_{i} = 1 \} $, $Y
  = \{ y \in \R^{n}_{\geq 0} | \sum_{i=1}^{n} y_{i} = 1 \}$. A
  \textbf{pure strategy} is a strategy that chooses an action with
  probability one. The expected payoffs from playing $(x, y) \in (X,
  Y)$ are given as $p(x, y) = x^{T} P y, q(x, y) = x^{T} Q y$.

  For two strategies $x, x' \in X$, $x$ is said to dominate $x'$ if
  for every strategy $y \in Y$ of the column player, $p(x, y) > p(x',
  y)$.

  A strategy that maximizes the payoff in the worst case, taken over
  all the other players strategies, is a \textbf{maximin strategy}, and
  the payoff is the player's \textbf{security level}.  It is
  sufficient to maximize the minimum payoff over all pure strategies
  of the other player (convex combinations), so choosing $x$ such that
  $\min_{j \in 1, \dots, n} \sum_{i=1}^{m} x_{i} p_{ij}$ is as large
  as possible.  Thus, this is a linear program $\max v$ subject to
  $\sum_{i=1}^{m} x_{i} p_{ij} \geq v$ for $j = 1, \dots, n$,
  $\sum_{i=1}^{m} x_{i} = 1$, $x \geq 0$.

  Strategy $x \in X$ of the row player is a \textbf{best response} to
  strategy $y \in Y$ of the column player if for all $x' \in X$, $p(x,
  y) \geq p(x', y)$.
\end{defn}

\begin{thm}
  \label{sec:non-coop-games-3}
  For the zero-sum game, where $q_{ij} = -p_{ij}$, we have \textbf{von
    Neumann's theorem}.

  $\max_{x \in X} \min_{y \in Y} p(x, y) = \min_{y \in Y} \max_{x \in
    X} p(x, y)$.

  This can be shown by adding a slack variable $z \in \R^{n}$ with $z
  \geq 0$ and obtaining the Lagrangian
  \begin{align}
    \label{eq:15}
    L(v, x, z, w, y) = v + \sum_{j=1}^{n} y_{j}(\sum_{i=1}^{m} x_{ij}
    - p_{ij} - z_{j} - v) - w(\sum_{i=1}^{m} x_{i} - 1) \\
    &= (1 - \sum_{j=1}^{n} y_{j}) v + \sum_{i=1}^{m} (\sum_{j=1}^{n}
    p_{ij} y_{j} - w)x_{i} - \sum_{j=1}^{n} y_{j} z_{j} + w
  \end{align} which has finite maximum  with $x \geq 0$ if and only if
  $\sum_{j=1}^{n} y_{j} = 1$, $\sum_{j=1}^{n} p_{ij} y_{j} \leq w$ for
  $ i = 1, \dots, m$, and $y \geq 0$. So the dual is $\min w$ such
  that $\\sum_{j=1}^{n} p_{ij} y_{j} \leq w$ for $i = 1, \dots, m$,
  $\sum_{j=1}^{n} y_{j} = 1$, and $y \geq 0$.  This has solution
  $\min_{y \in Y} \max_{x \in X} p(x, y)$ as required.  This solution
  is called the \textbf{value} of a matrix game with payoff matrix $P$.
\end{thm}


\section{Strategic Equilibrium}
\label{sec:strat-equil}

\begin{defn}
  \label{sec:strat-equil-1}
  A pair of strategies $(x, y) \in X \times Y$ with $x$ a best
  response to $y$ and $y$ a best response to $x$ is called an
  \textbf{equilibrium}.
\end{defn}

\begin{thm}
  \label{sec:strat-equil-2}
  A pair of strategies $(x, y) \in X \times Y$ is an equilibrium of
  the matrix game with payoff matrix $P$ if and only if $\min_{y \in
    Y} p(x, y') = \max_{x \in X} \min_{y' \in Y} p(x', y')$, and
  $\max_{x' \in X} p(x', y) = \min_{y' \in Y} \max_{x' \in X} p(x', y')$.
\end{thm}

\begin{thm}
  \label{sec:strat-equil-3}
  Let $(x, y), (x', y') \in X \times Y$ be equilibria of the matrix
  game with payoff matrix $P$.  Then $p(x, y) = p(x', y')$, and $(x,
  y')$ and $(x', y)$ are equilibria as well.
\end{thm}

\begin{thm}
  \label{sec:strat-equil-5}
  Let $f: S \rightarrow S$ be a continuous function, where $S
  \subseteq \R^{n}$ is closed, bounded, and convex. Then $f$ has a
  fixed point.
\end{thm}

\begin{thm}[Nash's Theorem]
  \label{sec:strat-equil-4}
  We show bimatrix game has an equilibrium.   

  To show this, define $X, Y$ as before, and $X \times Y$ is closed,
  bounded, and convex.  Then for $x \in X$, $y \in Y$, define
  $s_{i}(x, y)$ and $t_{i} (x, y)$ by the additionally payoff the two
  players could obtain by playing the $i$-th or $j$-th pure strategy
  instead of $x$ or $y$ - so $s_{i}(x, y) = \max \{ 0, p(e_{i}^{m}, y)
- p(x, y) \} $, and $t_{j}(x, y) = \max \{ 0, q(x, e_{j}^{n} - q(x,
y)) \}$, and define $f: X \times Y \rightarrow X \times Y$ by $f(x, y)
= (x', y')$, where
\begin{align}
  \label{eq:18}
  x_{i}' = \frac{x_{i} + s_{i}(x, y)}{1 = \sum_{k=1}^{m} s_{k}(x, y)}
\\
y'_{j} = \frac{y_{j} + t_{j}(x, y)}{1 + \sum_{k=1}^{n} t_{k}(x, y)}
\end{align}
Note also there must exist $i \in \{ 1, \dots, m \} $ with $x_{i} > 0$
and $s_{i}(x, y) = 0$, as otherwise $p(x, y) = \sum_{k=1}^{m} x_{k}
p(e_{k}^{m}, y) > \sum_{k=1}^{m} x_{k} p(x, y) = p(x, y)$.  Thus, and
as $(x, y)$ is a fixed point,
\begin{equation}
  \label{eq:19}
  x_{i} = \frac{x_{i} + s_{i}(x, y)}{1 + \sum_{k=1}^{m} s_{k}(x, y)}
\end{equation}
and so $\sum_{k=1}^{m} s_{k}(x, y) = 0$, so for $k = 1, \dots, m$,
$s_{k}(x, y) = 0$, and so $p(x, y) \geq p(e_{k}^{m}, y)$. So $p(x, y)
\geq p(x', y)$ for all $x' \in X$.  Analogously, $q(x, y) \geq q(x,
y')$ for all $y' \in Y$, so $(x, y)$ must be an equilibrium.
\end{thm}

\begin{thm}
  \label{sec:strat-equil-6}
  Given a bimatrix game, it is $NP$-complete to decide whether
  \begin{enumerate}
  \item it has at least two equilibria
  \item an equilibrium in which the expected payoff of the row player
    is at least a given amount,
  \item an equilibrium in which the expected sum of the payoff of the
    two players is at least a given amount
  \item an equilibrium with supports of a given minimum size,
  \item an equilibrium whose support includes a given pure strategy,
  \item or an equilibrium whose support does not include a given pure
    strategy.
  \end{enumerate}
\end{thm}

\section{Equilibrium Computation}
\label{sec:equil-comp}

\begin{defn}
  \label{sec:equil-comp-1}
  Consider a bimatrix game with payoffs $P, Q \in \R^{m \times n}$,
  and assume WLOG that $P, Q > 0$. Then $M = \{ 1, \dots, m \}$ and $N
  = m+1, \dots, m+n$, and define the sets $X, Y$ of strategies
  accordingly.
  
  A pair $(x, y) \in X \times Y$ is an equilibrium if and only if all
  pure strategies in $S(x)$ are best responses to $y$ and all pure
  strategies in $S(y)$ are best responses to $x$ - so if for all $i
  \in M$, $x_{i} > 0$implies $(Py)_{i} = \max_{k \in M}(Py)_{k}$, and
  for all $j \in N$, $y_{j} > 0$ implies $(Q^{T} x)_{j} = \max_{k \in
    N} (Q^{T}x)_{k}$.
\end{defn}

\todo{Fill in Lemke-Howson?}

\section{Cooperative Games}
\label{sec:cooperative-games}

\begin{defn}
  \label{sec:cooperative-games-1}
  A coalitional game is given by a set $N = \{ 1, \dots, n \} $ of
  players, a characteristic function $\nu: 2^{N} \rightarrow \R$ that
  maps each coalition to is value - the payoff the coalition can obtain
  by working together.  An \textbf{imputation} of a game $(N, \nu)$ is a
  vector $x \in \R^{n}$ such that $x_{i} \geq \nu(\{ i \} )$ for all $i
  \in N$, and $\sum_{i=1}^{n} x_{i} = \nu(N)$.  The first condition
  \textbf{individual rationality}, requires each player obtains the
  same payoff it would be able to obtain on it's own.  The second is
  \textbf{economic efficiency} - no payoff is wasted.
\end{defn} 


\begin{defn}
  \label{sec:cooperative-games-4}
  An imputation $x$ is in the core of game $(N, \nu)$ if $\sum_{i \in
    S}^{} x_{i} \geq v(S)$ for all $S \subseteq N$.
\end{defn}

\begin{defn}
  \label{sec:cooperative-games-2}
  A function $\lambda: 2^{N} \rightarrow [0, 1]$ is \textbf{balanced}
  if for every player the weights of all coalitions containing that
  player sum to $1$ - so, for all $i \in N$, $\sum_{S \subseteq N
    \backslash \{ i \} }^{} \lambda(S \cup \{ i \}) = 1$.  A game $(N,
  \nu)$ is balanced if for every balanced function $\lambda$, $\sum_{S
    \subseteq N}^{} \lambda(S) \nu(S) \leq \nu(N)$.
\end{defn}

\begin{thm}
  \label{sec:cooperative-games-3}
  A game has a non-empty core if and only if it is balanced.
\end{thm}



\begin{defn}
  \label{sec:cooperative-games-5}
  The \textbf{excess} $e(S, x)$ of coalition $S \subseteq N$ for
  imputation $x$ is the gain from leaving the grand coalition - $e(S,
  x) = \nu(S) - \sum_{i \in S}^{} x_{i}$.

  For a given imputation $x$, let $S_{1}^{x}, \dots, S_{2^{n}-1}^{x}$
  be an ordering of the coalitions such that $e(S^{x}_{k}, x) \geq
  e(S_{k+1}^{x}, x)$ for $k = 1, \dots, 2^{n} - 2$, and let $E(x) \in
  R^{2^{n} - 1}$ be the vector given by $E_{k}(x) = e(S_{k}^{x}, x)$.
  We say that $E(x)$ is lexicographically smaller than $E(y)$ if there
  exists $i \in \{ 1, \dots, 2^{n} - 1 \} $ such that $E_{k}(x) =
  E_{k}(y)$ for $k = 1, \dots, i=1$ and $E_{i} (x) < E_{i}(y)$. The
  nucleolus is then defined as the set of imputations $x$ for which
  $E(x)$ is lexicographically minimal.
\end{defn}

\begin{thm}
  \label{sec:cooperative-games-6}
  The nucleolus of any coalitional game is a singleton.
\end{thm}

\begin{defn}
  \label{sec:cooperative-games-7}
  Call player $i \in N$ a \textbf{dummy} if its contribution to every
  coalition is exactly its value - if $\nu(S \cup \{ i \} ) = \nu(S) +
  \nu(\{ i \} )$ for all $S \subseteq N \backslash \{ i \} $. Call two
  players $i, j$ interchangable if they contribute the same to every
  coalition - so $\nu(S \cup \{ i \} ) = \nu(S \cup \{ j \} )$ for all
  $S \subseteq N \backslash \{ i, j \} $. Let a \textbf{solution} be a
  function $\phi: \R^{2^{n}} \rightarrow \R^{n}$ that maps every
  characteristic function $\nu$ to an imputatoin $\phi(\nu)$.
  Solutoins $\phi$ is said to satisfy
  \begin{enumerate}
  \item \textbf{dummies} if $\phi_{i}(v) = v(\{ i \} )$ whenever $i$
    is a dummy,
  \item \textbf{symmetry} if $\phi_{i}(v) = \phi_{j}(v)$ whenever $i,
    j$ are interchangable, and \textbf{additivity} if $\phi(v + w) =
    \phi(v) + \phi(w)$.
  \end{enumerate}
\end{defn}
\begin{thm}
  \label{sec:cooperative-games-8}
  The \textbf{Shapley value}, given by
  \begin{equation}
    \label{eq:21}
    \phi_{i}(v) = \sum_{S \subseteq N \backslash \{ i \} }^{}
    \frac{|S|!(|N| - |S| - 1)!}{|N|!} (\nu(S \cup \{ i \} ) - \nu(S))
  \end{equation} is the unique solution that satisfies dummies,
  symmetry, and additivity.
\end{thm}

\section{Bargaining}
\label{sec:bargaining}

\begin{defn}
  \label{sec:bargaining-1}
  A two-player bargaining problem is a pair $F, d$ where $F \subseteq
  \R^{n}$ is a convex set of feasible outcomes, and $d \in F$ is a
  disagreement point that results if players fail to agree on an
  outcome.  A \textbf{bargaining solution} is a function that assings
  to every bargaining problem $(F, d)$ a unique element of $F$.

  A two-player normal-form game with payoff matrices $P, Q \in \R^{m
    \times n}$ can be interpreteed as a bargaining problem where $F =
  \con \{ (p_{ij}, q_{ij}) | i \in M, j \in N \} $, $d_{1} = \max_{x
    \in X} \min_{y \in Y} p(x, y)$, and $d_{2} = \max_{y \in Y}
  \min_{x \in X} q(x, y)$.
\end{defn}

\begin{thm}
  \label{sec:bargaining-2}
  For a given bargaining problem, Nash proposes $\max (v_{1} -
  d_{1})(v_{2} - d_{2})$ such that $v \in F$, $v \geq de$.
\end{thm}

\begin{defn}
  \label{sec:bargaining-3}
  A bargaining solution $f$ is
  \begin{enumerate}
  \item \textbf{Pareto efficient} if $f(F, d)$ is not Pareto dominated
    in $F$ for any bargaining problem $(F, d)$
  \item \textbf{symmetric} if $(f(F, d))_{1} = (f(F, d))_{2}$ for
    every bargaining problem $(F, d)$ such that $(y, x) \in F$
    whenever $(x, y) \in F$ and $d_{1} = d_{2}$.
  \item \textbf{Invariant under positive affine transforms} if $f(F, d) =
    \alpha \circ f(F, d) + \beta$ for $\alpha, \beta \in \R^{2}$ with
    $\alpha > 0$ and any two bargaining problems $(F, d)$ and $(F',
    d')$ with $F' = \alpha F + \beta$, $d' = \alpha d + \beta$,
  \item \textbf{independence or irrelevant alternatives} if $f(F, d) =
    f(F', d)$ for any two bargaiining problems $(F, d)$ and $(F', d)$
    such that $F' \subseteq F$ with $d \in F'$ and $f(F, d) \in F'$.
  \end{enumerate}
\end{defn}

\begin{thm}
  \label{sec:bargaining-4}
  Nash's bargaining solution is the unique bargaining solution that is
  Pareto efficient, symmetric, invariant under positive affine
  transformations, and independent or irrelevant alternatives.
\end{thm}

\section{Stable Matchings}
\label{sec:stable-matchings}

\begin{defn}
  \label{sec:stable-matchings-1}
  Consider a set $S$ of students and a set $A$ of potential advisors.
  Each studnet $i \in S$ has a strict linear order $\succgeq_{i}
  \subseteq A \times a$, each advisor $j \in A$ has a strict linear
  order $\succgeq_{j} \subseteq S \times S$.  A matching is a function
  $\mu: S \cup A \rightarrow S \cup A$ such that $\mu(\mu(i)) = i$,
  $\mu(i) \in A$ for all $i \in S$, and $\mu(j) \in S$ for all $j \in
  A$.  

  A pair $(i, j) \in S \times A$  is a \textbf{blocking pair} for
  matching $\mu$ if $i, j$ would rather be matched to each other than
  respective partners in $\mu$ - so $j \succgeq_{i} \mu(i)$ and $i
  \succgeq_{j} \mu(j)$.  A matching is called \textbf{stable} if it
  doesn ot have any blocking pairs.
\end{defn}

\begin{thm}
  \label{sec:stable-matchings-2}
  Consider the \textbf{deferred acceptance procedure},
  \begin{enumerate}
  \item Let each student $i \in S$ propse to the advisor it ranks highest.
  \item\label{item:9} Match advisor $j \in A$ tentatively to the highest-ranked
    amongst the students that have propsed to it, if any, and reject
    the others for good.
  \item Let each studnet that has been rejected but has not been
    rejected by all advisors propse to the next advisor. IF there are
    no such students, stop and return the matched pairs.  Otherwise,
    return to \ref{item:9}.
  \end{enumerate}

  This procedure always terminates, and yields a stable matching when
  it does.
\end{thm}

\begin{thm}
  \label{sec:stable-matchings-3}
  Call $j \in A$ achievable for $i \in S$ if there exists a stable
  matching $\mu$ such that $\mu(i) = j$.  A stable matching $\mu$
  is called  \textbf{student-optimal} if for all $i \in S$, $\mu(i)$
  is most preferred among the advisors achievalbe for $i$.

  Students-propse deferred acceptance yields a student-optimal stable matching.
\end{thm}

\begin{thm}
  \label{sec:stable-matchings-4}
  A matching $\mu$ is advisor pessimal if for all $j \in A$, $\mu(j)$
  is least preferred among the students achievable for $j$.
  
  Every student-optimal stable matching is advisor-pessimal.
\end{thm}

\begin{thm}
  \label{sec:stable-matchings-5}
  Fix a stable matching $\mu$ and let $x_{ij} = 1$ for $i \in S$ and
  $j \in A$ if $\mu(i) = j$, and $x_{ij} = 0$ otherwise.  Then the
  constraints $\sum_{j \in A}^{} x_{ij} = 1$ for all $i \in S$,
  $\sum_{i \in S}^{} x_{ij} = 1$ for all $j \in A$, $x_{ij} + \sum_{k:
  j \succgeq_{i} k}^{} x_{ik} + \sum_{k: i \succgeq_{j} k}^{} x_{kj}
\leq 1$ for all $i \in S$, $j \in A$, $x_{ij} \geq 0$ for all $i \in
S$, $j \in A$.

Then $P$ is the polytope descriebed by the above constraints, and $P
\neq \emptyset$.  Moreover, a vector is an extreme point of $P$ if and
only if it is a stable matching.
\end{thm}




\bibliographystyle{plainnat}
\bibliography{../../common/bibliography}
\end{document}
