
\chapter{Linear Programming}
\label{cha:linear-programming}

\section{Convexity and Strong Duality}
\label{sec:conv-strong-dual-1}


\begin{defn}
  \label{defn:linear_programming:1}
  Let $S \subseteq R^{n}$.  $S$ is convex if for all $\delta \in [0,
  1]$, $x, y \in S$ implies that $\delta x + (1 - \delta) y \in S$.

  $f: S \rightarrow R$ is convex if for all $x, y \in S$ and $\delta
  \in [0, 1]$, $\delta f(x) + (1-\delta) f(y) \geq f(\delta x + (1 -
  \delta)y$.

  Visually, the area under the function is a convex set.
\end{defn}

\begin{defn}
  \label{defn:linear_programming:2}
  $x \in S$ is an interior point of $S$ if there exists $\epsilon > 0$
  such $\{ y: \| y - x \|_{2} \leq \epsilon \} \subseteq S$.
  
  $x \in S$ is an extreme point of $S$ if for all $y, z \in S$ and $S
  \in (0, 1)$, $x = \delta y + (1 - \delta) z$ implies $x = y = z$.
\end{defn}

\begin{thm}[Supporting Hyperplane]
  \label{thm:linear_programming:supporting-hyperplane}
  Suppose that our function $\phi$ is convex and $b$ lies in the
  interior of the set of points where $\phi$ is finite.
  Then there is a (non-vertical) supporting hyperplane to $\phi$ at
  $b$.
\end{thm}

\begin{thm}
  Let $X(b) = \{ x \in X : h(x) \leq b \}$, $\phi(b) = \inf_{x \in
    X(b)} f(x)$.  Then $\phi$ is convex if $X$, $f$, and $h$ are convex.
\end{thm}

\begin{proof}
  Let $b_{1}, b_{2} \in R^{m}$ such that $\phi(b_{1})$, $\phi(b_{2})$
  are defined.  Let $\delta \in [0, 1]$ and $b = \delta b_{1} + (1 -
  \delta) b_{2}$.  Consider $x_{1} \in X(b_{1})$, $x_{2} \in X(b_{2})$
  and let $x = \delta x_{1} + (1 - \delta) x_{2}$.

  By convexity of $Y$, $x \in X$.  By convexity of $h$,
  \begin{align*}
    h(x)    & = h(\delta x_{1} + (1-\delta) x_{2})         \\
            & \leq \delta h(x_{1}) + (1-\delta) h(x_{2})   \\
            & \leq \delta b_{1} + (1 - \delta) b_{2} = b
  \end{align*}
  Thus $x \in X(b)$, and by convexity of $f$,
  \begin{align*}
    \phi(b) & \leq f(x)                                    \\
            & = f(\delta x_{1} + (1 - \delta) x_{2})       \\
            & \leq \delta f(x_{1}) + (1 - \delta) f(x_{2}) \\
            & \leq \delta \phi(b_{1}) + (1 - \delta) \phi(b_{2})
  \end{align*}
\end{proof}

\section{Linear Programs}
\label{sec:linear-programs}

\begin{defn}
  \label{defn:linear_programming:3}
  General form of a linear program is
  \begin{equation}
    \label{exmp:linear_programming:1}
    \min\{c^{T} x : Ax \geq b, x \geq 0 \}
  \end{equation}

  Standard form of a linear program is
  \begin{equation}
    \label{exmp:linear_programming:3}
    \min\{c^{T}x : Ax = b, x \geq 0 \}
  \end{equation}
\end{defn}

\section{Linear Program Duality}
\label{sec:line-progr-dual}

Introduce slack variables to turn problem $\Prob$ into the form
\begin{equation}
  \label{eq:11}
  \min \{ c^{T} x : Ax - z = b, x, z \geq 0 \}
\end{equation}

We have $X = \{ (x, z) : x, z \geq 0 \} \subseteq R^{m + n}$.  
The Lagrangian is
\begin{align}
  L((x, z), \lambda) & = c^{T} x - \lambda^{T}(Ax - z - b) \\
                     & = (c^{T} - \lambda^{T}A)x + \lambda^{T} z + \lambda^{T} b
\end{align}

and has a finite minimum if and only if
\begin{equation}
  \label{eq:13}
  \lambda \in Y = \{ \lambda : c^{T} - \lambda^{T} A \geq 0, \lambda
  \geq 0 \}
\end{equation}

For a fixed  $\lambda \in Y$, the minimum of $L$ is satisfied when
$(c^{T} - \lambda^{T} A) x = 0$ and $\lambda^{T} z = 0$, and thus
\begin{equation}
  \label{eq:14}
  g(\lambda) = \inf_{(x, z) \in X} L((x, z), \lambda) = \lambda^{T} b
\end{equation}

We obtain that the dual problem
\begin{equation}
  \label{eq:15}
  \max \{ \lambda^{T} b : A^{T} \lambda \leq c, \lambda \geq 0 \}
\end{equation}

and it can be shown (exercise) that the dual of the dual of a linear
program is the original program.


\section{Complementary Slackness}
\label{sec:compl-slackn}

\begin{thm}
  \label{defn:linear_programming:4}
  Let $x$ and $\lambda$ be feasible for $\Prob$ and its dual.  Then
  $x$ and $\lambda$ are optimal if and only if
  \begin{align}
    \label{eq:16}
    (c^{T} - \lambda^{T}A)x & = 0 \\
    \lambda^{T}(Ax - b)     & = 0
  \end{align}
\end{thm}

%%% Local Variables: 
%%% mode: latex
%%% TeX-master: "master"
%%% End: 
