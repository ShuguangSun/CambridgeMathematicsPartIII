
\chapter{Compexity of Problems and Algorithms}
\label{cha:comp-probl-algor}

\section{Asymptotic Complexity}
\label{sec:asympt-compl}

We measure complexity as a function of input size.  The input of a
linear programming problem: $c \in R^{n}, A \in R^{m \times n}, b \in
\R^{m}$ is represented in $(n + m \cdot n + m) \cdot k$ bits if we
represent each number using $k$ bits.

For two functions $f: \R \rightarrow \R$ and $g: \R \rightarrow \R$
write
\begin{equation}
  \label{eq:27}
  f(n) = \mathcal{O}(g(n))
\end{equation} if there exists $c, n_{0}$ such that for all $n \geq n_{0}$,
\begin{equation}
  \label{eq:27}
  f(n) \leq c \cdot g(n)
\end{equation}, ... (similarly for $\Omega \rightarrorw \geq$, and
$\Theta \rightarrow (\Omega + \mathcal{O})$)


%%% Local Variables: 
%%% mode: latex
%%% TeX-master: "master"
%%% End: 
