
\chapter{Constrained Optimization}
\label{cha:constr-optim}

Minimize $f(x)$ subject to $h(x) = b, x \in X$.  

Objective function $f: R^{n} \rightarrow R$
Vector $x \in R^{n}$ of decision variables,
Functional constraint where $h: R^{n} -> R^{m}, b \in R^{m}$
Regional constraint where $X \subseteq R^{n}$.

\begin{defn}
  \label{defn:1}
  The feasible set is $X(b) = \{ x \in X : h(x) = b \}$.
\end{defn}

An inequality of the form $g(x) \leq b$ can be written as $g(x) + z =
b$, where $z \in R^{m}$ called a slack variable with regional
constraint $z \geq 0$.

\section{Lagrangian Multipliers}
\label{sec:lagr-mult}

\begin{defn}
  \label{defn:2}
  Define the Lagrangian of a problem  as
  \begin{equation}
    \label{eq:1}
    L(x, \lambda) = f(x) - \lambda^{T}(h(x) - b)
  \end{equation} where $\lambda \in R^{m}$ is a vector of
  \textbf{Lagrange multipliers}
\end{defn}

\begin{thm}[Lagrangian Sufficiency Theorem]
  Let $x \in X$ and $\lambda \in R^{m}$ such that
  \begin{equation}
    \label{eq:2}
    L(x, \lambda) = \inf_{x' \in X} L(x', \lambda)
  \end{equation} and $h(x) = b$.  Then $x$ is optimal for $P$.
\end{thm}

\begin{proof}
  \begin{align}
    \label{eq:3}
    \min_{x' \in X(b)} f(x') &= \min_{x' \in X(b)} [f(x) - \lambda^{T}(h(x') - b)] \\
                             &\geq \min_{x' \in X} [ f(x') - \lambda^{T}(h(x') - b)] \\
                             &= f(x) - \lambda^{T}(h(x) - b) \\
                             &= f(x)
  \end{align}
\end{proof}

\section{Lagrange Dual}
\label{sec:lagrange-dual}

\begin{defn}
  \label{defn:3}

  Let
  \begin{equation}
    \label{eq:6}
    \phi(b) = \inf_{x \in X(b)} f(x).
  \end{equation}

  Define the Lagrange dual function $g: R^{m} \rightarrow R$ with
  \begin{equation}
    \label{eq:4}
    g(\lambda) = inf_{x \in X} L(x, \lambda)
  \end{equation}
  
  Then, for all $\lambda \in R^{m}$,
  \begin{equation}
    \label{eq:5}
    \inf_{x \in X(b)} f(x) = \inf_{x \in X(b)} L(x, \lambda) \geq
    \inf_{x \in X} L(x, \lambda) = g(\lambda)
  \end{equation} That is, $g(\lambda)$ is a lower bound on our
  optimization function.
\end{defn}

This motivates the \textbf{dual problem} to maximize $g(\lambda)$ subject to
$\lambda \in Y$, where $Y = \{ \lambda \in R^{m} : g(\lambda) >
-\infty \}$.

\begin{thm}[Duality]
  From \eqref{eq:5}, we see that the optimal value of the primal is
  always greater than the optimal value of the dual.  This is
  \textbf{weak duality}. 
\end{thm}

\section{Supporting Hyperplanes}
\label{sec:supp-hyperpl}


Fix $b \in R^{m}$ and consider $\phi$ as a function of $c \in R^{m}$.
Further consider the hyperplane given by $\alpha: R^{m} \rightarrow R$
with
\begin{equation}
  \label{eq:7}
  \alpha(c) = \beta - \lambda^{T}(b - c)
\end{equation}

Now, try to find $\phi(b)$ as follow.
\begin{enumerate}
\item For each $\lambda$, find
  \begin{equation}
    \label{eq:8}
    \beta_{\lambda} = \max \{ \beta: \alpha(c) \leq \phi(c), \forall c
    \in R^{m} \}
  \end{equation}
\item Choose $\lambda$ to maximize $\beta_{\lambda}$

\end{enumerate}

\begin{defn}
  \label{defn:4}
  Call $\alpha: R^{m} \rightarrow R$ a \textbf{supporting hyperplane}
  to $\phi$ at $b$ if
  \begin{equation}
    \label{eq:9}
    \alpha(c) = \phi(b) - \lambda^{T}(b - c)
  \end{equation} and 
  \begin{equation}
    \label{eq:10}
    \phi(c) \geq \phi(b) - \lambda^{T}(b - c)
  \end{equation} for all $c \in R^{m}$.
\end{defn}

\begin{thm}
  The following are equivalent
  \begin{enumerate}
  \item There exists a (non-vertical) supporting hyperplane to $\phi$
    at $b$,
  \item XXXXXXXXXXXXXXXXXXXXXXXXXX
  \end{enumerate}
\end{thm}

%%% Local Variables: 
%%% mode: latex
%%% TeX-master: "master"
%%% End: 
