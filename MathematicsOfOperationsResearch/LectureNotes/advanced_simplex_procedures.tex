
\chapter{Advanced Simplex Procedures}
\label{cha:advanc-simpl-proc}

\section{The Two-Phase Simplex Method}
\label{sec:two-phase-simplex}

Finding an initial \bfs is easy if the constraints have the form $Ax =
b$ where $b \geq 0$, as
\begin{equation}
  \label{eq:22}
  Ax + z = b, (x, z) = (0, b)
\end{equation}

\section{Gomory's Cutting Plane Method}
\label{sec:gomorys-cutt-plane}

Used in integer programming (IP).  This is a linear program where in
addition some of the variables are required to be integral.

Assume that for a given integer program we have found an optimal
fractional solution $x^{\star}$ with basis $B$ and let $a_{ij} =
(A_{B}^{-1}A_{j})$ and $a_{i0} = (A_{B}^{-1}b)$ be the entries of the
final tableau.  If $x^{\star}$ is not integral, the for some row $i$,
$a_{i0}$ is not integral.  For every feasible solution $x$,
\begin{equation}
  \label{eq:25}
  x_{i} = \sum_{j \in \N} \lfloor a_{ij} \rfloor x_{j} \leq x_{i} +
  \sum_{j \in \N} a_{ij} x_{j} = a_{i0}.
\end{equation}  If $x$ is integral, then the left hand side is
integral as well, and the inequality must still hold if the right hand
side is rounded down.

Thus,
\begin{equation}
  \label{eq:26}
  x_{i} + \sum_{j \in \N} \lfloor a_{ij} \rfloor x_{j} \leq \lfloor a_{i0} \rfloor.
\end{equation}

Then, we can add this constraint to the problem and solve the
augmented program. One can show that this procedure converges in a
finite number of steps.

%%% Local Variables: 
%%% mode: latex
%%% TeX-master: "master"
%%% End: 
