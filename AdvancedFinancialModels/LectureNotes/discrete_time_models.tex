\chapter{Discrete Time Models}
\label{cha:discrete-time-models}

\section{Standing Assumptions}
\label{sec:standing-assumptions}

\begin{enumerate}
\item Zero dividends
\item Zero tick size
\item Zero transaction costs
\item Infinitely divisible transactions
\item No short-selling constraints
\item No bid-ask spread
\item No market impact (infinitely deep market)
\end{enumerate}

\section{Setup}
\label{sec:setup}

Consider a probability space $(\Omega, \mathcal{F}, \mathbb{P})$.

\begin{defn}
  \label{defn:1}
  A random variable is a measurable map $X: \Omega \rightarrow \mathbb{R}$
\end{defn}

\begin{defn}
  \label{defn:2}
  A stochastic process $Y = (Y_{t})_{t \in I}$ is a collection of
  random variables.  For us, $I = \{ 0, 1, \dots \}$ or $[0, \infty)$. 
\end{defn}


\begin{defn}
  \label{defn:3}
  A filtration $\mathbb{F} = (\mathcal{F})_{t \geq 0}$ is a collection
  of sub-$\sigma$-algebras on $\mathcal{F}$ such that $\mathcal{F}_{s}
  \subseteq \mathcal{F}_{t}$ for all $0 \leq s \leq t$ (discrete and
  continuous time).
\end{defn}

\begin{exmp}
  Tossing coins.
  \begin{enumerate}
  \item $\Omega = \{ HH, HT, TH, TT \}$
  \item $\mathcal{F}$ is all 16 subsets of $\Omega$
  \item $\mathbb{P}(A) = \frac{|A|}{4}$
  \end{enumerate}

  Possible filtration
  \begin{enumerate}
  \item $\mathcal{F}_{0} = \{ \emptyset, \Omega \}$
  \item $\mathcal{F}_{1} = \{ \emptyset, \Omega, \{ HH, HT \}, \{ TH,
    TT \} \}$
  \item $\mathcal{F}_{2} = \mathcal{F}$
  \end{enumerate}
\end{exmp}

\begin{defn}
  \label{defn:4}
  A process $Y$ is adapted if and only if $Y_{t}$ is $\mathcal{F}_{t}$-measurable.
\end{defn}

Throughout the course, $\mathcal{F}_{0}$ is assumed trivial.

\begin{defn}
  \label{defn:discrete_time_models:1}
  Given a filtration $\mathbb{F} = (\mathcal{F}_{t})_{t \geq 0}$ in
  discrete time, a process $X = (X_{t})_{t \geq 1}$ is predictable if
  and only if $X_{t}$ is $\mathcal{F}_{t-1}$-measurable.

  Sometimes we need $X_{0}$ to be defined, so we just ask that $X_{0}$
  is $\mathcal{F}_{0}$-measurable.
\end{defn}

\begin{defn}
  \label{defn:discrete_time_models:2}
  Given $P = (P_{t})_{t \geq 0}$ prices process in discrete time.  An
  investment/consumption strategy is a predictable process $(H, c)$
  where $H_{t}$ takes values in $R^{n}$ and $c_{t} \geq 0$ and
  satisfies the \textbf{self-financing condition}
  \begin{equation}
    \label{eq:1}
    H_{t - 1} - P_{t - 1} = H_{t} \cdot P_{t} + c_{t}
  \end{equation} for all $t \geq 1$.
\end{defn}

$H_{t}$ models the portfolio during $(t - 1, t]$, and $c_{t}$ models
the consumption during $(t-1, t]$.

\begin{notation}
  $X_{t}(H) = H_{t} \cdot P_{t}$ is the wealth at time $t$.
  Note that given $H$, we can find $C$ by solving the self-financing
  condition.
\end{notation}

If $c_{t} = 0$ a.s. for all $t$ then $H$ is a pure investment
strategy. 
\begin{exmp}
  \label{defn:discrete_time_models:3}
  Given an initial wealth $x > 0$, find $(H, c)$ to maximize
  \begin{equation}
    \label{eq:2}
    \sum_{i=1}^{T} \E{U(c_{t}))}
  \end{equation} subject to $X_{T}(H) = 0$ where $T > 0$ is not
  random.

  Assume that $U$ is strictly increasing, strongly concave, and
  bounded from above.
\end{exmp}

\section{A Detour into Martingales}
\label{sec:deto-into-mart}


\begin{proposition}
  Let $X$ be integrable and $\mathcal{G} \subseteq \mathcal{F}$. Then
  there exists an integrable, $\mathcal{G}$-measurable random variable
  $\bar{X}$ such that
  \begin{equation}
    \label{eq:3}
    \E{X \I{G}} = \E{\bar{X} \I{G})}
  \end{equation} for all $G \in \mathcal{G}$.  Moreover, it is unique
  in the sense that if $\bar{\bar{X}}$ has the same property, then $\bar
  X = \bar{\bar{X}}$.
\end{proposition}

\begin{defn}
  \label{defn:discrete_time_models:4}
  Such $\bar X$ is written $\E{X | \mathcal{G}}$, the conditional
  expectation of $X$ given $\mathcal{G}$.
\end{defn}

Useful properties of conditional expectation:
\begin{enumerate}
\item If $X$ is $\mathcal{G}$-measurable, then $\E{X | \mathcal{G}} = X$.
\item If $X$ is independent of $\mathcal{G}$ (that is, $X$ and $\I{G}$
  are independent for all $G \in \mathcal{G}$), then $\E{X |
    \mathcal{G}} = \E{X}$.
\item (Tower property) If $\mathcal{H} \subseteq \mathcal{G}
  \subseteq \mathcal{F}$, then
  \begin{equation}
    \label{eq:4}
    \E{\E{X | \mathcal{G}} | \mathcal{H}} = \E{ \E{X | \mathcal{H}} |
      \mathcal{G}} = \E{X | \mathcal{H}}
  \end{equation}
\item (Slot property) If $Y$ is $\mathcal{G}$-measurable and $XY$ is
  integrable, then
  \begin{equation}
    \label{eq:5}
    \E{XY | \mathcal{G}} = Y \E{X | \mathcal{G}}
  \end{equation}
\end{enumerate}

\begin{defn}
  \label{defn:discrete_time_models:5}
  A martingale $(X_{t})_{t \geq 0}$ with respect to a filtration
  $\mathbb{F}$ has the properties
  \begin{itemize}
  \item $\E{|X_{t}|} < \infty$ for all $t$,
  \item $\E{X_{t} | \mathcal{F}_{s}} = X_{s}$ for all $0 \leq s \leq t$.
  \end{itemize}

  Note that $X$ is automatically adapted.
\end{defn}

\begin{exer}
  Suppose $X$ is an integrable discrete-time process such that
  $\E{X_{t}| \mathcal{F}_{t-1}} = X_{t-1}$ for all $t \geq 1$. Show
  that $X$ is a martingale.
\end{exer}

\begin{exmp}
  \label{defn:discrete_time_models:6}
  Let $\xi_{i}$, $i = 1, 2, ...$ be independent, integrable random
  variables with $\E{\xi_{i}} = 0$.  Let $\mathcal{F}_{t} =
  \sigma(\xi_{1}, \dots, \xi_{t}), X_{t} = \xi_{1} + \xi_{2} + \dots +
    \xi_{t}$.

    Then $X$ is a martingale.
\end{exmp}

\begin{exmp}
  \label{defn:discrete_time_models:7}
  Let $\xi$ be integrable and let $\mathbb{F}$ be a filtration, and
  $X_{t} = \E{\xi | \mathcal{F}_{t}}$
\end{exmp}

\begin{proof}
  Integrability comes from integrability of conditional expectations.

  \begin{align*}
    \E{X_{t} | \mathcal{F}_{s}} & = \E{\E{\xi | \mathcal{F}_{t}} |
      \mathcal{F}_{s}}                                        \\
                                & = \E{\xi | \mathcal{F}_{s}} \\ 
                                & = X_{s}
  \end{align*}
\end{proof}


\begin{exmp}
  \label{defn:discrete_time_models:8}
  Suppose $X$ is a discrete-time martingale and $Y$ is predictable and
  bounded.  Let $Z_{t} = \sum_{s=1}^{t} Y_{s} (X_{s} - X_{s-1})$. Then
  $Z$ is a martingale.
\end{exmp}

\begin{proof}
  Integrability checked by integrability of $X$ and boundedness of
  $Y$.

  $Z_{t-1}$ is $\mathcal{F}_{t-1}$ measurable since measurability
  respects algebraic operations.
  \begin{align*}
    \E{Z_{t} | \mathcal{F}_{t-1}} & = \E{Z_{t-1} + Y_{t}(X_{t} -
      X_{t-1}) | \mathcal{F}_{t-1}} \\
                                  & = Z_{t-1} + \underbrace{Y_{t}}_{\text{slot property}} \E{\underbrace{X_{t} - X_{t-1}}_{=0} | \mathcal{F}_{t-1}}
  \end{align*}
\end{proof}

\begin{thm}
  \label{defn:discrete_time_models:9}
  Suppose $u: [0, \infty) \rightarrow \R$ is strictly increasing,
  strictly concave, differentiable, bounded from above.  Suppose there
  exists investment strategy $H^{\star}$ and consumption
  $c^{\star}_{t}= (H^{\star}_{t-1} - H^{\star}_{t}) \cdot P_{t-1}$,
  and a state price density $Y^{\star}$ such that $u'(c^{\star}_{t}) =
  Y^{\star}_{t-1}$.  Then $(H^{\star}, c^{\star})$ is optimal for the
  problem $\max \sum_{t=1}^{T} \E{u(c_{t})}$, subject to $X_{0}(H) =
  x, X_{T}(H) = 0$.
\end{thm}

\begin{proof}
  We consider the case where $\Omega$ is finite.

  Let $L(H, c, Y) = \E{\sum_{t=1}^{T} \left( u(c_{t}) +
      Y_{t+1}(H_{t+1} P(t+1)
      - c_{t} - H_{t} \cdot P_{t-1}) \right)}$
  Note that $L(H, c, Y)$ is the objective when $(H, c)$ is feasible.
  Then
  \begin{multline}
    L(H, c, Y) = \E{\sum_{t=1}^{T} \left(u(c_{t}) - c_{t}
        Y_{t-1}\right)} +  Y_{0} X - Y_{t-1} H_{t} P_{t-1} \\
    + \sum_{t=1}^{T-1} H_{t}(Y_{t} P_{t} - Y_{t-1} P_{t-1})
  \end{multline}

  First note that $u(c^{\star}_{t}) - Y^{\star}_{t-1} c^{\star}_{t}
  \geq u(c_{t}) - Y^{\star}_{t-1} c_{t}$ since $u'(c^{\star}_{t}) =
  Y^{\star}_{t-1}$ (first order condition for the maximum of the
  concave function $c \mapsto u(c) - yc$).

  Second, by definition, $YP$ is a martingale, and by finiteness of
  $\Omega$, the predictable process $H$ is bounded. Therefore, $M_{t}
  = \sum_{s=1}^{t} H_{s}(Y_{s} P_{s} - Y_{s-1} P_{s-1})$ is a
  martingale and $E(M_{t}) = M_{s} = 0$.

  Putting this together, $L(H, c, Y^{\star}) \leq L(H^{\star},
  c^{\star}, Y^{\star})$.
\end{proof}

\begin{thm}
  \label{defn:discrete_time_models:10}
  An absolute arbitrage is an investment/consumption strategy $(H, c)$
  such that $X_{0}(H) = 0, X_T(H) = 0$, at some non-random time horizon
  $T > 0$, and $\Prob{\sum_{t=1}^T c_t > 0} > 0$.
\end{thm}

\begin{defn}
  \label{defn:discrete_time_models:11}
  A numeraire asset is one whose price is strictly positive almost surely.
\end{defn}

\begin{exmp}
  \label{defn:discrete_time_models:12}
  Here is a market without a numeraire. $P_{0} = 1, P_{0} = -1, P_{2}
  = 1$.

  Arbitrage:
  \begin{align*}
    H_{1} = -1, c_{1} = 1
    X_{1}= 1, c_{2} = 1, H_{2} = 0
    X_{2}= 0
  \end{align*}
\end{exmp}

\begin{exer}
  Suppose $H_{1}$ is an arbitrage and the market has a numeraire.
  Then there exists a pure investment strategy $H'$ and a time horizon $T'$ such that
  $X_{0}(H') = 0, X_{T'}(H') \geq 0$ a.s., and $\Prob{X_{T'}(H') > 0}
    > 0$.
\end{exer}

\begin{thm}
  \label{defn:discrete_time_models:13}
  A market model has no arbitrage if and only if there exists a state
  price density.
\end{thm}

\begin{proof}
  $T = 1$ case.  Suppose there exists a state price density
  $(Y_{t})_{t = 0, 1}$ without loss $Y_{0} = 1$.  Let $Y = Y_{1}$ for
  clarity, $Y > 0$ a.s.

  Suppose $(H_{t})_{t = 1} = H_{1} = H$ (non-random vector) is a
  candidate arbitrage, so $H \cdot P_{0} \leq 0$ and $H \cdot P_{1}
  \geq 0$ a.s.  We must show $H \cdot P_{0} = 0 = H \cdot P_{1}$ a.s.

  Since $Y > 0$, $H \cdot P_{1} \geq 0 \Rightarrow \E{Y H P_{1}} \geq
    0$, but $H \underbrace{\E{Y P_{1}}}_{\text{state price density}} =
    H P_{0} \leq 0$.

  By the pigeonhole principle, if $Z \geq 0$ a.s and $E(Z) = 0$, then
  $Z = 0$ a.s.

  Thus, $Y H \cdot P_{1} = 0$ a.s., and since $Y > 0$ a.s., $H_{0}
  P_{1} = 0 = H P_{0} = 0$ a.s.

  Now consider the other direction.  Let $\mathcal{Y} = \{ Y > 0 a.s,
  \E{Y \| P_{1} \|} < a \}$.  $\mathcal{Y}$ is non-empty since $Y_{0}
  = e^{-\| P_{1} \|} \in \mathcal{Y}$ and $\mathcal{Y}$ is convex.
  Let $\mathcal{C} = \{ \E{YP_{1}}, y \in \mathcal{Y} \}$.  Suppose
  $P_{0} \notin \mathcal{C}$.

  By the separating hyperplane theorem, there exists $H \in \R^{n}$ such
  that
  \begin{enumerate}
  \item For all $c \in \mathcal{C}$, $H(c - P_{0}) \geq 0$.
  \item There exists $c^{\star} \in \mathcal{C}$, $H(c^{\star} - P_{0}) > 0$.
  \end{enumerate}

  This implies
  \begin{enumerate}
  \item For all $Y \in \mathcal{Y}$, $\E{YH \cdot P_{1}} \geq H \cdot
    P_{0}$
  \item There exists $Y^{\star} \in \mathcal{Y}$, $\E{Y H \cdot P_{1}} > H
    \cdot P_{0}$.
  \end{enumerate}

  Let $y = \{ Y > 0: \E{Y \| P_{1} \|} \infty \}$.  Let $\mathcal{P} =
  \{ \E{YP_{1}} : Y \in \mathcal{Y} \} \subseteq \R^{n}$. Suppose
  $P_{0} \notin \mathcal{P}$.

  By the \textbf{separating/supporting hyperplane theorem} there
  exists a vector $H \in \R^{n}$ such that
  \begin{enumerate}
  \item For all $p \in \mathcal{P}$, $H \cdot (p - P_{0}) \geq 0$,
  \item There exists $p^{\star} \in \mathcal{P}$ such that $H \cdot
    (p^{\star} - P_{0}) > 0$.
  \end{enumerate}

  If $p \in \mathcal{P}$ then $p = \E{YP_{1}}$ for some $Y$.  Then
  \begin{equation}
    \label{eq:6}
    H \cdot p = \E{Y \underbrace{H \cdot P_{1}}_{\text{$X$,
          time 1 wealth}}}, H \cdot P_{0} =
    \underbrace{-c}_{\text{consumption in $(0, 1]$}}
  \end{equation}

  Restating, we then have:
  \begin{enumerate}
  \item \label{item:1} For all $Y \in \mathcal{Y}$, $\E{YH \cdot P_{1}} \geq H \cdot
    P_{0}$
  \item \label{item:2} There exists $Y^{\star} \in \mathcal{Y}$, $\E{Y H \cdot P_{1}} > H
    \cdot P_{0}$.
  \end{enumerate}
  

  We need to show that $X \geq 0$ a.s., $c \geq 0$, $\Prob{X + c > 0}
  > 0$.  Let $Y_{0} = e^{-\| P_{0} \|} \in \mathcal{Y}$.  For
  $\epsilon > 0$, let $Y = \epsilon Y_{0}$ in \ref{item:1}, then $\epsilon
  \E{Y_0 X} \geq c \Rightarrow c \geq 0$ by taking $\epsilon
  \rightarrow 0$.

  Let $Y = ( \frac{1}{\epsilon} \I{X < 0} + 1) Y_{0}$ in \ref{item:1}, which
  implies
  \begin{equation}
    \label{eq:7}
    \E{Y_{0} X \I{X < 0}} \geq - \epsilon (\E{X_{0} Y} + c)
    \rightarrow 0
  \end{equation} as $\epsilon \rightarrow 0$.

  Then $Y_{0} > 0$, $X\I{X < 0} \leq 0$ by pigeonhole principle,
  \begin{equation}
    \label{eq:8}
    \Prob{X < 0} = 0 \Rightarrow X \geq 0
  \end{equation} a.s.

  By \ref{item:2}, $\Prob{X = 0, c = 0} < 1$.
\end{proof}

\begin{defn}
  \label{defn:discrete_time_models:14}
  An integrable adapted process $X$ is a supermartingale is a
  supermartingale if
  \begin{equation}
    \label{eq:9}
    \E{X_{t} | \mathcal{F}_{s}} \leq X_{s}
  \end{equation} for all $0 \leq s \leq t$.
\end{defn}

\begin{proposition}
  If $X$ is a supermartingale and $\E{X_{T}} = X_{0}$ for some
  non-random $T > 0$, then $(X_{t})_{0 \leq t \leq T}$ is a martingale.
\end{proposition}

\begin{proof}
  Let $Y_{s, t} = X_{s} - \E{X_{t} | \mathcal{F}_{s}} \geq 0$ by
  assumption. Then
  \begin{align*}
    \E{Y_{s, t}} & = \E{X_{s} - \E{\E{X_{T} | \mathcal{F}_{s}}}} \\
                 & = \E{X_{s}} - \E{X_{T}}                       \\
                 & \leq \underbrace{X_{0}}_{\text{supermartingale}} -
    \underbrace{X_{0}}_{\text{by assumption}}
  \end{align*}

  By pigeonhole, $Y_{s, T} = 0$ a.s.  Then $X_{s} = \E{X_{T} |
    \mathcal{F}_{s}}$ for all $0 \leq s \leq T$.  So by the tower
  property, $(X_{s})_{0 \leq s \leq T}$ is a martingale.
\end{proof}

\begin{proof}[Easy direction of 1FTAP]
  Let $T > 1$, and finite sample space. Let $H$ be a strategy, and $X
  = X(H)$ be a corresponding wealth process. Let $Y$ be a state price
  density. Then $XY$ is a supermartingale, as\footnote{This relies on
    the finiteness of $\Omega$ since this guarantees that $H$ is
    bounded, and so we call use the slot property}

  \begin{align*}
    \E{X_{t} Y_{t} | \mathcal{F}_{t-1}} & = \E{H_{t} \cdot P_{t} Y_{t}
      | \mathcal{F}_{t-1}}                                                                                                      \\
                                        & = \underbrace{H_{t}}_{\text{slot property}} \cdot \E{P_{t} Y_{t} | \mathcal{F}_{t-1}} \\
                                        & = H_{t} \cdot P_{t-1} Y_{t-1}                                                         \\
                                        & = (H_{t-1} P_{t-1} - c_{t}) Y_{t-1}                                                   \\
                                        & \leq X_{t-1} Y_{t-1}.
  \end{align*}

  Suppose $H$ is such that $X_{0} = 0$ and $X_{T}= 0$ a.s. for some
  non-random $T > 0$. Then
  \begin{equation}
    \label{eq:10}
    \E{Y_{T} X_{T}} = 0 = Y_{0} X_{0}
  \end{equation} and so $XY$ is a martingale by the previous
  proposition.  This implies $Y_{t} X_{t} = \E{Y_{t} X_{t} |
    \mathcal{F}_{t}} = 0$, which implies $X_{t} = 0$ for all $t$.

  By the calculation,
  \begin{align*}
    \E{X_{t} Y_{t} | \mathcal{F}_{t-1}} & = (X_{t-1} + c_{t}) Y_{t-1}
 \\
                                        & \Rightarrow c_{t} = 0
  \end{align*} for all $t$.
\end{proof}

\begin{defn}
  \label{defn:discrete_time_models:15}
  A stopping time for a filtration $(F_{t})_{t \in \mathbb{T}}$ is a
  random variable $\tau: \Omega \rightarrow \mathbb{T} \cup \{ \infty
  \}$ such that $\{ \tau \leq t \} \in \mathcal{F}_{t}$ for all $t \in
  \mathbb{T}$ (discrete or continuous time). 
\end{defn}

\begin{notation}
  $M_{t \wedge \tau} = M_{t}^{\tau}$ is the martingale $M$ stopped at $\tau$.
\end{notation}

\begin{proposition}
  Let $M$ be a martingale and $\tau$ a stopping time, and let $N_{t} =
  M_{t \wedge \tau}$.  Then $N$ is also a martingale.
\end{proposition}

\begin{proof}
  \begin{align}
    \label{eq:11}
    N_{t} = M_{0} + \sum_{s=1}^{t} \I{s \leq \tau} (M_{s} - M_{s-1})
  \end{align} and $\I{\tau \leq s-1}$ is
  $\mathcal{F}_{s-1}$-measurable and bounded.
\end{proof}

\begin{defn}
  \label{defn:discrete_time_models:16}
  A local martingale is an adapted process $X$ such that there exists
  an increasing sequence of stopping times $\tau_{n} \uparrow \infty$
  such that $X^{\tau_{n}}$ is a martingale for all $n$.
\end{defn}

\begin{remark}
  Martingales are local martingales.
\end{remark}

\begin{proposition}
  Let $X$ be a local martingale (discrete time). Let $K$ be
  predictable and let $Y_{t} = \sum_{s=1}^{t} K_{s}(X_{s} - X_{s-1})$.
  Then $Y$ is a local martingale.
\end{proposition}

\begin{proof}
  Since $X$ is a local martingale, there exists a sequence $\sigma_{n}
  \rightarrow \infty$ stopping times such that $X^{\sigma_{n}}$ is a
  martingale. Let
  \begin{equation}
    \label{eq:13}
    \tau_{n} = \inf \{ t \geq 0 : |K_{t+1}| > N \}
  \end{equation} Then we have
  \begin{align}
    \label{eq:14}
    X_{t \wedge (\underbrace{\sigma_{n} \wedge \tau_{n}}_{\text{stopping time}})} = \sum_{s = 1}^{t} \underbrace{K_{s} \I{s
      \leq \tau_{n}}}_{\text{bounded and predictable}} (\underbrace{X_{s}^{\tau_{n}} -
      X_{s-1}^{\tau_{n}}}_{\text{martingale difference}})
  \end{align}
\end{proof}

\begin{exmp}
  \label{defn:discrete_time_models:17}
  Let $\nu, \xi$ be random variables with $\xi$ integrable and $\E{\xi}
  = 0$.  Let $\mathcal{F}_{1} = \sigma(\nu), \mathcal{F}_{2} =
  \sigma(\nu, \xi)$.  Let $X_{1} = 0$, $X_{2} = \nu \xi$.  Then $X$ is
  a local martingale.

  If the product $\nu \xi$ is also integrable, then $X$ is a true
  martingale, otherwise $\E{X_{2} | \mathcal{F}_{1}}$ is not defined.
\end{exmp}

\begin{proposition}
  Let $X$ be a local martingale such that there exists an integrable
  process $Y$ such that $Y_{t} \geq |X_{s} |$ for all $0 \leq s \leq
  t$. Then $X$ is a true martingale.
\end{proposition}

\begin{proof}
  By assumptions there exists a sequence $\tau_{N} \rightarrow \infty$
  such that $X^{\tau_{N}}$ is a martingale.  Also, $|X_{t \wedge
    \tau_{N}} \leq Y_{t}$ which is integrable.  Then
  \begin{align}
    \label{eq:15}
    \E{X_{t} | \mathcal{F}_{s}} &= \E{\lim_{N \rightarrow \infty} X_{t
        \wedge \tau_{N}} | \mathcal{F}_{s}} \\
    &= \lim_{N \rightarrow \infty} \E{X_{t \wedge \tau_{N}} |
      \mathcal{F}_{s}} \\
    &= \lim_{N \rightarrow \infty} X_{s \wedge \tau_{N}} \\
    &= X_{s}
  \end{align}
\end{proof}

\begin{corollary}
  \textbf{In discrete time}, if $X$ is a local martingale and
  $\E{|X_{t}|} < \infty$ for all $t \geq 0$ then $X$ is a martingale.
\end{corollary}

\begin{proof}
  Let $Y_{t} = \sum_{s=0}^{t} | X_{s} |$, and $Y$ is integrable by
  assumption.
\end{proof}

\begin{proposition}
  If $X$ is a local martingale (in discrete or continuous time) and
  $X_{t} \geq 0$ almost surely for all $t$, then $X$ is a
  supermartingale.
\end{proposition}

\begin{proof}
  First, $X_{t}$ is integrable, since
  \begin{align}
    \label{eq:17}
    \E{|X_{t}|} &= \E{X_{t}} \\
    &= \E{\lim_{N \rightarrow \infty} X_{t \wedge \tau_{N}}} \\
    &\leq \liminf_{N \rightarrow \infty} \E{X_{t \wedge \tau_{N}}} \\
    &= \liminf_{N \rightarrow \infty} X_{0 \wedge \tau_{n}} \\
    &= X_{0} < \infty.
  \end{align}
  
  Now,
  \begin{align}
    \label{eq:16}
    \E{X_{t} | \mathcal{F}_{s}} &= \E{\lim X_{t \wedge \tau_{N}} |
      \mathcal{F}_{s}} \\
    &\leq \liminf \E{X_{t \wedge \tau_{N}} | \mathcal{F}_{s}} \\
    &= \liminf X_{s \wedge \tau_{N}} \\
    &= X_{s}
  \end{align}
\end{proof}

\begin{corollary}
  \textbf{In discrete time}, non-negative local martingales in
  discrete time are martingales.
\end{corollary}

\begin{proof}
  Let $X$ be the local martingale. Then $\E{|X_{t}|} < \infty$ for all
  $t \geq 0$ by Fatau. The result follows from the last corollary.
\end{proof}

\begin{thm}
  \label{defn:discrete_time_models:18}
  Let $X$ be a \textbf{discrete} time local martingale. Fix $T > 0$
  non-random. Then $(X_{t})_{0 \leq t \leq T}$ is a true martingale if
  either
  \begin{enumerate}
  \item $\E{|X_{T}|} < \infty$, or
  \item $X_T \geq 0$ 
  \end{enumerate}
\end{thm}

\todo{Lecture on Wednesday 23 October}

\section{Contingent Claims}
\label{sec:contigent-claims}

Setup - $P$ is a price process ($n$-dimensional space, adapted).

Two types of claims
\begin{enumerate}
\item European - specified by a time horizon $T$
  (maturity date or expiry) and a $\mathcal{F}_{T}$-measurable random
  variable $\xi_{T}$ (the payout of the claim).
\item American - specified maturity date $T$ and an adapted process
  $(\xi_{t})_{0 \leq t \leq T}$ where $\xi_{t}$ is the payout if owner
  of claim chooses to exercise at time $t \leq T$.
\end{enumerate}


\begin{exmp}
  \label{defn:discrete_time_models:19}
  A call option is the right, but not the obligation, to buy a certain
  stock at a fixed price sometime in the future.

  \begin{align}
    \label{eq:12}
    \xi_{T} = (S_{T} - k)^{+} \\
    \xi_{t} = (S_{t} - k)^{+}
  \end{align} for all $0 \leq t \leq T$.
\end{exmp}

\begin{defn}
  \label{defn:discrete_time_models:20}
  A European contingent claim is \textbf{attainable} or
  \textbf{replicable} if there exists a pure investment strategy $H$
  such that $X_{T}(H) = \xi_{T}$ almost surely.
\end{defn}

\begin{thm}
  \label{defn:discrete_time_models:21}
  Suppose $\xi_{t}$ is the price of attainable claim for $0 \leq t
  \leq T$.  If the augmented market $(P, \xi)$ has no arbitrage then
  $\xi_{t} = X_{t}(H)$ a.s.
\end{thm}

\begin{proof}
  Let $\tau = \inf \{ t \geq 0: X_{t} \neq \xi_{t} \}$.  Let $\bar
  H_{t} = \text{sign}(\xi_{t}, X_{t}) \I{t \geq \tau}(H_{t}, -1)$.

  Then $c_{\tau+1} = |\xi_{\tau} - X_{\tau}|$, $\bar X_{t}(\bar H) =
  \bar H_{t} \cdot (P_{t}, \xi_{t})$, $\bar X_{0}(\bar H) = 0, \bar
  X_{T}(\bar H) = 0$, and $c_{t} = 0$ for all $t$ if and only if there
  is no arbitrage.
\end{proof}

\begin{thm}
  \label{defn:discrete_time_models:22}
  Suppose $Y$ is a state price density of the original market with
  prices $P$.  Suppose $\xi_{T}$ is the payout of an attainable claim,
  suppose either
  \begin{enumerate}
  \item $\E{|\xi_{T}|Y_{T}} < \infty$, or
  \item $\xi_{T} \geq 0$ a.s.
  \end{enumerate}

  If the augmented market $(P, \xi)$ has no arbitrage, then
  \begin{equation}
    \label{eq:18}
    \xi_{t} = \frac{1}{Y_{t}} \E{Y_{T} \xi_{T} | \mathcal{F}_{t}}
  \end{equation} for all $0 \leq t \leq T$.
\end{thm}

\begin{proof}
  By the previous result, there exists $H$ (pure investment strategy)
  such that $X_{t}(H) = \xi_{t}$ for all $t$.  But $XY$ is a local
  martingale.  From before, if either $X_{T} Y_{T}$ is integrable or
  non-negative, the process $XY$ is a true martingale.

  \begin{equation}
    \label{eq:19}
    \xi_{t} Y_{t} = X_{t} Y_{t} = \E{X_{T} Y_{T} | \mathcal{F}_{t}} =
    \E{\xi_{T} Y_{T} | \mathcal{F}_{t}}
  \end{equation} as required.
\end{proof}

\begin{remark}
  When our price process can be decomposed into a numeraire, so $P =
  (N, S)$, we can let $\Q$ be an equivalent martingale measure.  If
  either $\E{\frac{\xi_{T}}{N_{T}}}{\Q} < \infty$, or $\xi_{T} \geq 0$,
  then
  \begin{equation}
    \label{eq:20}
    \xi_{t} = N_{t} \E{\frac{\xi_{T}}{N_{T}} | \mathcal{F}_{t}}{\Q}
  \end{equation}
\end{remark}

\begin{thm}
  \label{defn:discrete_time_models:23}
  Suppose $\xi_{t}$ is the price of a contingent claim at time $t$
  (not necessarily attainable).  Suppose that the augmented market
  $(P, \xi)$ has no arbitrage.  Then there exists a positive process
  $Y$ such that
  \begin{align}
    \label{eq:21}
    P_{t} = \frac{1}{Y_{t}} \E{Y_{T} P_{T} | \mathcal{F}_{t}} \\
    \label{eq:22}
    \xi_{t} = \frac{1}{Y_{t}} \E{Y_{T} \xi_{T} | \mathcal{F}_{t}}
  \end{align}
  Here, \eqref{eq:21} shows $Y$ is a state price density for the original
  market, and \eqref{eq:22} shows $Y$ is a state price density for the
  augmented market.
\end{thm}

\begin{proof}
  The proof is just \textsc{1ftap} applied to the augmented market.
\end{proof}

\begin{exmp}
  \label{defn:discrete_time_models:24}
  Let $P_{t} = (B_{t, T}, S_{t})$.  $B_{t, T}$ is price of bond
  maturing at $T$, with $B_{T, T} = 1$ almost surely.  $S_{t}$ is a
  stock with $S_{t} \geq 0$ for all $t$.  Let $c_{t}$ be the price of
  a call with payout $(S_{T} - K)^{+}$.  Suppose $(B_{t, T}, S_{t},
  C_{t})_{t \in [0, T]}$ has no arbitrage.

  In general, since the payout of the call is non-negative then $c_{t}
  \geq 0$.  Also, $(S_{T} - K)^{+} \geq S_{T} - K = S_{T} - K B_{T, T}
  = (-K, 1) \cdot (B_{t, T}, S_{t})$.

  This implies
  \begin{equation}
    \label{eq:23}
    c_{t} \geq S_{t} - K B_{t, T}
  \end{equation}

  Then $c_{t} \geq (S_{t} - KB_{t, T})^{+}$, and $(S_{T} - K)^{+} <
  S_{T}$, thus $c_{t} \leq S_{t}$.

  If there exists a state price density $Y$ for $(B, S)$ such that
  \begin{align}
    \label{eq:152}
    c_{t} = \frac{1}{Y_{t}} \E{Y_{T}(S_{T} - K)^{+} |
      \mathcal{F}_{t}}.
  \end{align}
\end{exmp}

\begin{exmp}
  \label{defn:discrete_time_models:25}
  A put option is equivalent to $(K - S_{T})^{+} = K - S_{T} + (S_{T}
  - K)^{+} = (K, -1, 1) \cdot (B_{T, T}, S_{T}, C_{T})$.  If $p_{t}$
  is a no-arbitrage price of the put, then
  \begin{equation}
    \label{eq:24}
    p_{t} = K B_{t, T} - S_{t} + c_{t}.
  \end{equation}
\end{exmp}

\begin{defn}
  \label{defn:discrete_time_models:26}
  A market is \textbf{complete} if and only if every European
  contingent claim is attainable. A market that is not complete is
  \textbf{incomplete}.
\end{defn}

\begin{thm}[Second fundamental theorem of asset pricing]
  \label{defn:discrete_time_models:27}
  A market with no arbitrage is complete if and only if there exists a
  unique (up to scaling) state price density.
\end{thm}

\begin{proof}
  Suppose the market is complete.  Let $Y, Y'$ be state price
  densities with $Y_{0} = Y'_{0}  = 1$.  Fix $T > 0$ and let $\xi_{T}
  \geq 0$ be $\mathcal{F}_{T}$-measurable.  By completeness, there
  exists a pure investment strategy $H$ such that $X_{T}(H) =
  \xi_{T}$.

  From before,
  \begin{equation}
    \label{eq:25}
    \E{Y_{T} \xi_{T}} = X_{0}(H) = \E{Y'_{T} \xi_{T}}
  \end{equation} and thus $\E{\xi_{T}(Y_{T} - Y'_{T})} = 0$.  Let
  $\xi_{T} = \I{Y_{T} > Y'_{T}}$. Then $Y_{T} \leq Y'_{T}$ almost
  surely, and so by symmetry, $Y_{T} = Y'_{T}$.

  A claim with payout $\xi_{T} \geq 0$ is attainable if there exists
  $x \geq 0$ such that $\E{\frac{Y_{T} \xi_{T}}{Y_{0}}} = x =
  X_{0}(H)$ for all state price densities.\sidenote{Proof in example
    sheet}

  Given there exists a unique state price density, every non-negative
  claim is attainable.  The conclusion follows by observing $\xi_{T} =
  \xi_{T}^{+} - \xi_{T}^{-}$.
\end{proof}

\begin{thm}
  \label{defn:discrete_time_models:28}
  Suppose that the price process $P$ is $n$-dimensional and the market
  is complete.  The for each $t \geq 0$, there are no more than
  $n^{t}$ disjoint sets of positive probability
  $\mathcal{F}_{t}$-measurable sets of positive probability.  In
  particular, the random vector $P_{t}$ takes on at most $n^{t} values$.
\end{thm}

\begin{proof}
  Consider the $t=1$ case.  Let $A_{1}, \dots, A_{k}$ be  disjoint
  $\mathcal{F}_{1}$-measurable sets with $\Prob{A_{i}} > 0$.  We claim
  the set $\{ \I{A_{i}} \}$ is linearly independent.

  Suppose $\sum_{i} a_{i} \I{A_{i}} = 0$.  Multiplying by $\I{A_{j}}$
  implies $a_{j} \I{A_{j}} = 0$ almost surely by disjointness.  Since
  $\Prob{A_{j}} > 0$ by assumption we have $a_{j} = 0$.

  By completeness, each $\I{A_{i}}$ is attainable, so
  \begin{equation}
    \label{eq:26}
    \vecspan \{ \I{A_{i}} \} \subseteq \{ H \cdot P_{1}, H \in \R^{n} \}
    =  \vecspan \{ P^{1}_{1}, \dots, P^{n}_{1} \}
  \end{equation} 
\end{proof}

\section{American Claims}
\label{sec:american-claims}

Recall that the payoff of an American claim is specified by an adapted
process $(\xi_{t})_{0 \leq t \leq T}$ where $\xi_{t}$ is the payout if
the claim is executed at time $t$.

\begin{thm}
  \label{defn:discrete_time_models:29}
  Suppose the market is complete.  Then there exists a (pure
  investment) strategy such that $X_{t}(H) \geq \xi_{t}$ for all $0
  \leq t \leq T$, and there exists a stopping time $\tau^{\star}$ such
  that $X_{\tau^{\star}}(H) = \xi_{\tau^{\star}}$.

  Furthermore, $X_{0}(H) = \sup_{\text{stopping time $\tau \leq T$}}
  \E{Y_{\tau} \xi_{\tau}}$ where $Y$ is the unique state price density
  such that $Y_{0} = 1$.
\end{thm}

\begin{defn}
  \label{defn:discrete_time_models:30}
  Let $Z$ be an adapted integrable process $(Z_{t})_{0 \leq t \leq
    T}$.  The Snell envelope of $Z$is the process $U$ defined by
  $U_{T}= Z_{T}$, $U_{t} = \max \{ Z_{t}, \E{U_{t + 1} |
    \mathcal{F}_{t} } \}$ for $0 \leq t \leq T - 1$.
\end{defn}

\begin{remark}
  Note that $U_{t} \geq Z_{t}$for all $t$, and $U$ is a supermartingale
  since $U_{t} \geq \E{U_{t+1} | \mathcal{F}_{t}}$.
\end{remark}

\begin{thm}[Doob decomposition]
  \label{defn:discrete_time_models:31}
  Let $U$ be a discrete-time supermartingale.  Then there exists a
  martingale $M$ with $M_{0} = 0$, and a non-decreasing process $A$
  with $A_{0} = 0$ such that $U_{t} = U_{0} + M_{t} - A_{t}$.
\end{thm}

\begin{proof}
  Let $M_{0} = A_{0} = 0$, $M_{t+1} = M_{t} + U_{t+1} - \E{U_{t+1} |
    \mathcal{F}_{t}}$, and $A_{t+1} = A_{t} + U_{t} - \E{U_{t+1} |
    \mathcal{F}_{t}}$.  By induction, $A_{t}$ is predictable.  $A$ is
  non-decreasing as $U$ is a supermartingale.

  Now, we show uniqueness.  Suppose $U = U_{0} + M - A = U_{0} + M' -
  A'$ .  Then $M - M' = A - A'$, and as $A - A'$ is predictable, we
  have $M - M'$ is a predictable martingale.  In discrete time,
  predictable martingales are almost surely constant.  Thus, $M_{t} -
  M'_{t} = M_{0} - M'_{0} = 0$, and thus we have demonstrated uniqueness.
\end{proof}

\begin{thm}
  \label{defn:discrete_time_models:32}
  Let $Z$ be integrable and adapted, $U$ is a Snell envelope, with
  Doob decomposition $U = U_{0} + M - A$.  Let $\tau^{\star} = \inf \{
  t \geq 0 | A_{t + 1} > 0 \}$ with the convention $\tau^{\star} = T$
  on $\{ A_{t} = 0 \forall t \}$.

  Then $U_{\tau^{\star}} = U_{0} + M_{\tau^{\star}} = Z_{\tau^{\star}}$.
\end{thm}

\begin{remark}
  $\tau^{\star}$ is a stopping time since $A$ is predictable.
\end{remark}

\begin{proof}
  Note that $A_{\tau^{\star}} = 0$ but $A_{\tau^{\star} + 1} > 0$ .
  We have
  \begin{align}
    \label{eq:153}
    U_{t} &= U_{0} + M_{t} - A_{t} \\
    &= \max \{ Z_{t}, \E{U_{t+1} | \mathcal{F}_{t}}  \} \\
    &= \max \{ Z_{t}, U_{0} + M_{t} - A_{t+1} \}.
  \end{align}


  So $U_{0} + M_{\tau^{\star}} = \max \{ Z_{\tau^{\star}}, U_{0} +
  M_{\tau^{\star}} - A_{\tau^{\star} - 1} \}$, which implies $U_{0} +
  M_{\tau^{\star}}= Z_{\tau^{\star}} = U_{\tau^{\star}}$ as required.
\end{proof}

\begin{thm}
  \label{defn:discrete_time_models:33}
  Under the same hypothesis as before,
  \begin{equation}
    \label{eq:27}
    U_{0} = \sup_{\text{stopping times $\tau \leq T$}} \E{Z_{\tau}}.
  \end{equation}
\end{thm}

\begin{proof}
  By the optional stopping theorem, $U_{0} \geq \E{U_{\tau}} \leq \E{Z_{t}}$ for any
  stopping time $\tau \leq T$, and since $U_{t} \geq Z_{t} \forall t$.

  But $U_{0} = \E{U_{0} + M_{\tau^{\star}}} = \E{Z_{\tau^{\star}}}$.
\end{proof}

We now give a proof of the existence of the minimal super-replicating
strategy. Let $U$ be the Snell envelope of $(Y_{t} \xi_{t})_{0 \leq t
  \leq T}$. Let $U = U_{0} +M - A$ be its Doob decomposition.

By completeness, there exists a strategy $H$ such that
\begin{equation}
  \label{eq:28}
  X_{T}(H) = \frac{U_{0} + M_{T}}{Y_{T}}.
\end{equation}  Since $XY$ is a martingale ($XY$ is a local martingale
in general but by completeness all processes are bounded).  So
\begin{align}
  \label{eq:29}
  X_{t} Y_{T} &= U_{0} + M_{t} \\
  &\geq U_{0} + M_{t} - A_{t} \\
  &= U_{t} \\
  &\geq Y_{t} \xi_{t}.
\end{align} Thus $X_{t} \geq \xi_{t}$ for all $0 \leq t \leq T$.

Also, at $\tau^{\star} = \inf \{ t \geq 0 | A_{t+1} > 0 \}$, we have

\begin{equation}
  \label{eq:154}
  X_{\tau^{\star}} Y_{\tau^{\star}} = U_{0} + M_{\tau^{\star}} =
  U_{\tau^{\star}} = Y_{\tau^{\star}} \xi_{\tau^{\star}},
\end{equation} and so $X_{\tau^{\star}} = \xi_{\tau^{\star}}$.

Note also that $X_{0} = \E{U_{0} + M_{T}} = U_{0} = \sup_{\tau \leq T}
\E{\xi_{\tau} Y_{\tau}}$.
 

%%% Local Variables: 
%%% mode: latex
%%% TeX-master: "master"
%%% End: 