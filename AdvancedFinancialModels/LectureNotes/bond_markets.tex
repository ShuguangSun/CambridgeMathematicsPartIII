\chapter{Bond Markets and Interest Rates}
\label{cha:bond-mark-inter}

\begin{defn}
  \label{defn:bond_markets:1}
  A zero coupon bond is a contingent claim that pays exactly one unit
  of money at maturity.

  We assume that $\xi_{T}$, the payment of the bond, is 1 a.s. - that
  is, there is no credit risk.
\end{defn}

\begin{defn}
  \label{defn:bond_markets:3} $P(t, T)$is the price at time $t$ for a
  bond maturing at time $T$.
\end{defn}

\begin{defn}
  \label{defn:bond_markets:4}
  The yield $y(t, T)$ is defined as
  \begin{align}
    \label{eq:145}
    y(t, T) = - \frac{1}{T - t} \log P(t, T)
  \end{align} or equivalently
  \begin{align}
    \label{eq:146}
    P(t, T)= e^{-(T-t) y(t, T)}
  \end{align}
\end{defn}

\begin{defn}
  \label{defn:bond_markets:5}
  We call $\lim_{T \downarrow t} y(t, T) = r_{t}$ the ``spot'' or
  ``short'' rate.

  We call $\lim_{T \uparrow \infty} y(t, T)$ if it exists.
\end{defn}

\begin{defn}
  \label{defn:bond_markets:6}
  The forward rate $f(t, T)$ is defined
  \begin{align}
    \label{eq:147}
    f(t, T) = - \frac{\partial}{\partial T} \log P(t, T)
  \end{align} or equivalently
  \begin{align}
    \label{eq:148}
    P(t, T) = - \int_{t}^{T} f(t, u) du
  \end{align}
\end{defn}

\begin{thm}
  \label{defn:bond_markets:2}
  There is no arbitrage in the market prices $(P(t, T_{1}), P(t,
  T_{2}), \dots, P(t, T_{n}))$ if $Y_{t} P(t, T)_{t \in [0, T]}$ is a local
  martingale  for all $T$, where $Y$ is a state price
  density.\sidenote{Recall relative arbitrage, admissible class $D$,
    etc.}

  In particular, there is no arbitrage if $P(t, T) = \frac{1}{Y_{t}}
  \E{Y_{T} | \mathcal{F}_{t}}$
\end{thm}

Introduce the bank account $dB_{t} = B_{t} r_{t} dt \iff B_{t} = B_{0}
e^{\int_{0}^{t}r_{s} ds}$ where $r$ is the short rate.  Define an
equivalent martingale measure with density $\frac{d\Q}{d\Prob} =
\frac{B_{T} Y_{T}}{B_{0} Y_{0}}$.  Rewrite
\begin{align}
  \label{eq:149}
  P(t, T) = B_{t} \E{\frac{1}{B_{T}} | \mathcal{F}_{t}}{\Q} =
  \E{e^{-\int_{t}^{T} r_{s} ds} | \mathcal{F}_{t}}{\Q}
\end{align}

By the law of one price,
\begin{align}
  \label{eq:150}
  f(t, T) &= - \frac{\partial}{\partial T} \log \E{ e^{-\int_{t}^{T}
      r_{s} ds} | \mathcal{F}_{t}}{\Q} \\
  &= \frac{\E{r_{T} e^{-\int_{t}^{T} r_{s} ds} |
      \mathcal{F}_{t}}{\Q}}{\E{e^{-\int_{t}^{T} r_{s} ds} |
      \Phi_{t}}{\Q}},
\end{align} and so $f(t, T)$ can be seen as the ``market weighted
conditional expectation of $r_{T}$ given at $\mathcal{F}_{t}$.''

Alternatively, we have
\begin{align}
  \label{eq:151}
  \E{(f(t, T) - r_{T}) e^{-\int_{t}^{T} r_{s} ds} |
    \mathcal{F}_{t}}{\Q} = 0
\end{align} and so the forward rate is such that the claim with payout
$f(t, T) - r_{T}$ has price 0 at time $T$.

\todo{Fill in missing lecture from Monday 2 December}

There are two approaches to bond market pricing:

\begin{enumerate}
\item Let $(r_{t})_{t \geq 0}$ be fundamental, derive everything else:
  $f(t, T)$, etc.
\item Model $(f(t, T))_{0 \leq t \leq T}$ directly - the
  \citet{heath1992bond} approach.
\end{enumerate}

\section{The \citet{heath1992bond} Model}
\label{sec:citetheath1992bond}

\begin{thm}
  \label{defn:bond_markets:7}
  Suppose $df(t, T) = a(t, T) dt + \sigma(t, T) \cdot d\hat W_{t}$ for
  a $d$-dimensional Brownian motion $\hat W$ where $\sigma(t, T)$ is
  suitably measurable and integrable, and
  \begin{align}
    \label{eq:76}
    a(t, T) = \sigma(t, T) \cdot \int_{t}^{T} \sigma(t, u) du
  \end{align}

  Define $r_{t} = f(t, t)$ and $P(t, T) = e^{-\int_{t}^{T} f(t, u)
    du}$. Then
  \begin{align}
    \label{eq:157}
    \left( e^{-\int_{0}^{t} r_{s} ds} P(t, T) \right)_{0 \leq t \leq T}
  \end{align} is a local martingale.
\end{thm}


\begin{remark}
  \begin{align}
    \label{eq:158}
    f(t, T) = f(0, T) + \int_{0}^{t} a(s, T) ds + \int_{0}^{t}
    \sigma(s, T) \cdot d\hat W_{s}.
  \end{align}
\end{remark}

\begin{proof}
  Recall that if $d \log M_{t} = -\frac{|b_{t}|^{2}}{2} dt + b_{t}
  \cdot d \hat W_{t}$, then $M$ is a local martingale if and only if
  $M_{t} = M_{0} e^{-\frac{1}[2] \int_{0}^{t} |b_{s}|^{2} ds +
    \int_{0}^{t} b_{s} \cdot d \hat W_{s}}$.

  By differentiation, we have
  \begin{align}
    \label{eq:159}
    d \left( - \int_{0}^{t} r_{s} ds - \int_{t}^{T} f(t, u) du \right)
    &=
    -r_{t} dt + f(t, t) dt - \int_{t}^{T} df(t, u) du \\
    &= - (\int_{t}^{T} a(t, u) du) dt - (\int_{t}^{T} \sigma(t, u) du)
    \cdot d \hat W_{t}.
  \end{align}
  noting that
  \begin{align}
    \label{eq:160}
    \int_{t}^{T} a(t, u) du = \frac{1}{2} \| \int_{t}^{T} \sigma(t, u)
    du \|^{2} 
  \end{align} gives the required result.
\end{proof}

\begin{exmp}[\citet{ho1986term}]
  \label{defn:bond_markets:8}
  Assume $d = 1$, $\sigma(t, T) = \sigma_{0}$ constant. Then
  \begin{align}
    \label{eq:161}
    df(t, T) &= ((T - t) \sigma_{0}^{2}) dt + \sigma_{0} d \hat W_{t}
    \\
    f(t, T) &= f(0, T) + \int_{0}^{t} (T - s) \sigma_{0}^{2} ds +
    \sigma_{0} d \hat W_{t} \\
    r_{t} &= f(0, t) + \frac{1}{2} \sigma_{0}^{2} t^{2} + \sigma_{0}
    \hat W_{t}
  \end{align}
\end{exmp}

\begin{exmp}[\citet{hull1990pricing}]
  \label{defn:bond_markets:9}
  Again, assume $d = 1$, $\sigma(t, T) = \sigma_{0} e^{- \lambda(T - t)}$.
  \begin{align}
    \label{eq:162}
    df(t, T) &= \sigma_{0}^{2} e^{-\lambda(T - t)} (1 - e^{- \lambda(T
      - t)}) dt + \sigma_{0} e^{- \lambda (T - t)} d \hat W_{t} \\
    dr_{t} &= \lambda \left( \frac{f'_{0}(t)}{\lambda} + f_{0}(t) +
      \frac{\sigma_{0}^{2}}{2 \lambda^{2}}(1 - e^{-\lambda t}) - r_{t}
    \right) + \sigma_{0} d \hat W_{t}.
  \end{align}
\end{exmp}

\begin{exmp}[\citet{kennedy1997characterizing}]
  \label{defn:bond_markets:10}
  This is a Gaussian random field model.  Suppose $\sigma(t, T)$ is
  not random, so
  \begin{align}
    \label{eq:163}
    f(t, T) = f(0, T) + \int_{0}^{T} a(s, T) ds + \int_{0}^{t}
    \sigma(s, T) d \hat W_{s}
  \end{align} is Gaussian.  Then
  \begin{align}
    \label{eq:164}
    \E{f(t, T)}{\Q} = f(0, T) + \int_{0}^{t} a(s, T) ds \\
    \Cov{f(s, S)}{f(t, T)} = \int_{0}^{s \wedge t} \sigma(u, S) \cdot
    \sigma(u, T) du
  \end{align}

  Turning this around, we can model 

  \begin{align}
    \label{eq:167}
    (f(t, T))_{0 \leq t \leq T}
  \end{align}
  as a Gaussian random field with
  \begin{align}
    \label{eq:165}
    \Cov{f(s, S)}{f(t, T)} &= c_{s \wedge t}(S, T) \\
    \E{f(t, T)} &= f(0, T) + \int_{0}^{T} c_{s \wedge t}(s, T) ds,
  \end{align} and thus there is no need to introduce a Brownian
  motion.  For instance,
  \begin{align}
    \label{eq:168}
    d \IP{f(t, S), f(t, T)} &= \sigma(t, S) \cdot \sigma(t, T) dt \\
    &= \sigma_{0} e^{-\beta|T - S|}
  \end{align} and so we have an exponentially decaying correlation
  between forward rates of different maturities.
\end{exmp}

\begin{exmp}
  \label{defn:bond_markets:11}
  The HJM equation
  \begin{align}
    \label{eq:169}
    df(t, T) &= a(t, T) dt + \sigma(t, T) d W_{t} \\
    T = t + x, f_{t}(x) = f(t, t + x) \\
    df_{t}(x) = (\frac{\partial f}{\partial x} + a_{t}(x)) dt +
    \sigma_{t}(x) dW_{t} 
  \end{align}

  Fix a separable Hilbert space $F = \{ f: \mathbb{R}_{+} \rightarrow
  \R \}$.  Then (dropping the $x$),
  \begin{align}
    \label{eq:170}
    df_{t} = \left(A f_{t} + \alpha_{t} \right) dt + \sigma_{t} dW_{t}
  \end{align}
  can be interpreted as an evolution equation in this function space.
  In the simplest case, $\sigma_{t}$ is a constant vector $F \otimes
  \R^{d}$, $\alpha_{t}$ is a constant vector in $F$, then $(f_{t})_{t
    \geq 0}$ is an $F$-valued Ornstein-Uhlenbeck process.

  We can apply techniques from statistics (e.g. PCA) if this model has
  an invariant measure --- shown in early 2000's.
\end{exmp}

%%% Local Variables: 
%%% mode: latex
%%% TeX-master: "master"
%%% End: 
