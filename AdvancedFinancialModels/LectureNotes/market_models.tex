\begin{exmp}
  \label{defn:market_models:1}
  A market model with no absolute arbitrage but with a
  relative arbitrage.

  Consider $P = (1, S)$, where $dS_{t} = S_{t} \sigma_{t} dW_{t}$, $n
  = d = 1$, $\sigma_{t} > 0$ for all $t$. On the filtration generated
  by $W$ and $S$ is a strictly local martingale, $\E{S_T} < S_0$
  (recall that all positive local martingales are supermartingales)
  which implies $\E{\max_{0 \leq t \leq T} S_t} = \infty$.
\end{exmp}

\begin{defn}
  \label{defn:market_models:1}
  Let $Y_t = 1$  for all $t$ be a state price density.  If $L$ is of
  class $D$ locally, there exist $L$-admissible absolute arbitrages.
\end{defn}

\begin{defn}
  \label{defn:market_models:1}
  Let $\Q = \Prob$.  This is an EMM for the cash numeraire.  If $L$ is
  of class $D$ locally, there are no relative arbitrages.
\end{defn}

\begin{defn}
  \label{defn:market_models:2}
  By existential replication theorem, there exists $H$ such that
  $X_{T}(H) = S_{T}$.  Notice that $X_{0}(H) = \E{X_{T}} < S_{0}$ (!)
\end{defn}

Note that $\frac{X_{T}}{S_{T}} = 1$ a.s.\ but $\frac{X_{0}}{S_{0}} = p
< 1$ (so we have a relative arbitrage).  Let $\tilde H = H - p
\begin{pmatrix}
  0  \\
  1
\end{pmatrix}
$.  Then
\begin{align}
  \label{eq:86}
  X_{0}(\tilde H) = \E{S_{T}} - pS_{0} = 0 \\
  X_{T}(\tilde H) = S_{T} - p S_{T} > 0
\end{align}

$X_{t}(\tilde H)$ is \textbf{not} of class $D$.  So only admissible if
$L$ is wild.

\chapter{Black-Scholes}
\label{cha:black-scholes-1}

Consider the market model
\begin{align}
  \label{eq:87}
  dB_{t} &= B_{t} r dt \\
  dS_{t} &= S_{t}(\mu dt + \sigma dW_{t}) 
\end{align}
 
Then $B_t = B_{0} e^{rt}$, $S_{t} = S_{0} e^{(\mu -
  \frac{\sigma^{2}}{2})t + \sigma W_{t}}$, and $Y_{t} = e^{-(r -
  \lambda^{2}{2})t - \lambda W_{t}}$ is the unique state price density
with $Y_{0} = 1$, where $\lambda = \frac{\mu - r}{\sigma}$.

Our goal is to replicate a European claim with payout $\xi_{T} =
g(S_{T})$ where $g \geq 0$ and suitably integrable.  By our
replication theorem, there exists a 0-admissible strategy $H$ such
that $X_{t}(H) = \frac{1}{Y_{t}} \E{Y_{T} g(S_{T}) |
  \mathcal{F}_{t}}$.

Let $\frac{d\Q}{d\Prob} = e^{-\frac{\lambda^{2}T}{2} - \lambda W_{T}}$
be the unique EMM.  By the Cameron-Martin-Girsanov theorem, $\hat
W_{t} = W_{t} + \lambda t$ is a $\Q$-Brownian motion.  Then
\begin{align}
  \label{eq:89}
  S_{T} &= S_{t} e^{(\mu - \frac{\sigma^{2}}{2})(T-t) + \sigma(W_{T} -
    W_{t})} \\
  &= S_{t} e^{(-r - \sigma^{2}{2})(T-t) + \sigma(\hat W_{T} - \hat W_{t})}
\end{align} and we have
\begin{align}
  \label{eq:88}
  X_{t} &= e^{-r(T-t)} \mathbb{E}^{\Q}(g(S_{T}) | \mathcal{F}_{t}) \\
  &= \int g(S_{t} e^{(r - \frac{\sigma^{2}}{2})(T-t) +
    \sigma\sqrt{T-t} Z}) \frac{e^{-\frac{z^{2}}{2}}}{\sqrt{2 \pi}} dz
\end{align}

Substituting in $g(x) = (x - K)^{+}$ corresponding to a call option,
we obtain the price
\begin{multline}
  \label{eq:90}
  C_{t}(T, K) = S_{t} \Phi(\frac{-\log \frac{K}{S_{t}}}{\sigma \sqrt{T
      - t}} + (\frac{r}{\sigma} + \frac{\sigma}{2})\sqrt{T-t}) \\ -
  Ke^{-r(T-t)} \Phi(\frac{-\log \frac{K}{S_{t}}}{\sigma \sqrt{T-t}} +
  (\frac{r}{\sigma} - \frac{\sigma}{2}) \sqrt{T-t})
\end{multline}

\todo{Fill in missing lecture --- Black-Scholes price as a solution to
  BS PDE}

\section{Black-Scholes Volatility}
\label{sec:black-schol-volat}

Assume we observe $(S_{t})_{-T \leq t \leq 0}$ at some discrete intervals $(\frac{t}{n} -
1)T$ for $i = 0, \dots, n$, with 
\begin{align}
  \label{eq:156}
  Y_{i} &= \log \frac{S_{t_{i}}}{S_{t_{i-1}}} \\
  &= (\mu - \frac{\sigma^{2}}{2})(t_{i} - t_{i-1}) + \sigma(W_{t_{i}}
  - W_{t_{i-1}}) \\
  &\sim N(a\frac{T}{n}, \frac{\sigma^{2}T}{n}).
\end{align}
 The MLE is then
\begin{align}
  \label{eq:91}
  \hat a &= \frac{1}{T} \sum_{i=1}^{n} Y_{i} = \frac{1}{T} \log \frac{S_{0}}{S_{-T}} \\
  \hat \sigma^{2} &= \frac{1}{T} \sum_{i=1}^{n} (Y_{i} - \frac{\hat a
    T}{n})
\end{align} and $\Var{\hat \sigma^{2}} = \frac{2\sigma^{4}}{n}
\rightarrow 0$ as $n \rightarrow \infty$.

\section{Calibration}
\label{sec:calibration}

Black-Scholes model prediction, a call price 
\begin{equation}
  \label{eq:92}
  C_{t}(T, K) = C^{BS}(t, T, K, S_{t}, r, \sigma).
\end{equation}
The Black-Scholes implied volatility for strike $K$, maturity $T$ at
time $t$ is the unique $\sigma$ which solves~\eqref{eq:92}, denoted
$\sum_{t}(T, K)$.

Black-Scholes predicts there is a unique number $\sigma$ such that
$\sum_{t}(T, K) = \sigma$ for all $t, T, K$.  This fails in most
markets.

\section{Robustness}
\label{sec:robustness}

Consider a payout of claim $g(S_{T})$.  Assume we believe in
Black-Scholes, and so we believe the price
\begin{align}
  \label{eq:93}
  V(0, S, \sigma)
\end{align} where
\begin{align}
  \label{eq:94}
  V(t, S, \sigma) = e^{-r(T-t)} \int
  g(Se^{(r-\frac{\sigma^{2}}{2})(T-t) + \sigma \sqrt{T-t}z})
  \frac{e^{-\frac{z^{2}}{2}}}{\sqrt{2 \pi}} dz
\end{align} for some $\sigma$.  Pick $\hat \sigma$ to solve $V(0,
S_{0}, \hat \sigma) = \xi_{0}$, the initial price of the claim.

Now, try to replicate the claim with portfolio $(\phi, \pi)$ with
\begin{align}
  \label{eq:95}
  \pi_{t} &= \frac{\partial V}{\partial S}(t, S, \hat \sigma) \\
  \phi_{t} &= \frac{X_{t} - \pi_{t} S_{t}}{B_{t}}
\end{align}  Notice the equation
\begin{align}
  \label{eq:96}
  X_{0} &= V(0, S_{0}, \hat \sigma) \\
  dX_{t} &= r(X_{t} - \pi_{t} S_{t}) dt + \pi_{t} ds
\end{align} has a unique solution given by
\begin{align}
  \label{eq:97}
  X_{t} = X_{0} e^{rt} + e^{rt} \int_{0}^{t} \pi_{s} d(e^{-rs} S_{s})
\end{align} so given $\pi$, we can solve for $X$.

In the real model,
\begin{align}
  \label{eq:98}
  dB_{t} &= r B_{t} dt \\
  dS_{t} &= S_{t} (\mu dt + \sigma_{t} dW_{t})
\end{align} for $r, \mu$ constant but $\sigma_{t}$ a stochastic process.

Then
\begin{align}
  \label{eq:99}
  dV(t, S_{t}, \hat \sigma) &= \frac{\partial V}{\partial t} dt +
  \frac{\partial V}{\partial S} dS + \frac{1}{2} \frac{\partial^{2}
    V}{\partial S^{2}} d \IP{S} \\
  &= (\frac{\partial V}{\partial t} + \frac{1}{2} \frac{\partial^{2}
    V}{\partial^{2} S} \sigma^{2}_{t} S_{t}^{2}) dt + \pi_{t} dS_{t}
\\
&= (rV - rS \frac{\partial V}{\partial S} - \frac{1}{2}
\frac{\partial^{2} V}{\partial S^{2}} S^{2} \hat \sigma^{2} +
\frac{1}{2} \frac{\partial^{2} V}{\partial S^{2}} \sigma^{2}_{t}
S^{2}_{t}) dt + \pi_{t} dS_{t}
\end{align} and so
\begin{align}
  \label{eq:100}
  d(X_{t} - V(t, S_{t}, \hat \sigma)) = r (X - V) dt + \frac{1}{2}
  S^{2} (\hat \sigma^{2} - \sigma^{2}_{t}) \frac{\partial^{2}
    V}{\partial S^{2}} dt
\end{align} and so
\begin{align}
  \label{eq:101}
  X_{T} - V(T, S_{T}, \hat \sigma) - X_{0} + V(0, S_{0}, \hat \sigma)
  &= X_{T} - g(S_{T}) \\
  &= \frac{1}{2}\int_{0}^{T} e^{-r(T-s)} S^{2}_{s}
  (\hat \sigma^{2} - \sigma_{s}^{2}) \frac{\partial^{2} V}{\partial
    S^{2}} ds
\end{align}
and so we can estimate the difference between the option and the
replicating portfolio by a weighted average of the gamma multiplied by
the difference in implied and realized volatility over the time period.

\chapter{Local Volatility Models}
\label{cha:local-volat-models}

Consider
\begin{align}
  \label{eq:102}
  dB_{t} &= r B_{t} dt \\
  dS_{t} &= S_{t} (\mu(t, S_{t}) dt + \sigma(t, S_{t}) dW_{t})  \\
  &= S_{t}(r dt + \sigma(t, S_{t}) d \hat W_{t})
\end{align} with $d \hat W_{t} = dW_{t} + \frac{\mu(t, S_{t}) -
  r}{\sigma(t, S_{t})} dt$ is a Brownian motion under the equivalent
martingale measure $\Q$.

\begin{thm}[Dupire]
  \label{defn:market_models:3}
  Suppose $C_{0}(T, K) = \E{e^{-rT}(S_{T} - K)^{+}}{\Q}$.  Then
  \begin{align}
    \label{eq:103}
    \frac{\partial C_{0}}{\partial T} + rK \frac{\partial
      C_{0}}{\partial K} = \frac{\sigma(T, K)^{2}}{2} K^{2}
    \frac{\partial^{2} C_{0}}{\partial K^{2}}
  \end{align} with $C_{0}(0, K) = (S_{0} - K)^{+}$ with
  \begin{align}
    \label{eq:104}
    \sigma(T, K) = \sqrt{\frac{2(\frac{\partial C_{0}}{\partial T} +
        rK \frac{\partial C_{0}}{\partial K})}{K^{2}
        \frac{\partial^{2} C}{\partial K^{2}}}}
  \end{align}
\end{thm}

\begin{exer}
  If
  \begin{align}
    \label{eq:105}
    C_{0}(T, K) = C^{BS}(t = 0, \sigma, T, S_{0}, K, r, \sigma_{0})
  \end{align}
  show that
  \begin{align}
    \label{eq:106}
    \sigma(T, K) = \sigma_{0} 
  \end{align} for all $T, K$.
\end{exer}

\begin{lem}[Breden-Litzenberger, 1978]
  Suppose $S_{T}$ has density $f$ (under $\Q$). Then
  \begin{align}
    \label{eq:107}
    C_{0}(T, K) &= e^{-rT} \int_{K}^{\infty} f_{S_{T}}(y)(y-K) dy \\
    \frac{\partial C_{0}}{\partial K} &= -e^{-rT} \int_{K}^{\infty}
    f_{S_{T}}(y) dy \\
    \frac{\partial^{2} C_{0}}{\partial K^{2}} &= e^{-rT} f_{S_{T}}(K)
  \end{align}
\end{lem}

\begin{proof}[Proof of Theorem~\ref{defn:market_models:3}]
  By \ito's formula,
  \begin{align}
    \label{eq:108}
    (S_{T} - K^{+}) &= (S_{0} - K)^{+} + \int_{0}^{T} \I{S_{t} \geq K}
    dS_{t} + \frac{1}{2} \int_{0}^{T} \delta_{K} d \IP{S} \\
    &= (S_{0} - K)^{+} + \int_{0}^{T} S_{t} r \I{S_{t} \geq K} +
    \frac{1}{2} S_{t}^{2} \sigma(t, S_{t})^{2} \delta_{K}(S_{t}) dt + 
    \int_{0}^{T} S_{t} \sigma(t, S_{t}) \I{S_{t} \geq K} d\hat W_{t}.
  \end{align}

  Taking $\mathbb{E}^{\Q}$ on both sides, we obtain
  \begin{multline}
    \label{eq:109}
    e^{rT}C_{0}(T, K) = (S_{0} - K)^{+} + \int_{0}^{T}
    \left( \int_{K}^{\infty} f_{S_{t}}(y) y r dy \right) dt \\
    + \frac{1}{2} \int_{0}^{T} f_{S_{t}}(K) K^{2} \sigma(t, K)^{2} dt
  \end{multline} which gives
  \begin{align}
    \label{eq:110}
    e^{rT} \frac{\partial C_{0}}{\partial T} + re^{rT}C_{0} =
    \int_{K}^{\infty} f_{S_{T}}(y) y r dy + \frac{1}{2} f_{S_{T}}(K)
    K^{2} \sigma(T, K)^{2}
  \end{align}

  Writing $y = (y - K) + K$ and applying the previous lemma, we obtain
  the required result.
\end{proof}

\begin{remark}
  Given a call surface $\{ C_{0}(T, K), T, K > 0 \}$ where $C_{0}(T,
  \cdot)$ is smooth, we find the density of $S_{T}$ by
  \begin{align}
    \label{eq:111}
    \frac{\partial^{2} C_{0}}{\partial K^{2}} = e^{-rT} f_{S_{T}}(K)
  \end{align} and hence
  \begin{align}
    \label{eq:112}
    \mathbb{E}^{\Q}(e^{-rT} g(S_{T})) = \int_{0}^{\infty} g(y)
    \frac{\partial^{2} C_{0}}{\partial K^{2}}(T, y) dy
  \end{align}

  If $g$ is convex and smooth, then
  \begin{align}
    \label{eq:113}
    g(S_{T}) &= g(a) + g'(a)(S-a) + \int_{0}^{a} g''(K) (K) (K -
    S_{T})^{+} dK + \int_{a}^{\infty} g''(K) (S_{T} - K)^{+} dK \\
    &= \sum_{K_{i} \leq a} g''(K_{i})(K_{i} - S_{T})^{+} \Delta K_{i}
    + \sum_{K_{i} \geq a} g''(K_{i}) (S_{T} - K_{i}) \Delta K_{i}
  \end{align}
\end{remark}

\section{Computing Moment Generating Functions}
\label{sec:comp-moment-gener}

Consider a model with $B_{t} = B_{0} e^{rT}$, $S$ positive such that
$(e^{-rT}S_{t})_{t \geq 0}$ is a $\Q$-martingale.

Consider
\begin{align}
  \label{eq:115}
  \Theta = \{ p + qi | 0 \leq p \leq i, q \in \R \} \subseteq \mathbb{C}
\end{align} with $i = \sqrt{-1}$.

Let $M_{t}(\theta) = \mathbb{E}^{\Q} e^{\theta \log S_{t}}$ be the
moment generating function of log $S_{t}$, with $\theta = p + iq$, $0
\leq p \leq 1$, and so
\begin{align}
  \label{eq:116}
  \mathbb{E}^{\Q}|e^{\theta \log S_{t}}| &= \mathbb{E}^{\Q}(S_{t}^{p})
  \leq (\mathbb{E}^{\Q}S_{t})^{p} = (e^{rt} S_{0})^{p} < \infty
\end{align} and so $M_{t}(\theta)$ is well defined for $\theta \in
\Theta$.

\begin{thm}
  \label{defn:market_models:4}
  \begin{align}
    \label{eq:117}
    \mathbb{E}^{\Q}(e^{-rT}(S_{T} - K)^{+}) = S_{0} - \frac{e^{-rT}
      K^{1 - p}}{2 \pi} \int_{-\infty}^{\infty} \frac{M_{T}(p + ix)
      e^{-ix \log K}}{(x - ip)(x + i(1-p))} dx
  \end{align}
  for all $0 < p < 1$.
\end{thm}

\begin{thm}
  \label{defn:market_models:6}
  \begin{align}
    \label{eq:114}
    C_{0}(T, K) = S_{0}\frac{e^{-rT}K^{1-p}}2 \pi
  \int_{-\infty}^{\infty} \frac{M_{T}(p + ix) e^{-ix \log K}}{(x -
    ip)(x + i(1-p))} dx
  \end{align}
\end{thm}

\begin{lem}
  \begin{align}
    \label{eq:118}
    \frac{1}{2\pi} \int_{-\infty}^{\infty} \frac{e^{-iax}}{x-ip}{x +
      i(1-p)} =
    \begin{cases}
      e^{-ap} & a \geq 0 \\
      a^{a(1-p)} & a < 0
    \end{cases}
  \end{align}
  which can be shown via contour integration.

  Let $\gamma_{R}$ be the semi-circle of radius $R$ above the $x$-axis
  in the complex plane. Then
  \begin{align}
    \label{eq:119}
    \int_{\gamma_{R}} \frac{e^{iax}}{(x-ip)(x+i(1-p))} dx = 2\pi
    \res_{x=ip} = 2\pi e^{-ap}.
  \end{align}

  and we have
  \begin{align}
    \label{eq:120}
    \int_{-R}^{R} + \int_{\phi=0}^{\pi} \frac{e^{ia(R \cos \phi + i
        \sin \phi)}}{(Re^{i \phi} - ip)(Re^{i \phi} + i(1-p))} d\phi
    \leq \frac{e^{-aR\sin \phi}}{\frac{1}{2}R} \rightarrow 0
  \end{align} and so we obtain our required result.
\end{lem}

\begin{proof}[Proof of \ref{defn:market_models:6}]
  We have
  \begin{multline}
    \label{eq:121}
    e^{-rT}(S_{T} - K)^{+} = e^{-rT} S_{T} \\ - \frac{K^{1-p}
      e^{-rT}}{2\pi} \int_{-\infty}^{\infty} \frac{e^{p\log S_{T} +
        ix\log S_{T} - ix \log K}}{(x - ip)(x + i(1-p))} dx 
  \end{multline}

  Now computing $\mathbb{E}^{\Q}$, using Fubini's theorem to justify
  the interchange as
  \begin{align}
    \label{eq:122}
    \E{\int \left| \frac{e^{(p + ix) \log S_{T} - ix \log K}}{(x - ip)(x +
        i(1-p))}\right| dx}  = M_{T}(p) \int \frac{1}{\sqrt{(x^{2} +
        p^{2})(x^{2} + (1 - p)^{2})}} < \infty
  \end{align}
\end{proof}

\begin{remark}
  By Holder's inequality, $p \mapsto \log M_{T}(p) = \Lambda_{T}(p)$ is
  convex.  $\Lambda_{T}(0) = 0, \Lambda_{T}(1) = \log S_{0} + rT$, and
  $p \mapsto \Lambda_{T}(p)$ is smooth.  It has a minimal point $p =
  p^{\star} \in (0, 1)$ at
  \begin{align}
    \label{eq:123}
    \Lambda_{T}(p^{\star} + ix) \approx \Lambda_{T}(p^{\star}) +
    \Lambda_{T}^{'}(p^{\star})(ix) + \frac{1}{2}
    \underbrace{\Lambda^{''}}_{\text{$\geq 0$ by convexity}}(p^{\star})(ix)^{2} \\
    &= ...
  \end{align}  by Taylor's theorem.

  Then
  \begin{align}
    \label{eq:124}
    \int \frac{M_{T}(p^{\star} + ix) e^{-ix \log K}}{(x - ip)(x +
      i(1-p))} &\approx M_{T}(p^{\star}) \int
    \frac{e^{-\Lambda_{T}^{''}(p^{\star}) x^{2}}}{p(1-p)} dx \\
    &= \frac{M_{T}(p^{\star})}{p(1-p)} \sqrt{\frac{2\pi}{\Lambda_{T}^{''}(p^{\star})}}
  \end{align}
\end{remark}

\section{The Heston Model}
\label{sec:heston-model}

\begin{align}
  \label{eq:125}
  dB_{t} &= B_{t} r dt \\
  dS_{t} &= S_{t} (r dt + \sqrt{v_{t} dW_{t}^{S}}) \\
  dv_{t} &= \lambda (\overline v - v_{t}) dt + c \sqrt{v_{t}}
  dW^{V}_{t}
\end{align}

$W^{S}, W^{v}$ are Brownian motions under some EMM $\Q$, with
correlation $\rho$.  For instance, $W_{t}^{v} = \rho W_{t}^{s} +
\sqrt{1-\rho^{2}} d_{t}^{\perp}$ with $W^{s}, W^{\perp}$ independent.

$\overline v > 0$ is the mean-reversion level.  $\lambda > 0$ is the
mean reversion rate.  We have $v_{t} \geq 0$ almost surely \citep{cox1985theory}.

Our goal is fix $T > 0, \theta \in \Theta$, want to compute
$\E{e^{\theta \log S_{T}}}$. 

Idea: Let $(V(t, S_{t}, v_{t}))_{0 \leq t \leq T}$ be chosen so that
it is a martingale with $V(T, S_{T}, V_{T}) = e^{\theta \log S_{T}}$.
The moment generating function is then $V(t=0, S_{0}, v_{0})$.

By \ito,
\begin{align}
  \label{eq:126}
  dV(t, S_{t, v_{t}}) = \frac{\partial V}{\partial t} dt +
  \frac{\partial V}{\partial S} dS + \frac{1}{2} \frac{\partial^{2}
    V}{\partial S^{2}} d \IP{S} + \frac{\partial V}{\partial v} dv +
  \frac{1}{2} \frac{\partial^{2}}{\partial v^{2}} d \IP{v} +
  \frac{\partial^{2} V}{\partial v \partial s} d \IP{S, v}.
\end{align}

We seek to make the $dt$ terms vanish.  Thus,
\begin{align}
  \label{eq:127}
  \frac{\partial V}{\partial t} + \frac{\partial V}{\partial S} rS +
  \frac{1}{2} \frac{\partial^{2} V}{\partial S^{2}} S^{2} v +
  \frac{\partial V}{\partial v} \lambda (\overline v - v) +
  \frac{1}{2} \frac{\partial^{2} V}{\partial v^{2}} c^{2} v +
  \frac{\partial^{2} V}{\partial S \partial v} \rho S v c = 0.
\end{align}

The inspired idea is to look for solutions of the form
\begin{align}
  \label{eq:128}
  V(t, S, v) = e^{\theta \log S + R(T-t)v + Q(T-t)}
\end{align} with $R(0) = Q(0) = 0$.

Substituting this functional form in, we obtain 
\begin{align}
  \label{eq:129}
  R'v - Q' + r \theta + \frac{1}{2} \theta (\theta - 1) v + R \lambda
  (\overline v - v) + \frac{1}{2} R^{2} c^{2} v + \theta R \rho v c =
  0
\end{align}  Collecting terms, we have
\begin{align}
  \label{eq:130}
  \begin{cases}
    R' = \frac{1}{2} \theta (\theta - 1) + \frac{1}{2} R^{2} c^{2} +
    (\theta p c - \lambda) R \\
    Q' = r \theta = R \lambda \overline v
  \end{cases}
\end{align} which are Riccati equations, which have an explicit solution.


\section{American Options (Guest Lecture)}
\label{sec:amer-opti-guest}

Suppose we have some assets $d$ and our bank account $B_{t}$.  The random
assets evolve as
\begin{align}
  \label{eq:131}
  dS^{i}_{t} S^{i}_{t}(\mu^{i}_{t} dt + \sum_{j=1}^{d} \sigma_{ij}(t,
  S_{t}) dW_{t}^{j})
\end{align}

The option we want to price pays $g(S_{\tau})$ if
exercised at time $\tau$. The exercise time $\tau$ must be a stopping
time, with $\tau \leq T$, the expiration time.

For technical reasons, suppose $g$ is bounded.  For examples sake, we
assume we have one sock, and consider an American put $g(S) = (K -
S)^{+}$.

If there are $d$ assets, we might have a min-put, we have
\begin{align}
  \label{eq:132}
  g(S) = (K - \min_{1 \leq i \leq d} S^{i})^{+} = \max_{1 \leq i \leq d} (K
  - S^{i})^{+}
\end{align}

To solve this pricing problem, write
\begin{align}
  \label{eq:133}
  \mathcal{L} f = \frac{1}{2} \sum_{i, j} S_{i} S_{j} a_{ij}(t, S)
  \frac{\partial^{2} f}{\partial S_{i} \partial S_{j}} + \sum_{i} r
  S_{i} \frac{\partial f}{\partial S_{i}} - rf + \frac{\partial
    f}{\partial t}
\end{align}  where $a = \sigma \sigma^{T}$, and suppose we can find
some $V(t, S) \in C^{1, 2}$ such that
\begin{align}
  \label{eq:134}
  \max \{ \mathcal{L} V, g - V \} = 0, V(T, \cdot) = g(\cdot).
\end{align}

Then
\begin{align}
  \label{eq:135}
  V(0, S_{0}) = \sup_{\tau \leq T} \E{e^{-r\tau} g(S_{\tau}) | S_{0}}
\end{align}

Why is this true?  Consider
\begin{align}
  \label{eq:136}
  d(V(t, S_{t}) e^{-rt}) = V_{s}(t, S_{t}) S_{t} \sigma_{t} dW_{t} + \mathcal{L} V(t,
  S_{t}) dt
\end{align}

If we let $\tau$ be any stopping time $\leq T$, and we let $T_{}
\uparrow \infty$ be a sequence of stopping times ``rediscovering'' the local
martingale $V_{S}(t, S) S \sigma dW$, and we shall then have
\begin{align}
  \label{eq:137}
  V(0, S_{0}) &= \E{e^{-r \tau_{n}} V(\tau_{n}, S_{\tau_{n}}) -
    \int_{0}^{\tau_{n}} \mathcal{L} V(u, S_{u}) du} \\
  &\geq \E{e^{-r \tau_{n}} V(\tau_{n}, S_{\tau_{n}})} \\
  &\geq \E{e^{-r\tau_{n}} g(S_{\tau_{n}})}.
\end{align} since $\mathcal{L} V \leq 0$.

If we let $n \rightarrow \infty, \tau_{n} \uparrow \tau$, we must have that
\begin{align}
  \label{eq:138}
  V(0, S_{0}) \geq \sup_{0 \leq \tau \leq T} \E{e^{-r \tau} g(S_{\tau})}.
\end{align}

To show that there is equality, consider
\begin{align}
  \label{eq:139}
  \tau^{\star} = \inf \{ t | V(t, S_{t}) = g(S_{t}) \}
\end{align}
We know that $V(T, \cdot) = g(\cdot)$, and so $\tau^{\star} \leq T$.
We also notice that in $[0, \tau)$, $\mathcal{L} V = 0$ because in
$[0, \tau)$, $g - V < 0$, and $\max \{ \mathcal{L} V, g - V \} = 0$.
Now going back to the first calculation, if we write $\tau^{\star}_{n}
= \tau^{\star} \wedge T_{n}$.

\begin{align}
  \label{eq:137}
  V(0, S_{0}) &= \E{e^{-r \tau^{\star}_{n}} V(\tau^{\star}_{n}, S_{\tau^{\star}_{n}}) -
    \int_{0}^{\tau^{\star}_{n}} \mathcal{L} V(u, S_{u}) du} \\
  &= \E{e^{-r \tau_{n}} V(\tau_{n}, S_{\tau_{n}})} \\
  &= \E{e^{-r\tau^{\star}} V(\tau^{\star}, S_{\tau^{\star}}) :
    \tau^{\star} \leq T_{n}} + \E{e^{-r T_{n}} V(T_{n}, S_{T_{n}}):
    \tau^{\star} > T_{n}} \\
  &= \E{e^{-r \tau^{\star}} g(S_{\tau^{\star}}) | \tau^{\star} \leq
    T_{n} } + \E{e^{-r T_{n}} V(T_{n}, S_{T_{n}}): \tau^{\star} >
    T_{n}} \\
  &\rightarrow \E{e^{-r \tau^{\star}} g(S_{\tau^{\star}})}.
\end{align} 
n
We need to show that the $V$ we found is bounded.

\begin{exmp}
  \label{defn:market_models:5}
  American puts in one dimension.

  We have an envelope $V$.

  We find $V$ by solving
  \begin{align}
    \label{eq:140}
    0 = -rV = \frac{1}{2} \sigma^{2} S^{2} V_{SS} + rSV_{s}
  \end{align} for $S = q$ with boundary condition
  \begin{align}
    \label{eq:141}
    V(q) = (K-q)^{+}
  \end{align}

  This we can write as
  \begin{align}
    \label{eq:142}
    V(S) = AS + BS^{-2r/\sigma^{2}}
  \end{align} with the boundary condition $V(q) = (K - q)^{+}$.

  Suppose we let $q$ be a parameter of the stopping rule, work out the
  value and optimize over $q$. The value is
  \begin{align}
    \label{eq:143}
    V(S) = (K - q) (\frac{S}{q})^{-\frac{2r}{\sigma^{2}}} =
    S^{-\frac{2r}{\sigma^{2}}} q^{\frac{2r}{\sigma^{2}}}(K - q)
  \end{align}

  Optimizing over $q$, we have
  \begin{align}
    \label{eq:144}
    \frac{2r}{\sigma^{2} q} = \frac{1}{K-q} \Rightarrow q =
    \frac{2rk}{\sigma^{2} + 2r}.
  \end{align}

  We can check, if we use this value of $q$, then $V'(q) = -1 =
  \frac{\partial}{\partial S}(K - S)|_{s=q}$.

  It can be shown that $\sup_{0 \leq \tau \leq T} \E{e^{-r \tau}
    g(S_{\tau})} \leq \min_{M \in \mathcal{M}_{0}} \E{\sup_{...}}$ \todo{Fill in from lecture notes.?}
\end{exmp}





%%% Local Variables: 
%%% mode: latex
%%% TeX-master: "master"
%%% End: 
