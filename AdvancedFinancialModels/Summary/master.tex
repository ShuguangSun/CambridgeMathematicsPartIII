\documentclass[8pt, reqno, oneside, twocolumn]{amsart} 

% Utility packages
\usepackage{etoolbox}
\newtoggle{tufte}
\togglefalse{tufte}
\newtoggle{summary}
\togglefalse{summary}
\usepackage{amsthm}
\usepackage{amsmath}
\usepackage{amssymb}
% \usepackage[landscape]{geometry}
\usepackage[top=1cm, bottom=1cm, left=1cm, right=1cm]{geometry}
% \usepackage[inner=1cm, outer=1cm]{geometry}
\usepackage{setspace}
\usepackage{graphicx}
\usepackage{enumerate}
\usepackage{listings}
\usepackage{booktabs}
\usepackage{hyperref}
\usepackage{thmtools}
\usepackage{xparse}
\usepackage{xspace}
\usepackage{enumitem}
\usepackage{todonotes}
\usepackage{dcolumn}
\usepackage{array}
\usepackage{booktabs}
\usepackage{colortbl, xcolor}
\usepackage{longtable}
\usepackage{natbib}
\usepackage{python}
% \usepackage{a4wide}

\hypersetup{colorlinks=true}
\numberwithin{equation}{section}

\declaretheorem[numbered=no,name=Theorem,style=definition]{thm}
\declaretheorem[numbered=no,name=Lemma]{lem}
\declaretheorem[numbered=no,name=Corollary]{corollary}
\declaretheorem[numbered=no,name=Definition]{defn}
\declaretheorem[numbered=no,name=Proposition]{proposition}
\declaretheorem[numbered=no,name=Example]{exmp}
\declaretheorem[numbered=no,name=Exercise]{exer}
\declaretheorem[numbered=no,name=History]{history}
\declaretheorem[numbered=no,name=Question]{question}
\declaretheorem[numbered=no,name=Remark]{remark}
\declaretheorem[numbered=no,name=Notation]{notation}
\declaretheorem[numbered=no,name=Theorem]{boxthm}

\let\proof\relax
\declaretheorem[style=remark,numbered=no,qed=\qedsymbol]{proof}

\DeclareMathOperator*{\argmax}{arg\,max}
\DeclareMathOperator*{\dist}{dist}
\DeclareMathOperator*{\V}{V}
\DeclareMathOperator*{\saddlepoint}{sp}
\DeclareMathOperator*{\divergence}{div}
\DeclareMathOperator*{\sym}{Sym}
\DeclareMathOperator*{\vol}{Vol}
\DeclareMathOperator*{\argmin}{arg\,min}
\DeclareMathOperator*{\proj}{\Pi}
\DeclareMathOperator*{\bnd}{bnd}
\DeclareMathOperator*{\supp}{supp}
\DeclareMathOperator*{\aff}{aff}
\DeclareMathOperator*{\rint}{rint}
\DeclareMathOperator*{\lev}{lev}
\DeclareMathOperator*{\tr}{tr}
\DeclareMathOperator*{\interior}{int}
\DeclareMathOperator*{\epi}{epi}
\DeclareMathOperator*{\dom}{dom}
\DeclareMathOperator*{\vecspan}{span}
\DeclareMathOperator*{\con}{con}
\DeclareMathOperator*{\cl}{cl}
\DeclareMathOperator*{\grad}{\nabla}
\DeclareMathOperator*{\crit}{crit}
\DeclareMathOperator*{\diag}{diag}
\DeclareMathOperator*{\res}{Res}
\DeclareMathOperator*{\boundary}{boundary}
\DeclareMathOperator*{\range}{range}
\DeclareMathOperator*{\Tr}{Tr}
\DeclareMathOperator*{\sign}{sign}
\setlist[enumerate]{label=(\roman*)}

\newlist{exercises}{enumerate}{1}
\setlist[exercises]{label=Ex. \arabic*}


\setcounter{secnumdepth}{2}
\setcounter{tocdepth}{2}
% \onehalfspacing

\author{Andrew Tulloch}


% Shortcuts
% convergence in probability
\newcommand{\cd}{\overset{d}{\rightarrow}}    
% convergence in distribution
\newcommand{\cp}{\overset{p}{\rightarrow}}
% convergence almost surely
\newcommand{\cas}{\overset{as}{\rightarrow}}
\newcommand{\R}{\mathbb{R}}
\newcommand{\N}{\mathbb{N}}
\newcommand{\Z}{\mathbb{Z}}
\newcommand{\np}{\textsc{np}\xspace}
\newcommand{\iid}{\textsc{iid}\xspace}
\newcommand{\knn}{$k$-\textsc{nn}\xspace}

% Expectation
\NewDocumentCommand{\E}{gg}{%
  \IfNoValueTF{#1}
    {%
      \mathbb{E}
    }    
    {%
      \IfNoValueTF{#2}
      {%
        \mathbb{E}\!\left(#1\right)
      }
      {%
        \mathbb{E}_{#2}\!\left(#1\right)
      }
    }%
}

% Probability with measure P
\NewDocumentCommand{\Prob}{gg}{%
  \IfNoValueTF{#1}
    {%
      \mathbb{P}
    }    
    {%
      \IfNoValueTF{#2}
      {%
        \mathbb{P}\!\left(#1\right)
      }
      {%
        \mathbb{P}_{#2}\!\left(#1\right)
      }
    }%
}

% Probability with measure Q
\NewDocumentCommand{\Q}{g}{%
  \IfNoValueTF{#1}
    {%
      \mathbb{Q}
    }    
    {%
      \mathbb{Q}\!\left(#1\right)
    }%
}


% Indicator
\NewDocumentCommand{\I}{g}{%
  \IfNoValueTF{#1}
    {%
      \mathbb{I}
    }    
    {%
      \mathbb{I}\!\left(#1\right)
    }%
}


\NewDocumentCommand{\Var}{g}{%
  \IfNoValueTF{#1}
    {%
      \mathbb{V}
    }    
    {%
      \mathbb{V}\!\left(#1\right)
    }%
}


\NewDocumentCommand{\Cov}{gg}{%
  \IfNoValueTF{#1}
    {%
      \text{Cov}
    }   
    {%
      \text{Cov}\!\left(#1, #2\right)
    }%
}

% Inner product
\NewDocumentCommand{\IP}{g}{%
    {%
      \left \langle #1 \right \rangle
    }%
}
%%% Local Variables: 
%%% mode: latex
%%% End: 


\title{Advanced Financial Models Summary}

\begin{document}

\maketitle

\section{Arbitrage Theory}
\label{sec:arbitrage-theory}

\begin{defn}
  \label{sec:arbitrage-theory-1}
  An investment/consumption strategy is a predictable process $H$
  satisfying the \textbf{self-financing} condition
  \begin{equation}
    \label{eq:1}
    H_{t-1} \cdot P_{t-1} \geq H_{t} \cdot P_{t-1}
  \end{equation}
  The corresponding consumption process $c_{t}$ is given as
  \begin{equation}
    \label{eq:2}
    c_{t} = H_{t-1} \cdot P_{t-1} - H_{t} \cdot P_{t-1}
  \end{equation}
\end{defn}

\begin{defn}
  \label{sec:arbitrage-theory-2}
  $X_{t}$ is \textbf{predictable} if $X_{t}$ is
  $\mathcal{F}_{t-1}$-measurable for all $t \geq 1$.
\end{defn}

\begin{defn}
  \label{sec:arbitrage-theory-3}
  A state price density is a strictly positive adapted process $Y$
  such that the process $Y_{t} P_{t}$ is a martingale.
\end{defn}

\begin{defn}
  \label{sec:arbitrage-theory-4}
  An \textbf{absolute arbitrage} is a strategy $H$ such that there
  exists a non-random time $T > 0$ with the properties
  \begin{enumerate}
  \item $X_{0}(H) = 0 = X_{T}(H)$ almost surely, and
  \item $\Prob{\sum_{t=1}^{T} c_{t} > 0} > 0$.
  \end{enumerate}
\end{defn}

\begin{defn}
  \label{sec:arbitrage-theory-5}
  An asset is a \textnormal{numeraire} if its price is strictly
  positive for all time, almost surely.
\end{defn}

\begin{thm}
  \label{sec:arbitrage-theory-6}
  If a numeraire exists, then we have that if an
  investment/consumption strategy is an arbitrage for the market
  model, there exists a pure investment strategy $H'$ and a non-random
  time horizon $T'$ such that
  \begin{enumerate}
  \item $X_{0}(H') = 0$,
  \item $X_{T'}(H') \geq 0$ almost surely,
  \item $\Prob{X_{T'}(H') > 0} > 0$.
  \end{enumerate}
\end{thm}

\begin{thm}
  \label{sec:arbitrage-theory-7}
  A market model has no arbitrage if and only if there exists a state
  price density.
\end{thm}

\begin{proof}
  $(\Rightarrow)$ $H_{0} = \E{Y P_{1}}$, so if $0 \leq \E{Y H \cdot P_{1}} = H
  \cdot \E{Y P_{1}} = H \cdot P_{0}  = 0$, so by pigeonhole $H \cdot
  P_{1} = 0$.
  $(\Leftarrow)$ By separating hyperplane argument, we have $P = \{
  \E{Y P_{1}} | Y > 0, \E{Y \| P_{1} \|} < \infty \} $, so either
  $P_{0} \in P$ (and so state price density exists), or there exists
  $H$ with for all $p \in P$, $H \cdot (p - P_{0}) \geq 0$ (with
  $p^{\star} \in P$) with $H \cdot (p^{\star} - P_{0}) > 0$.

  Then setting $Y = \epsilon Y_{0}$, we have a pigeonhole argument
  showing that $P(X - H \cdot P_{0}) = 0$, with $X \geq 0$ a.s.
\end{proof}

\begin{defn}
  \label{sec:arbitrage-theory-14}
  A supermartingale is an adapted integrable process such that
  \begin{equation}
    \label{eq:3}
    \E{X_{t} | \mathcal{F}_{s}} \leq X_{s}
  \end{equation} for all $0 \leq s \leq t$.
\end{defn}

\begin{defn}
  \label{sec:arbitrage-theory-8}
  A stopping time for a filtration $\mathcal{F}_{t}$ is a random
  variable $\tau$  such that the event $\{ \tau \leq t \}$ is
  $\mathcal{F}_{t}$ measurable for all $t$.
\end{defn}

\begin{defn}
  \label{sec:arbitrage-theory-9}
  For an adapted process $X_{t}$ and a stopping time $\tau$, the
  stopped process $X^{\tau}$ is given by $X_{t \wedge \tau}$.
\end{defn}

\begin{thm}
  \label{sec:arbitrage-theory-10}
  Let $X$ be a martingale and let $\tau$ be a stopping time, then
  $X^{\tau}$ is a martingale.
\end{thm}

\begin{proof}
  The process $K_{t} = \I{t \leq \tau}$  is predictable, bounded, so
  $X^{\tau}$ is a martingale transform and hence a martingale.
\end{proof}

\begin{defn}
  \label{sec:arbitrage-theory-11}
  A local martingale is an adapted process $X_{t}$ such that there
  exists an increasing sequence of stopping times $\tau_{n}$ with
  $\tau_{n} \uparrow \infty$ such that the stopped process
  $X^{\tau_{n}}$ is a martingale for each $N$.
\end{defn}

\begin{thm}
  \label{sec:arbitrage-theory-12}
  Martingales are local martingales..
\end{thm}

\begin{thm}
  \label{sec:arbitrage-theory-13}
  Let $X$ be a local martingale, with $|X_{s}| < Y_{t}$ a.s for all $0
  \leq s \leq t$.  If $\E{Y_{t}} < \infty$ for all $t \geq 0$, then
  $X$ is a true martingale.
\end{thm}

\begin{proof}
  Conditional dominated convergence theorem, $X_{t \wedge \tau_{n}}$
  is a martingale.
\end{proof}

\begin{thm}
  \label{sec:arbitrage-theory-15}
  Let $X$ be a local martingale, with $X_{t} \geq 0$ for all $t \geq
  0$.  Then $X$ is a supermartingale.
\end{thm}

\begin{proof}
  Fatau's Lemma.
\end{proof}

\begin{thm}
  \label{sec:arbitrage-theory-16}
  If $X$ is a discrete-time local martingale with $X_{t} \geq 0$ for
  all $t \geq 0$. Then $X$ is a martingale.
\end{thm}

\begin{thm}
  \label{sec:arbitrage-theory-18}
  The probability measure $\Q$ is equivalent to the measure $\Prob$ if
  and only if there exists a positive random variable $\xi$ such that
  $\Q{A} = \E^{P}(\xi \I{A})$.

  The random variable $\xi$ is call the density, or Radon-Nikodym
  derivative of $\Q$ with respect to $\Prob$.
\end{thm}

\begin{defn}
  \label{sec:arbitrage-theory-19}
  A numeraire is an asset with a strictly positive price at all times.
\end{defn}

\begin{defn}
  \label{sec:arbitrage-theory-20}
  An equivalent martingale measure is any probability measure $\Q$
  equivalent opt $\Prob$ such that the discounted price process
  $\frac{S_{t}}{N_{t}}$ is a martingale under $\Q$, where $N_{t}$ is
  the numeraire price process.
\end{defn}

\begin{defn}
  \label{sec:arbitrage-theory-21}
  Let $Y$ be a state price density, and fix a time horizon $T > 0$.
  Then
  \begin{equation}
    \label{eq:4}
    \frac{d\Q}{d\Prob} = \frac{Y_{T} N_{T}}{Y_{0} N_{0}}
  \end{equation}
  is an equivalent martingale measure relative to $N$ for the model.
\end{defn}

\begin{defn}
  \label{sec:arbitrage-theory-22}
  Suppose $\Q$ is an equivalent martingale measure for the market
  $P_{t}$. Then
  \begin{equation}
    \label{eq:5}
    Y_{t} = \frac{N_{0}}{N_{T}}  \E^{\Prob}(\frac{d\Q}{d\Prob} | \mathcal{F}_{t})
  \end{equation} is a state price density.
\end{defn}

\section{Pricing and Hedging Contingent Claims}
\label{sec:pric-hedg-cont}

\begin{defn}
  \label{sec:pric-hedg-cont-1}
  A European claim with payout $\xi_{T}$ is \textbf{replicable} (or
  \textbf{attainable}) if there exists a pure investment strategy $H$
  such that $X_{T}(H) = \xi_{T}$ almost surely.
\end{defn}

\begin{thm}
  \label{sec:pric-hedg-cont-3}
  Suppose that our market has no arbitrage.  Let $\xi_{T}$ be the
  payout of a European option, and let $H$ be the replicating strategy.
  Suppose that the option is priced at $\xi_{t}$ for $0 \leq t \leq
  T$.  Then if the augmented market with the option has no arbitrage,
  \begin{equation}
    \label{eq:7}
    \xi_{t} = X_{t}(H)
  \end{equation} for all $0 \leq t \leq T$.
\end{thm}

\begin{proof}
  First fundamental theorem applied to $(P, \xi)$.
\end{proof}

\begin{thm}
  \label{sec:pric-hedg-cont-4}
  Suppose that the market model has no arbitrage, and let $Y$ be a
  state price density process. Let $\xi_{T}$ be the payout of an
  attainable European contingent claim with maturity date $T > 0$.
  Suppose the claim has price $\epsilon_{t}$ for $0 \leq t \leq T$ and
  that the augmented market with the option has no arbitrage.  If
  either $Y_{T} \xi_{T}$ is integrable of $\xi_{T} \geq 0$ almost
  surely, then
  \begin{align}
    \label{eq:8}
    \xi_{t} = \frac{1}{Y_{t}} \E{\xi_{T} Y_{T} | \mathcal{F}_{t}}
  \end{align} or (if a numeraire exists),
  \begin{align}
    \label{eq:9}
    \frac{d\Q}{d\Prob} = \frac{N_{T} Y_{T}}{N_{0} Y_{0}}
  \end{align}
\end{thm}

\begin{defn}
  \label{sec:pric-hedg-cont-5}
  A market is \textbf{complete} if every European contingent claim is
  attainable, and \textbf{incomplete} otherwise.
\end{defn}

\begin{thm}
  \label{sec:pric-hedg-cont-6}
  An arbitrage-free market model is complete if and only if there
  exists a unique state price density $Y$ such that $Y_{0} = 1$.
\end{thm}

\begin{proof}
  Uniqueness follows from $Y = Y'$, then considering $\xi = \I{Y_{T} >
  Y'_{T}}$ and pigeonhole.
\end{proof}

\begin{thm}
  \label{sec:pric-hedg-cont-7}
  Suppose the adapted process $\xi_{t}$ specifies the payout of an
  American claim maturing at $T > 0$. Then there exists a trading
  strategy $H$ such that
  \begin{enumerate}
  \item $X_{t}(H) \geq \xi_{t}$ for all $0 \le qt \leq T$,
  \item $X_\tau{^\star} = \xi_{\tau^{\star}}$ for some stopping time
    $\tau^{\star}$, and
  \item $X_{0}(H) = \sup_{\tau \leq T} \E{Y_{\tau} \xi_{\tau}}$.
  \end{enumerate}
\end{thm}

\begin{thm}
  \label{sec:pric-hedg-cont-8}
  Let $U$ be a discrete-time supermartingale.  Then there is a unique
  decomposition
  \begin{equation}
    \label{eq:10}
    U_{t}= U_{0}+ M_{t} - A_{t}
  \end{equation} where $M$is a martingale and $A$is a predictable
  non-decreasing process with $M_{0} = A_{0} = 0$.
\end{thm}

\begin{proof}
  $M_{0} = 0 = A_{0}$,
  \begin{align}
    \label{eq:30}
    M_{t+1} = M_{t} + U_{t+1} - \E{U_{t+1} | \mathcal{F}_{t}} \\
    \label{eq:31}
    A_{t+1} = A_{t} + U_{t} - \E{U_{t+1} | \mathcal{F}_{t}}.
  \end{align} and telescope.
\end{proof}

\begin{defn}
  \label{sec:pric-hedg-cont-9}
  Let $Z_{t}$ be an integrable adapted discrete-time process. Let
  $U_{t}$ be given by the recursion
  \begin{align}
    \label{eq:11}
    U_{T} = Z_{T} \\
    U_{t} = \max (Z_{t}, \E{U_{t+1} | \mathcal{F}_{t}})
  \end{align}
  $U_{t}$ is called the Snell envelope of $Z_{t}$.  It is the smallest
  supermartingale that dominates the process $Z_{t}$.
\end{defn}

\begin{thm}
  \label{sec:pric-hedg-cont-11}
  Let $Z_{t}$ be an integrable adapted process, with $U_{t}$ its Snell
  envelope with Doob decomposition $U_{t} = U_{0} + M_{t} - A_{t}$.
  Let $\tau^{\star} = \min \{ t \in \{0, \dots, T\}: A_{t+1} > 0 \}$
  with the convention $\tau^{\star} = T$ on $\{ A_{t} = 0 \forall t
  \}$.  Then $\tau^{\star}$ is a stopping time, with
  \begin{align}
    \label{eq:12}
    U_{\tau^{\star}} = U_{0} + M_{\tau^{\star}} = Z_{\tau^{\star}}.
  \end{align}
\end{thm}

\begin{thm}
  \label{sec:pric-hedg-cont-12}
  Let $Z$ be an adapted integrable process and let $U$ be its Snell
  envelope.  Then
  \begin{align}
    \label{eq:13}
    U_{0} = \sup_{\tau \leq T} \E{Z_{\tau}}.
  \end{align}
\end{thm}

\section{Brownian Motion and Stochastic Calculus}
\label{sec:brown-moti-stoch}

\begin{defn}
  \label{sec:brown-moti-stoch-1}
  A Brownian motion is a collection of random variables such that
  \begin{enumerate}
  \item $W_{0}(\omega) = 0$ for all $\omega \in \Omega$,
  \item For all $0 \leq t_{0} < t_{1} \dots < t_{n}$, $W_{t_{i+1}} -
    W_{t_{i}}$ are independent with distribution $N(0, |t_{i+1} - t_{i}|)$,
  \item The sample path $t \mapsto W_{t}(\omega)$ is continuous for
    all $\omega \in \Omega$.
  \end{enumerate}
\end{defn}

\begin{defn}
  \label{sec:brown-moti-stoch-2}
  A simple predictable process is an adapted process $\alpha$ of the
  form $\alpha_{t}(\omega) = \sum_{n=1}^{N} \I{(t_{n-1}, t_{n}]}(t)
  a_{n}(\omega)$ where $a_{n}$ are bounded and
  $\mathcal{F}_{t_{n-1}}$-measurable for $0 \leq t_{0} < \cdots < t_{n}
  < \infty$.

  Define the stochastic integral by the formula
  \begin{align}
    \label{eq:14}
    \int_{0}^{\infty} \alpha_{s} dW_{s} = \sum_{n=1}^{N}  a_{n}
    (W_{t_{n} - W_{t_{n-1}}})
  \end{align}
\end{defn}

\begin{thm}
  \label{sec:brown-moti-stoch-3}
  For a simple predictable integrand $\alpha$, we have
  \begin{align}
    \label{eq:15}
    \E{(\int_{0}^{\infty} \alpha_{s} dW_{s})^{2}} =
    \E{\int_{0}^{\infty} \alpha_{s}^{2} ds}
  \end{align}
\end{thm}

\begin{defn}
  \label{sec:brown-moti-stoch-4}
  If $\alpha$ is predictable with $\E{\int_{0}^{\infty} \alpha_{s}^{2}
    ds} < \infty$, then $\int_{0}^{\infty} \alpha_{s} dW_{s} = \lim_{k}
  \int_{0}^{\infty} \alpha_{s}^{(k) dW_{s}}$ where the limit is in
  $L^{2}(\Omega)$ where $\alpha^{(k)}$ is a sequence of simple
  predictable processes converging to $\alpha$ in $L^{2}(R_{+} \times \Omega)$.
\end{defn}

\begin{thm}
  \label{sec:brown-moti-stoch-5}
  For every predictable $\alpha$ such that $\E{\int_{0}^{t}
    \alpha_{s}^{2} ds} < \infty$ for all $t \geq 0$, there exists a
  continuous martingale $X$ such that $X_{t} = \int_{0}^{\infty}
  \alpha_{s} \I{s \leq t} dW_{s}$.
\end{thm}

\begin{thm}
  \label{sec:brown-moti-stoch-6}
  If $\alpha$ is an adapted continuous process then $X_{t} =
  \int_{0}^{t} \alpha_{s} dW_{s}$ is a continuous local martingale. If
  we have $\E{\int_{0}^{t} \alpha_{s}^{2} ds} < \infty$ for all $t
  \geq 0$, then $X$ is a true martingale.
\end{thm}

\newcommand{\ito}{Ito\xspace}

\begin{defn}
  \label{sec:brown-moti-stoch-7}
  An \ito process $X$ is an adapted process of the form
  \begin{align}
    \label{eq:16}
    X_{t} = X_{0} + \int_{0}^{t} \alpha_{s} dW_{s} + \int_{0}^{t}
    \beta_{s} ds
  \end{align} where $X_{0}$ is a fixed real number and $\alpha_{t}$,
  $\beta_{t}$ are predictable real-valued processes such that
  $\int_{0}^{t} \alpha_{s}^{2} ds < \infty$ and $\int_{0}^{t}
  |\beta_{s}| ds < \infty$ almost surely for all $t \geq 0$.
\end{defn}

\begin{thm}
  \label{sec:brown-moti-stoch-8}
  Let $X$ be an \ito process and $f: \R \rightarrow \R$ twice
  continuously differentiable.  Then
  \begin{align}
    \label{eq:17}
    f(X_{t}) = f(X_{0}) + \int_{0}^{t} f'(X_{s}) \alpha_{s} dW_{s} +
    \int_{0}^{t} [f'(X_{s}) \beta_{s} + \frac{1}{2} f''(X_{s})
    \alpha_{s}^{2}] ds
  \end{align}
\end{thm}

\begin{thm}
  \label{sec:brown-moti-stoch-9}
  Let $X$ be an \ito process. There exists a continuous non-decreasing
  process $\IP{X}$ called the quadratic variation of $X$, such that
  \begin{align}
    \label{eq:18}
    \IP{X}_{t} = \lim_{N} \sum_{n=1}^{N} (X_{\frac{nt}{N}} - X_{\frac{(n-1)t}{N}})^{2}
  \end{align} for each $t \geq 0$, where the limit is in probability.
  If
  \begin{align}
    \label{eq:19}
    dX_{t} = \alpha_{t} dW_{t} + \beta_{t} dt,
  \end{align} then
  \begin{align}
    \label{eq:20}
    d \IP{X}_{t} = \alpha_{t}^{2} dt
  \end{align}
\end{thm}

\begin{thm}
  \label{sec:brown-moti-stoch-10}
  Let $f : \R_{+} \times \R^{n} \rightarrow \R$ where $(t, x) \mapsto
  f(t, x)$ is continuously differentiable in $t$ and
  twice-continuously differentiable in $x$.  Then
  \begin{align}
    \label{eq:21}
    df(t, X_{t}) = \frac{\partial f}{\partial t} (t, X_{t}) dt +
    \sum_{i=1}^{n} \frac{\partial f}{\partial x_{i}} (t, X_{t})
    dX_{t}^{(i)} + \frac{1}{2} \sum_{i=1}^{n} \sum_{j=1}^{n}
    \frac{\partial^{2} f}{\partial x_{i} \partial x_{j}}(t, X_{t}) d
    \IP{X^{(i)}, X^{(j)}}
  \end{align}
\end{thm}

\begin{thm}
  \label{sec:brown-moti-stoch-11}
  Let $W_{t}$ be an $m$-dimensional Brownian motion, with
  \begin{align}
    \label{eq:22}
    Z_{t} = \exp(-\frac{1}{2} \int_{0}^{t} \| \alpha_{s} \|^{2} ds +
    \int_{-}^{t} \alpha_{s} \cdot dW_{s})
  \end{align} and $Z_{t}$be a martingale.  Let $\Q$ be the equivalent
  measure with density $\frac{d\Q}{d\Prob} = Z_{T}$.  Then the
  $m$-dimensional process $(\hat W_{t})$ defined by $\hat W_{t} =
  W_{t} - \int_{0}^{t} \alpha_{s} ds$ is a Brownian motion on
  $(\Omega, \mathcal{F}_{T}, \Q)$.
\end{thm}

\begin{thm}
  \label{sec:brown-moti-stoch-12}
  If $\E{\exp(\frac{1}{2} \int_{0}^{T} \| \alpha_{s} \|^{2} ds)} <
  \infty$, then $\E{\exp(-\frac{1}{2} \int_{0}^{T} \|\alpha_{s}\|^{2}
    ds + \int_{0}^{T} \alpha_{s} \cdot dW_{s})} = 1$.
\end{thm}

\begin{thm}
  \label{sec:brown-moti-stoch-13}
  Let $(\Omega, \mathcal{F}, \Pi)$ be a probability space with an
  $m$-dimensional Brownian motion $W$ and filtration $\mathcal{F}_{t}$
  generated by $W$.  Let $X$ be a continuous local martingale. Then
  there exists a unique predictable $m$-dimensional process
  $\alpha_{t}$ such that $\int_{0}^{t} \|\alpha_{s}\|^{2} ds < \infty$
  almost surely for all $t \geq 0$ and $X_{t} = X_{0} + \int_{0}^{t}
  \alpha_{s} \cdot dW_{s}$. Furthermore, if $X_{t} > 0$ for all $t
  \geq 0$ then there exists a predictable $\beta$ such that
  $\int_{0}^{t} \| \beta_{s}\|^{2} < \infty$ and
  \begin{align}
    \label{eq:23}
    X_{t} = X_{0} \exp(-\frac{1}{2} \int_{0}^{T} \| \beta_{s}\|^{2} ds
    + \int_{0}^{T} \beta_{s} \cdot dW_{s}).
  \end{align}
\end{thm}

\section{Arbitrage Theory for Continuous-Time Models}
\label{sec:arbitr-theory-cont}

\begin{defn}
  \label{sec:arbitr-theory-cont-2}
  An $(n+1)$-dimensional predictable process $(H, c)$ such that $H$ is
  $P$-integrable is a self-financing investment/consumption strategy
  if and only if $d(H_{t} \cdot P_{t}) = H_{t} \cdot dP_{t} - c_{t}
  dt$.  The wealth associated with a self-financing strategy $H$ is
  $X_{t} = H_{t} \cdot P_{t} = X_{0}(H) + \int_{0}^{t} H_{s} \cdot
  dP_{s} - \int_{0}^{t} c_{s} ds$.
\end{defn}

\begin{defn}
  \label{sec:arbitr-theory-cont-3}
  A trading strategy $H$ is $L$-admissible if and only if the
  associated wealth process $X(H)$ is such that $X_{t}(H) \geq -L_{t}$
  for all $t \geq 0$ a.s. where $L$ is a given continuous non-negative
  adapted process.
\end{defn}

\begin{defn}
  \label{sec:arbitr-theory-cont-4}
  An admissible investment/consumption strategy $(H, c)$ is called an
  absolute arbitrage if and only if there is a non-random time $T$
  such that $X_{0}(H) = 0 = X_{T}(H)$ a.s. and $\Prob{\int_{0}^{T}
    c_{s} ds > 0} > 0$.
\end{defn}

\begin{defn}
  \label{sec:arbitr-theory-cont-5}
  A state price density is a positive \ito process $Y$ such that $YP$
  is an $n$-dimensional local martingale.
\end{defn}

\begin{thm}
  \label{sec:arbitr-theory-cont-6}
  If there exists a state price density $Y$ such that $YL$ is locally
  of class $D$, then there are no $L$-admissible absolute arbitrages.
\end{thm}

\begin{proof}
  Using the self-financing condition $d X_{t} = d (H_{t} \cdot dP_{t})
  = H_{t} \cdot dP_{t} - c_{t} dt$, we obtain $d(X_{t} Y_{t}) = H_{t}
  \cdot d(Y_{t} P_{t}) - Y_{t} c_{t} dt$.

  Then if $YL$ is of class $D$, we can show $\E{\int_{0}^{T} H_{s}
    \cdot d(Y_{s} P_{s}) + L_{T} Y_{T}} = \E{L_{T} Y_{T}}$ be Fatau's,
  stopped local martingales, and uniform integrability.  Then we show
  $\E{\int_{0}^{T} Y_{s} c_{s} ds} = \E{\int_{0}^{T} H_{s} \cdot
  (dY_{s} P_{s})} \leq 0$ and so $c_{t} = 0$ a.s.
\end{proof}

\begin{defn}
  \label{sec:arbitr-theory-cont-7}
  A family of random variables $\mathcal{Z}$ is called uniformly
  integrable if and only if $\lim_{k \rightarrow \infty} \sup_{Z \in
    \mathcal{Z}} \E{|Z| \I{|Z| > k}} = 0$.
\end{defn}

\begin{thm}
  \label{sec:arbitr-theory-cont-8}
  Let $Z_{1}, \dots, Z_{n}$ be a family of integrable random
  variables. The following statements are equivalent:
  \begin{enumerate}
  \item $Z_{n} \rightarrow Z_{\infty}$ in $L^{1}$, and
  \item $(Z_{n})$is uniformly integrable and $Z_{n} \rightarrow
    Z_{\infty}$ in probability.
  \end{enumerate}
\end{thm}

\begin{defn}
  \label{sec:arbitr-theory-cont-9}
  A continuous adapted process $Z$ is of class $D$ if the family of
  random variables $\{ Z_{\tau} \}$ with $\tau$ a finite stopping time
  is uniformly integrable.  A process is locally of class $D$ if $\{
  Z_{\tau \wedge t} \} $ for $\tau$ a stopping time is uniformly
  integrable for each $t \geq 0$.

  If $\E{\sup_{0 \leq s \leq t}|Z_{s}|} < \infty$ for each $t \geq 0$,
  then $Z$ is locally of class $D$.  If $Z$ is a martingale, then $Z$
  is locally of class $D$.
\end{defn}

\begin{thm}
  \label{sec:arbitr-theory-cont-1}
  Suppose $(H, c)$ is a self-financing investment/consumption strategy
  and let $X_{t} = H_{t} \cdot P_{t} = X_{0} + \int_{0}^{t} H_{s}
  \cdot dP_{s} - \int_{0}^{t} c_{s} ds$.  Then $d(X_{t} Y_{t}) = H_{t}
  \cdot d(Y_{t} P_{t}) - Y_{t} c_{t} dt$ for any \ito process $Y$.
\end{thm}

\begin{defn}
  \label{sec:arbitr-theory-cont-10}
  A relative arbitrage is a pure investment strategy with wealth
  process $X$ such that there is a non-random time $T > 0$ satisfying
  $\frac{X_{T}}{N_{T}} \geq \frac{X_{0}}{N_{0}}$ a.s and
  $\Prob{\frac{X_{T}}{N_{T}} > \frac{X_{0}}{N_{0}}} > 0$.
\end{defn}

\begin{defn}
  \label{sec:arbitr-theory-cont-11}
  An equivalent (local) martingale measure relative to the numeraire
  with price $N$ is a probability measure $\Q$ equivalent to $\Prob$ such
  that $\frac{S}{N}$ is a local martingale.
\end{defn}

\begin{thm}
  \label{sec:arbitr-theory-cont-12}
  Let $\Q$ be an equivalent local martingale measure.  Suppose
  $\frac{L}{N}$ is locally in $\Q$-class $D$.  Then there are no
  $L$-admissible relative arbitrages.
\end{thm}

\begin{thm}
  \label{sec:arbitr-theory-cont-13}
  There exist continuous time markets that have relative arbitrage but
  no absolute arbitrage.
\end{thm}

\begin{thm}
  \label{sec:arbitr-theory-cont-14}
  Let $\lambda$ be a predictable $m$-dimensional process such that
  $\int_{0}^{t} \| \lambda_{s} \|^{2}ds < \infty$ a.s. for all $t \geq
  0$ and that $\sigma_{t} \lambda_{t} = \mu_{t} - r_{t} \mathbf{1}$.
  Let $Y_{t} = Y_{0} \exp(-\int_{0}^{t} (r_{s} + \frac{\|
    \lambda_{s}\|^{2}}{2}) ds - \int_{0}^{t} \lambda_{s} \cdot
  dW_{s})$ for a constant $Y_{0} > 0$ - or equivalently, $dY_{t} =
  Y_{t}(-r_{t} dt - \lambda_{t} \cdot dW_{t})$.  Then $Y$ is a
  state-price density.  Furthermore, if the filtration is generated by
  the $m$-dimensional Brownian motion $W$, all state price densities
  have this form.
\end{thm}

\begin{proof}
  Show $YB$ and $YS$ are local martingales. 
  
  Martingale representation theorem shows that all are of the form
  $M_{t} = M_{0} \exp(-\frac{1}{2} \int_{0}^{t} \| \lambda_{s} \|^{2}
  ds - \int_{0}^{t} \lambda_{s} \cdot dW_{s})$. 
\end{proof}
\begin{thm}
  \label{sec:arbitr-theory-cont-15}
  Suppose $\lambda$ is a predictable process with $\sigma_{t}
  \lambda_{t} = \mu_{t} - r_{t} \mathbf{1}$.  If $M_{t} =
  e^{-\frac{1}{2} \int_{-}^{t} \| \lambda_{s} \|^{2}ds - \int_{0}^{t}
    \lambda_{s} \cdot dW_{s}}$ is a true martingale, then the measure
  $\Q$ defined by $\frac{d\Q}{d\Prob} = M_{T}$ is an equivalent
  martingale measure.  In particular, the stock price dynamics
  are given by
  \begin{align}
    \label{eq:6}
    dS^{i}_{t} = S^{i}_{t}(r_{t} dt + \sum_{j} \sigma^{ij}_{t} d\hat W_{t}^{j})
  \end{align} where $\hat W_{t} = W_{t} + \int_{0}^{t} \lambda_{s} ds$
  is a $\Q$ Brownian motion.
\end{thm}

\section{Hedging Contingent Claims in Continuous Time Models}
\label{sec:hedg-cont-claims}
\begin{thm}
  \label{sec:hedg-cont-claims-1}
  Suppose $m=d$ and the $d \times d$ matrix $\sigma_{t}$ is invertible
  for all $t, \omega$ so that in particular, there is a unique (up to
  scaling) state price density $Y$ of the form $dY_{t} = Y_{t}(-r_{t}
  dt - \lambda_{t} \cdot dW_{t})$ where $\lambda_{t} =
  \sigma_{t}^{-1}(\mu_{t} - r_{t} \mathbf{1})$.

  Let $\xi_{T}$ be non-negative, $\mathcal{F}_{T}$-measurable, and
  such that $\xi_{T} Y_{T}$ is integrable.  Then there exists a
  $0$-admissible strategy $H$ with initial cost $X_{0}(H) = \E{Y_{T}
    \xi_{T}}{Y_{0}}$ which replicates the European claim with payout
  $\xi_{T}$.

  Furthermore, if $LY$ is locally of class $D$and $\tilde H$ is an
  $L$-admissible strategy replicating the claim, then $X_{0}(\tilde H)
  \geq X_{0}(H)$.
\end{thm}


\begin{defn}
  \label{sec:hedg-cont-claims-2}
  The Black-Scholes model is given by the pair of equations
  \begin{align}
    \label{eq:24}
    dB_{t} = B_{t} r dt \\
    dS_{t} = S_{t}(\mu dt + \sigma dW_{t})
  \end{align}

  Consider pricing a European option with payoff $\xi_{T} = g(S_{T})$.
  The unique state price density with $Y_{0} = 1$ is given by $Y_{t} =
  \exp((r - \frac{\lambda^{2}}{2})t - \lambda W_{t}$ with $\lambda =
  \frac{\mu - r}{\sigma}$.


  Thus, the is a trading strategy $H$ which replicates the payout with
  \begin{align}
    \label{eq:25}
    X_{t}(H) = \frac{1}{Y_{t}} \E{Y_{T} g(S_{T}) | \mathcal{F}_{t}}.
  \end{align}

  The EMM $\Q$ is given by the density $\frac{d\Q}{d\Prob} =
  \exp(-\frac{\lambda^{2} T}{2} - \lambda W_{T})$.
\end{defn}

\begin{thm}
  \label{sec:hedg-cont-claims-3}
  Suppose that the function $V: [0, T] \times \R^{d} \rightarrow [0,
  \infty)$ satisfies the PDE
  \begin{align}
    \label{eq:26}
    \frac{\partial V}{\partial t} + \sum_{i=1}^{d} r S^{i}
    \frac{\partial V}{\partial S^{i}} + \frac{1}{2} \sum_{i=1}^{d}
    \sum_{j=1}^{d} a_{i, j} S^{i} S^{j} \frac{\partial^{2} V}{\partial
      S^{i} \partial S^{j}} = rV
  \end{align} and $V(T, S) = g(S)$.

  Then there exists a $0$-admissible strategy $H$ such that $X_{t}(H)
  = V(t, S_{t})$. In particular, this strategy replicates the
  contingent claim with payout $g(S_{T})$.

  Furthermore, if $H = (\phi, \pi)$, then the strategy can be
  calculated as
  \begin{align}
    \label{eq:27}
    \pi_{t} = \grad V(t, S_{t}) = (\frac{\partial V}{\partial
      S_{1}}(t, S_{t}), \dots, \frac{\partial V}{\partial S^{d}}(t,
    S_{t}))
  \end{align} and $\phi_{t} = \frac{V(t, S_{t}) - \pi_{t} \cdot
    S_{t}}{B_{t}}$.
\end{thm}

\begin{proof}
  Ito's formula on $dV(t, S_{t})$, and show $V(t, S_{t})= \phi_{t}
  B_{t} + \pi_{t} S_{t}$, and $dV(t, S_{t}) = \phi_{t} dB_{t} =
  \pi_{t} \cdot dS_{t}$, so $H = (\phi, \pi)$ is a self-financing
  strategy that replicates $V(t, S_{t})$ as required.
\end{proof}

\begin{thm}
  \label{sec:hedg-cont-claims-4}
  Suppose that $C_{0}(T, K) = \E{\Q}{e^{-rT}(S_{T} - K)^{+}}$.  Then
  $\frac{\partial C_{0}}{\partial T}(T, K) + rK \frac{\partial
    C_{0}}{\partial K}(T, K) = \frac{\sigma(T, K)^{2}}{2} K^{2}
  \frac{\partial^{2} C_{0}}{\partial K^{2}}(T, K)$.
\end{thm}

\begin{thm}
  Assume that a banker hedges an option assuming constant volatility,
  and delta hedges with wealth evolving with $dX_{t} = r(X_{t} -
  \pi_{t} S_{t}) + \pi_{t} S_{t}$, with $\pi_{t} = V_{S}(t, S_{t},
  \hat \sigma)$.  If the true dynamics are $dS_{t} = S_{t}(\mu dt +
  \sigma_{t} dW_{t})$, then using the fact that $V$ solves the BS PDE
  and that $dV_{t} = rV dt + \pi_{t}(dS_{t} - r S_{t} dt) +
  \frac{1}{2} S_{t}^{2}(\sigma_{t}^{2} - \hat \sigma^{2}) V_{SS} dt$,
  we obtain
  \begin{equation}
    \label{eq:32}
    X_{T} - g(S_{T}) = \frac{1}{2} \int_{0}^{T} e^{r(T-t)}(\hat
    \sigma^{2} - \sigma^{2}_{t}) S_{t}^{2} V_{SS}(t, S_{t}, \hat
    \sigma) dt
  \end{equation}
\end{thm}

\section{Interest Rate Models}
\label{sec:interest-rate-models}
\begin{defn}
  \label{sec:interest-rate-models-1}
  A zero-coupon bond with maturity $T$ is a European contingent claim
  that pays one unit of currency at time $T$.  $P(t, T)$ is the price
  at time $t \in [0, T]$ of the bond.

  The yield $y(t, T)$ is defined by $y(t, T) = - \frac{1}{T- t} \log
  P(t, T)$.

  The forward rate $f(t, T)$ is defined by $f(t, T) = -
  \frac{\partial}{\partial T} \log P(t, T)$.

  Note that $P(t, T) = e^{-(T-t) y(t, T)} = e^{-\int_{t}^{T} f(t, s)
    ds}$.
\end{defn}

\begin{thm}
  \label{sec:interest-rate-models-3}
  Let $dB_{t} = B_{t} r_{t} dt$ where $r_{t}$ is the sport interest
  rate.  Then there is no arbitrage relative to the numeraire if there
  exists an equivalent measure $\Q$ such that the discounted bond
  price process $\frac{P(t, T)}{B_{t}}_{t \in 0, T}$ is a local
  martingale for all $T > 0$.  IN particular, there is no arbitrage if
  $P(t, T) = \E{\Q}{\exp(-\int_{t}^{T} r_{s} ds) | \mathcal{F}_{t}}$
  for all $0 \leq t \leq T$.
\end{thm}

\begin{defn}
  \label{sec:interest-rate-models-4}
  The Vasicek model is $dr_{t} = \lambda(\overline r - r_{t}) dt +
  \sigma d\hat W_{t}$.

  We have $\E{\Q}(r_{t}) = e^{-\lambda t} r_{0} + (1 - e^{-\lambda t})
  \overline r$, $\Var^{\Q}(r_{t}) = \int_{0}^{t} e^{-2\lambda(t-s)}
  \sigma^{2} ds = \frac{\sigma^{2}}{2\lambda} (1 - e^{-2\lambda t})$.

  Indeed, we can deduce that
  \begin{align}
    \label{eq:28}
    f(t, t + x) = r_{t} e^{-\lambda x} + \overline r(1 - e^{-\lambda
      x}) - \frac{\sigma^{2}}{2 \lambda^{2}} (1 - e^{-\lambda x})^{2}
  \end{align}
\end{defn}

\begin{thm}
  \label{sec:interest-rate-models-5}
  Consider now where the short rate is Markovian, and so $dr_{t} =
  \alpha(t, r_{t}) dt + \beta(t, r_{t}) d \hat W_{t}$ for non-random
  function $\alpha, \beta$.

  If we fix $T > 0$ and let $V: [0, T] \times \R \rightarrow \R$
  satisfy the PDE
  \begin{align}
    \label{eq:29}
    \frac{\partial V}{\partial t}(t, r) + \alpha(t, r) \frac{\partial
      V}{\partial r}(t, r) + \frac{1}{2} \beta(t, r)^{2}
    \frac{\partial^{2} V}{\partial r^{2}}(t, r) = rV(t, r)
  \end{align} with $V(T, r) = 1$.  Assume $P(t, T) = V(t, r_{t})$.
  Then the discounted price process $\exp(-\int_{-}^{t} r_{s} ds) P(t,
  T)$ is a $\Q$-local martingale.
\end{thm}

\bibliographystyle{plainnat}
\bibliography{../../common/bibliography}
\end{document}
