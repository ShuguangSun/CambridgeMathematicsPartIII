\chapter{Introduction}
\label{cha:introduction}

Consider observations $X_{1}, \dots, X_{n}$ are copies of a random
variable (r.v.) with distribution $$F(t) = P(X \leq t), t \in
\mathbb{R}.$$

\begin{defn}
  A statistical model is a family of candidate distributions 
  $$\mathbb{P}_{\Theta} = \{ P_{\theta} | \theta \in \Theta \}$$ where
  $\Theta$ is a parameter space.
\end{defn}
\begin{exmp}[Linear Regression]
  Consider
  \begin{equation}
    \label{eq:1}
    \mathbf{Y} = \mathbf{X \Theta} + \epsilon
  \end{equation}
  $\mathbf{X}$ is our design matrix, $\mathbf{X}_{i}$ are our
  explanatory variables, $\mathbf{Y}$ is our response, $\epsilon$ is
  our measurement error (e.g. $\epsilon ~ N(0, \sigma^{2})$).
\end{exmp}


\begin{exmp}[Nonlinear regression]
  \begin{equation}
    \label{eq:2}
    \mathbf{Y} = g(\mathbf{X}, \mathbf{\Theta}) + \epsilon
  \end{equation}
\end{exmp}

\begin{exmp}[High dimensional linear model]
  \begin{equation}
    \label{eq:3}
    \mathbf{Y} = \mathbf{X \Theta} + \epsilon
  \end{equation}
  with $\mathbf{\Theta}$ being sparse.

  If $\mathbf{X}$ has some ``properties'' (restricted isometry property), then ``miracle happens''.
\end{exmp}

\section{Asymptotic Statistics}
\label{sec:asympt-stat}

Investigate statistical problems where the sample size $n$ is large
($n \rightarrow \infty$).
\begin{enumerate}
\item Quick intuitions (about the complexity of $\Theta$.
\item Large sample approximation can often proven to be good
  (concentration of measure).
\item Nice mathematics.
\item For $n$ finite, $\Theta$ can be useless.
\item Does not consider computation cost.
\end{enumerate}

\section{Statistical Inference}
\label{sec:stat-infer}

\begin{enumerate}
\item Estimation: construct $\hat{\theta}_{n} - \hat{\theta}(X_{1},
  \dots, X_{n})$, that estimates (approximates) $\theta$ well when
  $X_{i} \sim P_{\theta}$.
\item Hypothesis testing: $H_{0}: \theta = \theta_{0}$ vs $H_{1}:
  \theta \neq \theta_{0}$.  Find test/decision rule $\psi_{n} =
  \psi(X_{1}, \dots, X_{n})$ such that $\psi_{n} = 0$ if $H_{0}$ is true.
\item Confidence Sets: Find $C_{n} = C(X_{1}, \dots, X_{n}, \alpha)
  \subseteq \Theta, 0 \leq \alpha \leq 1$ such that $P_{\theta}(\theta
  \in C_{n}) = 1-\alpha$ for all $n \in \mathbb{N}$.
  This is \textbf{uncertainty quantification}.
\end{enumerate}

%%% Local Variables: 
%%% mode: latex
%%% TeX-master: "master"
%%% End: 
