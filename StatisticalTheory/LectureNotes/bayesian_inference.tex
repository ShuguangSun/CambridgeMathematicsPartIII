
\chapter{Bayesian Inference}
\label{cha:bayesian-inference}

In any parametric model $\{ f(\cdot, \theta), \theta \in \Theta \}$,
we can consider a prior distribution $\Pi$ on $\Theta$, and model the
observations $X_{1}, \dots, X_{n}$ as IID copies of the random
variable $X|\theta \sim f(\cdot, \theta)$, where $\theta \sim \Pi$.
The posterior distribution is the law of $\theta | X_{1}, \dots, X_{n}$.

Formally, if $\mathcal{X}$ is the sample space that $X$ takes values
in, consider on $\mathcal{X} \times \Theta$ the probability
distribution $Q$ with pdf/pmf by $dQ(x, \theta) = f(x, \theta)
\Pi(\theta) dx d\theta$ by the laws/definition of conditional
probability,
\begin{equation}
  \label{eq:100}
  X|\theta \sim \frac{f(x, \theta) \Pi(\theta)
    dx}{\int_{\mathcal{X}}f(x, \theta) dx \Pi(\theta)} = f(x, \theta) dx
\end{equation} and conversely
\begin{equation}
  \label{eq:100}
  \theta | X \sim \frac{f(x, \theta) \Pi(\theta)
    d\theta}{\int_{\Theta} f(x, \theta) \Pi(\theta) d\theta} =
  \Pi(\theta | X).
\end{equation} In particular, for $(X_{i}, i = 1, \dots, n)$ \iid
copies of $X|\theta$, the posterior distribution equals
\begin{equation}
  \label{eq:100}
  \theta | X_{1}, \dots, X_{n} \sim \frac{\prod_{i=1}^{n} f(x_{i},
    \theta) \Pi(\theta)}{\int_{\Theta} \prod_{i=1}^{n}f(X_{i}, \theta)
  \Pi(\theta) d\theta}
\end{equation}


The posterior distribution can be used for all purposes of statistical
inference on $\theta$:

\begin{enumerate}
\item
  \begin{equation}
    \label{eq:100}
    \bar \theta(X_{1}, \dots, X_{n}) = \E{\theta | X_{1}, \dots, X_{n}}
  \end{equation} estimates $\theta$,
\item
  \begin{equation}
    \label{eq:100}
    C_{n} = \{ \theta \in \Theta : \| \theta - \bar \theta \| \leq
    R_{n} \}
  \end{equation} where $R_{n}$ is such that $\Pi(C_{n}| X_{1}, \dots,
  X_{n}) = 1 - \alpha$, giving a credible set for $\Theta$.
\end{enumerate}

\begin{thm}[Bernstein-von Mises theorem]
  \label{defn:bayesian_inference:1}
  The Bernstein-von Mises theorem states that in LAN-models $\{
  f(\cdot, \theta), \theta \in \Theta \}$ and for \textbf{any} prior
  that has a positive continouus density at $\theta_{0}$, we have
  \begin{align}
    \label{eq:100}
    \Pi{\cdot, X_{1}, \dots, X_{n}} \approx N(\hat \theta_{MLE}, \frac{1}{n}I(\theta_{0})^{-1})
  \end{align} under $P_{\theta_{0}}^{n}$ (in total variation
  distance), which in particular implies that any credible set $C_{n}$
  such that $\Pi(C_{n} | X_{1}, \dots, X_{n}) = 1 - \alpha$ satisfies
  $P_{\theta_{0}}^{n}(\theta_{0} \in C_{n}) \rightarrow 1 - \alpha$ as
  $n \rightarrow \infty$.  In particular, asymptotic Bayesian inference
  coincides with asymptotic inference based on the MLE.
\end{thm}



%%% Local Variables: 
%%% mode: latex
%%% TeX-master: "master"
%%% End: 
