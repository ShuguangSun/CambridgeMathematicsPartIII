\documentclass[8pt, reqno, oneside, twocolumn]{amsart} 

% Utility packages
\usepackage{etoolbox}
\newtoggle{tufte}
\togglefalse{tufte}
\newtoggle{summary}
\togglefalse{summary}
\usepackage{amsthm}
\usepackage{amsmath}
\usepackage{amssymb}
\usepackage{comment}
% \usepackage[landscape]{geometry}
\usepackage[top=1cm, bottom=1cm, left=1cm, right=1cm]{geometry}
% \usepackage[inner=1cm, outer=1cm]{geometry}
\usepackage{setspace}
\usepackage{graphicx}
\usepackage{enumerate}
\usepackage{listings}
\usepackage{booktabs}
\usepackage{hyperref}
\usepackage{thmtools}
\usepackage{xparse}
\usepackage{xspace}
\usepackage{enumitem}
\usepackage{todonotes}
\usepackage{dcolumn}
\usepackage{array}
\usepackage{booktabs}
\usepackage{colortbl, xcolor}
\usepackage{longtable}
\usepackage{accents}
\usepackage{natbib}
\usepackage{python}
% \usepackage{a4wide}

\hypersetup{colorlinks=true}
\numberwithin{equation}{section}

\declaretheorem[numbered=no,name=Theorem,style=definition]{thm}
\declaretheorem[numbered=no,name=Lemma]{lem}
\declaretheorem[numbered=no,name=Corollary]{corollary}
\declaretheorem[numbered=no,name=Definition]{defn}
\declaretheorem[numbered=no,name=Proposition]{proposition}
\declaretheorem[numbered=no,name=Example]{exmp}
\declaretheorem[numbered=no,name=Exercise]{exer}
\declaretheorem[numbered=no,name=History]{history}
\declaretheorem[numbered=no,name=Question]{question}
\declaretheorem[numbered=no,name=Remark]{remark}
\declaretheorem[numbered=no,name=Notation]{notation}
\declaretheorem[numbered=no,name=Theorem]{boxthm}

\let\proof\relax
\declaretheorem[style=remark,numbered=no,qed=\qedsymbol]{proof}

\DeclareMathOperator*{\argmax}{arg\,max}
\DeclareMathOperator*{\dist}{dist}
\DeclareMathOperator*{\V}{V}
\DeclareMathOperator*{\saddlepoint}{sp}
\DeclareMathOperator*{\divergence}{div}
\DeclareMathOperator*{\sym}{Sym}
\DeclareMathOperator*{\vol}{Vol}
\DeclareMathOperator*{\argmin}{arg\,min}
\DeclareMathOperator*{\proj}{\Pi}
\DeclareMathOperator*{\bnd}{bnd}
\DeclareMathOperator*{\supp}{supp}
\DeclareMathOperator*{\aff}{aff}
\DeclareMathOperator*{\rint}{rint}
\DeclareMathOperator*{\lev}{lev}
\DeclareMathOperator*{\tr}{tr}
\DeclareMathOperator*{\interior}{int}
\DeclareMathOperator*{\epi}{epi}
\DeclareMathOperator*{\dom}{dom}
\DeclareMathOperator*{\vecspan}{span}
\DeclareMathOperator*{\con}{con}
\DeclareMathOperator*{\cl}{cl}
\DeclareMathOperator*{\grad}{\nabla}
\DeclareMathOperator*{\crit}{crit}
\DeclareMathOperator*{\diag}{diag}
\DeclareMathOperator*{\res}{Res}
\DeclareMathOperator*{\boundary}{boundary}
\DeclareMathOperator*{\range}{range}
\DeclareMathOperator*{\Tr}{Tr}
\DeclareMathOperator*{\sign}{sign}
\setlist[enumerate]{label=(\roman*)}

\newlist{exercises}{enumerate}{1}
\setlist[exercises]{label=Ex. \arabic*}


\setcounter{secnumdepth}{2}
\setcounter{tocdepth}{2}
% \onehalfspacing

\author{Andrew Tulloch}


% Shortcuts
% convergence in probability
\newcommand{\cd}{\overset{d}{\rightarrow}}
% convergence in distribution
\newcommand{\cp}{\overset{p}{\rightarrow}}
% convergence almost surely
\newcommand{\cas}{\overset{as}{\rightarrow}}
\newcommand{\R}{\mathbb{R}}
\newcommand{\N}{\mathbb{N}}
\newcommand{\Z}{\mathbb{Z}}
\newcommand{\np}{\textsc{np}\xspace}
\newcommand{\iid}{\textsc{iid}\xspace}
\newcommand{\knn}{$k$-\textsc{nn}\xspace}
\newcommand{\leqtilde}{\underset{\sim}{<}}

% Expectation
\NewDocumentCommand{\E}{gg}{%
  \IfNoValueTF{#1}
    {%
      \mathbb{E}
    }    
    {%
      \IfNoValueTF{#2}
      {%
        \mathbb{E}\!\left(#1\right)
      }
      {%
        \mathbb{E}_{#2}\!\left(#1\right)
      }
    }%
}

% Probability with measure P
\NewDocumentCommand{\Prob}{gg}{%
  \IfNoValueTF{#1}
    {%
      \mathbb{P}
    }    
    {%
      \IfNoValueTF{#2}
      {%
        \mathbb{P}\!\left(#1\right)
      }
      {%
        \mathbb{P}_{#2}\!\left(#1\right)
      }
    }%
}

% Probability with measure Q
\NewDocumentCommand{\Q}{g}{%
  \IfNoValueTF{#1}
    {%
      \mathbb{Q}
    }    
    {%
      \mathbb{Q}\!\left(#1\right)
    }%
}


% Indicator
\NewDocumentCommand{\I}{g}{%
  \IfNoValueTF{#1}
    {%
      \mathbb{I}
    }    
    {%
      \mathbb{I}\!\left(#1\right)
    }%
}


\NewDocumentCommand{\Var}{g}{%
  \IfNoValueTF{#1}
    {%
      \mathbb{V}
    }    
    {%
      \mathbb{V}\!\left(#1\right)
    }%
}


\NewDocumentCommand{\Cov}{gg}{%
  \IfNoValueTF{#1}
    {%
      \text{Cov}
    }   
    {%
      \text{Cov}\!\left(#1, #2\right)
    }%
}

% Inner product
\NewDocumentCommand{\IP}{g}{%
    {%
      \left \langle #1 \right \rangle
    }%
}
%%% Local Variables: 
%%% mode: latex
%%% End: 
